\documentclass[12pt]{article}
\usepackage{pmmeta}
\pmcanonicalname{MonomorphismsOfCategoryOfSets}
\pmcreated{2013-03-22 16:42:41}
\pmmodified{2013-03-22 16:42:41}
\pmowner{rspuzio}{6075}
\pmmodifier{rspuzio}{6075}
\pmtitle{monomorphisms of category of sets}
\pmrecord{7}{38927}
\pmprivacy{1}
\pmauthor{rspuzio}{6075}
\pmtype{Theorem}
\pmcomment{trigger rebuild}
\pmclassification{msc}{18-00}

\endmetadata

% this is the default PlanetMath preamble.  as your knowledge
% of TeX increases, you will probably want to edit this, but
% it should be fine as is for beginners.

% almost certainly you want these
\usepackage{amssymb}
\usepackage{amsmath}
\usepackage{amsfonts}

% used for TeXing text within eps files
%\usepackage{psfrag}
% need this for including graphics (\includegraphics)
%\usepackage{graphicx}
% for neatly defining theorems and propositions
\usepackage{amsthm}
% making logically defined graphics
%%%\usepackage{xypic}

% there are many more packages, add them here as you need them

% define commands here
\newtheorem{theorem}{Theorem}
\begin{document}
\begin{theorem}
Every monomorphism in the category of sets is an injection.
\end{theorem}

\begin{proof}
Assume $f \colon A \to B$ is a monomorphism.  Then, by definition of monomorphism,
given any two maps $g,h \colon C \to A$, if $f \circ g = f \circ h$, then $g = h$.
Suppose $x$ and $y$ are two elements of $A$ such that $f(x) = f(y)$.  Let $C$ be 
a set with one element, let $g$ be the map which sends this one element to $x$ and
let $h$ be the map which sends this one element to $y$.  Because $f(x) = f(y)$, we
have $f \circ g = f \circ h$.  Since $f$ is a monomorphism, $g = h$, so $x = y$.
This implies that $f$ is injective.
\end{proof}

\begin{theorem}
Every injection is a split monomorphism.
\end{theorem}

\begin{proof}
Assume $f \colon A \to B$ is injection.  If $A$ is empty, the result is 
trivial, so we assume that $A$ is not empty; let $z$ be an element of $A$.
Set
\[
g = \{ (f(x),x) \mid x \in A \} \cup
\{(x,z) \mid x \in B \land (\forall y \in A) x \neq f(y)\}
\]
We claim that $g$ is a function from $B$ into $A$.  Suppose that $x$ is 
an element of $B$.  If $x \neq f(y)$ for any $y \in A$, then we have exactly
one element of $g$ with $x$ as the first element, namely $(x,z)$.  If $x = 
f(y)$ for some $y \in A$, then we the pair $(x,y)$ with $x$ as first element;
were there another pair with $x$ as first element, then we would have 
$(f(x_1),x_1) = (f(x_2),x_2)$ but, as $f$ is an injection, $f(x_1) = f(x_2)$
would imply $x_1 = x_2$, so this would not be a distinct pair.  Hence $g$ is
a function.  Furthermore, by construction $g \circ f (x) = x$ for all $x \in A$,
so $f$ is a split monomorphism.
\end{proof}

Note that the second theorem is stronger than a simple converse to the first
theorem --- it states that an injection is not just a monomorphism, but that
it is actually a split monomorphism.  In particular, this means that, in the
category of sets, all monomorphisms are actually split monomorphisms.
%%%%%
%%%%%
\end{document}
