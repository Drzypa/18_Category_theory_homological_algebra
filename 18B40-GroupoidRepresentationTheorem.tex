\documentclass[12pt]{article}
\usepackage{pmmeta}
\pmcanonicalname{GroupoidRepresentationTheorem}
\pmcreated{2013-03-22 18:21:47}
\pmmodified{2013-03-22 18:21:47}
\pmowner{bci1}{20947}
\pmmodifier{bci1}{20947}
\pmtitle{groupoid representation theorem}
\pmrecord{8}{41001}
\pmprivacy{1}
\pmauthor{bci1}{20947}
\pmtype{Feature}
\pmcomment{trigger rebuild}
\pmclassification{msc}{18B40}
\pmclassification{msc}{81T10}
\pmclassification{msc}{81T05}
\pmclassification{msc}{22A22}
\pmclassification{msc}{20L05}
\pmclassification{msc}{20N02}
\pmsynonym{Hahn groupoid representations theorem}{GroupoidRepresentationTheorem}
%\pmkeywords{groupoid representation theorem}
%\pmkeywords{Hahn groupoid representations theorem}
%\pmkeywords{groupoid and group representations}
%\pmkeywords{Haar measure}
%\pmkeywords{quantum symmetries}
%\pmkeywords{measured groupoids}
%\pmkeywords{Haar measure systems}
%\pmkeywords{locally compact (quantum) groups--not Hopf algebras}
%\pmkeywords{locally compact gro}
\pmrelated{GroupoidAndGroupRepresentationsRelatedToQuantumSymmetries}
\pmdefines{groupoid representation}

\endmetadata

% this is the default PlanetMath preamble.  
\usepackage{amssymb}
\usepackage{amsmath}
\usepackage{amsfonts}

% define commands here
\usepackage{amsmath, amssymb, amsfonts, amsthm, amscd, latexsym, enumerate}
\usepackage{xypic, xspace}
\usepackage[mathscr]{eucal}
\usepackage[dvips]{graphicx}
\usepackage[curve]{xy}

\setlength{\textwidth}{6.5in}
%\setlength{\textwidth}{16cm}
\setlength{\textheight}{9.0in}
%\setlength{\textheight}{24cm}

\hoffset=-.75in     %%ps format
%\hoffset=-1.0in     %%hp format
\voffset=-.4in

\theoremstyle{plain}
\newtheorem{lemma}{Lemma}[section]
\newtheorem{proposition}{Proposition}[section]
\newtheorem{theorem}{Theorem}[section]
\newtheorem{corollary}{Corollary}[section]

\theoremstyle{definition}
\newtheorem{definition}{Definition}[section]
\newtheorem{example}{Example}[section]
%\theoremstyle{remark}
\newtheorem{remark}{Remark}[section]
\newtheorem*{notation}{Notation}
\newtheorem*{claim}{Claim}

\renewcommand{\thefootnote}{\ensuremath{\fnsymbol{footnote}}}
\numberwithin{equation}{section}

\newcommand{\Ad}{{\rm Ad}}
\newcommand{\Aut}{{\rm Aut}}
\newcommand{\Cl}{{\rm Cl}}
\newcommand{\Co}{{\rm Co}}
\newcommand{\DES}{{\rm DES}}
\newcommand{\Diff}{{\rm Diff}}
\newcommand{\Dom}{{\rm Dom}}
\newcommand{\Hol}{{\rm Hol}}
\newcommand{\Mon}{{\rm Mon}}
\newcommand{\Hom}{{\rm Hom}}
\newcommand{\Ker}{{\rm Ker}}
\newcommand{\Ind}{{\rm Ind}}
\newcommand{\IM}{{\rm Im}}
\newcommand{\Is}{{\rm Is}}
\newcommand{\ID}{{\rm id}}
\newcommand{\grpL}{{\rm GL}}
\newcommand{\Iso}{{\rm Iso}}
\newcommand{\rO}{{\rm O}}
\newcommand{\Sem}{{\rm Sem}}
\newcommand{\SL}{{\rm Sl}}
\newcommand{\St}{{\rm St}}
\newcommand{\Sym}{{\rm Sym}}
\newcommand{\Symb}{{\rm Symb}}
\newcommand{\SU}{{\rm SU}}
\newcommand{\Tor}{{\rm Tor}}
\newcommand{\U}{{\rm U}}

\newcommand{\A}{\mathcal A}
\newcommand{\Ce}{\mathcal C}
\newcommand{\D}{\mathcal D}
\newcommand{\E}{\mathcal E}
\newcommand{\F}{\mathcal F}
%\newcommand{\grp}{\mathcal G}
\renewcommand{\H}{\mathcal H}
\renewcommand{\cL}{\mathcal L}
\newcommand{\Q}{\mathcal Q}
\newcommand{\R}{\mathcal R}
\newcommand{\cS}{\mathcal S}
\newcommand{\cU}{\mathcal U}
\newcommand{\W}{\mathcal W}

\newcommand{\bA}{\mathbb{A}}
\newcommand{\bB}{\mathbb{B}}
\newcommand{\bC}{\mathbb{C}}
\newcommand{\bD}{\mathbb{D}}
\newcommand{\bE}{\mathbb{E}}
\newcommand{\bF}{\mathbb{F}}
\newcommand{\bG}{\mathbb{G}}
\newcommand{\bK}{\mathbb{K}}
\newcommand{\bM}{\mathbb{M}}
\newcommand{\bN}{\mathbb{N}}
\newcommand{\bO}{\mathbb{O}}
\newcommand{\bP}{\mathbb{P}}
\newcommand{\bR}{\mathbb{R}}
\newcommand{\bV}{\mathbb{V}}
\newcommand{\bZ}{\mathbb{Z}}

\newcommand{\bfE}{\mathbf{E}}
\newcommand{\bfX}{\mathbf{X}}
\newcommand{\bfY}{\mathbf{Y}}
\newcommand{\bfZ}{\mathbf{Z}}

\renewcommand{\O}{\Omega}
\renewcommand{\o}{\omega}
\newcommand{\vp}{\varphi}
\newcommand{\vep}{\varepsilon}

\newcommand{\diag}{{\rm diag}}
\newcommand{\grp}{{\mathsf{G}}}
\newcommand{\dgrp}{{\mathsf{D}}}
\newcommand{\desp}{{\mathsf{D}^{\rm{es}}}}
\newcommand{\grpeod}{{\rm Geod}}
%\newcommand{\grpeod}{{\rm geod}}
\newcommand{\hgr}{{\mathsf{H}}}
\newcommand{\mgr}{{\mathsf{M}}}
\newcommand{\ob}{{\rm Ob}}
\newcommand{\obg}{{\rm Ob(\mathsf{G)}}}
\newcommand{\obgp}{{\rm Ob(\mathsf{G}')}}
\newcommand{\obh}{{\rm Ob(\mathsf{H})}}
\newcommand{\Osmooth}{{\Omega^{\infty}(X,*)}}
\newcommand{\grphomotop}{{\rho_2^{\square}}}
\newcommand{\grpcalp}{{\mathsf{G}(\mathcal P)}}

\newcommand{\rf}{{R_{\mathcal F}}}
\newcommand{\grplob}{{\rm glob}}
\newcommand{\loc}{{\rm loc}}
\newcommand{\TOP}{{\rm TOP}}

\newcommand{\wti}{\widetilde}
\newcommand{\what}{\widehat}

\renewcommand{\a}{\alpha}
\newcommand{\be}{\beta}
\newcommand{\grpa}{\grpamma}
%\newcommand{\grpa}{\grpamma}
\newcommand{\de}{\delta}
\newcommand{\del}{\partial}
\newcommand{\ka}{\kappa}
\newcommand{\si}{\sigma}
\newcommand{\ta}{\tau}

\newcommand{\med}{\medbreak}
\newcommand{\medn}{\medbreak \noindent}
\newcommand{\bign}{\bigbreak \noindent}

\newcommand{\lra}{{\longrightarrow}}
\newcommand{\ra}{{\rightarrow}}
\newcommand{\rat}{{\rightarrowtail}}
\newcommand{\ovset}[1]{\overset {#1}{\ra}}
\newcommand{\ovsetl}[1]{\overset {#1}{\lra}}
\newcommand{\hr}{{\hookrightarrow}}

\newcommand{\<}{{\langle}}

%\newcommand{\>}{{\rangle}}

%\usepackage{geometry, amsmath,amssymb,latexsym,enumerate}
%%%\usepackage{xypic}

\def\baselinestretch{1.1}


\hyphenation{prod-ucts}

%\grpeometry{textwidth= 16 cm, textheight=21 cm}

\newcommand{\sqdiagram}[9]{$$ \diagram  #1  \rto^{#2} \dto_{#4}&
#3  \dto^{#5} \\ #6    \rto_{#7}  &  #8   \enddiagram
\eqno{\mbox{#9}}$$ }

\def\C{C^{\ast}}

\newcommand{\labto}[1]{\stackrel{#1}{\longrightarrow}}

%\newenvironment{proof}{\noindent {\bf Proof} }{ \hfill $\Box$
%{\mbox{}}

\newcommand{\quadr}[4]
{\begin{pmatrix} & #1& \\[-1.1ex] #2 & & #3\\[-1.1ex]& #4&
 \end{pmatrix}}
\def\D{\mathsf{D}}
\begin{document}
\subsection{Groupoid representation theorem}

  We shall briefly consider a main result due to Hahn (1978) that relates 
groupoid and associated groupoid algebra representations:

\begin{theorem} {\rm (source: \cite{Hahn78, Hahn2}.)}~
  Any representation of a groupoid $\grp$ with Haar measure $(\nu,
\mu)$ in a separable Hilbert space $\H$ induces a *-algebra
representation $f \mapsto X_f$ of the associated groupoid algebra
$ \Pi \grp, \nu)$ in $L^2 (U_{\grp} , \mu, \H )$ with the
following properties:
\begin{itemize}
\item[(1)]
For any $l,m \in H $ , one has that $\left|<X_f(u
\mapsto l), (u \mapsto m)>\right|\leq \left\|f_l\right\| \left\|l
\right\| \left\|m \right\|$ and

\item[(2)]  $M_r (\alpha) X_f = X_{f \alpha \circ r}$,  where
$M_r: L^\infty (U_{\grp}, \mu, \H)   \longrightarrow \mathcal L (
L^2 (U_{\grp}, \mu, \H))$, with $M_r (\alpha)j = \alpha \cdot j$.
\end{itemize}
Conversely, any *-algebra representation with the above two
properties induces a groupoid representation, X, as follows:
\begin{equation}
\left\langle X_f , j, k\right\rangle ~ = ~ \int
f(x)[X(x)j(d(x)),k(r(x))d \nu (x)],
\end{equation}
(cf. p. 50 of Hahn, 1978).
\end{theorem}

\subsection{Remarks}

 Furthermore, according to Seda (1986, on p.116) the continuity of a
Haar system is equivalent to the continuity of the convolution
product $f*g$ for any pair $f, g$ of continuous functions with
compact support. One may thus conjecture that similar results
could be obtained for functions with \textit{locally compact}
support in dealing with convolution products of either locally
compact groupoids or quantum groupoids. Seda's result also implies
that the convolution algebra $C_{conv}(\grp)$ of a groupoid $\grp$ is
closed with respect to the convolution {*} if and only if the fixed Haar
system associated with the measured groupoid $\grp$ is
\textit{continuous} (Buneci, 2003).

 In the case of groupoid algebras of transitive groupoids, Buneci
(2003) showed that representations of a measured groupoid
$({\grp, [\int \nu ^u d \tilde{ \lambda}(u)]=[\lambda]})$ on a
separable Hilbert space $\H$ induce  \textit{non-degenerate}
$*$--representations  $f \mapsto  X_f$ of the associated groupoid
algebra $ \Pi (\grp, \nu,\tilde{\lambda})$ with properties
formally similar to (1) and (2) above.  Moreover, as in the case
of groups, \textit{there is a correspondence between the unitary
representations of a groupoid and its associated C*--convolution
algebra representations} (p.182 of Buneci, 2003), the latter
involving however fiber bundles of Hilbert spaces instead of
single Hilbert spaces. Therefore, groupoid representations appear
as the natural construct for algebraic quantum field theories (AQFT) in
which nets of local observable operators in Hilbert space fiber
bundles were introduced by Rovelli (1998).

\begin{thebibliography}{9}

\bibitem{GR02}
R. Gilmore: \emph{Lie Groups, Lie Algebras and Some of Their Applications.},
Dover Publs., Inc.: Mineola and New York, 2005.

\bibitem{Hahn78}
P. Hahn: Haar measure for measure groupoids., \textit{Trans. Amer. Math. Soc}. \textbf{242}: 1--33(1978).
(Theorem 3.4 on p. 50).

\bibitem{Hahn2}
P. Hahn: The regular representations of measure groupoids., \textit{Trans. Amer. Math. Soc}. \textbf{242}:34--72(1978).

\bibitem{HeynLifsctz}
R. Heynman and S. Lifschitz. 1958. \emph{Lie Groups and Lie Algebras}., New York and London: Nelson Press.

\bibitem{HLS2k8}
C. Heunen, N. P. Landsman, B. Spitters.: A topos for algebraic quantum theory, (2008);
  arXiv:0709.4364v2 [quant--ph]

\end{thebibliography}
%%%%%
%%%%%
\end{document}
