\documentclass[12pt]{article}
\usepackage{pmmeta}
\pmcanonicalname{KernelPair}
\pmcreated{2013-03-22 18:20:34}
\pmmodified{2013-03-22 18:20:34}
\pmowner{CWoo}{3771}
\pmmodifier{CWoo}{3771}
\pmtitle{kernel pair}
\pmrecord{10}{40976}
\pmprivacy{1}
\pmauthor{CWoo}{3771}
\pmtype{Definition}
\pmcomment{trigger rebuild}
\pmclassification{msc}{18A30}
\pmrelated{KernelOfAHomomorphismBetweenAlgebraicSystems}
\pmdefines{cokernel pair}

\usepackage{amssymb,amscd}
\usepackage{amsmath}
\usepackage{amsfonts}
\usepackage{mathrsfs}

% used for TeXing text within eps files
%\usepackage{psfrag}
% need this for including graphics (\includegraphics)
%\usepackage{graphicx}
% for neatly defining theorems and propositions
\usepackage{amsthm}
% making logically defined graphics
%%%\usepackage{xypic}
%\usepackage{pst-plot}
\usepackage[curve]{xypic}

% define commands here
\newcommand*{\abs}[1]{\left\lvert #1\right\rvert}
\newtheorem{prop}{Proposition}
\newtheorem{thm}{Theorem}
\newtheorem{ex}{Example}
\newcommand{\real}{\mathbb{R}}
\newcommand{\pdiff}[2]{\frac{\partial #1}{\partial #2}}
\newcommand{\mpdiff}[3]{\frac{\partial^#1 #2}{\partial #3^#1}}
\begin{document}
Let $f:A\to B$ be a morphism in a category $\mathcal{C}$.  The \emph{kernel pair} of $f$ is defined as the pair of morphisms $(k_1: K\to A, k_2:K\to A)$ such that 
$$\xymatrix@+=4pc{
{K}\ar[r]^{k_1}\ar[d]_{k_2} &{A}\ar[d]^{f} \\
{A}\ar[r]_{f}&{B}
}
$$
is a pullback diagram.

Since 
$$\xymatrix@+=4pc{
{A}\ar[r]^{1_A}\ar[d]_{1_A} &{A}\ar[d]^{f} \\
{A}\ar[r]_{f}&{B}
}
$$
is a commutative diagram, we have a unique morphism $g:A\to K$ such that
$$\xymatrix@+=4pc{
A\ar@/^1ex/[rrd]^{1_A} \ar@/_1ex/[rdd]_{1_A} \ar[rd]^g & & \\
& K \ar[d]^{k_2} \ar[r]_{k_1} & A\ar[d]^f \\
& A\ar[r]_f & B.
}
$$
is commutative.  As a result, $k_1$ and $k_2$ are both monomorphisms: if $k_1\circ h_1 = k_1\circ h_2$, then $$h_1 = 1_A \circ h_1 = (g\circ k_1) \circ h_1 = g\circ (k_1 \circ h_1) =g\circ (k_1 \circ h_2) = (g\circ k_1) \circ h_2 = 1_A \circ h_2 = h_2.$$

For example, in \textbf{Set}, the category of sets, the kernel pair of a function $f:A\to B$ is the pair $p_1:K\to A$ and $p_2:K\to A$, given by $$K=\lbrace (a,b) \in A\times A \mid f(a)=f(b) \rbrace,$$ and $p_1$ and $p_2$ are given by $$p_1(a,b)=a \qquad \mbox{and} \qquad p_2(a,b)=b.$$
This is just the kernel of a function, in the sense of universal algebra.  Please see \PMlinkname{this entry}{KernelOfAHomomorphismBetweenAlgebraicSystems} for more details.

The notion of \emph{cokernel pair} is dually defined.

\textbf{Remark}.  $f:A\to B$ is a monomorphism iff the kernel pair of $f$ is $(1_A,1_A)$.  Dually, $f$ is an epimorphism iff the cokernel pair of $f$ is $(1_A,1_A)$.


\begin{thebibliography}{9}
\bibitem{fb} F. Borceux \emph{Basic Category Theory, Handbook of Categorical Algebra I}, Cambridge University Press, Cambridge (1994)
\end{thebibliography}
%%%%%
%%%%%
\end{document}
