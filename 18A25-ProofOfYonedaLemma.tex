\documentclass[12pt]{article}
\usepackage{pmmeta}
\pmcanonicalname{ProofOfYonedaLemma}
\pmcreated{2013-03-22 18:30:16}
\pmmodified{2013-03-22 18:30:16}
\pmowner{GodelsTheorem}{21277}
\pmmodifier{GodelsTheorem}{21277}
\pmtitle{proof of Yoneda lemma}
\pmrecord{7}{41188}
\pmprivacy{1}
\pmauthor{GodelsTheorem}{21277}
\pmtype{Proof}
\pmcomment{trigger rebuild}
\pmclassification{msc}{18A25}

% this is the default PlanetMath preamble.  as your knowledge
% of TeX increases, you will probably want to edit this, but
% it should be fine as is for beginners.

% almost certainly you want these
\usepackage{amssymb}
\usepackage{amsmath}
\usepackage{amsfonts}

% used for TeXing text within eps files
%\usepackage{psfrag}
% need this for including graphics (\includegraphics)
%\usepackage{graphicx}
% for neatly defining theorems and propositions
%\usepackage{amsthm}
% making logically defined graphics
%%%\usepackage{xypic}

% there are many more packages, add them here as you need them

% define commands here

\begin{document}
We give a proof of Yoneda's Lemma. Thus, we have to show that  $\mathcal{C}\to\hat{\mathcal{C}}$ is a faithful functor. Let $X$ and $Y$ be two objects belonging to $\mathcal{C}$, we want to show that 


$\psi : {\rm Hom}(X,Y)\to {\rm Hom}(X(.),Y(.))$  

$f\mapsto (f_K:X(K)\to Y(K))_K$      is bijective.

Let's start with injectivity. Let $f$ and $g$ be two morphisms from $X$ to $Y$ which are having the same mappings for the points $f_K=g_K$ for all K. Let's show that $f=g$. What happens for the $X$-points? For the $X$-points, we have $f=g$ and the range of $f_X$ and of $g_X$ of the $X$-point of X which is $Id_X$ is exactly the $X$-points of $Y$ which are $f$ and $g$. Hence $f=g$.

Now for surjectivity: let $\psi: X(.)\to Y(.)$ a morphism of functors. We need to show that this morphism comes from an arrow $f$ which should be the range of $Id_X$ by the map $\phi_X$. Thus, let $f=\psi_X(Id_X)$. 
Let's verify that $f_K=\psi_K$ for all $K$. Let $p:K\to X$ be a $K$-point of $X$. $p$ is a morphism between the two types of points $K$ and $X$ and in this case we have the following commutative diagram:

$I don't know how to do diagrams with latex, it's too hard$

If you make $Id(X)$ turn in the diagram one verifies that $\psi_K(p)=f\circ p=f_K(p)$ which proves the surjectivity.
%%%%%
%%%%%
\end{document}
