\documentclass[12pt]{article}
\usepackage{pmmeta}
\pmcanonicalname{NonAbelianTheory}
\pmcreated{2013-03-22 18:12:31}
\pmmodified{2013-03-22 18:12:31}
\pmowner{bci1}{20947}
\pmmodifier{bci1}{20947}
\pmtitle{non-Abelian theory}
\pmrecord{57}{40789}
\pmprivacy{1}
\pmauthor{bci1}{20947}
\pmtype{Topic}
\pmcomment{trigger rebuild}
\pmclassification{msc}{18-00}
\pmclassification{msc}{18A15}
\pmclassification{msc}{03G12}
\pmclassification{msc}{03G30}
\pmclassification{msc}{03G20}
\pmsynonym{non-Abelian}{NonAbelianTheory}
\pmsynonym{nonabelian}{NonAbelianTheory}
%\pmkeywords{quantum non--Abelian algebraic topology (QNAAT)}
\pmrelated{NonAbelianStructures}
\pmrelated{AbelianCategory}
\pmrelated{AxiomsForAnAbelianCategory}
\pmrelated{GeneralizedVanKampenTheoremsHigherDimensional}
\pmrelated{AxiomaticTheoryOfSupercategories}
\pmrelated{AlgebraicCategoryOfLMnLogicAlgebras}
\pmrelated{CategoricalOntology}
\pmrelated{NonCommutingGraphOfAGroup}
\pmrelated{TopicEntryOnTheAlgebra}
\pmdefines{non-Abelian character}
\pmdefines{non-Abelian theories}
\pmdefines{nonabelian examples}

\endmetadata

 
% this is the default PlanetMath preamble. as your knowledge
% of TeX increases, you will probably want to edit this, but
% it should be fine as is for beginners.

% almost certainly you want these
\usepackage{amssymb}
\usepackage{amsmath}
\usepackage{amsfonts}

% used for TeXing text within eps files
%\usepackage{psfrag}
% need this for including graphics (\includegraphics)
%\usepackage{graphicx}
% for neatly defining theorems and propositions
%\usepackage{amsthm}
% making logically defined graphics
%%%\usepackage{xypic}

% there are many more packages, add them here as you need them

% define commands here
\usepackage{amsmath, amssymb, amsfonts, amsthm, amscd, latexsym}
%%\usepackage{xypic}
\usepackage[mathscr]{eucal}

\setlength{\textwidth}{6.5in}
%\setlength{\textwidth}{16cm}
\setlength{\textheight}{9.0in}
%\setlength{\textheight}{24cm}

\hoffset=-.75in %%ps format
%\hoffset=-1.0in %%hp format
\voffset=-.4in

\theoremstyle{plain}
\newtheorem{lemma}{Lemma}[section]
\newtheorem{proposition}{Proposition}[section]
\newtheorem{theorem}{Theorem}[section]
\newtheorem{corollary}{Corollary}[section]

\theoremstyle{definition}
\newtheorem{definition}{Definition}[section]
\newtheorem{example}{Example}[section]
%\theoremstyle{remark}
\newtheorem{remark}{Remark}[section]
\newtheorem*{notation}{Notation}
\newtheorem*{claim}{Claim}

\renewcommand{\thefootnote}{\ensuremath{\fnsymbol{footnote%%@
}}}
\numberwithin{equation}{section}

\newcommand{\Ad}{{\rm Ad}}
\newcommand{\Aut}{{\rm Aut}}
\newcommand{\Cl}{{\rm Cl}}
\newcommand{\Co}{{\rm Co}}
\newcommand{\DES}{{\rm DES}}
\newcommand{\Diff}{{\rm Diff}}
\newcommand{\Dom}{{\rm Dom}}
\newcommand{\Hol}{{\rm Hol}}
\newcommand{\Mon}{{\rm Mon}}
\newcommand{\Hom}{{\rm Hom}}
\newcommand{\Ker}{{\rm Ker}}
\newcommand{\Ind}{{\rm Ind}}
\newcommand{\IM}{{\rm Im}}
\newcommand{\Is}{{\rm Is}}
\newcommand{\ID}{{\rm id}}
\newcommand{\GL}{{\rm GL}}
\newcommand{\Iso}{{\rm Iso}}
\newcommand{\Sem}{{\rm Sem}}
\newcommand{\St}{{\rm St}}
\newcommand{\Sym}{{\rm Sym}}
\newcommand{\SU}{{\rm SU}}
\newcommand{\Tor}{{\rm Tor}}
\newcommand{\U}{{\rm U}}

\newcommand{\A}{\mathcal A}
\newcommand{\Ce}{\mathcal C}
\newcommand{\D}{\mathcal D}
\newcommand{\E}{\mathcal E}
\newcommand{\F}{\mathcal F}
\newcommand{\G}{\mathcal G}
\newcommand{\Q}{\mathcal Q}
\newcommand{\R}{\mathcal R}
\newcommand{\cS}{\mathcal S}
\newcommand{\cU}{\mathcal U}
\newcommand{\W}{\mathcal W}

\newcommand{\bA}{\mathbb{A}}
\newcommand{\bB}{\mathbb{B}}
\newcommand{\bC}{\mathbb{C}}
\newcommand{\bD}{\mathbb{D}}
\newcommand{\bE}{\mathbb{E}}
\newcommand{\bF}{\mathbb{F}}
\newcommand{\bG}{\mathbb{G}}
\newcommand{\bK}{\mathbb{K}}
\newcommand{\bM}{\mathbb{M}}
\newcommand{\bN}{\mathbb{N}}
\newcommand{\bO}{\mathbb{O}}
\newcommand{\bP}{\mathbb{P}}
\newcommand{\bR}{\mathbb{R}}
\newcommand{\bV}{\mathbb{V}}
\newcommand{\bZ}{\mathbb{Z}}

\newcommand{\bfE}{\mathbf{E}}
\newcommand{\bfX}{\mathbf{X}}
\newcommand{\bfY}{\mathbf{Y}}
\newcommand{\bfZ}{\mathbf{Z}}

\renewcommand{\O}{\Omega}
\renewcommand{\o}{\omega}
\newcommand{\vp}{\varphi}
\newcommand{\vep}{\varepsilon}

\newcommand{\diag}{{\rm diag}}
\newcommand{\grp}{{\mathbb G}}
\newcommand{\dgrp}{{\mathbb D}}
\newcommand{\desp}{{\mathbb D^{\rm{es}}}}
\newcommand{\Geod}{{\rm Geod}}
\newcommand{\geod}{{\rm geod}}
\newcommand{\hgr}{{\mathbb H}}
\newcommand{\mgr}{{\mathbb M}}
\newcommand{\ob}{{\rm Ob}}
\newcommand{\obg}{{\rm Ob(\mathbb G)}}
\newcommand{\obgp}{{\rm Ob(\mathbb G')}}
\newcommand{\obh}{{\rm Ob(\mathbb H)}}
\newcommand{\Osmooth}{{\Omega^{\infty}(X,*)}}
\newcommand{\ghomotop}{{\rho_2^{\square}}}
\newcommand{\gcalp}{{\mathbb G(\mathcal P)}}

\newcommand{\rf}{{R_{\mathcal F}}}
\newcommand{\glob}{{\rm glob}}
\newcommand{\loc}{{\rm loc}}
\newcommand{\TOP}{{\rm TOP}}

\newcommand{\wti}{\widetilde}
\newcommand{\what}{\widehat}

\renewcommand{\a}{\alpha}
\newcommand{\be}{\beta}
\newcommand{\ga}{\gamma}
\newcommand{\Ga}{\Gamma}
\newcommand{\de}{\delta}
\newcommand{\del}{\partial}
\newcommand{\ka}{\kappa}
\newcommand{\si}{\sigma}
\newcommand{\ta}{\tau}
\newcommand{\med}{\medbreak}
\newcommand{\medn}{\medbreak \noindent}
\newcommand{\bign}{\bigbreak \noindent}
\newcommand{\lra}{{\longrightarrow}}
\newcommand{\ra}{{\rightarrow}}
\newcommand{\rat}{{\rightarrowtail}}
\newcommand{\oset}[1]{\overset {#1}{\ra}}
\newcommand{\osetl}[1]{\overset {#1}{\lra}}
\newcommand{\hr}{{\hookrightarrow}}  
\begin{document}
\begin{definition}
A {\em non-Abelian theory} is one that does not satisfy one, several, or all of the axioms
of an Abelian theory, such as, for example, those for an Abelian category theory. 
\end{definition}

\subsection{Examples}  
ETAC and ETAS axiom interpretations that do not satisfy--in addition to the ETAC or ETAS axioms-- the $Ab1$ to $Ab6$ axioms for an \PMlinkname{abelian category}{AbelianCategory} are all examples on non-Abelian categories; a more detailed list is also presented next.

\begin{remark}
 In a general sense, any Abelian category (or {\em abelian category}) can be regarded as a `good' model for the category of Abelian, or commutative, groups. Furthermore, in an Abelian category $Ab$ every class, or set, of morphisms
$Hom_{Ab}(-,-)$ forms an Abelian (or commutative) group. There are several strict definitions of Abelian
categories involving 3, 4 or up to 6 axioms defining the Abelian character of a category. 
To illustrate non-Abelian theories it is useful to consider non-Abelian structures so that
specific properties determined by the non-Abelian set of axioms become `transparent' in terms
of the properties of objects for example for concrete categories that have objects; such examples
are presented separately as {\em non-Abelian structures}. 
\end{remark}

\subsection{Further examples of non-Abelian theories}

 The following is only a short list of non-Abelian theories:

\begin{enumerate}

\item Non-Abelian algebraic topology, including also non-Abelian homological algebra;
\PMlinkexternal{non-Abelian algebraic topology overview}{http://www.bangor.ac.uk/~mas010/nonab-a-t.html} and
\PMlinkexternal{R. Brown 2008 preprint}{http://arxiv.org/abs/math/0212274}, (\cite{RBetal2k7,RB2k8}).\\
(See also the \PMlinkexternal{recent book exposition}{2008 http://planetmath.org/?op=getobj&from=lec&id=75} with the title {\em ``Nonabelian Algebraic Topology''} vol. 1 by Brown and Sivera,(respectively, vol. 2 with Higgins, \emph{in preparation}). 

\item Non-Abelian quantum algebraic topology;

\item Non-Abelian gauge field theory (in Quantum Physics);

\item Noncommutative geometry;

\item The axiomatic theory of supercategories (ETAS); 

\item Higher dimensional algebra (HDA)
\item $LM_n$ Logic algebras;

\item Non-Abelian categorical ontology (\cite{BBG2k7}). 

\end{enumerate}

\subsection{Remarks}
 The following alternative definition by Barry Mitchell of an Abelian category should also be mentioned as {\em ``an exact additive category with finite products.''}. 

 He also published in his textbook the following theorem:
(\textbf{Theorem 20.1}, on p.33 of Barry Mitchell in ``Theory of Catgeories'', 1965, Academic Press: 
New York and London): 

\begin{theorem} 
 ``{\em The following statements are equivalent}:
\begin{itemize}
\item (a) $Ab$ is an abelian category; 
\item (b) $Ab$ has kernels, cokernels, finite products, finite coproducts,
 and is both normal and conormal; 
\item (c) $Ab$ has pushouts and pullbacks and is both normal and conormal.''
\end{itemize}
\end{theorem}

\begin{thebibliography}{9}

\bibitem{RBetal2k7}
R. Brown et al. 2008. {\em ``Non-Abelian Algebraic Topology''}. vols. 1 and 2. ({\em Preprint}).

\bibitem{RB2k8}
R. Brown. 2008. {\em Higher Dimensional Algebra Preprint as pdf and ps docs. at $arXiv:math/0212274v6 [math.AT]$}

\bibitem{BBG2k7}
I. C. Baianu, R. Brown and J. F. Glazebrook. 2007,  A Non--Abelian Categorical Ontology and Higher Dimensional Algebra of Spacetimes and Quantum Gravity., {\em Axiomathes }, \textbf{17}: 353-408. 


\end{thebibliography}
%%%%%
%%%%%
\end{document}
