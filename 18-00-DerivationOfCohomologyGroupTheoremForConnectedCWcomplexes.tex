\documentclass[12pt]{article}
\usepackage{pmmeta}
\pmcanonicalname{DerivationOfCohomologyGroupTheoremForConnectedCWcomplexes}
\pmcreated{2013-03-22 18:17:20}
\pmmodified{2013-03-22 18:17:20}
\pmowner{bci1}{20947}
\pmmodifier{bci1}{20947}
\pmtitle{derivation of cohomology group theorem for connected CW-complexes}
\pmrecord{51}{40899}
\pmprivacy{1}
\pmauthor{bci1}{20947}
\pmtype{Derivation}
\pmcomment{trigger rebuild}
\pmclassification{msc}{18-00}
\pmclassification{msc}{55P20}
\pmclassification{msc}{55N33}
\pmclassification{msc}{55N20}
\pmsynonym{group cohomology}{DerivationOfCohomologyGroupTheoremForConnectedCWcomplexes}
\pmsynonym{reduced cohomology theorem}{DerivationOfCohomologyGroupTheoremForConnectedCWcomplexes}
%\pmkeywords{fundamental cohomology theorem}
%\pmkeywords{derivation of the cohomology group theorem for connected CW complexes}
%\pmkeywords{fundamental cohomology group theorem}
\pmrelated{CohomologyGroupTheorem}
\pmrelated{GroupCohomology}
\pmrelated{EilenbergMacLaneSpace}
\pmrelated{OmegaSpectrum}
\pmdefines{fundamental class}
\pmdefines{cohomology group}

% this is the default PlanetMath preamble.  as your knowledge
% of TeX increases, you will probably want to edit this, but
% it should be fine as is for beginners.

% almost certainly you want these
\usepackage{amssymb}
\usepackage{amsmath}
\usepackage{amsfonts}

% used for TeXing text within eps files
%\usepackage{psfrag}
% need this for including graphics (\includegraphics)
%\usepackage{graphicx}
% for neatly defining theorems and propositions
%\usepackage{amsthm}
% making logically defined graphics
%%%\usepackage{xypic}

% there are many more packages, add them here as you need them

% define commands here
\usepackage{amsmath, amssymb, amsfonts, amsthm, amscd, latexsym,color,enumerate}
%%\usepackage{xypic}
\xyoption{curve}
\usepackage[mathscr]{eucal}

\setlength{\textwidth}{7.1in}
%\setlength{\textwidth}{16cm}
\setlength{\textheight}{9.2in}
%\setlength{\textheight}{24cm}

\hoffset=-1.0in     %%ps format
%\hoffset=-1.0in     %%hp format
\voffset=-.30in

%the next gives two direction arrows at the top of a 2 x 2 matrix

\newcommand{\directs}[2]{\def\objectstyle{\scriptstyle}  \objectmargin={0pt}
\xy
(0,4)*+{}="a",(0,-2)*+{\rule{0em}{1.5ex}#2}="b",(7,4)*+{\;#1}="c"
\ar@{->} "a";"b" \ar @{->}"a";"c" \endxy }

\theoremstyle{plain}
\newtheorem{lemma}{Lemma}[section]
\newtheorem{proposition}{Proposition}[section]
\newtheorem{theorem}{Theorem}[section]
\newtheorem{corollary}{Corollary}[section]
\newtheorem{conjecture}{Conjecture}[section]

\theoremstyle{definition}
\newtheorem{definition}{Definition}[section]
\newtheorem{example}{Example}[section]
%\theoremstyle{remark}
\newtheorem{remark}{Remark}[section]
\newtheorem*{notation}{Notation}
\newtheorem*{claim}{Claim}


\theoremstyle{plain}
\renewcommand{\thefootnote}{\ensuremath{\fnsymbol{footnote}}}
\numberwithin{equation}{section}
\newcommand{\Ad}{{\rm Ad}}
\newcommand{\Aut}{{\rm Aut}}
\newcommand{\Cl}{{\rm Cl}}
\newcommand{\Co}{{\rm Co}}
\newcommand{\DES}{{\rm DES}}
\newcommand{\Diff}{{\rm Diff}}
\newcommand{\Dom}{{\rm Dom}}
\newcommand{\Hol}{{\rm Hol}}
\newcommand{\Mon}{{\rm Mon}}
\newcommand{\Hom}{{\rm Hom}}
\newcommand{\Ker}{{\rm Ker}}
\newcommand{\Ind}{{\rm Ind}}
\newcommand{\IM}{{\rm Im}}
\newcommand{\Is}{{\rm Is}}
\newcommand{\ID}{{\rm id}}
\newcommand{\GL}{{\rm GL}}
\newcommand{\Iso}{{\rm Iso}}
\newcommand{\Sem}{{\rm Sem}}
\newcommand{\St}{{\rm St}}
\newcommand{\Sym}{{\rm Sym}}
\newcommand{\SU}{{\rm SU}}
\newcommand{\Tor}{{\rm Tor}}
\newcommand{\U}{{\rm U}}

\newcommand{\A}{\mathcal A}
\newcommand{\D}{\mathcal D}
\newcommand{\E}{\mathcal E}
\newcommand{\F}{\mathcal F}
\newcommand{\G}{\mathcal G}
\newcommand{\R}{\mathcal R}
\newcommand{\cS}{\mathcal S}
\newcommand{\cU}{\mathcal U}
\newcommand{\W}{\mathcal W}

\newcommand{\Ce}{\mathsf{C}}
\newcommand{\Q}{\mathsf{Q}}
\newcommand{\grp}{\mathsf{G}}
\newcommand{\dgrp}{\mathsf{D}}

\newcommand{\bA}{\mathbb{A}}
\newcommand{\bB}{\mathbb{B}}
\newcommand{\bC}{\mathbb{C}}
\newcommand{\bD}{\mathbb{D}}
\newcommand{\bE}{\mathbb{E}}
\newcommand{\bF}{\mathbb{F}}
\newcommand{\bG}{\mathbb{G}}
\newcommand{\bK}{\mathbb{K}}
\newcommand{\bM}{\mathbb{M}}
\newcommand{\bN}{\mathbb{N}}
\newcommand{\bO}{\mathbb{O}}
\newcommand{\bP}{\mathbb{P}}
\newcommand{\bR}{\mathbb{R}}
\newcommand{\bV}{\mathbb{V}}
\newcommand{\bZ}{\mathbb{Z}}

\newcommand{\bfE}{\mathbf{E}}
\newcommand{\bfX}{\mathbf{X}}
\newcommand{\bfY}{\mathbf{Y}}
\newcommand{\bfZ}{\mathbf{Z}}

\renewcommand{\O}{\Omega}
\renewcommand{\o}{\omega}
\newcommand{\vp}{\varphi}
\newcommand{\vep}{\varepsilon}

\newcommand{\diag}{{\rm diag}}
\newcommand{\desp}{{\mathbb D^{\rm{es}}}}
\newcommand{\Geod}{{\rm Geod}}
\newcommand{\geod}{{\rm geod}}
\newcommand{\hgr}{{\mathbb H}}
\newcommand{\mgr}{{\mathbb M}}
\newcommand{\ob}{\operatorname{Ob}}
\newcommand{\obg}{{\rm Ob(\mathbb G)}}
\newcommand{\obgp}{{\rm Ob(\mathbb G')}}
\newcommand{\obh}{{\rm Ob(\mathbb H)}}
\newcommand{\Osmooth}{{\Omega^{\infty}(X,*)}}
\newcommand{\ghomotop}{{\rho_2^{\square}}}
\newcommand{\gcalp}{{\mathbb G(\mathcal P)}}

\newcommand{\rf}{{R_{\mathcal F}}}
\newcommand{\glob}{{\rm glob}}
\newcommand{\loc}{{\rm loc}}
\newcommand{\TOP}{{\rm TOP}}

\newcommand{\wti}{\widetilde}
\newcommand{\what}{\widehat}

\renewcommand{\a}{\alpha}
\newcommand{\be}{\beta}
\newcommand{\ga}{\gamma}
\newcommand{\Ga}{\Gamma}
\newcommand{\de}{\delta}
\newcommand{\del}{\partial}
\newcommand{\ka}{\kappa}
\newcommand{\si}{\sigma}
\newcommand{\ta}{\tau}


\newcommand{\lra}{{\longrightarrow}}
\newcommand{\ra}{{\rightarrow}}
\newcommand{\rat}{{\rightarrowtail}}
\newcommand{\oset}[1]{\overset {#1}{\ra}}
\newcommand{\osetl}[1]{\overset {#1}{\lra}}
\newcommand{\hr}{{\hookrightarrow}}


\newcommand{\hdgb}{\boldsymbol{\rho}^\square}
\newcommand{\hdg}{\rho^\square_2}

\newcommand{\med}{\medbreak}
\newcommand{\medn}{\medbreak \noindent}
\newcommand{\bign}{\bigbreak \noindent}

\renewcommand{\leq}{{\leqslant}}
\renewcommand{\geq}{{\geqslant}}

\def\red{\textcolor{red}}
\def\magenta{\textcolor{magenta}}
\def\blue{\textcolor{blue}}
\def\<{\langle}
\def\>{\rangle}
\begin{document}
\subsection{Introduction}


 Let $X_g$ be a general CW-complex and consider the set 
$$\left\langle{X_g, K(G,n)}\right\rangle $$ 
of basepoint preserving homotopy classes of maps from $X_g$ to Eilenberg-MacLane spaces $K(G, n)$ for $n \geq 0 $, with $G$ being an Abelian group. 

\begin{theorem}(Fundamental, [or reduced] Cohomology Theorem, \cite{AllenHatcher2k1}).
  
 There exists a natural group isomorphism:
\begin{equation}
\iota : \left\langle{X_g, K(G,n)}\right\rangle \cong \overline{H}^n (X_g;G)
\end{equation}
 for all CW-complexes $X_g$ , with $G$ any Abelian group and all $n \geq 0$. Such a group isomorphism
has the form $\iota ([f]) = f^*(\Phi)$ for a certain distinguished class in the cohomology group
$\Phi \in \overline{H}^n (X_g;G)$, (called a \emph{fundamental class}). 

\end{theorem}

\subsection{Derivation of the Cohomology Group Theorem for Connected CW-complexes.}
 For connected CW-complexes, $X$, the set $\left\langle{X_g, K(G,n)}\right\rangle$ of basepoint preserving homotopy classes maps from $X_g$ to Eilenberg-MacLane spaces $K(G, n)$ is replaced by the set of non-basepointed homotopy classes $[X, K(\pi,n)]$, for an Abelian group $G = \pi$ and all $n \geq 1$, because every map $X \to K(\pi,n)$ can be homotoped to take basepoint to basepoint, and also every  homotopy between basepoint -preserving maps can be homotoped to be basepoint-preserving when the image space $K(\pi,n)$ is simply-connected. 

 Therefore, the {\em natural group isomorphism} in {\bf Eq. (0.1)} becomes:
\begin{equation}
\iota : [X, K(\pi,n)] \cong \overline{H}^n (X;\pi)
\end{equation}

When $n =1$ the above group isomorphism results immediately from the condition that 
$\pi = G$ is an Abelian group. QED 

\subsection{Remarks}
\begin{enumerate}
\item A direct but very tedious proof of the (reduced) cohomology theorem can be obtained by constructing maps and homotopies cell-by-cell. 

\item An alternative, categorical derivation {\em via} duality and generalization of the proof of the cohomology group theorem (\cite{May1999}) is possible by employing the categorical definitions of a limit, colimit/cocone, the definition of Eilenberg-MacLane spaces (as specified under related), and by verification of the axioms for reduced cohomology groups (pp. 142-143 in Ch.19 and p. 172 of ref. \cite{May1999}).
This also raises the interesting question of the propositions that hold for non-Abelian groups G, and generalized cohomology theories.
\end{enumerate}

\begin{thebibliography} {9}

\bibitem{AllenHatcher2k1}
Hatcher, A. 2001. \PMlinkexternal{Algebraic Topology.}{http://www.math.cornell.edu/~hatcher/AT/AT.pdf}, Cambridge University Press; Cambridge, UK., (Theorem 4.57, pp.393-405).

\bibitem{May1999}
May, J.P. 1999, \emph{A Concise Course in Algebraic Topology.}, The University of Chicago Press: Chicago

\end{thebibliography}
%%%%%
%%%%%
\end{document}
