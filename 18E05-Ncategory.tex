\documentclass[12pt]{article}
\usepackage{pmmeta}
\pmcanonicalname{Ncategory}
\pmcreated{2013-03-22 18:18:33}
\pmmodified{2013-03-22 18:18:33}
\pmowner{bci1}{20947}
\pmmodifier{bci1}{20947}
\pmtitle{n-category}
\pmrecord{37}{40931}
\pmprivacy{1}
\pmauthor{bci1}{20947}
\pmtype{Definition}
\pmcomment{trigger rebuild}
\pmclassification{msc}{18E05}
\pmclassification{msc}{18-00}
\pmsynonym{higher order categories}{Ncategory}
\pmsynonym{higher dimensional algebra}{Ncategory}
%\pmkeywords{n-category}
%\pmkeywords{higher order categories}
%\pmkeywords{higher dimensional algebra}
%\pmkeywords{supercateories}
\pmrelated{2Category}
\pmrelated{FunctorCategories}
\pmrelated{RSupercategory}
\pmrelated{Supercategories3}
\pmrelated{AxiomsOfMetacategoriesAndSupercategories}
\pmrelated{HigherDimensionalAlgebraHDA}
\pmrelated{VariableTopology3}
\pmrelated{CategoryTheory}
\pmrelated{OverviewArticleForAlgebraicTopology}
\pmdefines{higher order category}
\pmdefines{(n-1)-supercategory}

\endmetadata

% this is the default PlanetMath preamble.  as your knowledge
% of TeX increases, you will probably want to edit this, but
% it should be fine as is for beginners.

% almost certainly you want these
\usepackage{amssymb}
\usepackage{amsmath}
\usepackage{amsfonts}

% used for TeXing text within eps files
%\usepackage{psfrag}
% need this for including graphics (\includegraphics)
%\usepackage{graphicx}
% for neatly defining theorems and propositions
%\usepackage{amsthm}
% making logically defined graphics
%%%\usepackage{xypic}

% there are many more packages, add them here as you need them

% define commands here
\usepackage{amsmath, amssymb, amsfonts, amsthm, amscd, latexsym}
%%\usepackage{xypic}
\usepackage[mathscr]{eucal}

\setlength{\textwidth}{6.5in}
%\setlength{\textwidth}{16cm}
\setlength{\textheight}{9.0in}
%\setlength{\textheight}{24cm}

\hoffset=-.75in     %%ps format
%\hoffset=-1.0in     %%hp format
\voffset=-.4in

\theoremstyle{plain}
\newtheorem{lemma}{Lemma}[section]
\newtheorem{proposition}{Proposition}[section]
\newtheorem{theorem}{Theorem}[section]
\newtheorem{corollary}{Corollary}[section]

\theoremstyle{definition}
\newtheorem{definition}{Definition}[section]
\newtheorem{example}{Example}[section]
%\theoremstyle{remark}
\newtheorem{remark}{Remark}[section]
\newtheorem*{notation}{Notation}
\newtheorem*{claim}{Claim}

\renewcommand{\thefootnote}{\ensuremath{\fnsymbol{footnote%%@
}}}
\numberwithin{equation}{section}

\newcommand{\Ad}{{\rm Ad}}
\newcommand{\Aut}{{\rm Aut}}
\newcommand{\Cl}{{\rm Cl}}
\newcommand{\Co}{{\rm Co}}
\newcommand{\DES}{{\rm DES}}
\newcommand{\Diff}{{\rm Diff}}
\newcommand{\Dom}{{\rm Dom}}
\newcommand{\Hol}{{\rm Hol}}
\newcommand{\Mon}{{\rm Mon}}
\newcommand{\Hom}{{\rm Hom}}
\newcommand{\Ker}{{\rm Ker}}
\newcommand{\Ind}{{\rm Ind}}
\newcommand{\IM}{{\rm Im}}
\newcommand{\Is}{{\rm Is}}
\newcommand{\ID}{{\rm id}}
\newcommand{\GL}{{\rm GL}}
\newcommand{\Iso}{{\rm Iso}}
\newcommand{\Sem}{{\rm Sem}}
\newcommand{\St}{{\rm St}}
\newcommand{\Sym}{{\rm Sym}}
\newcommand{\SU}{{\rm SU}}
\newcommand{\Tor}{{\rm Tor}}
\newcommand{\U}{{\rm U}}

\newcommand{\A}{\mathcal A}
\newcommand{\Ce}{\mathcal C}
\newcommand{\D}{\mathcal D}
\newcommand{\E}{\mathcal E}
\newcommand{\F}{\mathcal F}
\newcommand{\G}{\mathcal G}
\newcommand{\Q}{\mathcal Q}
\newcommand{\R}{\mathcal R}
\newcommand{\cS}{\mathcal S}
\newcommand{\cU}{\mathcal U}
\newcommand{\W}{\mathcal W}

\newcommand{\bA}{\mathbb{A}}
\newcommand{\bB}{\mathbb{B}}
\newcommand{\bC}{\mathbb{C}}
\newcommand{\bD}{\mathbb{D}}
\newcommand{\bE}{\mathbb{E}}
\newcommand{\bF}{\mathbb{F}}
\newcommand{\bG}{\mathbb{G}}
\newcommand{\bK}{\mathbb{K}}
\newcommand{\bM}{\mathbb{M}}
\newcommand{\bN}{\mathbb{N}}
\newcommand{\bO}{\mathbb{O}}
\newcommand{\bP}{\mathbb{P}}
\newcommand{\bR}{\mathbb{R}}
\newcommand{\bV}{\mathbb{V}}
\newcommand{\bZ}{\mathbb{Z}}

\newcommand{\bfE}{\mathbf{E}}
\newcommand{\bfX}{\mathbf{X}}
\newcommand{\bfY}{\mathbf{Y}}
\newcommand{\bfZ}{\mathbf{Z}}

\renewcommand{\O}{\Omega}
\renewcommand{\o}{\omega}
\newcommand{\vp}{\varphi}
\newcommand{\vep}{\varepsilon}

\newcommand{\diag}{{\rm diag}}
\newcommand{\grp}{{\mathbb G}}
\newcommand{\dgrp}{{\mathbb D}}
\newcommand{\desp}{{\mathbb D^{\rm{es}}}}
\newcommand{\Geod}{{\rm Geod}}
\newcommand{\geod}{{\rm geod}}
\newcommand{\hgr}{{\mathbb H}}
\newcommand{\mgr}{{\mathbb M}}
\newcommand{\ob}{{\rm Ob}}
\newcommand{\obg}{{\rm Ob(\mathbb G)}}
\newcommand{\obgp}{{\rm Ob(\mathbb G')}}
\newcommand{\obh}{{\rm Ob(\mathbb H)}}
\newcommand{\Osmooth}{{\Omega^{\infty}(X,*)}}
\newcommand{\ghomotop}{{\rho_2^{\square}}}
\newcommand{\gcalp}{{\mathbb G(\mathcal P)}}

\newcommand{\rf}{{R_{\mathcal F}}}
\newcommand{\glob}{{\rm glob}}
\newcommand{\loc}{{\rm loc}}
\newcommand{\TOP}{{\rm TOP}}

\newcommand{\wti}{\widetilde}
\newcommand{\what}{\widehat}

\renewcommand{\a}{\alpha}
\newcommand{\be}{\beta}
\newcommand{\ga}{\gamma}
\newcommand{\Ga}{\Gamma}
\newcommand{\de}{\delta}
\newcommand{\del}{\partial}
\newcommand{\ka}{\kappa}
\newcommand{\si}{\sigma}
\newcommand{\ta}{\tau}
\newcommand{\med}{\medbreak}
\newcommand{\medn}{\medbreak \noindent}
\newcommand{\bign}{\bigbreak \noindent}
\newcommand{\lra}{{\longrightarrow}}
\newcommand{\ra}{{\rightarrow}}
\newcommand{\rat}{{\rightarrowtail}}
\newcommand{\oset}[1]{\overset {#1}{\ra}}
\newcommand{\osetl}[1]{\overset {#1}{\lra}}
\newcommand{\hr}{{\hookrightarrow}}
\begin{document}
\begin{definition} a small $n$-category , $\mathcal{C}_n$, is the $n$-th order category 
of (small) $n$-categories $n$-$\mathcal{C}at$ constructed by induction on $n$ in two main stages:
\begin{enumerate}
\item define the category $0$-Cat as the category $\mathcal{S}et$ of sets and functions;
 
\item define the category $(n+1)-\mathcal{C}at$ as the category of ($n$) categories enriched over the category 
$\mathcal{C}_n$. The construction is simplified by beginning with the definition of the 2-category.
 
The following, more detailed \emph{recursive construction} of $n-\mathcal{C}at$ utilizes the fact that \emph{if a category $\mathcal{C}$ has finite products, the category of $\mathcal{C}$-enriched categories also has finite products}.

\end{enumerate}

\begin{enumerate}
\item define $\mathcal{C}at$ , or category $1-\mathcal{C}at$ as the category of small categories and functors;
\item define a class of objects $A, B,...$ in $\mathcal{C}at$ called \emph{`$0$-cells'};
\item for all `$0$-cells' $A$, $B$, consider the set $Hom_{\mathcal{C}_2}(A, B)$, or 
\PMlinkname{$\mathcal{C}_2(A,B)$, organized as a small category}{FunctorCategories}, whose $2$-morphisms, or `$1$-cells', are defined as natural transformations called `$2$-cells', $\eta: F \to G$ for any two `morphisms' of $\mathcal{C}at$, with $F$ and $G$ being functors between the `$0$-cells' $A$ and $B$, $F,G: A \to B$);
\item the 2-categorical composition is denoted as ``$ \bullet$" and is called the vertical composition;
\item a \emph{horizontal composition}, ``$\circ$", is defined for all triples of $0$-cells, $A$, $B$ and
$C$ in $\mathcal{C}at$ as the functor $\circ: \mathcal{C}_2(B,C) \times \mathcal{C}_2(A,B) = \mathcal{C}_2(A,C)$;
which is \emph{associative};
\item  the identities under horizontal composition are the identities of the $2$-cells of $1_X$ for any $X$ in $\mathcal{C}at$;
\item for any object $A$ in $\mathcal{C}at$ there is a functor from the one-object/one-arrow category \textbf{$1$} (terminal object) to $\mathcal{C}_2(A,A)$. 
\item repeat the last $(n-1)$ steps to define `3'-cells, ..., to $n$-cells; the resulting structure is called an $n$-category, but it is in fact a metagraph, metacategory, or more generally, a $\S_{n-1}$-supercategory with \emph{$n$ composition laws} and it is also called more recently a \emph{higher order category} or a \emph{higher dimensional algebra}.
\end{enumerate}
\end{definition}

\textbf{Note} Because the 2-cells can be considered as 2-morphisms between 1-morphisms, they are also written as: $\eta : F \Rightarrow G$, and are depicted as labelled faces in the plane determined by their domains and codomains.
%%%%%
%%%%%
\end{document}
