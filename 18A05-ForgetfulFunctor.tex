\documentclass[12pt]{article}
\usepackage{pmmeta}
\pmcanonicalname{ForgetfulFunctor}
\pmcreated{2013-03-22 12:38:43}
\pmmodified{2013-03-22 12:38:43}
\pmowner{RevBobo}{4}
\pmmodifier{RevBobo}{4}
\pmtitle{forgetful functor}
\pmrecord{4}{32910}
\pmprivacy{1}
\pmauthor{RevBobo}{4}
\pmtype{Definition}
\pmcomment{trigger rebuild}
\pmclassification{msc}{18A05}
\pmrelated{AdjointFunctor}
\pmdefines{forgetful}

% this is the default PlanetMath preamble.  as your knowledge
% of TeX increases, you will probably want to edit this, but
% it should be fine as is for beginners.

% almost certainly you want these
\usepackage{amssymb}
\usepackage{amsmath}
\usepackage{amsfonts}

% used for TeXing text within eps files
%\usepackage{psfrag}
% need this for including graphics (\includegraphics)
%\usepackage{graphicx}
% for neatly defining theorems and propositions
%\usepackage{amsthm}
% making logically defined graphics
%%%\usepackage{xypic} 

% there are many more packages, add them here as you need them

% define commands here
\begin{document}
Let $\mathcal{C}$ and $\mathcal{D}$ be categories such that each object $c$ of $\mathcal{C}$ can be regarded an object of $\mathcal{D}$ by suitably ignoring structures $c$ may have as a $\mathcal{C}$-object but not a $\mathcal{D}$-object. A functor $U:\mathcal{C} \to \mathcal{D}$ which operates on objects of $\mathcal{C}$ by ``forgetting'' any imposed mathematical structure is called a \emph{forgetful functor}. The following are examples of forgetful functors:
\begin{enumerate}
\item $U:\mathbf{Grp} \to \mathbf{Set}$ takes groups into their underlying sets and group homomorphisms to set maps.
\item $U:\mathbf{Top} \to \mathbf{Set}$ takes topological spaces into their underlying sets and continuous maps to set maps.
\item $U:\mathbf{Ab} \to \mathbf{Grp}$ takes abelian groups to groups and acts as identity on arrows.
\end{enumerate}
Forgetful functors are often instrumental in studying adjoint functors.
%%%%%
%%%%%
\end{document}
