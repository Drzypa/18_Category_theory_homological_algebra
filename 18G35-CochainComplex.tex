\documentclass[12pt]{article}
\usepackage{pmmeta}
\pmcanonicalname{CochainComplex}
\pmcreated{2013-03-22 19:03:52}
\pmmodified{2013-03-22 19:03:52}
\pmowner{rm50}{10146}
\pmmodifier{rm50}{10146}
\pmtitle{cochain complex}
\pmrecord{5}{41948}
\pmprivacy{1}
\pmauthor{rm50}{10146}
\pmtype{Definition}
\pmcomment{trigger rebuild}
\pmclassification{msc}{18G35}
\pmclassification{msc}{16E05}
\pmdefines{cocycle}
\pmdefines{coboundary}

\usepackage{amssymb}
\usepackage{amsmath}
\usepackage{amsfonts}

% used for TeXing text within eps files
%\usepackage{psfrag}
% need this for including graphics (\includegraphics)
%\usepackage{graphicx}
% for neatly defining theorems and propositions
%\usepackage{amsthm}
% making logically defined graphics
%%%\usepackage{xypic}

% there are many more packages, add them here as you need them

% define commands here
\newcommand{\BQ}{\mathbb{Q}}
\newcommand{\BR}{\mathbb{R}}
\newcommand{\BZ}{\mathbb{Z}}
\DeclareMathOperator{\im}{im}
\begin{document}
\PMlinkescapeword{adjacent}
\PMlinkescapeword{complex}
\PMlinkescapeword{equivalent}
\PMlinkescapeword{relation}
\PMlinkescapeword{satisfies}

Let $R$ be a ring.
A sequence of \PMlinkname{$R$-modules}{Module} and homomorphisms
\[
 \cdots \rightarrow
 A^{n-1} \xrightarrow{d_{n-1}} A^n \xrightarrow{d_n} A^{n+1} \xrightarrow{d_{n+1}}
 \cdots
\]
is said to be a \emph{cochain complex}
(or \emph{$R$-complex}, or just \emph{complex})
if each pair of adjacent homomorphisms $(d_{n-1}, d_n)$
satisfies the relation $d_n\circ d_{n-1} = 0$.
This is equivalent to saying that
$\im d_{n-1} \subseteq \ker d_n$.
We often denote such a complex by $(\mathcal{A}, d)$, or simply $\mathcal{A}$.

Compare this to the notion of an exact sequence,
which requires $\im d_{n-1} = \ker d_n$. Compare also to the notion of a chain complex, in which the arrows go in the opposite direction.

The homomorphisms $d_n$ in the chain complex
are called \emph{coboundary operators}, or \emph{coboundary maps}. Elements of $\ker d_n$ are known as \emph{cocycles}; elements of $\im d_{n-1}$ as \emph{coboundaries}.
%%%%%
%%%%%
\end{document}
