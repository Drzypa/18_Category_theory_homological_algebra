\documentclass[12pt]{article}
\usepackage{pmmeta}
\pmcanonicalname{AGroupHomomorphismIsInjectiveIffTheKernelIsTrivial}
\pmcreated{2013-03-22 14:10:15}
\pmmodified{2013-03-22 14:10:15}
\pmowner{alozano}{2414}
\pmmodifier{alozano}{2414}
\pmtitle{a group homomorphism is injective iff the kernel is trivial}
\pmrecord{7}{35596}
\pmprivacy{1}
\pmauthor{alozano}{2414}
\pmtype{Theorem}
\pmcomment{trigger rebuild}
\pmclassification{msc}{18Exx}
\pmclassification{msc}{18A30}
\pmclassification{msc}{20K30}
\pmclassification{msc}{13B10}
\pmclassification{msc}{16W20}
\pmclassification{msc}{15A03}
%\pmkeywords{injective}
%\pmkeywords{homomorphism}
\pmrelated{KernelOfAGroupHomomorphism}

\endmetadata

% this is the default PlanetMath preamble.  as your knowledge
% of TeX increases, you will probably want to edit this, but
% it should be fine as is for beginners.

% almost certainly you want these
\usepackage{amssymb}
\usepackage{amsmath}
\usepackage{amsthm}
\usepackage{amsfonts}

% used for TeXing text within eps files
%\usepackage{psfrag}
% need this for including graphics (\includegraphics)
%\usepackage{graphicx}
% for neatly defining theorems and propositions
%\usepackage{amsthm}
% making logically defined graphics
%%%\usepackage{xypic}

% there are many more packages, add them here as you need them

% define commands here

\newtheorem*{thm}{Theorem}
\newtheorem{defn}{Definition}
\newtheorem*{prop}{Proposition}
\newtheorem*{lemma}{Lemma}
\newtheorem{cor}{Corollary}

% Some sets
\newcommand{\Nats}{\mathbb{N}}
\newcommand{\Ints}{\mathbb{Z}}
\newcommand{\Reals}{\mathbb{R}}
\newcommand{\Complex}{\mathbb{C}}
\newcommand{\Rats}{\mathbb{Q}}
\newcommand{\Ker}{\operatorname{Ker}}
\begin{document}
\begin{prop}
Let $G,H$ be groups, and let $f \colon G \to H$ be a group homomorphism. Then $f$ is injective if and only if $\Ker(f)=\{e_G\}$, where $e_G$ is the identity element of $G$, and $\Ker$ denotes the kernel of $f$ (see also Kernel of a group homomorphism).  
\end{prop}
\begin{proof}

First assume that $f$ is injective (i.e. $f(g_1)=f(g_2) \Rightarrow g_1=g_2$). Recall that:
$$\Ker(f)=\{ g\in G : f(g)=e_H \}$$
where $e_H$ is the identity element of $H$. Since $f$ is a group homomorphism, it follows that $f(e_G)=e_H$. Let $g\in \Ker(f)$, then $f(g)=e_H=f(e_G)$, which implies that $g=e_G$, by the injectivity of $f$. Thus $\Ker(f)=\{ e_G \}$.

For the converse, we assume that $\Ker(f)=\{ e_G \}$ and suppose that $f(g_1)=f(g_2)$, for some $g_1,g_2 \in G$. Since $f$ is a homomorphism:
$$f(g_1)=f(g_2) \Rightarrow f(g_1)\cdot f(g_2)^{-1}=e_H \Rightarrow f(g_1\cdot g_2^{-1})=e_H$$
Thus $g_1\cdot g_2^{-1} \in \Ker(f)$, and the kernel is trivial so $g_1\cdot g_2^{-1}=e_G$, therefore $g_1=g_2$.
\end{proof}
%%%%%
%%%%%
\end{document}
