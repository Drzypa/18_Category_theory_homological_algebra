\documentclass[12pt]{article}
\usepackage{pmmeta}
\pmcanonicalname{BibliographyForCategoryTheory}
\pmcreated{2013-03-22 16:40:03}
\pmmodified{2013-03-22 16:40:03}
\pmowner{rspuzio}{6075}
\pmmodifier{rspuzio}{6075}
\pmtitle{bibliography for category theory}
\pmrecord{70}{38874}
\pmprivacy{1}
\pmauthor{rspuzio}{6075}
\pmtype{Bibliography}
\pmcomment{trigger rebuild}
\pmclassification{msc}{18-01}
\pmrelated{BibliographyForGroupoidsAndAlgebraicTopology}

\endmetadata

% this is the default PlanetMath preamble.  as your knowledge
% of TeX increases, you will probably want to edit this, but
% it should be fine as is for beginners.

% almost certainly you want these
\usepackage{amssymb}
\usepackage{amsmath}
\usepackage{amsfonts}

% used for TeXing text within eps files
%\usepackage{psfrag}
% need this for including graphics (\includegraphics)
%\usepackage{graphicx}
% for neatly defining theorems and propositions
%\usepackage{amsthm}
% making logically defined graphics
%%%\usepackage{xypic}

% there are many more packages, add them here as you need them

% define commands here

\begin{document}
\begin{thebibliography}{199}

\bibitem{ahs}
J. Ad$\'a$mek, H. Herrlich, and G. Strecker. {\em Abstract and Concrete Categories}. Wiley, New York, 1990.

\begin{quote}
This is an introduction to category theory which is aimed at readers
who do not necessarrily have much (or any) background in such areas
as algebraic geometry, in particular, readers from other subjects
such as computer science and theoretical linguistics which make occasional
use of concepts and techniques of category theory.  In addition to being
an excellent intrpduction to the subject, this book also contains an
account of the category-theoretical approach to automata via dynamorphisms.
\end{quote}

\bibitem{AJetal90}
Adamek, J. et al., 1990, Abstract and Concrete Categories: The Joy of Cats, New York: Wiley. 
\bibitem{AJ94}
Adamek, J. et al., 1994, Locally Presentable and Accessible Categories, Cambridge: Cambridge University Press.
\bibitem{AAM75}
M.~A.~Arbib {\em Arrows, Structures, and Functors: The Categorical Imperative}.
Academic Press, 1975.
\bibitem{A-GZ999}
Arzi-Gonczaworski, Z., 1999, "Perceive This as That--Analogies, Artificial Perception, and Category Theory", Annals of Mathematics and Artificial Intelligence, 26, no. 1, 215-252.
\bibitem{AS-BC2k}
Awodey, S. \& Butz, C., 2000, "Topological Completeness for Higher Order Logic", Journal of Symbolic Logic, 65, 3, 
1168-1182. 
\bibitem{AS-RER2k2}
Awodey, S. \& Reck, E. R., 2002, "Completeness and Categoricity I. Nineteen-Century Axiomatics to Twentieth-Century Metalogic", History and Philosophy of Logic, 23, 1, 1-30.
\bibitem{AS-RER2k2}
Awodey, S. \& Reck, E. R., 2002, "Completeness and Categoricity II. Twentieth-Century Metalogic to Twenty-first-Century Semantics", History and Philosophy of Logic, 23, 2, 77-94.  
\bibitem{AS96}
Awodey, S., 1996, "Structure in Mathematics and Logic: A Categorical Perspective", Philosophia Mathematica, 3, 209-237. 
\bibitem{AS2k4}
Awodey, S., 2004, "An Answer to Hellman's Question: Does Category Theory Provide a Framework for Mathematical Structuralism", Philosophia Mathematica, 12, 54-64.
\bibitem{AS2k6}
Awodey, S., 2006, Category Theory, Oxford: Clarendon Press.
\bibitem{BAJ-DJ98a}
Baez, J. \& Dolan, J., 1998a, "Higher-Dimensional Algebra III. n-Categories and the Algebra of Opetopes", Advances in Mathematics, 135, 145-206.
\bibitem{BAJ-DJ98B}
Baez, J. \& Dolan, J., 1998b, "Categorification", Higher Category Theory, Contemporary Mathematics, 230, Providence: AMS, 1-36.
\bibitem{BAJ-DJ97}
Baez, J., 1997, "An Introduction to n-Categories", Category Theory and Computer Science, Lecture Notes in Computer Science, 1290, Berlin: Springer-Verlag, 1-33.
\bibitem{ICB4}
Baianu, I.C.: 1971b, Categories, Functors and Quantum Algebraic
Computations, in P. Suppes (ed.), \emph{Proceed. Fourth Intl. Congress Logic-Mathematics-Philosophy of Science}, September 1-4, 1971.
\bibitem{ICBs5}
Baianu, I.C. and D. Scripcariu: 1973, On Adjoint Dynamical Systems. \emph{Bulletin of Mathematical Biophysics}, \textbf{35}(4), 475-486.
\bibitem{ICB5}
Baianu, I.C.: 1973, Some Algebraic Properties of \emph{\textbf{(M,R)}} -- Systems. \emph{Bulletin of Mathematical Biophysics} \textbf{35}, 213-217.
\bibitem{ICBm2}
Baianu, I.C. and M. Marinescu: 1974, On A Functorial Construction of \emph{\textbf{(M,R)}}-- Systems. \emph{Revue Roumaine de Mathematiques Pures et Appliquees} \textbf{19}: 388-391.
\bibitem{ICB6}
Baianu, I.C.: 1977, A Logical Model of Genetic Activities in \L ukasiewicz
Algebras: The Non-linear Theory. \emph{Bulletin of Mathematical Biology},
\textbf{39}: 249-258.
\bibitem{ICB2}
Baianu, I.C.: 1980a, Natural Transformations of Organismic Structures.,
\emph{Bulletin of Mathematical Biology},\textbf{42}: 431-446.
\bibitem{ICB87}
Baianu, I. C.: 1986-1987a, Computer Models and Automata Theory in Biology and Medicine.,  in M. Witten (ed.), \emph{Mathematical Models in Medicine}, vol. 7., Pergamon Press, New York, 1513 -1577; URLs: \emph{CERN Preprint No. EXT-2004-072: } \\
\PMlinkexternal{on line}{http://en.scientificcommons.org/1857371}
\bibitem{Bgg2}
Baianu, I. C., Glazebrook, J. F. and G. Georgescu: 2004,
Categories of Quantum Automata and N-Valued \L ukasiewicz Algebras
in Relation to Dynamic Bionetworks, \textbf{(M,R)}--Systems and
Their Higher Dimensional Algebra, \PMlinkexternal{Abstract and Preprint of
Report on line}{http://www.ag.uiuc.edu/fs401/QAuto.pdf} and 
\PMlinkexternal{also on line Abstract}{http://www.medicalupapers.com/quantum+automata+math+categories+baianu/}
\bibitem{Bgb2}
Baianu, I. C., Brown, R. and J. F. Glazebrook: 2006a, Quantum Algebraic Topology and Field Theories. URL: \\
http://www.ag.uiuc.edu/fs40l/QAT.pdf. (\emph{Preprint--in subm.})
\bibitem{BBGG1}
Baianu I. C., Brown R., Georgescu G. and J. F. Glazebrook: 2006b, Complex Nonlinear Biodynamics in Categories, Higher Dimensional Algebra and \L ukasiewicz--Moisil Topos: Transformations of Neuronal, Genetic and Neoplastic Networks., \emph{Axiomathes}, \textbf{16} Nos. 1--2: 65--122.
\bibitem{Bbg3}
Baianu, I.C., R. Brown and J.F. Glazebrook. : 2007a, Categorical Ontology of Complex Spacetime Structures: The Emergence of Life and Human Consciousness, \emph{Axiomathes}, \textbf{17}: 35-168.
\bibitem{Bggb4}
Baianu, I.C.,  R. Brown and J. F. Glazebrook: 2007b, A Non-Abelian, Categorical Ontology of Spacetimes and Quantum Gravity, Axiomathes, 17: 169-225.
\bibitem{Ba-We2k}
M.~Barr and C.~Wells. {\em Toposes, Triples and Theories}. Montreal: McGill University, 2000.
\bibitem{Ba-We85}
Barr, M. \& Wells, C., 1985, Toposes, Triples and Theories, New York: Springer-Verlag.
\bibitem{BM-CW99}
Barr, M. \& Wells, C., 1999, Category Theory for Computing Science, Montreal: CRM. 
\bibitem{BaM98}
Batanin, M., 1998, "Monoidal Globular Categories as a Natural Environment for the Theory of Weak n-Categories", Advances in Mathematics, 136, 39-103.   
\bibitem{BJL81}
Bell, J. L., 1981, "Category Theory and the Foundations of Mathematics", British Journal for the Philosophy of Science, 32, 349-358. 
\bibitem{BJL82}
Bell, J. L., 1982, "Categories, Toposes and Sets", Synthese, 51, 3, 293-337. 
\bibitem{BJL86}
Bell, J. L., 1986, "From Absolute to Local Mathematics", Synthese, 69, 3, 409-426. 
\bibitem{BJL88} 
Bell, J. L., 1988, Toposes and Local Set Theories: An Introduction, Oxford: Oxford University Press. 
\bibitem{BG-MCLS99}
Birkoff, G. \& Mac Lane, S., 1999, Algebra, 3rd ed., Providence: AMS.  
\bibitem{BDK2k3}
Biss, D.K., 2003, "Which Functor is the Projective Line?", American Mathematical Monthly, 110, 7, 574-592. 
\bibitem{BA-SA83}
Blass, A. \& Scedrov, A., 1983, Classifying Topoi and Finite Forcing , Journal of Pure and Applied Algebra, 28, 
111-140. 
\bibitem{BA-SA89}
Blass, A. \& Scedrov, A., 1989, Freyd's Model for the Independence of the Axiom of Choice, 
Providence: AMS.  
\bibitem{BASA92}
Blass, A. \& Scedrov, A., 1992, "Complete Topoi Representing Models of Set Theory", Annals of Pure and Applied Logic , 57, no. 1, 1-26.  
\bibitem{BA84}
Blass, A., 1984, "The Interaction Between Category Theory and Set Theory", Mathematical Applications of Category Theory, 30, Providence: AMS, 5-29. 
\bibitem{BR-SP2k4}
Blute, R. \& Scott, P., 2004, "Category Theory for Linear Logicians", in Linear Logic in Computer 
\bibitem{BAJA81}
Boileau, A. \& Joyal, A., 1981, "La logique des topos", Journal of Symbolic Logic, 46, 1, 6-16.  

\bibitem{Borceux94}
Borceux, F.: 1994, \emph{Handbook of Categorical Algebra 1 ,2 \& 3-
Encyclopedia of Mathematics and its Applications} \textbf{50} \& \textbf{51}, Cambridge University Press.
\bibitem{Bourbaki1}
Bourbaki, N. 1961 and 1964: \emph{Alg\`{e}bre commutative.},
in \`{E}l\'{e}ments de Math\'{e}matique., Chs. 1-6., Hermann: Paris.
\bibitem (BJk4)
Brown, R. and G. Janelidze: 2004, Galois theory and a new homotopy
double groupoid of a map of spaces, \emph{Applied Categorical
Structures} \textbf{12}: 63-80.
\bibitem{BHR2}
Brown, R., Higgins, P. J. and R. Sivera,: 2007a, \emph{Non-Abelian
Algebraic Topology}, in preparation.\\
http://www.bangor.ac.uk/~mas010/nonab-a-t.html ; \\
http://www.bangor.ac.uk/~mas010/nonab-t/partI010604.pdf
\bibitem{BGB2k7b}
Brown, R., Glazebrook, J. F. and I.C. Baianu.: 2007b, A Conceptual, Categorical and Higher Dimensional Algebra Framework of Universal Ontology and the Theory of Levels for Highly Complex Structures and Dynamics., \emph{Axiomathes} (17): 321-379.
\bibitem{BPP2k4}
Brown, R., Paton, R. and T. Porter.: 2004, Categorical language and
hierarchical models for cell systems, in \emph{Computation in
Cells and Tissues - Perspectives and Tools of Thought}, Paton, R.;
Bolouri, H.; Holcombe, M.; Parish, J.H.; Tateson, R. (Eds.)
Natural Computing Series, Springer Verlag, 289-303.
\bibitem{BP2k3}
Brown R. and T. Porter: 2003, Category theory and higher
dimensional algebra: potential descriptive tools in neuroscience, In:
Proceedings of the International Conference on Theoretical
Neurobiology, Delhi, February 2003, edited by Nandini Singh,
National Brain Research Centre, Conference Proceedings 1, 80-92.
\bibitem{Br-Har-Ka-Po2k2}
Brown, R., Hardie, K., Kamps, H. and T. Porter: 2002, The homotopy
double groupoid of a Hausdorff space., \emph{Theory and
Applications of Categories} \textbf{10}, 71-93.
\bibitem{Br-Hardy76}
Brown, R., and Hardy, J.P.L.:1976, Topological groupoids I:
universal constructions, \emph{Math. Nachr.}, 71: 273-286.
\bibitem{Br-Po-analogy2k6}
Brown, R. and T. Porter: 2006, Category Theory: an abstract
setting for analogy and comparison, In: What is Category Theory?,
\emph{Advanced Studies in Mathematics and Logic, Polimetrica
Publisher}, Italy, (2006) 257-274.
\bibitem{Br-Sp76}
Brown, R. and Spencer, C.B.: 1976, Double groupoids and crossed
modules, \emph{Cah.  Top. G\'{e}om. Diff.} \textbf{17}, 343-362.
\bibitem{BL2k3}
Bunge, M. and S. Lack: 2003, Van Kampen theorems for toposes, \emph{Adv. in Math.} \textbf{179}, 291-317.
\bibitem{BM74} 
Bunge, M., 1974, "Topos Theory and Souslin's Hypothesis", Journal of Pure and Applied Algebra, 4, 159-187.  
\bibitem{BM84}
Bunge, M., 1984, "Toposes in Logic and Logic in Toposes", Topoi, 3, no. 1, 13-22. 
\bibitem{CRL94}
Crole, R.L., 1994, Categories for Types, Cambridge: Cambridge University Press.  
\bibitem{CJ-LJ91}
Couture, J. \& Lambek, J., 1991, "Philosophical Reflections on the Foundations of Mathematics", Erkenntnis, 34, 2, 187–209. 
\bibitem{CRL94}
Crole, R.L., 1994, Categories for Types, Cambridge: Cambridge University Press.  
\bibitem{DJ-ALEX60-71}
Dieudonné, J. \& Grothendieck, A., 1960, [1971], Éléments de Géométrie Algébrique, Berlin: Springer-Verlag.  
\bibitem{EC}
Ehresmann, C.: 1965, \emph{Cat\'egories et Structures}, Dunod,
Paris
\bibitem{EC}
Ehresmann, C.: 1966, Trends Toward Unity in Mathematics.,
\emph{Cahiers de Topologie et Geometrie Differentielle}
\textbf{8}: 1-7.
\bibitem{Eh}
Ehresmann, C.: 1959, Cat\'egories topologiques et cat\'egories
diff\'erentiables, \emph{Coll. G\'eom. Diff. Glob.} Bruxelles, pp.137-150.
\bibitem{Eh-quintettes}
Ehresmann, C.:1963, Cat\'egories doubles des quintettes: applications covariantes
, \emph{C.R.A.S. Paris}, \textbf{256}: 1891--1894.
\bibitem{Eh-Oe}
Ehresmann, C.: 1984, \emph{Oeuvres compl\`etes et  comment\'ees:
Amiens, 1980-84}, edited and commented by Andr\'ee Ehresmann.
\bibitem{EACV1}
Ehresmann, A. C. and J.-P. Vanbremersch: 1987, Hierarchical
Evolutive Systems: A mathematical model for complex systems,
\emph{Bull. of Math. Biol.} \textbf{49} (1): 13-50.
\bibitem{EACV2}
Ehresmann, A. C. and J.-P. Vanbremersch: 2006, The Memory Evolutive Systems as
a model of Rosen's Organisms, \emph{Axiomathes} \textbf{16} (1--2): 13-50.
\bibitem{EML1}
Eilenberg, S. and S. Mac Lane.: 1942, Natural Isomorphisms in Group Theory., \emph{American Mathematical Society 43}: 757-831.
\bibitem{EL}
Eilenberg, S. and S. Mac Lane: 1945, The General Theory of Natural Equivalences, \emph{Transactions of the American Mathematical Society} \textbf{58}: 231-294.
\bibitem{ES-CH56}
Eilenberg, S. \& Cartan, H., 1956, Homological Algebra, Princeton: Princeton University Press. 
\bibitem{ES-MCLS42}
Eilenberg, S. \& MacLane, S., 1942, Group Extensions and Homology, Annals of Mathematics, 43, 757–831. 
\bibitem{ES-MCLS45}
\bibitem{ES-SN52}
Eilenberg, S. \& Steenrod, N., 1952, Foundations of Algebraic Topology, Princeton: Princeton University Press. 
\bibitem{ED88}
Ellerman, D., 1988, Category Theory and Concrete Universals, {\em Synthese}, \textbf{28}, 409-429. 
\bibitem{FREYD64}
Freyd, P.: 1964, \emph{Abelian Categories}. Harper and Row: New York and London.
\bibitem{FP65}
Freyd, P., 1965, The Theories of Functors and Models. Theories of Models, Amsterdam: North Holland, 107–120. 
\bibitem{FP72}
Freyd, P., 1972, Aspects of Topoi, Bulletin of the Australian Mathematical Society, 7, 1–76.  
\bibitem{FP80}
Freyd, P., 1980, The Axiom of Choice, Journal of Pure and Applied Algebra, 19, 103-125. 
\bibitem{FP87}
Freyd, P., 1987, Choice and Well-Ordering, Annals of Pure and Applied Logic, 35, 2, 149-166.  
\bibitem{FP90}
Freyd, P., 1990, Categories, Allegories, Amsterdam: North Holland. 
\bibitem{FP2k2}
Freyd, P., 2002, Cartesian Logic, Theoretical Computer Science, 278, no. 1-2, 3-21.  
\bibitem{FP-FH-SA87}
Freyd, P., Friedman, H. \& Scedrov, A., 1987, Lindembaum Algebras of Intuitionistic Theories and Free Categories, Annals of Pure and Applied Logic, 35, 2, 167-172.  
\bibitem{FS77}
Feferman, S., 1977, Categorical Foundations and Foundations of Category Theory., Logic, Foundations of Mathematics and Computability, R. Butts (ed.), Reidel, 149-169.
\bibitem{Gablot}
Gablot, R. 1971. Sur deux classes de cat\'{e}gories de Grothendieck. Thesis..  Univ. de Lille.
\bibitem{Gabriel1}
Gabriel, P.: 1962, Des cat\'egories ab\'eliennes, \emph{Bull. Soc.
Math. France} \textbf{90}: 323-448.
\bibitem{Gabriel2}
Gabriel, P. and M. Zisman:. 1967: \emph{Category of fractions and homotopy theory}, \emph{Ergebnesse der math.} Springer: Berlin.
\bibitem{GabrielNP}
Gabriel, P. and N. Popescu: 1964, Caract\'{e}risation des cat\'egories ab\'eliennes
avec g\'{e}n\'{e}rateurs et limites inductives. , \emph{CRAS Paris} \textbf{258}: 4188-4191.
\bibitem{GR2k}
Galli, A. \& Reyes, G. \& Sagastume, M., 2000, "Completeness Theorems via the Double Dual Functor", Studia Logical, 64, no. 1, 61-81. 
\bibitem{GS-ZM2K2}
Ghilardi, S. \& Zawadowski, M., 2002, Sheaves, Games \& Model Completions: A Categorical Approach to Nonclassical Porpositional Logics, Dordrecht: Kluwer.  
\bibitem{gs89}
Ghilardi, S., 1989, Presheaf Semantics and Independence Results for some Non-classical first-order logics, Archive for Mathematical Logic, 29, no. 2, 125-136. 
\bibitem{GOR58}
Godement,R. 1958. Th\'{e}orie des faisceaux. Hermann: Paris.
\bibitem{GR79} 
Goldblatt, R., 1979, Topoi: The Categorical Analysis of Logic, Studies in logic and the foundations of mathematics, Amsterdam: Elsevier.
\bibitem{GRAY65}
Gray, C. W.: 1965. Sheaves with values in a category.,\emph {Topology}, 3: 1-18.  
\bibitem{ALEXsem}
Grothendieck, A. et al., Séminaire de Géométrie Algébrique, Vol. 1--7, Berlin: Springer-Verlag.
\bibitem{ALEX57}
Grothendieck, A., 1957, Sur Quelques Points d'algèbre homologique, Tohoku Mathematics Journal, 9, 119–221.  
\bibitem{Alex1}
Grothendieck, A.: 1971, Rev\^{e}tements \'Etales et Groupe Fondamental (SGA1),
chapter VI: Cat\'egories fibr\'ees et descente, \emph{Lecture Notes in Math.}
\textbf{224}, Springer--Verlag: Berlin.
\bibitem{Alex2}
Grothendieck, A.: 1957, Sur quelque point d-alg\'{e}bre homologique. , \emph{Tohoku Math. J.}, \textbf{9:} 119-121.
\bibitem{Alex3}
Grothendieck, A. and J. Dieudon\'{e}.: 1960, El\'{e}ments de geometrie alg\'{e}brique., \emph{Publ. Inst. des Hautes Etudes de Science}, \textbf{4}.
\bibitem{HWS82}
Hatcher, W. S., 1982, The Logical Foundations of Mathematics, Oxford: Pergamon Press. 

\bibitem{Heller58}
Heller, A. :1958, Homological algebra in Abelian categories., \emph{Ann. of Math.}
\textbf{68}: 484-525.

\bibitem{HellerRowe62}
Heller, A.  and K. A. Rowe.:1962, On the category of sheaves., \emph{Amer J. Math.}
\textbf{84}: 205-216.

\bibitem{HG2k3}
Hellman, G., 2003, "Does Category Theory Provide a Framework for Mathematical Structuralism?", Philosophia Mathematica, 11, 2, 129-157. 

\bibitem{HC-MM-PJ2K}
Hermida, C. \& Makkai, M. \& Power, J., 2000, "On Weak Higher-dimensional Categories I", Journal of Pure and Applied Algebra, 154, no. 1-3, 221-246. 

\bibitem{HC-MM-PI2K1}
Hermida, C. \& Makkai, M. \& Power, J., 2001, "On Weak Higher-dimensional Categories 2", Journal of Pure and Applied Algebra, 157, no. 2-3, 247-277.  

\bibitem{HC-MM-PI2K2}
Hermida, C. \& Makkai, M. \& Power, J., 2002, "On Weak Higher-dimensional Categories 3", Journal of Pure and Applied Algebra, 166, no. 1-2, 83-104.  

\bibitem{HPJbook}
Higgins, P. J.: 2005, \emph{Categories and groupoids}, Van
Nostrand Mathematical Studies: 32, (1971); \emph{Reprints in
Theory and Applications of Categories}, No. 7: 1-195.

\bibitem{HPJ2k5}
Higgins, Philip J. Thin elements and commutative shells in cubical
$\omega$-categories. Theory Appl. Categ. 14 (2005), No. 4, 60-74
(electronic). (Reviewer: Timothy Porter) 18D05.

\bibitem{HJ-RE-RG90}
Hyland,  J.M.E. \& Robinson,  E.P. \& Rosolini, G., 1990, "The Discrete Objects in the Effective Topos", Proceedings of the London Mathematical Society (3), 60, no. 1, 1-36. 

\bibitem{HJME82}
Hyland,  J.M.E., 1982, "The Effective Topos", Studies in Logic and the Foundations of Mathematics, 110, Amsterdam: North Holland, 165-216.  

\bibitem{HJME88}
Hyland, J. M..E., 1988, "A Small Complete Category", Annals of Pure and Applied Logic, 40, no. 2, 135–165. 

\bibitem{HJME91}
Hyland,  J. M .E., 1991, "First Steps in Synthetic Domain Theory", Category Theory (Como 1990), Lecture Notes in Mathematics, 1488, Berlin: Springer, 131-156.  

\bibitem{HJME2K2}
Hyland, J. M.E., 2002, "Proof Theory in the Abstract", Annals of Pure and Applied Logic, 114, no. 1–3, 43-78. 

\bibitem{JB99}
Jacobs, B., 1999, Categorical Logic and Type Theory, Amsterdam: North Holland.  

\bibitem{JPT77}
Johnstone, P. T., 1977, Topos Theory, New York: Academic Press. 

\bibitem{JPT79A}
Johnstone, P. T., 1979a, "Conditions Related to De Morgan's Law", Applications of Sheaves, Lecture Notes in Mathematics, 753, Berlin: Springer, 479-491. 

\bibitem{JPT79B}
Johnstone, P.T., 1979b, "Another Condition Equivalent to De Morgan's Law", Communications in Algebra, 7, no. 12, 
1309-1312.  

\bibitem{JPT81}
Johnstone, P. T., 1981, "Tychonoff's Theorem without the Axiom of Choice", Fundamenta Mathematicae, 113, no. 1,
21-35. 

\bibitem{JPT85}
Johnstone, P. T., 1985, "How General is a Generalized Space?", Aspects of Topology, Cambridge: Cambridge University Press, 77-111. 

\bibitem{JPT2K2A}
Johnstone, P. T., 2002a, Sketches of an Elephant: a Topos Theory Compendium. Vol. 1, Oxford Logic Guides, 43, Oxford: Oxford University Press.  

\bibitem{JAMI95}
Joyal, A. \& Moerdijk, I., 1995, Algebraic Set Theory, Cambridge: Cambridge University Press.  

\bibitem{TMMFJ84}
Groups Authors: João Faria Martins, Timothy Porter.,
On Yetter's Invariant and an Extension of the Dijkgraaf-Witten Invariant to Categorical
$math.QA/0608484 [abs, ps, pdf, other]$.

\bibitem{KDM58}
Kan, D. M., 1958, "Adjoint Functors", Transactions of the American Mathematical Society, 87, 294–329.  

\bibitem{Kleisli62}
Kleisli, H.: 1962, Homotopy theory in Abelian categories., Can. J. Math., 14: 139-169.

\bibitem{KA81}
Kock, A., 1981, Synthetic Differential Geometry, London Mathematical Society Lecture Note Series, 51, Cambridge: Cambridge University Press. 

\bibitem{LPRM94}
La Palme Reyes, M., et. al., 1994, "The non-Boolean Logic of Natural Language Negation", Philosophia Mathematica, 2, no. 1, 45-68.

\bibitem{LPRM99} 
La Palme Reyes, M., et. al., 1999, "Count Nouns, Mass Nouns, and their Transformations: a Unified Category-theoretic Semantics", Language, Logic and Concepts, Cambridge: MIT Press, 427-452.  

\bibitem{LJ-SPJ81}
Lambek, J. \& Scott, P. J., 1981, "Intuitionistic Type Theory and Foundations", Journal of Philosophical Logic, 10, 1, 101–115. 

\bibitem{LJ-SPJ86}
Lambek, J. \& Scott, P.J., 1986, Introduction to Higher Order Categorical Logic, Cambridge: Cambridge University Press. 

\bibitem{LJ68}
Lambek, J., 1968, "Deductive Systems and Categories I. Syntactic Calculus and Residuated Categories", Mathematical Systems Theory, 2, 287-318. 

\bibitem{LJ69}
Lambek, J., 1969, "Deductive Systems and Categories II. Standard Constructions and Closed Categories", Category Theory, Homology Theory and their Applications I, Berlin: Springer, 76-122. 

\bibitem{LJ72}
Lambek, J., 1972, "Deductive Systems and Categories III. Cartesian Closed Categories, Intuitionistic Propositional Calculus, and Combinatory Logic", Toposes, Algebraic Geometry and Logic, Lecture Notes in Mathematics, 274, Berlin: Springer, 57-82.  

\bibitem{LJ82} 
Lambek, J., 1982, "The Influence of Heraclitus on Modern Mathematics", Scientific Philosophy Today, J. Agassi and R.S. Cohen, eds., Dordrecht, Reidel, 111-122.  

\bibitem{LJ86}
Lambek, J., 1986, "Cartesian Closed Categories and Typed lambda calculi", Combinators and Functional Programming Languages, Lecture Notes in Computer Science, 242, Berlin: Springer, 136-175.   

\bibitem{LT89A}
Lambek, J., 1989A, "On Some Connections Between Logic and Category Theory", Studia Logica, 48, 3, 269-278. 

\bibitem{LJ89B}
Lambek, J., 1989B, "On the Sheaf of Possible Worlds", Categorical Topology and its relation to Analysis, Algebra and Combinatorics, Teaneck: World Scientific Publishing, 36-53. 

\bibitem{LJ94a}
Lambek, J., 1994a, "Some Aspects of Categorical Logic", Logic, Methodology and Philosophy of Science IX, Studies in Logic and the Foundations of Mathematics 134, Amsterdam: North Holland, 69-89. 
\bibitem{LJ94b}
Lambek, J., 1994b, "What is a Deductive System?", What is a Logical System?, Studies in Logic and Computation, 4, Oxford: Oxford University Press, 141-159.  
 
\bibitem{LJ2k4}
Lambek, J., 2004, "What is the world of Mathematics? Provinces of Logic Determined", Annals of Pure and Applied Logic, 126(1-3), 149–158. 

\bibitem{LaSc}
J.~Lambek and P.~J.~Scott. {\em Introduction to higher order categorical logic}. Cambridge University Press, 1986.

\bibitem{LE-MJP2k5}
Landry, E. \& Marquis, J.-P., 2005, "Categories in Context: Historical, Foundational and philosophical", Philosophia Mathematica, 13, 1-43.  

\bibitem{LE99}
Landry, E., 1999, "Category Theory: the Language of Mathematics", Philosophy of Science, 66, 3: supplement, S14–S27. 

\bibitem{LE99}
Landry, E., 2001, "Logicism, Structuralism and Objectivity", Topoi, 20, 1, 79-95.  

\bibitem{LFW64}
Lawvere, F. W., 1964, "An Elementary Theory of the Category of Sets", Proceedings of the National Academy of Sciences U.S.A., 52, 1506-1511. 

\bibitem{LFW65}
Lawvere, F. W., 1965, "Algebraic Theories, Algebraic Categories, and Algebraic Functors", Theory of Models, Amsterdam: North Holland, 413-418.  

\bibitem{LFW66}
Lawvere, F. W., 1966, "The Category of Categories as a Foundation for Mathematics", Proceedings of the Conference on Categorical Algebra, La Jolla, New York: Springer-Verlag, 1-21. 

\bibitem{LFW69a}
Lawvere, F. W., 1969a, "Diagonal Arguments and Cartesian Closed Categories", Category Theory, Homology Theory, and their Applications II, Berlin: Springer, 134-145.  

\bibitem{LFW69b}
Lawvere, F. W., 1969b, "Adjointness in Foundations", Dialectica, 23, 281-295.  

\bibitem{LFW70}
Lawvere, F. W., 1970, "Equality in Hyper doctrines and Comprehension Schema as an Adjoint Functor", Applications of Categorical Algebra, Providence: AMS, 1-14.  

\bibitem{LT271}
Lawvere, F. W., 1971, "Quantifiers and Sheaves", Actes du Congrès International des Mathématiciens, Tome 1, Paris: Gauthier-Villars, 329-334. 

\bibitem{LFW72}
Lawvere, F. W., 1972, "Introduction", Toposes, Algebraic Geometry and Logic, Lecture Notes in Mathematics, 274, Springer-Verlag, 1-12.  

\bibitem{LFW75}
Lawvere, F. W., 1975, "Continuously Variable Sets: Algebraic Geometry = Geometric Logic", Proceedings of the Logic Colloquium Bristol 1973, Amsterdam: North Holland, 135-153. 

\bibitem{LFW76}
Lawvere, F. W., 1976, "Variable Quantities and Variable Structures in Topoi", Algebra, Topology, and Category Theory, New York: Academic Press, 101-131. 

\bibitem{LFW97}
Lawvere, F. W. \& Schanuel, S., 1997, Conceptual Mathematics: A First Introduction to Categories, Cambridge: Cambridge University Press. 

\bibitem{LFW66}
Lawvere, F. W.: 1966, The Category of Categories as a Foundation for Mathematics., in
\emph{Proc. Conf. Categorical Algebra--La Jolla}., Eilenberg, S. et al., eds. Springer--Verlag:
Berlin, Heidelberg and New York., pp. 1-20.

\bibitem{LFW63}
Lawvere, F. W.: 1963, Functorial Semantics of Algebraic Theories,
\emph{Proc. Natl. Acad. Sci. USA, Mathematics}, \textbf{50}: 869-872.

\bibitem{LFW69}
Lawvere, F. W.: 1969, \emph{Closed Cartesian Categories}., Lecture held as a guest of the
Romanian Academy of Sciences, Bucharest.

\bibitem{LFW92}
Lawvere, F. W., 1992, "Categories of Space and of Quantity", The Space of Mathematics, Foundations of Communication and Cognition, Berlin: De Gruyter, 14-30.  

\bibitem{LFW94a}
Lawvere, F. W., 1994a, "Cohesive Toposes and Cantor's lauter Ensein ", Philosophia Mathematica, 2, 1, 5–15. 

\bibitem{LFW94b}
Lawvere, F. W., 1994b, "Tools for the Advancement of Objective Logic: Closed Categories and Toposes", The Logical Foundations of Cognition, Vancouver Studies in Cognitive Science, 4, Oxford: Oxford University Press, 43-56.  

\bibitem{LFW95}
Lawvere, H. W (ed.), 1995. Springer Lecture Notes in Mathematics 274,:13-42. 

\bibitem{LFW2k}
Lawvere, F. W., 2000, "Comments on the Development of Topos Theory", Development of Mathematics 1950-2000, Basel: Birkhäuser, 715–734. 

\bibitem{LFW2k2}
Lawvere, F. W., 2002, "Categorical Algebra for Continuum Micro Physics", Journal of Pure and Applied Algebra, 175, no. 1-3, 267-287. 

\bibitem{LFW-RR2k3}
Lawvere, F. W. \& Rosebrugh, R., 2003, Sets for Mathematics, Cambridge: Cambridge University Press.  

\bibitem{LFWk3}
Lawvere, F. W., 2003, "Foundations and Applications: Axiomatization and Education. New Programs and Open Problems in the Foundation of Mathematics", Bullentin of Symbolic Logic, 9, 2, 213–224. 
Lawvere, F.W., 1963, "Functorial Semantics of Algebraic Theories", Proceedings of the National Academy of Sciences U.S.A., 50, 869-872. 

\bibitem{LT2k2}
Leinster, T., 2002, "A Survey of Definitions of n-categories", Theory and Applications of Categories, (electronic), 10, 1-70. 

\bibitem{MCLSS69}
MacLane, S., 1969, "Foundations for Categories and Sets", Category Theory, Homology Theory and their Applications II, Berlin: Springer, 146–164. 

\bibitem{MCLS69}MacLane, S., 1969, "One Universe as a Foundation for Category Theory", Reports of the Midwest Category Seminar III, Berlin: Springer, 192-200. 

\bibitem{MCLS71}MacLane, S., 1971, "Categorical algebra and Set-Theoretic Foundations", Axiomatic Set Theory, Providence: AMS, 231-240. 

\bibitem{MCLS75}
MacLane, S., 1975, "Sets, Topoi, and Internal Logic in Categories", Studies in Logic and the Foundations of Mathematics, 80, Amsterdam: North Holland, 119-134. 

\bibitem{MCLS81}
MacLane, S., 1981, "Mathematical Models: a Sketch for the Philosophy of Mathematics", American Mathematical Monthly, 88, 7, 462-472.
 
\bibitem{MCLS86}
MacLane, S., 1986, Mathematics, Form and Function, New York: Springer. 

\bibitem{MCLS88}
MacLane, S., 1988, "Concepts and Categories in Perspective", A Century of Mathematics in America, Part I, Providence: AMS, 323-365. 

\bibitem{MCLS89}
MacLane, S., 1989, "The Development of Mathematical Ideas by Collision: the Case of Categories and Topos Theory", Categorical Topology and its Relation to Analysis, Algebra and Combinatorics, Teaneck: World Scientific, 1-9.

\bibitem{MLS-MI92}
MacLane, S. \& Moerdijk, I., 1992, Sheaves in Geometry and Logic, New York: Springer-Verlag. 

\bibitem{MCLS50}
MacLane, S., 1950, "Dualities for Groups", Bulletin of the American Mathematical Society, 56, 485-516. 

\bibitem{MCLS96} 
MacLane, S., 1996, "Structure in Mathematics. Mathematical Structuralism", Philosophia Mathematica, 4, 2, 174-183. 

\bibitem{MCLS98}
MacLane, S., 1997, Categories for the Working Mathematician, 2nd edition, New York: Springer-Verlag. 

\bibitem{MCLS97}
MacLane, S., 1997, "Categorical Foundations of the Protean Character of Mathematics", Philosophy of Mathematics Today, Dordrecht: Kluwer, 117-122. 

\bibitem{MaMo}
S.~Mac~Lane and I.~Moerdijk. {\em Sheaves and Geometry in Logic: A First Introduction to Topos Theory}, Springer-Verlag, 1992.

\bibitem{MM-RG95} 
Makkai, M. \& Paré, R., 1989, Accessible Categories: the Foundations of Categorical Model Theory, Contemporary Mathematics 104, Providence: AMS. 

\bibitem{MM98}
Makkai, M., 1998, "Towards a Categorical Foundation of Mathematics", Lecture Notes in Logic, 11, Berlin: Springer, 
153-190. 

\bibitem{MM99}
Makkai, M., 1999, "On Structuralism in Mathematics", Language, Logic and Concepts, Cambridge: MIT Press, 43-66. 

\bibitem{MM-RG77}
Makkai, M. \& Reyes, G., 1977, First-Order Categorical Logic, Springer Lecture Notes in Mathematics 611, New York: Springer. 

\bibitem{MM98}
Makkai, M., 1998, "Towards a Categorical Foundation of Mathematics", Lecture Notes in Logic, 11, 
Berlin: Springer, 153-190. 

\bibitem{MM99}
Makkai, M., 1999, "On Structuralism in Mathematics", Language, Logic and Concepts, Cambridge: 
MIT Press, 43-66. 

\bibitem{MM-RG95}
Makkei, M. \& Reyes, G., 1995, "Completeness Results for Intuitionistic and Modal Logic in a Categorical Setting", Annals of Pure and Applied Logic, 72, 1, 25-101. 

\bibitem{MJP93}
Marquis, J.-P., 1993, "Russell's Logicism and Categorical Logicisms", Russell and Analytic Philosophy, A. D. Irvine \& G. A. Wedekind, (eds.), Toronto, University of Toronto Press, 293-324.

\bibitem{MJP95}
Marquis, J.-P., 1995, "Category Theory and the Foundations of Mathematics: Philosophical Excavations", Synthese, 103, 421–447. 

\bibitem{MJP2K}
Marquis, J.-P., 2000, "Three Kinds of Universals in Mathematics?", Logical Consequence: Rival Approaches and New Studies in Exact Philosophy: Logic, Mathematics and Science, Vol. 2, Oxford: Hermes, 191-212. 

\bibitem{MJP2k}
Marquis, J.-P., 2000, "Three Kinds of Universals in Mathematics?", in Logical Consequence: Rival Approaches and New Studies in Exact Philosophy: Logic, Mathematics and Science, Vol. II, B. Brown \& J. Woods, eds., 
Oxford: Hermes, 191-212, 2000 ,

\bibitem{MJP2k6} 
Marquis, J.-P., 2006, "Categories, Sets and the Nature of Mathematical Entities", in The Age of Alternative Logics. Assessing philosophy of logic and mathematics today, J. van Benthem, G. Heinzmann, Ph. Nabonnand, M. Rebuschi, H.Visser, eds., Springer,181-192. 

\bibitem{MLC86}
Mc Larty, C., 1986, "Left Exact Logic", Journal of Pure and Applied Algebra, 41, no. 1, 63-66.

\bibitem{MLC91}
Mc Larty, C., 1991, "Axiomatizing a Category of Categories", Journal of Symbolic Logic, 56, no. 4, 
1243-1260. 

\bibitem{MLC92} 
Mc Larty, C., 1992, Elementary Categories, Elementary Toposes, Oxford: Oxford University Press.

\bibitem{MLC94}
Mc Larty, C., 1994, "Category Theory in Real Time", Philosophia Mathematica, 2, no. 1, 36-44.

\bibitem{2k4}
Mc Larty, C., 2004, "Exploring Categorical Structuralism", Philosophia Mathematica, 12, 37-53.

\bibitem{MLC2k5}
Mc Larty, C., 2005, "Learning from Questions on Categorical Foundations", 
Philosophia Mathematica, 13, 1, 44-60.

\bibitem{Mitchell1}
Mitchell, B.: 1965, \emph{Theory of Categories}, Academic Press:London.

\bibitem{Mitchell2}
Mitchell, B.: 1964, The full imbedding theorem. \emph{Amer. J. Math}. \textbf{86}: 619-637.

\bibitem{MI-P2k2}
Moerdijk, I. \& Palmgren, E., 2002, "Type Theories, Toposes and Constructive Set Theory: Predicative Aspects of AST", Annals of Pure and Applied Logic, 114, nos. 1-3, 155-201. 

\bibitem{MO98}
Moerdijk, I., 1998, "Sets, Topoi and Intuitionism", Philosophia Mathematica, 6, no. 2, 169-177.

\bibitem{OB69}
Oberst, U.: 1969, Duality theory for Grothendieck categories., \emph{Bull. Amer. Math. Soc.} \textbf{75}: 1401-1408.

\bibitem{ORT70}
Oort, F.: 1970. On the definition of an abelian category.
\emph{Proc. Roy. Neth. Acad. Sci}. \textbf{70}: 13-02.

\bibitem{NPop1}
Popescu, N.: 1973, \emph{Abelian Categories with Applications to
Rings and Modules.} New York and London: Academic Press., 2nd edn.
1975. \emph{(English translation by I.C. Baianu)}.

\bibitem{PB70}
Pareigis, B., 1970, Categories and Functors, New York: Academic Press. 

\bibitem{PMC2k4}
Pedicchio, M. C. \& Tholen, W., 2004, Categorical Foundations, Cambridge: Cambridge University Press. 

\bibitem{PB91}
Peirce, B., 1991, Basic Category Theory for Computer Scientists, Cambridge: MIT Press. 

\bibitem{PAM90}
Pitts, A. M., 1989, "Conceptual Completeness for First-order Intuitionistic Logic: an Application of Categorical Logic", Annals of Pure and Applied Logic, 41, no. 1, 33-81. 

\bibitem{PAM2k}
Pitts, A. M., 2000, "Categorical Logic", Handbook of Logic in Computer Science, Vol.5, Oxford: Oxford Unversity Press, 39-128.

\bibitem{PB2k} 
Plotkin, B., 2000, "Algebra, Categories and Databases", Handbook of Algebra, Vol. 2, Amsterdam: Elsevier, 79–148. 

\bibitem{RGZH91}
Reyes, G. \& Zolfaghari, H., 1991, "Topos-theoretic Approaches to Modality", Category Theory (Como 1990), Lecture Notes in Mathematics, 1488, Berlin: Springer, 359-378. 

\bibitem{RGZH96}
Reyes, G. \& Zolfaghari, H., 1996, "Bi-Heyting Algebras, Toposes and Modalities", Journal of Philosophical Logic, 25, no. 1, 25-43. 

\bibitem{RG74}
Reyes, G., 1974, "From Sheaves to Logic", in Studies in Algebraic Logic, A. Daigneault, ed., Providence: AMS. 

\bibitem{RG91}
Reyes, G., 1991, "A Topos-theoretic Approach to Reference and Modality", Notre Dame Journal of Formal Logic, 32, no. 3, 359-391.

\bibitem{RSE-KEP94} 
Rodabaugh, S. E. \& Klement, E. P., eds., Topological and Algebraic Structures in Fuzzy Sets: A Handbook of Recent Developments in the Mathematics of Fuzzy Sets, Trends in Logic, 20, Dordrecht: Kluwer. 

\bibitem{SPJ2k} 
Scott, P. J., 2000, "Some Aspects of Categories in Computer Science", Handbook of Algebra, Vol. 2, Amsterdam: North Holland, 3-77. 

\bibitem{SRAG84}
Seely, R. A. G., 1984, "Locally Cartesian Closed Categories and Type Theory", Mathematical Proceedings of the Cambridge Mathematical Society, 95, no. 1, 33-48. 

\bibitem{SHS2k5}
Shapiro, S., 2005, "Categories, Structures and the Frege-Hilbert Controversy: the Status of Metamathematics", Philosophia Mathematica, 13, 1, 61-77.

\bibitem{TP96} 
Taylor, P., 1996, "Intuitionistic sets and Ordinals", Journal of Symbolic Logic, 61, 705-744.
 
\bibitem{TP99}
Taylor, P., 1999, Practical Foundations of Mathematics, Cambridge: Cambridge University Press. 

\bibitem{TP72}
Tierney, M., 1972, "Sheaf Theory and the Continuum Hypothesis", Toposes, Algebraic Geometry and Logic, 

\bibitem{VdHG-MI84a}
Van der Hoeven, G. \& Moerdijk, I., 1984a, "Sheaf Models for Choice Sequences", Annals of Pure and Applied Logic, 27, no. 1, 63-107. 

\bibitem{VdHG-MI84b}
Van der Hoeven, G. \& Moerdijk, I., 1984b, "On Choice Sequences determined by Spreads", Journal of Symbolic Logic, 49, no. 3, 908-916. 

\bibitem{WRJ2k4}
Wood, R.J., 2004, " Ordered Sets via Adjunctions", Categorical Foundations, M. C. Pedicchio \& W. Tholen, eds., Cambridge: Cambridge University Press.
 
\end{thebibliography}

%%%%%
%%%%%
\end{document}
