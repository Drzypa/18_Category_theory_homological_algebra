\documentclass[12pt]{article}
\usepackage{pmmeta}
\pmcanonicalname{AlgebraicCategoriesAndClassesOfAlgebras}
\pmcreated{2013-03-22 18:15:29}
\pmmodified{2013-03-22 18:15:29}
\pmowner{bci1}{20947}
\pmmodifier{bci1}{20947}
\pmtitle{algebraic categories and classes of algebras}
\pmrecord{56}{40855}
\pmprivacy{1}
\pmauthor{bci1}{20947}
\pmtype{Topic}
\pmcomment{trigger rebuild}
\pmclassification{msc}{18-00}
\pmclassification{msc}{18E05}
\pmclassification{msc}{08A05}
\pmclassification{msc}{08A70}
\pmsynonym{algebras}{AlgebraicCategoriesAndClassesOfAlgebras}
\pmsynonym{class of specific algebraic structures}{AlgebraicCategoriesAndClassesOfAlgebras}
%\pmkeywords{algebras}
%\pmkeywords{classes of algebras}
%\pmkeywords{class of specific algebraic structures or general algebras}
%\pmkeywords{algebraic category}
%\pmkeywords{higher dimensional algebras (HDA)}
%\pmkeywords{classes of superalgebras}
\pmrelated{Algebras2}
\pmrelated{CategoricalAlgebras}
\pmrelated{CAlgebra3}
\pmrelated{HigherDimensionalAlgebraHDA}
\pmrelated{OmegaSpectrum}
\pmrelated{AlgebraicTopology}
\pmdefines{algebraic category}
\pmdefines{monadic functor}
\pmdefines{monad on a category}
\pmdefines{T-algebra}
\pmdefines{structure map}
\pmdefines{category of Eilenberg-Moore  algebras}

% this is the default PlanetMath preamble.  

% almost certainly you want these
\usepackage{amssymb}
\usepackage{amsmath}
\usepackage{amsfonts}
% used for TeXing text within eps files
%\usepackage{psfrag}
% need this for including graphics (\includegraphics)
%\usepackage{graphicx}
% for neatly defining theorems and propositions
%\usepackage{amsthm}
% making logically defined graphics
%%%\usepackage{xypic}

% there are many more packages

% define commands here
\usepackage{amsmath, amssymb, amsfonts, amsthm, amscd, latexsym}
%%\usepackage{xypic}
\usepackage[mathscr]{eucal}

\setlength{\textwidth}{6.5in}
%\setlength{\textwidth}{16cm}
\setlength{\textheight}{9.0in}
%\setlength{\textheight}{24cm}

\hoffset=-.75in     %%ps format
%\hoffset=-1.0in     %%hp format
\voffset=-.4in

\theoremstyle{plain}
\newtheorem{lemma}{Lemma}[section]
\newtheorem{proposition}{Proposition}[section]
\newtheorem{theorem}{Theorem}[section]
\newtheorem{corollary}{Corollary}[section]

\theoremstyle{definition}
\newtheorem{definition}{Definition}[section]
\newtheorem{example}{Example}[section]
%\theoremstyle{remark}
\newtheorem{remark}{Remark}[section]
\newtheorem*{notation}{Notation}
\newtheorem*{claim}{Claim}

\renewcommand{\thefootnote}{\ensuremath{\fnsymbol{footnote%%@
}}}
\numberwithin{equation}{section}

\newcommand{\Ad}{{\rm Ad}}
\newcommand{\Aut}{{\rm Aut}}
\newcommand{\Cl}{{\rm Cl}}
\newcommand{\Co}{{\rm Co}}
\newcommand{\DES}{{\rm DES}}
\newcommand{\Diff}{{\rm Diff}}
\newcommand{\Dom}{{\rm Dom}}
\newcommand{\Hol}{{\rm Hol}}
\newcommand{\Mon}{{\rm Mon}}
\newcommand{\Hom}{{\rm Hom}}
\newcommand{\Ker}{{\rm Ker}}
\newcommand{\Ind}{{\rm Ind}}
\newcommand{\IM}{{\rm Im}}
\newcommand{\Is}{{\rm Is}}
\newcommand{\ID}{{\rm id}}
\newcommand{\GL}{{\rm GL}}
\newcommand{\Iso}{{\rm Iso}}
\newcommand{\Sem}{{\rm Sem}}
\newcommand{\St}{{\rm St}}
\newcommand{\Sym}{{\rm Sym}}
\newcommand{\SU}{{\rm SU}}
\newcommand{\Tor}{{\rm Tor}}
\newcommand{\U}{{\rm U}}

\newcommand{\A}{\mathcal A}
\newcommand{\Ce}{\mathcal C}
\newcommand{\D}{\mathcal D}
\newcommand{\E}{\mathcal E}
\newcommand{\F}{\mathcal F}
\newcommand{\G}{\mathcal G}
\newcommand{\Q}{\mathcal Q}
\newcommand{\R}{\mathcal R}
\newcommand{\cS}{\mathcal S}
\newcommand{\cU}{\mathcal U}
\newcommand{\W}{\mathcal W}

\newcommand{\bA}{\mathbb{A}}
\newcommand{\bB}{\mathbb{B}}
\newcommand{\bC}{\mathbb{C}}
\newcommand{\bD}{\mathbb{D}}
\newcommand{\bE}{\mathbb{E}}
\newcommand{\bF}{\mathbb{F}}
\newcommand{\bG}{\mathbb{G}}
\newcommand{\bK}{\mathbb{K}}
\newcommand{\bM}{\mathbb{M}}
\newcommand{\bN}{\mathbb{N}}
\newcommand{\bO}{\mathbb{O}}
\newcommand{\bP}{\mathbb{P}}
\newcommand{\bR}{\mathbb{R}}
\newcommand{\bV}{\mathbb{V}}
\newcommand{\bZ}{\mathbb{Z}}

\newcommand{\bfE}{\mathbf{E}}
\newcommand{\bfX}{\mathbf{X}}
\newcommand{\bfY}{\mathbf{Y}}
\newcommand{\bfZ}{\mathbf{Z}}

\renewcommand{\O}{\Omega}
\renewcommand{\o}{\omega}
\newcommand{\vp}{\varphi}
\newcommand{\vep}{\varepsilon}

\newcommand{\diag}{{\rm diag}}
\newcommand{\grp}{{\mathbb G}}
\newcommand{\dgrp}{{\mathbb D}}
\newcommand{\desp}{{\mathbb D^{\rm{es}}}}
\newcommand{\Geod}{{\rm Geod}}
\newcommand{\geod}{{\rm geod}}
\newcommand{\hgr}{{\mathbb H}}
\newcommand{\mgr}{{\mathbb M}}
\newcommand{\ob}{{\rm Ob}}
\newcommand{\obg}{{\rm Ob(\mathbb G)}}
\newcommand{\obgp}{{\rm Ob(\mathbb G')}}
\newcommand{\obh}{{\rm Ob(\mathbb H)}}
\newcommand{\Osmooth}{{\Omega^{\infty}(X,*)}}
\newcommand{\ghomotop}{{\rho_2^{\square}}}
\newcommand{\gcalp}{{\mathbb G(\mathcal P)}}

\newcommand{\rf}{{R_{\mathcal F}}}
\newcommand{\glob}{{\rm glob}}
\newcommand{\loc}{{\rm loc}}
\newcommand{\TOP}{{\rm TOP}}

\newcommand{\wti}{\widetilde}
\newcommand{\what}{\widehat}

\renewcommand{\a}{\alpha}
\newcommand{\be}{\beta}
\newcommand{\ga}{\gamma}
\newcommand{\Ga}{\Gamma}
\newcommand{\de}{\delta}
\newcommand{\del}{\partial}
\newcommand{\ka}{\kappa}
\newcommand{\si}{\sigma}
\newcommand{\ta}{\tau}
\newcommand{\med}{\medbreak}
\newcommand{\medn}{\medbreak \noindent}
\newcommand{\bign}{\bigbreak \noindent}
\newcommand{\lra}{{\longrightarrow}}
\newcommand{\ra}{{\rightarrow}}
\newcommand{\rat}{{\rightarrowtail}}
\newcommand{\oset}[1]{\overset {#1}{\ra}}
\newcommand{\osetl}[1]{\overset {#1}{\lra}}
\newcommand{\hr}{{\hookrightarrow}}
\begin{document}
\subsection{Introduction}
Classes of algebras can be categorized at least in two types: either classes of \emph{specific
algebras}, such as: group algebras, K-algebras, groupoid algebras, logic algebras, and so on, 
or \emph{general} ones, such as general classes of: categorical algebras, higher dimensional algebra
(HDA), supercategorical algebras, universal algebras, and so on.

\subsection{Basic concepts and definitions}

\begin{itemize}
\item {\bf Class of algebras}
\begin{definition}
 A \emph{class of algebras} is defined in a precise sense as an algebraic object in the
\emph{groupoid category}.

\end{definition}

\item {\bf Monad on a category $\mathcal{C}$, and a T-algebra in $\mathcal{C}$}

\begin{definition}
Let us consider a category $\mathcal{C}$, two functors: $T: \mathcal{C} \to \mathcal{C}$ (called the monad functor) and $T^2: \mathcal{C} \to \mathcal{C} = T \circ T$, and two natural transformations:
$\eta: 1_ \mathcal{C} \to T$ and $\mu: T^2 \to T$. The triplet $(\mathcal{C},\eta,\mu)$
is called a {\em monad on the category $\mathcal{C}$}. Then, a {\em T-algebra} $(Y,h)$ is defined as an object $Y$ of a category $\mathcal{C}$ together with an arrow $h: TY \to Y $ called the {\em structure map} in  $\mathcal{C}$ such that:

\begin{enumerate}
\item $$Th: T^2 \to TY,$$ 

\item $$h \circ Th = h \circ \mu_Y,$$
 where: $\mu_Y: T^2 Y \to TY;$ and

\item $$ h \circ \eta_Y = 1_Y.$$
\end{enumerate}

\end{definition}

\item {\bf Category of Eilenberg-Moore algebras of a monad $T$}
An important definition related to abstract classes of algebras and universal algebras is that of the category of Eilenberg-Moore algebras of a monad $T$:
 
\begin{definition}
 The category $\mathcal{C}^T$ of $T$-algebras and their morphisms is called the {\em Eilenberg-Moore category} or {\em category of Eilenberg-Moore algebras} of the monad T.
\end{definition}

\end{itemize}

\subsection{Remarks}
\begin{itemize}
\item {\bf a. Algebraic category definition}

\begin{remark}
 With the above definition, one can also define a \emph{category of classes of algebras and their
associated groupoid homomorphisms} which is then an algebraic category. 
 
 Another example of algebraic category is that of the category of C*-algebras.

 Generally, a category $\mathcal{A}_C$ is called \emph{algebraic} if it is monadic over the category of sets and set-theoretical mappings, $Set$; thus, a functor $G: \mathcal{D} \to \mathcal{C}$ is called \emph{monadic} if it has a left adjoint 
$F: \mathcal{C}\to \mathcal{D}$ forming a {\em monadic adjunction} $(F,G,\eta,\epsilon)$ with $G$ and $\eta, \epsilon$
being, respectively, the unit and counit; such a {\em monadic adjunction} between categories 
$\mathcal{C}$ and $\mathcal{D}$ is defined by the condition that category $\mathcal{D}$ is equivalent to the to the Eilenberg-Moore category $\mathcal{C} ^T$ for the monad 
$$T = GF.$$

\end{remark}



\item b. Equivalence classes
\begin{remark}
 Although all classes can be regarded as equivalence, weak equivalence, etc., classes of 
algebras (either specific or general ones), do not define identical, or even isomorphic structures, as the notion of `equivalence' can have more than one meaning even in the algebraic case.

\end{remark} 
\end{itemize}

\subsection{Note:}
See also the entry about abstract and concrete algebras in Expositions.
%%%%%
%%%%%
\end{document}
