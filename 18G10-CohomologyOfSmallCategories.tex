\documentclass[12pt]{article}
\usepackage{pmmeta}
\pmcanonicalname{CohomologyOfSmallCategories}
\pmcreated{2013-03-22 17:55:21}
\pmmodified{2013-03-22 17:55:21}
\pmowner{whm22}{2009}
\pmmodifier{whm22}{2009}
\pmtitle{cohomology of small categories}
\pmrecord{7}{40415}
\pmprivacy{1}
\pmauthor{whm22}{2009}
\pmtype{Definition}
\pmcomment{trigger rebuild}
\pmclassification{msc}{18G10}
\pmrelated{inverselimit}
\pmrelated{IndexOfCategories}
\pmdefines{derived functors of inverse limit}

% this is the default PlanetMath preamble.  as your knowledge
% of TeX increases, you will probably want to edit this, but
% it should be fine as is for beginners.

% almost certainly you want these
\usepackage{amssymb}
\usepackage{amsmath}
\usepackage{amsfonts}

% used for TeXing text within eps files
%\usepackage{psfrag}
% need this for including graphics (\includegraphics)
%\usepackage{graphicx}
% for neatly defining theorems and propositions
%\usepackage{amsthm}
% making logically defined graphics
%%%\usepackage{xypic}

% there are many more packages, add them here as you need them

% define commands here

\begin{document}
Let $\mathcal{C}$ be a small category.  For $n \geq 0$ we have functors $\Delta_n: \mathcal{C}\to{\rm Ab}$ which send an object $X \in \mathcal{C}$ to the free abelian group generated by $n+1$-tuples of morphisms to $X$.  The action of $\Delta_n$ on a morphism $f:X \to Y$ is defined by: $$\Delta_n(f): (g_0,g_1,\cdots, g_n) \mapsto (fg_0,fg_1,\cdots,fg_n)$$ for any morphisms $g_0,g_1, \cdots,g_n \in \mathcal{C}$ with codomain $X$.

For $n>0$ the natural transformation $\partial_n:\Delta_n \to \Delta_{n-1}$ is defined by letting the homomorphism $[\partial_n]_X:\Delta_n(X) \to \Delta_{n-1}(X)$ be given by:  $$[\partial_n]_X (f_0,f_1,\cdots,f_n)$$$$=\,(f_1,\cdots f_n)\,-\,(f_0,f_2,\cdots,f_n)\,+\,\cdots\,+\,[-1^n](f_0,f_1,\cdots,f_{n-1})$$

Hence we have a \PMlinkescapetext{chain} of natural transformations:$$
\cdots \stackrel{\partial_{n+1}}\to \Delta_n \stackrel{\partial_n\,}\to \Delta_{n-1} \stackrel{\partial_{n-1}}\to \cdots \stackrel{\partial_{2}\,}\to \Delta_1 \stackrel{\partial_1\,}\to \Delta_{0}$$

For any functor $F:\mathcal{C} \to {\rm Ab}$, let $[\Delta_n,F]$ denote the abelian group of natural transformations $\Delta_n \to F$.  Also let $\partial^n:[\Delta_{n-1},F] \to [\Delta_n,F]$ denote the abelian group homomorphism sending $\eta \to \eta \partial_n$.

We have a chain complex:$$
\cdots \stackrel{\partial^{n+1}}\leftarrow [\Delta_{n},F] \stackrel{\,\partial^n}\leftarrow [\Delta_{n-1},F] \stackrel{\partial^{n-1}}\leftarrow \cdots \stackrel{\,\partial^{2}}\leftarrow [\Delta_{1},F] \stackrel{\,\partial^1}\leftarrow [\Delta_{0},F]$$

It is easily verified that $H_0([\Delta_{*},F],\partial^*)$ is just $ {\rm lim}_{\leftarrow}(F)$, the inverse limit of $F$.  This motivates the definition:$$ {\rm lim}^n_{\leftarrow}(F)=H_n([\Delta_{*},F],\partial^*)$$

Note that if $\mathcal{C}$ is a group $G$ (that is $\mathcal{C}$ has one object and all its morphisms are invertible) then $F$ may be regarded as a module $M$, over $G$.  In this case ${\rm lim}^n_{\leftarrow}(F)$ coincides with group cohomology:  $ {\rm lim}^n_{\leftarrow}(F)=H^n(G;M) $.
%%%%%
%%%%%
\end{document}
