\documentclass[12pt]{article}
\usepackage{pmmeta}
\pmcanonicalname{LocallyFiniteCategory}
\pmcreated{2013-03-22 16:33:59}
\pmmodified{2013-03-22 16:33:59}
\pmowner{mps}{409}
\pmmodifier{mps}{409}
\pmtitle{locally finite category}
\pmrecord{4}{38756}
\pmprivacy{1}
\pmauthor{mps}{409}
\pmtype{Definition}
\pmcomment{trigger rebuild}
\pmclassification{msc}{18A05}
\pmclassification{msc}{06A99}

% this is the default PlanetMath preamble.  as your knowledge
% of TeX increases, you will probably want to edit this, but
% it should be fine as is for beginners.

% almost certainly you want these
\usepackage{amssymb}
\usepackage{amsmath}
\usepackage{amsfonts}

% used for TeXing text within eps files
%\usepackage{psfrag}
% need this for including graphics (\includegraphics)
%\usepackage{graphicx}
% for neatly defining theorems and propositions
%\usepackage{amsthm}
% making logically defined graphics
%%%\usepackage{xypic}

% there are many more packages, add them here as you need them

% define commands here

\begin{document}
\PMlinkescapeword{implies}
\PMlinkescapeword{arrow}
\PMlinkescapeword{arrows}
\PMlinkescapeword{loops}
\PMlinkescapeword{decomposition}

A \emph{locally finite category} is a category $\mathcal{C}$ such that each \PMlinkname{arrow}{Category} can be written as a composition of non-identity arrows in only finitely many ways.  To get a picture of what this means, view $\mathcal{C}$ as a graph, where the nodes are objects and the directed edges are all the arrows besides the identity arrows.  Note that loops besides identity arrows are permitted.  Then a decomposition of an arrow $f\colon A\to B$ is a directed path from $A$ to $B$ in the graph.  So the statement that each arrow can be decomposed in only finitely many ways essentially means that there are only finitely many directed paths from $A$ to $B$ in the graph.  This analogy only breaks down in the case where $A$ and $B$ are the same object.

The notion of a locally finite category generalizes the notion of a locally finite poset.  The condition that each interval $[x,z]$ in a poset $P$ contain only finitely many elements $y$ of $P$ implies that a locally finite poset is locally finite when viewed as a category, since between two objects of $P$ there is at most one arrow.

If $\mathcal{C}$ is a locally finite category, then an \PMlinkname{algebra over the category}{AlgebraFormedFromACategory} arises as the dual of a coalgebra over the category.

\begin{thebibliography}{99}
\bibitem{JoRo}
S.~A.~Joni and G.-C.~Rota, {\it Coalgebras and bialgebras in combinatorics}, Stud.~Appl.~Math., 61 (1979), pp. 93--139.
\end{thebibliography}
%%%%%
%%%%%
\end{document}
