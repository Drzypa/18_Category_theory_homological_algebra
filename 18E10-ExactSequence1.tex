\documentclass[12pt]{article}
\usepackage{pmmeta}
\pmcanonicalname{ExactSequence1}
\pmcreated{2013-03-22 12:36:50}
\pmmodified{2013-03-22 12:36:50}
\pmowner{djao}{24}
\pmmodifier{djao}{24}
\pmtitle{exact sequence}
\pmrecord{10}{32872}
\pmprivacy{1}
\pmauthor{djao}{24}
\pmtype{Definition}
\pmcomment{trigger rebuild}
\pmclassification{msc}{18E10}
\pmrelated{ExactSequence}
\pmdefines{image morphism}

% this is the default PlanetMath preamble.  as your knowledge
% of TeX increases, you will probably want to edit this, but
% it should be fine as is for beginners.

% almost certainly you want these
\usepackage{amssymb,amsthm}
\usepackage{amsmath}
\usepackage{amsfonts}

% used for TeXing text within eps files
%\usepackage{psfrag}
% need this for including graphics (\includegraphics)
%\usepackage{graphicx}
% for neatly defining theorems and propositions
%\usepackage{amsthm}
% making logically defined graphics
\usepackage[all]{xypic} 

% there are many more packages, add them here as you need them

% define commands here
\newcommand{\A}{\mathcal{A}}
\renewcommand{\Im}{\operatorname{Im}}
\newcommand{\im}{\operatorname{im}}
\newcommand{\cok}{\operatorname{cok}}

\newtheorem{theorem}{Theorem}
\newtheorem{proposition}[theorem]{Proposition}
\newtheorem{lemma}[theorem]{Lemma}
\newtheorem{corollary}[theorem]{Corollary}

\theoremstyle{definition}
\newtheorem{definition}[theorem]{Definition}
\begin{document}
Let $\A$ be an abelian category. We begin with a preliminary definition.

\begin{definition}
For any morphism $f: A \longrightarrow B$ in $\A$, let $m: X \longrightarrow B$ be the morphism equal to $\ker(\cok(f))$. Then the object $X$ is called the {\em image} of $f$, and denoted $\Im(f)$. The morphism $m$ is called the {\em image morphism} of $f$, and denoted $\im(f)$.
\end{definition}

Note that $\Im(f)$ is not the same as $\im(f)$: the former is an object of $\A$, while the latter is a morphism of $\A$. We note that $f$ factors through $\im(f)$:
$$
\xymatrix{
A \ar[r]^-{e} \ar@/_1pc/[rr]_{f} & \Im(f) \ar[r]^-{m} & B
}
$$
The proof is as follows: by definition of cokernel, $\cok(f) f = 0$; therefore by definition of kernel, the morphism $f$ factors through $\ker(\cok(f)) = \im(f) = m$, and this factor is the morphism $e$ above. Furthermore $m$ is a monomorphism and $e$ is an epimorphism, although we do not prove these facts.

\begin{definition}
A sequence
$$
\xymatrix{
\cdots \ar[r] & A \ar[r]^-f & B \ar[r]^-g & C \ar[r] & \cdots
}
$$
of morphisms in $\A$ is {\em exact} at $B$ if $\ker(g) = \im(f)$.
\end{definition}
%%%%%
%%%%%
\end{document}
