\documentclass[12pt]{article}
\usepackage{pmmeta}
\pmcanonicalname{CategoricalDirectProduct}
\pmcreated{2013-03-22 12:36:08}
\pmmodified{2013-03-22 12:36:08}
\pmowner{djao}{24}
\pmmodifier{djao}{24}
\pmtitle{categorical direct product}
\pmrecord{10}{32855}
\pmprivacy{1}
\pmauthor{djao}{24}
\pmtype{Definition}
\pmcomment{trigger rebuild}
\pmclassification{msc}{18A30}
\pmsynonym{direct product}{CategoricalDirectProduct}
\pmrelated{CategoricalDirectSum}
\pmrelated{DirectProduct}
\pmrelated{DirectProduct2}
\pmdefines{product}
\pmdefines{categorical product}
\pmdefines{categorical direct product}
\pmdefines{projection morphism}

\endmetadata

% this is the default PlanetMath preamble.  as your knowledge
% of TeX increases, you will probably want to edit this, but
% it should be fine as is for beginners.

% almost certainly you want these
\usepackage{amssymb}
\usepackage{amsmath}
\usepackage{amsfonts}

% used for TeXing text within eps files
%\usepackage{psfrag}
% need this for including graphics (\includegraphics)
%\usepackage{graphicx}
% for neatly defining theorems and propositions
%\usepackage{amsthm}
% making logically defined graphics
%%\usepackage{xypic}

% there are many more packages, add them here as you need them

% define commands here
\begin{document}
Let $\{C_i\}_{i \in I}$ be a set of objects in a category $\mathcal{C}$. A \emph{direct product} of the collection $\{C_i\}_{i \in I}$ is an object $\prod_{i \in I} C_i$ of $\mathcal{C}$, with morphisms $\pi_i\colon \prod_{j \in I} C_j \to C_i$ for each $i \in I$, such that:

For every object $A$ in $\mathcal{C}$, and any collection of morphisms $f_i\colon A \to C_i$ for every $i \in I$, there exists a unique morphism $f\colon A \to \prod_{i \in I} C_i$ making the following diagram commute for all $i \in I$.
\[
\xymatrix{
A \ar@{-->}[dr]_{f} \ar[rr]^{f_i} & & C_i  \\
& \prod_{j \in I} C_j \ar[ur]_{\pi_i}
}
\]
The morphisms $\pi_i\colon \prod_{j \in I} C_j \to C_i$ are called \emph{projection morphisms}.

The direct product of a finite collection of sets $C_1, C_2, \ldots, C_n$ is often denoted $C_1 \times C_2 \times \cdots \times C_n$, in analogy with the Cartesian product.
%%%%%
%%%%%
\end{document}
