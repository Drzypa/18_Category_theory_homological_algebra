\documentclass[12pt]{article}
\usepackage{pmmeta}
\pmcanonicalname{YonedaLemma}
\pmcreated{2013-03-22 12:15:19}
\pmmodified{2013-03-22 12:15:19}
\pmowner{rspuzio}{6075}
\pmmodifier{rspuzio}{6075}
\pmtitle{Yoneda lemma}
\pmrecord{13}{31638}
\pmprivacy{1}
\pmauthor{rspuzio}{6075}
\pmtype{Theorem}
\pmcomment{trigger rebuild}
\pmclassification{msc}{18A25}
\pmsynonym{Yoneda's Lemma}{YonedaLemma}
\pmrelated{ConcreteCategory}
\pmdefines{Yoneda embedding}

\endmetadata

\usepackage{amssymb}
\usepackage{amsmath}
\usepackage{amsfonts}
\begin{document}
If $\mathcal{C}$ is a category, write $\hat{\mathcal{C}}$ for the category of contravariant functors from $\mathcal{C}$ to ${\bf Sets}$, the category of sets.  The morphisms in $\hat{\mathcal{C}}$ are natural transformations of functors.

(To avoid set theoretical concerns, one can take a universe $\mathcal{U}$ and take all categories to be $\mathcal{U}$-small.)

For any object $X$ of $\mathcal{C}$, $h_X = {\rm Hom}(-,X)$ is a contravariant functor from $\mathcal{C}$ to ${\bf Sets}$, and therefore is an object of $\hat{\mathcal{C}}$.  

\emph{Yoneda Lemma} says that $X\mapsto h_X$ is a covariant functor $\mathcal{C}\to\hat{\mathcal{C}}$, which embeds $\mathcal{C}$ faithfully as a full subcategory of $\hat{\mathcal{C}}$. This embedding is called the \emph{Yoneda embedding}.
%%%%%
%%%%%
%%%%%
\end{document}
