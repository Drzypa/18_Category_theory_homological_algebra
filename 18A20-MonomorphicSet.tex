\documentclass[12pt]{article}
\usepackage{pmmeta}
\pmcanonicalname{MonomorphicSet}
\pmcreated{2013-03-22 18:20:50}
\pmmodified{2013-03-22 18:20:50}
\pmowner{CWoo}{3771}
\pmmodifier{CWoo}{3771}
\pmtitle{monomorphic set}
\pmrecord{9}{40981}
\pmprivacy{1}
\pmauthor{CWoo}{3771}
\pmtype{Definition}
\pmcomment{trigger rebuild}
\pmclassification{msc}{18A20}
\pmdefines{monomorphic pair}
\pmdefines{epimorphic set}
\pmdefines{epimorphic pair}

\usepackage{amssymb,amscd}
\usepackage{amsmath}
\usepackage{amsfonts}
\usepackage{mathrsfs}

% used for TeXing text within eps files
%\usepackage{psfrag}
% need this for including graphics (\includegraphics)
%\usepackage{graphicx}
% for neatly defining theorems and propositions
\usepackage{amsthm}
% making logically defined graphics
%%\usepackage{xypic}
\usepackage{pst-plot}

% define commands here
\newcommand*{\abs}[1]{\left\lvert #1\right\rvert}
\newtheorem{prop}{Proposition}
\newtheorem{thm}{Theorem}
\newtheorem{ex}{Example}
\newcommand{\real}{\mathbb{R}}
\newcommand{\pdiff}[2]{\frac{\partial #1}{\partial #2}}
\newcommand{\mpdiff}[3]{\frac{\partial^#1 #2}{\partial #3^#1}}
\begin{document}
Let $\mathcal{C}$ be a category and $M: = \lbrace f_i: A \to B_i \mid i\in I\rbrace$ a set (indexed by a set $I$) of morphisms with common domain $A$ in $\mathcal{C}$.  Then $M$ is said to be a \emph{monomorphic set} if for any pair of morphisms $g,h: C\to A$, $f_i\circ g = f_i\circ h$ for all $i\in I$ imply that $g=h$.  A \emph{monomorphic pair} is a monomorphic set $M$ such that the cardinality of $M$ is 2.

Monomorphic sets are generalizations of monomorphisms.  Indeed, for if $\lbrace f:A\to B\rbrace$ is a monomorphic set, then $f$ is a monomorphic.

For example, in \textbf{Set}, the category of sets, let $R$ be an $n$-ary relation on a set $A$.  For each $i=1,\ldots, n$, let $p_i$ be the projection of the $i$-th coordinate of $R$ into $A$.  Then $$\lbrace p_i\mid i=1,\ldots, n\rbrace$$ is a monomorphic set in \textbf{Set}.  To see this, observe first that, since $R$ is a subset of $A^n$, any function $f:B\to R$ has $n$ components, $f_i: B\to A$, given by $f_i = p_i\circ f$.  Now, suppose $g,h:B\to R$ are functions, such that $p_i\circ g=p_i\circ h$.  Then $g_i=h_i$ for all $i$.  In other words, all components of $g$ and $h$ match.  Therefore $g=h$.

More generally, a relation $R$ between sets $A_1,\ldots, A_n$ is a subset of the cartesian product $A_1\times \cdots \times A_n$.  The set of projections $\lbrace p_i: R\to A_i\mid i=1,\ldots,n\rbrace$ is also a monomorphic set in $\textbf{Set}$.  Using this concept, one may generalize the notion of a relation on sets to a relation on objects in a category.

\textbf{Remark}.  One can dually define \emph{epimorphic sets} and \emph{epimorphic pairs}.

\begin{thebibliography}{9}
\bibitem{fb} F. Borceux \emph{Basic Category Theory, Handbook of Categorical Algebra I}, Cambridge University Press, Cambridge (1994)
\end{thebibliography}
%%%%%
%%%%%
\end{document}
