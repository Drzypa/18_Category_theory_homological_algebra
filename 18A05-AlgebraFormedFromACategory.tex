\documentclass[12pt]{article}
\usepackage{pmmeta}
\pmcanonicalname{AlgebraFormedFromACategory}
\pmcreated{2013-03-22 16:30:34}
\pmmodified{2013-03-22 16:30:34}
\pmowner{rspuzio}{6075}
\pmmodifier{rspuzio}{6075}
\pmtitle{algebra formed from a category}
\pmrecord{6}{38686}
\pmprivacy{1}
\pmauthor{rspuzio}{6075}
\pmtype{Definition}
\pmcomment{trigger rebuild}
\pmclassification{msc}{18A05}

\endmetadata

% this is the default PlanetMath preamble.  as your knowledge
% of TeX increases, you will probably want to edit this, but
% it should be fine as is for beginners.

% almost certainly you want these
\usepackage{amssymb}
\usepackage{amsmath}
\usepackage{amsfonts}

% used for TeXing text within eps files
%\usepackage{psfrag}
% need this for including graphics (\includegraphics)
%\usepackage{graphicx}
% for neatly defining theorems and propositions
%\usepackage{amsthm}
% making logically defined graphics
%%%\usepackage{xypic}

% there are many more packages, add them here as you need them

% define commands here

\begin{document}
Given a category $\mathcal{C}$ and a ring $R$, one can construct an
algebra $\mathcal{A}$ as follows. Let $\mathcal{A}$ be the set of
all formal finite linear combinations of the form
\[\sum_i c_i e_{a_i, b_i, \mu_i},\]
where the coefficients $c_i$ lie in $R$ and, to every pair of objects
$a$ and $b$ of $\mathcal{C}$ and every morphism $\mu$ from $a$ to $b$,
there corresponds a basis element $e_{a,b,\mu}$. Addition and scalar
multiplication are defined in the usual way. Multiplication of
elements of $\mathcal{A}$ may be defined by specifying how to multiply
basis elements. If $b \not= c$, then set $e_{a, b, \phi} \cdot
e_{c, d, \psi} = 0$; otherwise set $e_{a, b, \phi} \cdot e_{b, c, \psi}
= e_{a, c, \psi \circ \phi}$. Because of the associativity of
composition of morphisms, $\mathcal{A}$ will be an associative algebra
over $R$.

Two instances of this construction are worth noting.  If $G$ is a group,
we may regard $G$ as a category with one object.  Then this construction 
gives us the group algebra of $G$.  If $P$ is a partially ordered set,
we may view $P$ as a category with at most one morphism between any
two objects.  Then this construction provides us with the incidence 
algebra of $P$.
%%%%%
%%%%%
\end{document}
