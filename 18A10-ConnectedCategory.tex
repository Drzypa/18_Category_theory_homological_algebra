\documentclass[12pt]{article}
\usepackage{pmmeta}
\pmcanonicalname{ConnectedCategory}
\pmcreated{2013-03-22 18:38:29}
\pmmodified{2013-03-22 18:38:29}
\pmowner{CWoo}{3771}
\pmmodifier{CWoo}{3771}
\pmtitle{connected category}
\pmrecord{8}{41382}
\pmprivacy{1}
\pmauthor{CWoo}{3771}
\pmtype{Definition}
\pmcomment{trigger rebuild}
\pmclassification{msc}{18A10}
\pmrelated{BrandtGroupoid}
\pmdefines{strongly connected}

\usepackage{amssymb,amscd}
\usepackage{amsmath}
\usepackage{amsfonts}
\usepackage{mathrsfs}

% used for TeXing text within eps files
%\usepackage{psfrag}
% need this for including graphics (\includegraphics)
%\usepackage{graphicx}
% for neatly defining theorems and propositions
\usepackage{amsthm}
% making logically defined graphics
%%\usepackage{xypic}
\usepackage{pst-plot}

% define commands here
\newcommand*{\abs}[1]{\left\lvert #1\right\rvert}
\newtheorem{prop}{Proposition}
\newtheorem{thm}{Theorem}
\newtheorem{ex}{Example}
\newcommand{\real}{\mathbb{R}}
\newcommand{\pdiff}[2]{\frac{\partial #1}{\partial #2}}
\newcommand{\mpdiff}[3]{\frac{\partial^#1 #2}{\partial #3^#1}}
\begin{document}
Let $\mathcal{C}$ be a category.  Two objects $A,B$ in $\mathcal{C}$ are said to be \emph{joined} if there is a morphism with domain one object and codomain the other.  In other words, $\hom(A,B)\cup \hom(B,A)\ne \varnothing$.  Two objects $A,B$ are said to be \emph{connected} if there is a finite sequence of objects in $\mathcal{C}$ $$A=C_1, C_2, \ldots, C_n = B$$ such that $C_i,C_{i+1}$ are joined for $i=1,\ldots,n-1$.

A category is said to be \emph{connected} if every pair of objects are connected, and \emph{strongly connected} if every pair of objects are joined.

For example, every category with either an initial object or a terminal object is connected.  If a category has a zero object, it is strongly connected.

A small category may be viewed as a graph or a digraph.  Then the underlying graph of a small connected category is connected, and the underlying digraph of a small strongly connected category is strongly connected.  Conversely, the free category freely generated a connected graph is connected, and the free category freely generated by a strongly connected digraph is strongly connected.

The relation (on objects of $\mathcal{C}$) of being joined is in general not an equivalence relation (it is reflexive and symmetric, but not transitive).  Let us call this relation $R$.  The relation of being connected, on the other hand, is an equivalence relation, and is the transitive closure $R^*$ of $R$.  Therefore, we may partition the class of objects in $\mathcal{C}$ by $R^*$.  Furthermore, $R^*$ induces an equivalence relation $R'$ on the class of all morphisms in $\mathcal{C}$: for morphisms $f,g$, set $$f R' g\qquad\mbox{iff}\qquad \operatorname{dom}(f) R^* \operatorname{dom}(g).$$  If $A$ is an object of $\mathcal{C}$, denote $[A]$ the equivalence class containing $A$ under $R^*$, together with the equivalence class containing $1_A$ under $R'$.  Then $[A]$ is a connected full subcategory of $\mathcal{C}$.  $[A]$ is called a connected component of $\mathcal{C}$.  Every small category can be expressed as the disjoint union of its connected components.

\begin{thebibliography}{9}
\bibitem{sm} S. Mac Lane, {\em Categories for the Working Mathematician}, Springer, New York (1971).
\end{thebibliography}
%%%%%
%%%%%
\end{document}
