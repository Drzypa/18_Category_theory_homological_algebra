\documentclass[12pt]{article}
\usepackage{pmmeta}
\pmcanonicalname{QuantumTopos}
\pmcreated{2013-03-22 18:17:48}
\pmmodified{2013-03-22 18:17:48}
\pmowner{bci1}{20947}
\pmmodifier{bci1}{20947}
\pmtitle{quantum topos}
\pmrecord{25}{40913}
\pmprivacy{1}
\pmauthor{bci1}{20947}
\pmtype{Definition}
\pmcomment{trigger rebuild}
\pmclassification{msc}{18D25}
\pmclassification{msc}{18-00}
\pmclassification{msc}{55U99}
\pmclassification{msc}{81-00}
\pmclassification{msc}{81P05}
\pmclassification{msc}{81Q05}
\pmsynonym{quantum category}{QuantumTopos}
%\pmkeywords{quantum state spaces}
%\pmkeywords{commutative lattice}
%\pmkeywords{subobject classifier}
%\pmkeywords{pc-lattice}
%\pmkeywords{category of pc-lattices}
%\pmkeywords{quantum topoi}
%\pmkeywords{quantum logics}
%\pmkeywords{Heyting logic algebra}
%\pmkeywords{Heyting lattice}
%\pmkeywords{quantum categories}
%\pmkeywords{quantum spaces}
%\pmkeywords{quantum system}
%\pmkeywords{the fundamental concept of qu}
\pmrelated{NonCommutativeStructureAndOperation}
\pmrelated{Commutative}
\pmrelated{QuantumCategory}
\pmrelated{HeytingAlgebra}
\pmrelated{AxiomsOfTopoi}
\pmrelated{QuantumLogic}
\pmrelated{CategoricalDynamics}
\pmrelated{Lattice}
\pmrelated{IntuitionisticLogic}
\pmrelated{ConsistencyOfClassicNumberTheory}
\pmdefines{quantum state space}

% this is the default PlanetMath preamble.  as your knowledge
% of TeX increases, you will probably want to edit this, but
% it should be fine as is for beginners.

% almost certainly you want these
\usepackage{amssymb}
\usepackage{amsmath}
\usepackage{amsfonts}


% used for TeXing text within eps files
%\usepackage{psfrag}
% need this for including graphics (\includegraphics)
%\usepackage{graphicx}
% for neatly defining theorems and propositions
%\usepackage{amsthm}
% making logically defined graphics
%%%\usepackage{xypic}

% there are many more packages, add them here as you need them

% define commands here
\usepackage{amsmath, amssymb, amsfonts, amsthm, amscd, latexsym}
%%\usepackage{xypic}
\usepackage[mathscr]{eucal}
\usepackage{amssymb,amscd}
\usepackage{amsmath}
\usepackage{amsfonts}

% used for TeXing text within eps files
%\usepackage{psfrag}
% need this for including graphics (\includegraphics)
%\usepackage{graphicx}
% for neatly defining theorems and propositions
\usepackage{amsthm}
% making logically defined graphics


\setlength{\textwidth}{6.5in}
%\setlength{\textwidth}{16cm}
\setlength{\textheight}{9.0in}
%\setlength{\textheight}{24cm}

\hoffset=-.75in     %%ps format
%\hoffset=-1.0in     %%hp format
\voffset=-.4in

\theoremstyle{plain}
\newtheorem{lemma}{Lemma}[section]
\newtheorem{proposition}{Proposition}[section]
\newtheorem{theorem}{Theorem}[section]
\newtheorem{corollary}{Corollary}[section]

\theoremstyle{definition}
\newtheorem{definition}{Definition}[section]
\newtheorem{example}{Example}[section]
%\theoremstyle{remark}
\newtheorem{remark}{Remark}[section]
\newtheorem*{notation}{Notation}
\newtheorem*{claim}{Claim}

\renewcommand{\thefootnote}{\ensuremath{\fnsymbol{footnote
}}}
\numberwithin{equation}{section}

\newcommand{\Ad}{{\rm Ad}}
\newcommand{\Aut}{{\rm Aut}}
\newcommand{\Cl}{{\rm Cl}}
\newcommand{\Co}{{\rm Co}}
\newcommand{\DES}{{\rm DES}}
\newcommand{\Diff}{{\rm Diff}}
\newcommand{\Dom}{{\rm Dom}}
\newcommand{\Hol}{{\rm Hol}}
\newcommand{\Mon}{{\rm Mon}}
\newcommand{\Hom}{{\rm Hom}}
\newcommand{\Ker}{{\rm Ker}}
\newcommand{\Ind}{{\rm Ind}}
\newcommand{\IM}{{\rm Im}}
\newcommand{\Is}{{\rm Is}}
\newcommand{\ID}{{\rm id}}
\newcommand{\GL}{{\rm GL}}
\newcommand{\Iso}{{\rm Iso}}
\newcommand{\Sem}{{\rm Sem}}
\newcommand{\St}{{\rm St}}
\newcommand{\Sym}{{\rm Sym}}
\newcommand{\SU}{{\rm SU}}
\newcommand{\Tor}{{\rm Tor}}
\newcommand{\U}{{\rm U}}

\newcommand{\A}{\mathcal A}
\newcommand{\Ce}{\mathcal C}
\newcommand{\D}{\mathcal D}
\newcommand{\E}{\mathcal E}
\newcommand{\F}{\mathcal F}
\newcommand{\G}{\mathcal G}
\newcommand{\Q}{\mathcal Q}
\newcommand{\R}{\mathcal R}
\newcommand{\cS}{\mathcal S}
\newcommand{\cU}{\mathcal U}
\newcommand{\W}{\mathcal W}

\newcommand{\bA}{\mathbb{A}}
\newcommand{\bB}{\mathbb{B}}
\newcommand{\bC}{\mathbb{C}}
\newcommand{\bD}{\mathbb{D}}
\newcommand{\bE}{\mathbb{E}}
\newcommand{\bF}{\mathbb{F}}
\newcommand{\bG}{\mathbb{G}}
\newcommand{\bK}{\mathbb{K}}
\newcommand{\bM}{\mathbb{M}}
\newcommand{\bN}{\mathbb{N}}
\newcommand{\bO}{\mathbb{O}}
\newcommand{\bP}{\mathbb{P}}
\newcommand{\bR}{\mathbb{R}}
\newcommand{\bV}{\mathbb{V}}
\newcommand{\bZ}{\mathbb{Z}}

\newcommand{\bfE}{\mathbf{E}}
\newcommand{\bfX}{\mathbf{X}}
\newcommand{\bfY}{\mathbf{Y}}
\newcommand{\bfZ}{\mathbf{Z}}

\renewcommand{\O}{\Omega}
\renewcommand{\o}{\omega}
\newcommand{\vp}{\varphi}
\newcommand{\vep}{\varepsilon}

\newcommand{\diag}{{\rm diag}}
\newcommand{\grp}{{\mathbb G}}
\newcommand{\dgrp}{{\mathbb D}}
\newcommand{\desp}{{\mathbb D^{\rm{es}}}}
\newcommand{\Geod}{{\rm Geod}}
\newcommand{\geod}{{\rm geod}}
\newcommand{\hgr}{{\mathbb H}}
\newcommand{\mgr}{{\mathbb M}}
\newcommand{\ob}{{\rm Ob}}
\newcommand{\obg}{{\rm Ob(\mathbb G)}}
\newcommand{\obgp}{{\rm Ob(\mathbb G')}}
\newcommand{\obh}{{\rm Ob(\mathbb H)}}
\newcommand{\Osmooth}{{\Omega^{\infty}(X,*)}}
\newcommand{\ghomotop}{{\rho_2^{\square}}}
\newcommand{\gcalp}{{\mathbb G(\mathcal P)}}

\newcommand{\rf}{{R_{\mathcal F}}}
\newcommand{\glob}{{\rm glob}}
\newcommand{\loc}{{\rm loc}}
\newcommand{\TOP}{{\rm TOP}}

\newcommand{\wti}{\widetilde}
\newcommand{\what}{\widehat}

\renewcommand{\a}{\alpha}
\newcommand{\be}{\beta}
\newcommand{\ga}{\gamma}
\newcommand{\Ga}{\Gamma}
\newcommand{\de}{\delta}
\newcommand{\del}{\partial}
\newcommand{\ka}{\kappa}
\newcommand{\si}{\sigma}
\newcommand{\ta}{\tau}
\newcommand{\med}{\medbreak}
\newcommand{\medn}{\medbreak \noindent}
\newcommand{\bign}{\bigbreak \noindent}
\newcommand{\lra}{{\longrightarrow}}
\newcommand{\ra}{{\rightarrow}}
\newcommand{\rat}{{\rightarrowtail}}
\newcommand{\oset}[1]{\overset {#1}{\ra}}
\newcommand{\osetl}[1]{\overset {#1}{\lra}}
\newcommand{\hr}{{\hookrightarrow}}
\begin{document}
\textbf{Preliminary Data}. \\
There are several distinct definitions of \emph{quantum topos} in the
Mathematical Physics literature attempting to redefine the quantum logic
that was first introduced by von Neumann and Birkhoff for the foundation
of Quantum Mechanics. The definitions of quantum topoi published so far
are not, however, those of `quantum' categories (previously introduced as rigid monoidal categories) 
- with finite limits and power objects. \\

\begin{definition}
A \emph{quantum topos} was defined as a \emph{general model, or representation of quantum state spaces (QST) 
in a topos} with a \PMlinkname{(commutative)}{Commutative} Heyting logic algebra as a subobject \emph{(quantum logic)} classifier.The differences between the several published definitions of a quantum topos differ in the categorical
representation in QST' s, and in the choice of category, but not in the choice of quantum logic algebra
that was selected as a standard, Heyting logic algebra (or \emph{Heyting algebra}) which has a commutative 
\emph{Heyting lattice} structure; this choice is at variance with the original quantum logic introduced by von Neumann and Birkhoff. Thus instead of the orthomodular lattice of Birkhoff and von Neumann, the recent definitions of quantum topoi postulate an intuitionistic- Brouwer logic corresponding to a pseudocomplemented and rel. pseudocomplemented
lattice structure, as further explained in the next section.
\end{definition}


\subsection{Heyting Logic Concept and Algebraic Structure}

\begin{definition}
A \emph{Heyting lattice} $L$ is a Brouwer-intuitionistic logic lattice with a bottom, or lowest element $0$. 
In the more technical classification it is a \PMlinkname{commutative}{Commutative} lattice which is both `pseudocomplemented and also relatively pseudocomplemented'. The concept of \PMlinkname{relative pseudocomplementation}{RelativelyPseudocomplemented} coincides with the material implication operator, $\Rightarrow$, in symbolic  
propositional logic based on chryssippian or Boolean logic.

\end{definition}

\begin{definition}
A \emph{Heyting algebra} is a \emph{p-algebra (as defined next in \textbf{Definition 1.3} ) 
with the relative pseudocomplentation operation} $\to$ (which replaces the propositional implication $\Rightarrow$).\\
Given an element $a$ in a bounded lattice $L$, a \emph{complement} of $a$ is defined to be an element $b\in L$, if such an element exists, such that $$a\wedge b=0,\qquad{ and }\qquad a\vee b =1.$$ \\
To surmount the non-uniqueness of the complement, an alternative to the latter was defined-- 
the \emph{pseudocomplement} of an element. 
\begin{quote}
An element $b$ in a lattice $L$ with $0$ is a \emph{pseudocomplement} of $a\in L$ if 
\begin{enumerate}
\item $b\wedge a=0$
\item for any $c$ such that $c\wedge a=0$ then $c\le b$.
\end{enumerate}
In other words, $b$ is the maximal element in the set $\lbrace c\in L\mid c\wedge a=0\rbrace$.
\end{quote}
\end{definition}

\begin{definition}
A convenient modification of the pseudocomplemented (pc) lattice concept is  a \emph{p-algebra} 
(or pseudocomplemented algebra) which is a pc-lattice where $^*$ is regarded as an algebraic operator. 
Thus, a morphism  of pc--lattices is a proper lattice homomorphism, whereas a morphism between two p-algebras is 
a lattice homomorphism $f$ that also preserves the pc-algebraic operation $^*$, i.e.,  $f(a^*)= f(a)^*$. 
One can therefore define a \emph{category of p-algebras} by specifying the morphism between any pair of p-algebras 
(considered as objects of this algebraic logic category) as the $\lbrace 0,1\rbrace$-lattice homomorphism, with the following condition $f(1)=f(0^*)=f(0)^*=0^*=1$ being also satisfied. 
\end{definition}

\textbf{Remark}
Unlike the Heyting lattice, an $LM_n$-logic algebra has a \emph{non-commutative} lattice structure
and is therefore considered as a stronger candidate for quantum logics, including those based
on the orthomodular lattices of the original quantum logic of Birkhoff and von Neumann. Thus,
a generalized topos defined with a subobject classifier based on $LM_n$-logic algebra may
provide suitable representations of arbitrary quantum state spaces.  

\begin{thebibliography}{9}

%%bibitem{BIsham1}
Butterfield, J. and C. J. Isham: 2001, Space-time and the
Philosophical Challenges of Quantum Gravity., in C. Callender and
N. Hugget (eds. ) \emph{Physics Meets Philosophy at the Planck
scale.}, Cambridge University Press,pp.33--89.

%%bibitem{BIsham2}
Butterfield, J. and C. J. Isham: 1998, 1999, 2000--2002, A topos
perspective on the Kochen--Specker theorem I - IV, \emph{Int. J.
Theor. Phys}, \textbf{37}  No 11., 2669--2733 \textbf{38} No 3.,
827--859, \textbf{39} No 6., 1413--1436, \textbf{41} No 4.,
613--639.

\end{thebibliography}
%%%%%
%%%%%
\end{document}
