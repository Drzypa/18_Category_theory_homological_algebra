\documentclass[12pt]{article}
\usepackage{pmmeta}
\pmcanonicalname{AttachingPullbackSquares}
\pmcreated{2013-03-22 17:09:07}
\pmmodified{2013-03-22 17:09:07}
\pmowner{mps}{409}
\pmmodifier{mps}{409}
\pmtitle{attaching pullback squares}
\pmrecord{7}{39461}
\pmprivacy{1}
\pmauthor{mps}{409}
\pmtype{Result}
\pmcomment{trigger rebuild}
\pmclassification{msc}{18A30}
\pmrelated{CWComplexDefinitionRelatedToSpinNetworksAndSpinFoams}

% this is the default PlanetMath preamble.  as your knowledge
% of TeX increases, you will probably want to edit this, but
% it should be fine as is for beginners.

% almost certainly you want these
\usepackage{amssymb}
\usepackage{amsmath}
\usepackage{amsfonts}

% used for TeXing text within eps files
%\usepackage{psfrag}
% need this for including graphics (\includegraphics)
%\usepackage{graphicx}
% for neatly defining theorems and propositions
\usepackage{amsthm}
% making logically defined graphics
%%\usepackage{xypic}

% there are many more packages, add them here as you need them

% define commands here
\newtheorem*{proposition*}{Proposition}
\newcommand{\from}{\leftarrow}
\begin{document}
Pullback squares can be attached together to form larger pullback
squares.  The following proposition describes the main step involved
in this attaching procedure.

\begin{proposition*}
Let a commutative diagram
\[\xymatrix{
A\ar[r]\ar[d] & B\ar[r]\ar[d] & C\ar[d] \\
D\ar[r]       & E\ar[r]       & F
}\]
be given.

Assume that $B$ is a pullback for $E\to F\from C$.  Then the following
are equivalent:

\begin{itemize}
\item
$A$ is a pullback for $D\to E\from B$.

\item
$A$ is a pullback for $D\to F\from C$, where $D\to F$ is the composition $D\to E\to F$.
\end{itemize}
\end{proposition*}

\begin{proof}

First assume that $A$ is a pullback for $D\to E\from B$.  Let $Z$ be
the vertex of a limiting cone $D\from Z\to C$ over $D\to F\from C$.
We must find a unique filler for the diagram
\[\xymatrix{
Z\ar[rdd]\ar[rrrd]\ar@{.>}[rd] &               &               &         \\
                               & A\ar[r]\ar[d] & B\ar[r]\ar[d] & C\ar[d] \\
                               & D\ar[r]       & E\ar[r]       & F
}\]
If we let $Z\to E$ denote the composition $Z\to D\to E$, then $Z$ is
the vertex of a limiting cone $E\from Z\to C$ over $E\to F\from C$.
Since $B$ is a pullback for $E\to F\from C$, it follows that $Z$ must
factor uniquely through $B$, that is, there is a unique filler $Z\to
B$ for the diagram
\[\xymatrix{
Z\ar[rdd]\ar[rrrd]\ar@{.>}[rrd] &               &               &         \\
                                &               & B\ar[r]\ar[d] & C\ar[d] \\
                                & D\ar[r]       & E\ar[r]       & F
}\]
But now observe that $D\from Z\to B$ is a limiting cone over $D\to
E\from B$.  Since we assumed that $A$ is a pullback for $D\to E\from
B$, we find that $Z$ must factor through $A$.  The commutativity of
the intermediate diagrams implies that this filler $Z\to A$ completes
\[\xymatrix{
Z\ar[rdd]\ar[rrrd]\ar[rrd]\ar@{.>}[rd] &               &               &         \\
                                       & A\ar[r]\ar[d] & B\ar[r]\ar[d] & C\ar[d] \\
                                       & D\ar[r]       & E\ar[r]       & F
}\]
to a commutative diagram.  Hence $A$ is a pullback for $D\to F\from C$.

Now we must prove the converse.  The reasoning is similar.  Assume
that $A$ is a pullback for $D\to F\from C$.  Suppose $Z$ is the vertex
of a limiting cone $D\from Z\to B$ over $D\to E\from B$.  By composing
$Z\to B$ with $B\to C$, we get that $D\from Z\to C$ is a limiting cone
over $D\to F\from C$.  Applying the assumption that $A$ is a pullback
from $D\to F\from C$, it follows that there is a unique filler $Z\to
A$ for the diagram
\[\xymatrix{
Z\ar[rdd]\ar[rrrd]\ar@{.>}[rd] &               &               &         \\
                               & A\ar[r]\ar[d] & B\ar[r]\ar[d] & C\ar[d] \\
                               & D\ar[r]       & E\ar[r]       & F
}\]
Since $Z\to C$ is just the composition $Z\to B\to C$, we conclude that in fact $Z\to A$ is the unique filler
for the diagram
\[\xymatrix{
Z\ar[rdd]\ar[rrd]\ar@{.>}[rd] &               &         \\
                              & A\ar[r]\ar[d] & B\ar[d] \\
                              & D\ar[r]       & E       
}\]
Hence $A$ is a pullback for $D\to E\from B$.
\end{proof}

The left-to-right half of this proposition states that if two
pullback squares are attached along a common edge, the outer rectangle
of the resulting commutative diagram is also a pullback square.  The
right-to-left half says that if two commutative squares are attached
so that the outer rectangle and the right square are both pullback
squares, then the left square is a pullback square too.  It is
important to note what this does \emph{not} say.  One might ask
whether we could assume instead that the outer rectangle and the
\emph{left} square are pullback squares and obtain that the right
square is a pullback square.  In this case we would need to find a
filler for the diagram
\[\xymatrix{
Z\ar[rdd]\ar[rrd]\ar@{.>}[rd] &               &         \\
                              & B\ar[r]\ar[d] & C\ar[d] \\
                              & E\ar[r]       & F
}\]

In order to apply the assumption that the outer rectangle is a
pullback square, we would need to find a morphism $Z\to D$, and to be
able to exploit the commutativity of the diagram, this morphism must
be compatible with $Z\to E$ above.  These requirements together amount
to demanding that there is a lifting 
\[\xymatrix{
                    & D\ar[d] \\
Z\ar[r]\ar@{.>}[ur] & E
}\]
of $Z\to E$ to a morphism $Z\to D$.  But not every morphism can be
lifted, so this task is impossible in general.

\PMlinkescapeword{commutative}
\PMlinkescapeword{completes}
\PMlinkescapeword{edge}
\PMlinkescapeword{factor}
\PMlinkescapeword{moment}
\PMlinkescapeword{order}
\PMlinkescapeword{outer}
\PMlinkescapeword{rectangle}
\PMlinkescapeword{right}
\PMlinkescapeword{similar}
\PMlinkescapeword{square}
\PMlinkescapeword{squares}

%%%%%
%%%%%
\end{document}
