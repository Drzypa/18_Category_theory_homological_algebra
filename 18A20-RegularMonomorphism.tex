\documentclass[12pt]{article}
\usepackage{pmmeta}
\pmcanonicalname{RegularMonomorphism}
\pmcreated{2013-03-22 18:23:21}
\pmmodified{2013-03-22 18:23:21}
\pmowner{CWoo}{3771}
\pmmodifier{CWoo}{3771}
\pmtitle{regular monomorphism}
\pmrecord{12}{41034}
\pmprivacy{1}
\pmauthor{CWoo}{3771}
\pmtype{Definition}
\pmcomment{trigger rebuild}
\pmclassification{msc}{18A20}
\pmclassification{msc}{18-00}
\pmsynonym{regular monic}{RegularMonomorphism}
\pmsynonym{regular epi}{RegularMonomorphism}
\pmsynonym{regular epic}{RegularMonomorphism}
\pmrelated{PropertiesOfRegularAndExtremalMonomorphisms}
\pmdefines{regular epimorphism}

\usepackage{amssymb,amscd}
\usepackage{amsmath}
\usepackage{amsfonts}
\usepackage{mathrsfs}

% used for TeXing text within eps files
%\usepackage{psfrag}
% need this for including graphics (\includegraphics)
%\usepackage{graphicx}
% for neatly defining theorems and propositions
\usepackage{amsthm}
% making logically defined graphics
%%\usepackage{xypic}
\usepackage{pst-plot}

% define commands here
\newcommand*{\abs}[1]{\left\lvert #1\right\rvert}
\newtheorem{prop}{Proposition}
\newtheorem{thm}{Theorem}
\newtheorem{ex}{Example}
\newcommand{\real}{\mathbb{R}}
\newcommand{\pdiff}[2]{\frac{\partial #1}{\partial #2}}
\newcommand{\mpdiff}[3]{\frac{\partial^#1 #2}{\partial #3^#1}}
\begin{document}
Let $\mathcal{C}$ be a category.  Recall that the equalizer of a pair of morphisms is monomorphic.  Call a monomorphism $f:A\to B$ a \emph{regular monomorphism} if it is the equalizer of a pair of morphisms.

The dual notion of this is that of a \emph{regular epimorphism}: a morphism that is the coequalizer of a pair of morphisms.  As above, a regular epimorphism is an epimorphism.

For example, in \textbf{Set}, the category of sets, every monomorphism (epimorphism) is regular.

\begin{prop} Every split monomorphism is regular. \end{prop}
\begin{proof}  If the monomorphism $f:A\to B$ is split, then there is a morphism $g:B\to A$ such that
$$\xymatrix@1{A\ar[r]^f & B\ar[r]^g & A}=\xymatrix@1{A\ar[r]^{1_A} & A}.$$
Then $f$ is the equalizer of $f\circ g, 1_B:B\to B$.  First, $f$ equalizes $f\circ g$ and $1_B$:
$$\xymatrix@1{A\ar[r]^f & B\ar[r]^g & A\ar[r]^f & B}=\xymatrix@1{A\ar[r]^{1_A} & A \ar[r]^f & B}=\xymatrix@1{A \ar[r]^f & B}=\xymatrix@1{A\ar[r]^f & B \ar[r]^{1_B} & B}.$$
Furthermore, if $h:C\to B$ also equalizes $f\circ g$ and $1_B$:
$$\xymatrix@1{C\ar[r]^h & B\ar[r]^g & A\ar[r]^f & B}=\xymatrix@1{C\ar[r]^h & B \ar[r]^{1_B} & B},$$
then by defining $x: C\to A$ by $x:=g\circ h$, we see that $h=(f\circ g)\circ h = f\circ x$, or $h$ factors through $f$.  Furthermore, $x$ is uniquely determined by $g$ and $h$, showing that $f$ is indeed the equalizer of $f\circ g$ and $1_B$.
\end{proof}

\begin{prop} Every regular monomorphism is strong. \end{prop}
\begin{proof}  Suppose $f:A\to B$ is the equalizer of $s,t:B\to E$, and we have the following commutative diagram with $g$ epimorphic:
$$\xymatrix@+=3pc{
{C}\ar[r]^{g}\ar[d]_{x}&{D}\ar[d]^{y}\\
{A}\ar[r]_{f}&{B} \ar@<0.5ex>[r]^s \ar@<-0.5ex>[r]_t & E
}
$$
Now we do some diagram chasing.  Since $s\circ f = t\circ f$, we have $s\circ f\circ x=t\circ f\circ x$.  But $f\circ x = y\circ g$, we get $s\circ y\circ g = t\circ y\circ g$.  Since $g$ is epimorphic, $s\circ y=t\circ y$.  Since $f$ is the equalizer of $s$ and $t$, there is a unique morphism $u:D\to A$ such that the following triangle is commutative:
$$\xymatrix@+=3pc{& D \ar[d]^y \ar@{.>}[dl]_u \\ A \ar[r]_f & B}$$
As a result, $f\circ x = y\circ g= f\circ u \circ g$.  Since $f$ is monomorphic, $x=u\circ g$, yielding the following commutative diagram:
$$\xymatrix@+=3pc{
{C}\ar[r]^{g}\ar[d]_{x}&{D}\ar[d]^{y} \ar[dl]_u \\
{A}\ar[r]_{f}&{B} 
}
$$
which is the precise statement that $f$ is strong.
\end{proof}


\begin{thebibliography}{9}
\bibitem{fb} F. Borceux \emph{Basic Category Theory, Handbook of Categorical Algebra I}, Cambridge University Press, Cambridge (1994)
\end{thebibliography}
%%%%%
%%%%%
\end{document}
