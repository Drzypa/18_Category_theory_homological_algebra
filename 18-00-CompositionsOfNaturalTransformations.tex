\documentclass[12pt]{article}
\usepackage{pmmeta}
\pmcanonicalname{CompositionsOfNaturalTransformations}
\pmcreated{2013-03-22 18:27:02}
\pmmodified{2013-03-22 18:27:02}
\pmowner{CWoo}{3771}
\pmmodifier{CWoo}{3771}
\pmtitle{compositions of natural transformations}
\pmrecord{6}{41112}
\pmprivacy{1}
\pmauthor{CWoo}{3771}
\pmtype{Definition}
\pmcomment{trigger rebuild}
\pmclassification{msc}{18-00}
\pmclassification{msc}{18A25}
\pmclassification{msc}{18A05}
\pmsynonym{Godemont product}{CompositionsOfNaturalTransformations}
\pmdefines{vertical composition}
\pmdefines{horizontal composition}

\usepackage{amssymb,amscd}
\usepackage{amsmath}
\usepackage{amsfonts}
\usepackage{mathrsfs}

% used for TeXing text within eps files
%\usepackage{psfrag}
% need this for including graphics (\includegraphics)
%\usepackage{graphicx}
% for neatly defining theorems and propositions
\usepackage{amsthm}
% making logically defined graphics
%%\usepackage{xypic}
\usepackage{pst-plot}
\usepackage [2cell]{xy}

% define commands here
\newcommand*{\abs}[1]{\left\lvert #1\right\rvert}
\newtheorem{prop}{Proposition}
\newtheorem{thm}{Theorem}
\newtheorem{ex}{Example}
\newcommand{\real}{\mathbb{R}}
\newcommand{\pdiff}[2]{\frac{\partial #1}{\partial #2}}
\newcommand{\mpdiff}[3]{\frac{\partial^#1 #2}{\partial #3^#1}}
\begin{document}
The usual way to compose two natural transformation is what is known as the \emph{vertical composition}.  Given categories $\mathcal{C},\mathcal{D}$, functors $R,S,T$ from $\mathcal{C}$ to $\mathcal{D}$, and natural transformations $\tau:R\Rightarrow S$ and $\eta:S\Rightarrow T$, we have a natural transformation $\eta\bullet \tau:R\Rightarrow T$ given by $$(\eta \bullet \tau)_A:=\eta_A\circ \tau_A$$
The reason for calling $\bullet$ the ``vertical'' composition is illustrated by the diagram below:

\begin{Large}
$$
\UseAllTwocells
\xymatrix @+=3cm{\mathcal{C} \ruppertwocell<4.5>^{\stackrel{R}{}}{<0>_{\quad \eta\bullet\tau}} \rlowertwocell<-4.5>_{\stackrel{}{T}}{\omit} & \mathcal{D}} \quad{:=}\quad
\xymatrix @+=3cm{\mathcal{C} \ruppertwocell<9>^{\stackrel{R}{}}{<-2.5>_{\mbox{   } \tau}} \ar[r]|S \rlowertwocell<-9>_{\stackrel{}{T}}{<2.5>^{\mbox{  }\eta}} & \mathcal{D}}
$$
\end{Large}

However, there is another way to compose natural transformations: the so-called \emph{horizontal composition}.  Given categories, $\mathcal{B},\mathcal{C},\mathcal{D}$, functors $S_1,T_1: \mathcal{B}\to \mathcal{C}$, $S_2,T_2:\mathcal{C}\to \mathcal{D}$, and natural transformations, $\tau: S_1\Rightarrow T_1$ and $\eta: S_2 \Rightarrow T_2$ as in the following diagram

\begin{Large}
$$
\UseAllTwocells
\xymatrix @+=3cm{\mathcal{B} \ruppertwocell<5>^{\stackrel{S_2 S_1}{}}{<0>_{\quad \eta\circ \tau}} \rlowertwocell<-5>_{\stackrel{}{T_2 T_1}}{\omit} & \mathcal{D}} \quad{:=}\quad
\xymatrix @+=3cm{\mathcal{B} \ruppertwocell<4.5>^{\stackrel{S_1}{}}{<0>_{\quad \tau}} \rlowertwocell<-4.5>_{\stackrel{}{T_1}}{\omit} & \mathcal{C} \ruppertwocell<4.5>^{\stackrel{S_2}{}}{<0>_{\quad \eta}} \rlowertwocell<-4.5>_{\stackrel{}{T_2}}{\omit} & \mathcal{D}}
$$
\end{Large}

we define the \emph{horizontal composition}, or \emph{Godemont product}, of $\eta$ and $\tau$, written $\eta\circ \tau: S_2S_1\to T_2T_1$, as follows: first, pick any object $A$ in $\mathcal{B}$.  Because $\eta$ is a natural transformation, we have a commutative diagram (solid arrows) below
$$
\xymatrix @+=2cm{S_2S_1(A) \ar[r]^{\eta_{S_1(A)}} \ar[d]_{S_2(\tau_A)} \ar@{-->}[dr]^{(\eta\circ\tau)_A} & T_2S_1(A) \ar[d]^{T_2(\tau_A)} \\ 
S_2T_1(A) \ar[r]_{\eta_{T_1(A)}} & T_2T_1(A)}
$$
From this, we set $(\eta\circ \tau)_A$ to be the ``diagonal'' morphism (dotted arrow) from $S_2S_1(A)$ to $T_2T_1(A)$ in the diagram above:
$$(\eta\circ \tau)_A:= T_2(\tau_A)\circ \eta_{S_1(A)} = \eta_{T_1(A)}\circ S_2(\tau_A).$$

Below are some properties of $\circ$:
\begin{enumerate}
\item $\eta\circ \tau$ is a natural transformation.
\item $\circ$ is associative.
\item $\eta\circ 1_S=\eta$, and $1_S\circ \tau =\tau$, where $\eta$ and $\tau$ are described in the diagram above, and $1_S$ is the identity transformation on the functor $S:\mathcal{B}\to \mathcal{C}$, and $1_T$ is the identity transformation on the functor $T:\mathcal{C}\to \mathcal{D}$.
\item $\circ$ and $\bullet$ satisfy the interchange law.
\end{enumerate}

In fact, the first three properties above turn $\textbf{Cat}$, the category of small categories, into a category where the objects are small categories, morphisms are natural transformations, and the composition of morphisms is the horizontal composition.
%%%%%
%%%%%
\end{document}
