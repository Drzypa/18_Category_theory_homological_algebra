\documentclass[12pt]{article}
\usepackage{pmmeta}
\pmcanonicalname{FisomorphismsInCategories}
\pmcreated{2013-03-22 18:28:36}
\pmmodified{2013-03-22 18:28:36}
\pmowner{joking}{16130}
\pmmodifier{joking}{16130}
\pmtitle{F-isomorphisms in categories}
\pmrecord{7}{41149}
\pmprivacy{1}
\pmauthor{joking}{16130}
\pmtype{Definition}
\pmcomment{trigger rebuild}
\pmclassification{msc}{18A05}
\pmrelated{WeakHomotopyDoubleGroupoid}
\pmrelated{WeakHomotopyAdditionLemma}
\pmrelated{ThinEquivalenceRelation}

\endmetadata

% this is the default PlanetMath preamble.  as your knowledge
% of TeX increases, you will probably want to edit this, but
% it should be fine as is for beginners.

% almost certainly you want these
\usepackage{amssymb}
\usepackage{amsmath}
\usepackage{amsfonts}

% used for TeXing text within eps files
%\usepackage{psfrag}
% need this for including graphics (\includegraphics)
%\usepackage{graphicx}
% for neatly defining theorems and propositions
%\usepackage{amsthm}
% making logically defined graphics
%%%\usepackage{xypic}

% there are many more packages, add them here as you need them

% define commands here

\begin{document}
Let $\mathcal{C}$ and $\mathcal{D}$ be categories, $F:\mathcal{C}\to\mathcal{D}$ be a (covariant or contravariant) functor and let $\alpha\in\mathrm{Hom}(A,B)$ be a morphism, where $A,B\in\mathrm{Ob}(\mathcal{C})$.\\

\textbf{Definition}. A morphism $\alpha:A\to B$ is an $F$\textit{-isomorphism} if $F(\alpha)$ is an isomorphism in $\mathcal{D}$.\\

Note that each isomorphism in $\mathcal{C}$ is an $F$-isomorphism for each functor $F$. The converse is true in the following sense: if $\alpha$ is an $F$-isomorphism for each functor $F$ then $\alpha$ is an isomorphism. On the other hand there are $F$-isomorphisms which are not isomorphisms.

Also note, that if $\alpha:X\to Y$ is an $F$-isomorphism, then there does not have to exist morphism $\beta:Y\to X$ which is ,,$F$-inverse'' to $\alpha$. Indeed, there may be no $F$-isomorphism from $Y$ to $X$, see examples:\\

\textbf{Example}. $1)$ Let $X$ be an object in $\mathcal{D}$ and define $F_{X}:\mathcal{C}\to\mathcal{D}$ as follows: for $A\in\mathrm{Ob}(\mathcal{C})$ put $F_{X}(A)=X$ and for $\alpha\in\mathrm{Hom}(A,B)$ put $F_{X}(\alpha)=\mathrm{id}_{X}$. This is the constant functor and every morphism in $\mathcal{C}$ is an $F_{X}$-isomorphism (although it does not have to be an isomorphism).
\indent In particular, it may happen that for some objects $X$ and $Y$ in $\mathcal{C}$ there is a morphism from $X$ to $Y$ but no morphism from $Y$ to $X$. In this case there is an $F_X$-isomorphism from $X$ to $Y$ but not vice versa.\\ \\
$2)$ Let $\mathcal{T}\mathrm{op}^{*}$ be the category of pointed topological spaces and continous maps preserving based point, $\mathcal{S}\mathrm{et}$ be the category of sets and functions, $\mathcal{G}\mathrm{r}$ be the category of groups and homomorphisms. Consider the functor $\pi:\mathcal{T}\mathrm{op}^{*}\to\mathcal{S}\mathrm{et}\times\mathcal{G}\mathrm{r}\times\mathcal{G}\mathrm{r}\times\cdots$ defined by:
$$\pi(X,x_{0})=(\pi_{0}(X,x_{0}),\pi_{1}(X,x_{0}),\pi_{2}(X,x_{0}),\pi_{3}(X,x_{0}),\ldots);$$
$$\pi(f)=(\pi_{0}(f),\pi_{1}(f),\pi_{2}(f),\pi_{3}(f),\ldots),$$
where $\pi_{n}$ is the $n$-th homotopy group functor. Then $\pi$-isomorphism is a weak homotopy equivalence and it is known (due to Whitehead) that each weak homotopy equivalence between pointed $\mathrm{CW}$-complexes is the homotopy equivalance. On the other hand there are weak homotopy equivalences which are not homotopy equivalences.

The concept of $F$-isomorphism is especially important in representation theory, where $F$ is the homology functor from category of complexes over an abelian category $\mathcal{C}$ to $\mathcal{C}$.
%%%%%
%%%%%
\end{document}
