\documentclass[12pt]{article}
\usepackage{pmmeta}
\pmcanonicalname{LeraySpectralSequence}
\pmcreated{2013-03-22 12:03:08}
\pmmodified{2013-03-22 12:03:08}
\pmowner{bwebste}{988}
\pmmodifier{bwebste}{988}
\pmtitle{Leray spectral sequence}
\pmrecord{7}{31099}
\pmprivacy{1}
\pmauthor{bwebste}{988}
\pmtype{Theorem}
\pmcomment{trigger rebuild}
\pmclassification{msc}{18G40}
%\pmkeywords{Grothendieck spectral sequence}
%\pmkeywords{spectral sequence}
\pmrelated{GrothendieckSpectralSequence}

\usepackage{amssymb}
\usepackage{amsmath}
\usepackage{amsfonts}
\usepackage{graphicx}
%%%\usepackage{xypic}
\begin{document}
The \emph{Leray spectral sequence} is a special case of the Grothendieck spectral sequence regarding composition of functors.

If $f : X\to Y$ is a continuous map of topological spaces, and if $\mathcal{F}$ is a sheaf of abelian groups on $X$, then there is a spectral sequence, called the Leray spectral sequence, given by

$E_2^{pq} = H^p(Y,{\rm R}^q f_* \mathcal{F})\implies H^{p+q}(X,\mathcal{F})$

where $f_*$ is the direct image functor.
%%%%%
%%%%%
%%%%%
\end{document}
