\documentclass[12pt]{article}
\usepackage{pmmeta}
\pmcanonicalname{SuperalgebroidsAndHigherDimensionalAlgebroids}
\pmcreated{2013-03-22 18:30:02}
\pmmodified{2013-03-22 18:30:02}
\pmowner{bci1}{20947}
\pmmodifier{bci1}{20947}
\pmtitle{superalgebroids and higher dimensional algebroids}
\pmrecord{10}{41185}
\pmprivacy{1}
\pmauthor{bci1}{20947}
\pmtype{Topic}
\pmcomment{trigger rebuild}
\pmclassification{msc}{18A25}
\pmclassification{msc}{18A20}
\pmclassification{msc}{18A15}
\pmclassification{msc}{55U40}
\pmclassification{msc}{55U35}
\pmclassification{msc}{18D05}
\pmsynonym{HDA superstructure}{SuperalgebroidsAndHigherDimensionalAlgebroids}
%\pmkeywords{higher dimensional algebroids}
%\pmkeywords{groupoids}
%\pmkeywords{superalgebroids or generalized superalgebra}
%\pmkeywords{graded Lie algebroids}
\pmrelated{RAlgebroid}
\pmrelated{HDA}
\pmrelated{HigherDimensionalAlgebraHDA}
\pmrelated{SuperfieldsSuperspace}
\pmdefines{generalized superalgebra}
\pmdefines{graded Lie algebroid}
\pmdefines{HDA structure}

\endmetadata

% this is the default PlanetMath preamble.  
\usepackage{amssymb}
\usepackage{amsmath}
\usepackage{amsfonts}

% define commands here
\usepackage{amsmath, amssymb, amsfonts, amsthm, amscd, latexsym, enumerate,color}
\usepackage{xypic, xspace}
\usepackage[mathscr]{eucal}
\usepackage[dvips]{graphicx}
\usepackage[curve]{xy}
\def\blue{\textcolor{blue}}

\theoremstyle{plain}
\newtheorem{lemma}{Lemma}[section]
\newtheorem{proposition}{Proposition}[section]
\newtheorem{theorem}{Theorem}[section]
\newtheorem{corollary}{Corollary}[section]

\theoremstyle{definition}
\newtheorem{definition}{Definition}[section]
\newtheorem{example}{Example}[section]
%\theoremstyle{remark}
\newtheorem{remark}{Remark}[section]
\newtheorem*{notation}{Notation}
\newtheorem*{claim}{Claim}

\renewcommand{\thefootnote}{\ensuremath{\fnsymbol{footnote}}}
\numberwithin{equation}{section}

\newcommand{\Ad}{{\rm Ad}}
\newcommand{\Aut}{{\rm Aut}}
\newcommand{\Cl}{{\rm Cl}}
\newcommand{\Co}{{\rm Co}}
\newcommand{\DES}{{\rm DES}}
\newcommand{\Diff}{{\rm Diff}}
\newcommand{\Dom}{{\rm Dom}}
\newcommand{\Hol}{{\rm Hol}}
\newcommand{\Mon}{{\rm Mon}}
\newcommand{\Hom}{{\rm Hom}}
\newcommand{\Ker}{{\rm Ker}}
\newcommand{\Ind}{{\rm Ind}}
\newcommand{\IM}{{\rm Im}}
\newcommand{\Is}{{\rm Is}}
\newcommand{\ID}{{\rm id}}
\newcommand{\grpL}{{\rm GL}}
\newcommand{\Iso}{{\rm Iso}}
\newcommand{\rO}{{\rm O}}
\newcommand{\Sem}{{\rm Sem}}
\newcommand{\SL}{{\rm Sl}}
\newcommand{\St}{{\rm St}}
\newcommand{\Sym}{{\rm Sym}}
\newcommand{\Symb}{{\rm Symb}}
\newcommand{\SU}{{\rm SU}}
\newcommand{\Tor}{{\rm Tor}}
\newcommand{\U}{{\rm U}}

\newcommand{\A}{\mathcal A}
\newcommand{\Ce}{\mathcal C}
\newcommand{\D}{\mathcal D}
\newcommand{\E}{\mathcal E}
\newcommand{\F}{\mathcal F}
%\newcommand{\grp}{\mathcal G}
\renewcommand{\H}{\mathcal H}
\renewcommand{\cL}{\mathcal L}
\newcommand{\Q}{\mathcal Q}
\newcommand{\R}{\mathcal R}
\newcommand{\cS}{\mathcal S}
\newcommand{\cU}{\mathcal U}
\newcommand{\W}{\mathcal W}

\newcommand{\bA}{\mathbb{A}}
\newcommand{\bB}{\mathbb{B}}
\newcommand{\bC}{\mathbb{C}}
\newcommand{\bD}{\mathbb{D}}
\newcommand{\bE}{\mathbb{E}}
\newcommand{\bF}{\mathbb{F}}
\newcommand{\bG}{\mathbb{G}}
\newcommand{\bK}{\mathbb{K}}
\newcommand{\bM}{\mathbb{M}}
\newcommand{\bN}{\mathbb{N}}
\newcommand{\bO}{\mathbb{O}}
\newcommand{\bP}{\mathbb{P}}
\newcommand{\bR}{\mathbb{R}}
\newcommand{\bV}{\mathbb{V}}
\newcommand{\bZ}{\mathbb{Z}}

\newcommand{\bfE}{\mathbf{E}}
\newcommand{\bfX}{\mathbf{X}}
\newcommand{\bfY}{\mathbf{Y}}
\newcommand{\bfZ}{\mathbf{Z}}

\renewcommand{\O}{\Omega}
\renewcommand{\o}{\omega}
\newcommand{\vp}{\varphi}
\newcommand{\vep}{\varepsilon}

\newcommand{\diag}{{\rm diag}}
\newcommand{\grp}{\mathcal G}
\newcommand{\dgrp}{{\mathsf{D}}}
\newcommand{\desp}{{\mathsf{D}^{\rm{es}}}}
\newcommand{\grpeod}{{\rm Geod}}
%\newcommand{\grpeod}{{\rm geod}}
\newcommand{\hgr}{{\mathsf{H}}}
\newcommand{\mgr}{{\mathsf{M}}}
\newcommand{\ob}{{\rm Ob}}
\newcommand{\obg}{{\rm Ob(\mathsf{G)}}}
\newcommand{\obgp}{{\rm Ob(\mathsf{G}')}}
\newcommand{\obh}{{\rm Ob(\mathsf{H})}}
\newcommand{\Osmooth}{{\Omega^{\infty}(X,*)}}
\newcommand{\grphomotop}{{\rho_2^{\square}}}
\newcommand{\grpcalp}{{\mathsf{G}(\mathcal P)}}

\newcommand{\rf}{{R_{\mathcal F}}}
\newcommand{\grplob}{{\rm glob}}
\newcommand{\loc}{{\rm loc}}
\newcommand{\TOP}{{\rm TOP}}

\newcommand{\wti}{\widetilde}
\newcommand{\what}{\widehat}

\renewcommand{\a}{\alpha}
\newcommand{\be}{\beta}
\newcommand{\grpa}{\grpamma}
%\newcommand{\grpa}{\grpamma}
\newcommand{\de}{\delta}
\newcommand{\del}{\partial}
\newcommand{\ka}{\kappa}
\newcommand{\si}{\sigma}
\newcommand{\ta}{\tau}

\newcommand{\med}{\medbreak}
\newcommand{\medn}{\medbreak \noindent}
\newcommand{\bign}{\bigbreak \noindent}

\newcommand{\lra}{{\longrightarrow}}
\newcommand{\ra}{{\rightarrow}}
\newcommand{\rat}{{\rightarrowtail}}
\newcommand{\ovset}[1]{\overset {#1}{\ra}}
\newcommand{\ovsetl}[1]{\overset {#1}{\lra}}
\newcommand{\hr}{{\hookrightarrow}}

\newcommand{\<}{{\langle}}

%\newcommand{\>}{{\rangle}}
%\usepackage{geometry, amsmath,amssymb,latexsym,enumerate}
%%%\usepackage{xypic}

\def\baselinestretch{1.1}


\hyphenation{prod-ucts}

%\grpeometry{textwidth= 16 cm, textheight=21 cm}

\newcommand{\sqdiagram}[9]{$$ \diagram  #1  \rto^{#2} \dto_{#4}&
#3  \dto^{#5} \\ #6    \rto_{#7}  &  #8   \enddiagram
\eqno{\mbox{#9}}$$ }

\def\C{C^{\ast}}

\newcommand{\labto}[1]{\stackrel{#1}{\longrightarrow}}

%\newenvironment{proof}{\noindent {\bf Proof} }{ \hfill $\Box$
%{\mbox{}}
\newcommand{\midsqn}[1]{\ar@{}[dr]|{#1}}
\newcommand{\quadr}[4]
{\begin{pmatrix} & #1& \\[-1.1ex] #2 & & #3\\[-1.1ex]& #4&
 \end{pmatrix}}
\def\D{\mathsf{D}}
\begin{document}
\subsubsection{Definitions of double, and higher dimensional, algebroids, superalgebroids and generalized superalgebras}

\emph{Double algebroids}


\begin{definition} A \emph{double $R$--algebroid} consists of a 
\PMlinkname{double category $D$}{HomotopyDoubleGroupoidOfAHausdorffSpace}, 
as detailed in ref.\cite{BS1}, such that each category structure has the additional structure of
an $R$--algebroid.  More precisely, a double \PMlinkname{$R$--algebroid}{RAlgebroid} $\D$ involves four related 
\PMlinkname{$R$--algebroids}{RAlgebroid}:

\begin{equation}
\begin{aligned} (D,D_1,\del^0_1 ,\del^1_1 , \vep_1 , +_1 , \circ _1 , ._1)
,\qquad  &(D,D_2,\del^0_2 , \del ^1_2 , \vep_2 , +_2 , \circ _2 ,
._2 )\\
 (D_1,D_0, \delta^0_1 ,\delta^1_1 , \vep , + , \circ  , .) ,\qquad
 &(D_2 , D_0 , \delta^0_2 , \delta^1_2 , \vep , + , \circ  , .)
\end{aligned}
\end{equation}
that satisfy the following rules:

\begin{itemize}
\item[i)]
$\delta^i_2 \del^j_2 = \delta ^j_1 \del ^i_1$  for $i,j \in
\{0,1\}$

\med
\item[ii)]
\begin{equation}
\begin{aligned} \del^i_2 ( \a +_1 \be) = \del^i_2\a
+\del^i_2\be , \qquad &\del^i_1 ( \a +_2 \be) = \del^i_1 \a
+\del^i_1\be \\ \del^i_2 ( \a \circ _1 \be) = \del^i_2 \a \circ
\del^i_2 \be ,\qquad  &  \del^i_1 (\a \circ _2 \be ) = \del^i_1\a
\circ \del^i_1\be
\end{aligned}
\end{equation}
for $i = 0,1 , \a,\be  \in D$ and both sides are defined.

\med
\item[iii)]
\begin{equation}
\begin{aligned}
r ._1 (\a+_2 \be) = (r ._1 \a) +_2 (r ._1\be ) ,\qquad  & r ._2
(\a  +_1 \be  ) = (r ._2 \a ) +_1 (r ._2\be  )\\ r ._1 (\a  \circ
_2 \be ) = (r ._1\a ) \circ _2  (r ._1 \be )  , \qquad & r ._2 (\a
\circ _1 \be  ) = (r ._2 \a ) \circ _1  (r ._2\be )\\
 r ._1 ( s ._2 \be  ) &= s ._2 ( r._1 \be  )
\end{aligned}
\end{equation}
for all $\a ,\be  \in  D, ~r,s \in R$ and both sides are defined.

\med
\item[iv)]
\begin{equation}
\begin{aligned}
(\a  +_1 \be  ) +_2  (\gamma  +_1 \lambda )& = (\a  +_2 \gamma )
+_1 (\be   +_2  \lambda ) ,\\
 (\a  \circ _1 \be ) \circ _2  (\gamma  \circ _1  \lambda )& =
(\a  \circ _2  \gamma ) \circ _1  (\be   \circ _2  \lambda
 )\\ (\a  +_i \be  ) \circ _j (\gamma  +_i \lambda ) &= (\a  \circ _j
\gamma ) +_i (\be   \circ _j \lambda )
\end{aligned}
\end{equation}
for  $i \neq j$, whenever both sides are defined.
\end{itemize}

\end{definition}

The definition of a double algebroid specified above was introduced by Brown and Mosa \cite{RBM1986}. 
Two functors can be then constructed, one from the category of double algebroids to the
category of crossed modules of algebroids, whereas the reverse functor is the unique adjoint (up to natural
equivalence). The construction of such functors requires the following definition.

\subsection{Category of Double Algebroids}

A \textit{morphism $f : \D \to \E$ of double algebroids} is then
defined as a morphism of truncated cubical sets which commutes
with all the algebroid structures. Thus, one can construct a
category $\mathbf{DA}$ of double algebroids and their morphisms.
The main construction in this subsection is that of two functors
$\eta,\eta'$ from this category $\mathbf{DA}$ to the category
$\mathbf{CM}$ of crossed modules of algebroids.

Let ${D}$ be a double algebroid. One can associate to ${D}$ a
crossed module $\mu : M  \lra {D}_1$. Here $M(x,y)$ will consist
of elements $m$ of ${D}$ with boundary of the form:
                                                                                                0                         1
\begin{equation}
\del m = \quadr{a}{1_y}{1_x}{ 0_{xy}}~,
\end{equation}
that is
$M(x,y) = \{ m \in D : \del^1_1 m = 0_{xy} , \del^0_2 m =
1_x,\del^1_2  m = 1_y \}$.

\subsection{Cubic and Higher dimensional algebroids}

One can extend the above notion of double algebroid to cubic and higher dimensional algebroids.
The concepts of 2-algebroid, 3-algebroid,..., $n$--algebroid and superalgebroid are however
quite distinct from those of double, cubic,..., n--tuple algebroid, and have technically less complicated
definitions. 


\begin{thebibliography}{9}

\bibitem{RBM1986}
R. Brown and G. H. Mosa: Double algebroids and crossed modules of algebroids, University of Wales--Bangor, Maths Preprint, 1986.


\bibitem{BS1}
R. Brown and C.B. Spencer: Double groupoids and crossed modules, \emph{Cahiers Top. G\'eom.Diff.} \textbf{17}: 343--362 (1976).

\end{thebibliography}
%%%%%
%%%%%
\end{document}
