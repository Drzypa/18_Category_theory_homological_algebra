\documentclass[12pt]{article}
\usepackage{pmmeta}
\pmcanonicalname{AdjointFunctor}
\pmcreated{2013-03-22 12:28:50}
\pmmodified{2013-03-22 12:28:50}
\pmowner{mps}{409}
\pmmodifier{mps}{409}
\pmtitle{adjoint functor}
\pmrecord{37}{32691}
\pmprivacy{1}
\pmauthor{mps}{409}
\pmtype{Definition}
\pmcomment{trigger rebuild}
\pmclassification{msc}{18A40}
\pmsynonym{left adjoint}{AdjointFunctor}
\pmsynonym{right adjoint}{AdjointFunctor}
%\pmkeywords{adjoint functor pairs}
%\pmkeywords{adjointness}
%\pmkeywords{adjunction}
%\pmkeywords{adjoint dynamical systems}
%\pmkeywords{natural equivalence}
%\pmkeywords{natural isomorphism}
%\pmkeywords{adjointness theorems}
%\pmkeywords{limit and colimit preserving functors}
\pmrelated{ForgetfulFunctor}
\pmrelated{UniversalProperty}
\pmrelated{SimilarityAndAnalogousSystemsDynamicAdjointness2}
\pmrelated{GaloisConnection}
\pmrelated{SectionFunctor}
\pmrelated{NaturalEquivalence}
\pmrelated{FOUNDATIONSOFMATHEMATICSOVERVIEW}
\pmdefines{adjoint}
\pmdefines{adjoint pair}
\pmdefines{adjunction}

\endmetadata

% this is the default PlanetMath preamble.  as your knowledge
% of TeX increases, you will probably want to edit this, but
% it should be fine as is for beginners.

% almost certainly you want these
\usepackage{amssymb}
\usepackage{amsmath}
\usepackage{amsfonts}

% used for TeXing text within eps files
%\usepackage{psfrag}
% need this for including graphics (\includegraphics)
%\usepackage{graphicx}
% for neatly defining theorems and propositions
%\usepackage{amsthm}
% making logically defined graphics
%%\usepackage{xypic} 

% there are many more packages, add them here as you need them

% define commands here
\DeclareMathOperator{\Hom}{Hom}
\begin{document}
\PMlinkescapeword{properties}

Let $\mathcal{C}$ and $\mathcal{D}$ be (small) categories, and let $T:\mathcal{C} \to \mathcal{D}$ and $S:\mathcal{D} \to \mathcal{C}$ be covariant functors. $T$ is said to be a \emph{left adjoint functor} to $S$ (equivalently, $S$ is a \emph{right adjoint functor} to $T$) if there is a natural equivalence
\[
\nu\colon \Hom_{\mathcal{D}}(T(-),-) \overset{\cdot}{\longrightarrow} \Hom_{\mathcal{C}}(-,S(-)).
\]
Here the functor $\Hom_{\mathcal{D}}(T(-),-)$ is a bifunctor $\mathcal{C}\times\mathcal{D}\to\mathbf{Set}$ which is contravariant in the first variable, is covariant in the second variable, and sends an object $(C,D)$ to $\Hom_{\mathcal{D}}(T(C),D)$.  The functor $\Hom_{\mathcal{C}}(-,S(-))$ is defined analogously.

This definition needs additional explanation.  Essentially, it says that for every object $C$ in $\cal{C}$ and every object $D$ in $\cal{D}$ there is a function 
\[
\nu_{C,D} \colon \Hom_{\mathcal{D}}(T(C),D) \overset{\sim}{\longrightarrow} \Hom_{\mathcal{C}}(C,S(D)) 
\]
which is a natural bijection of hom-sets.  Naturality means that if $f\colon C'\to C$ is a morphism in $\mathcal{C}$ and $g\colon D\to D'$ is a morphism in $\mathcal{D}$, then the diagram
\[\xymatrix{
\Hom_{\mathcal{D}}(T(C),D)\ar[dd]_{(Tf,g)}\ar[rr]^{\nu_{C,D}} &&
\Hom_{\mathcal{C}}(C,S(D))\ar[dd]^{(f,Sg)} \\ && \\
\Hom_{\mathcal{D}}(T(C'),D')\ar[rr]^{\nu_{C',D'}} &&
\Hom_{\mathcal{C}}(C',S(D')) \\
}\] 
is a commutative diagram.  If we pick any $h:T(C)\to D$, then we have the equation $$Sg\circ \nu_{C,D}(h)\circ f= \nu_{C',D'}(g\circ h\circ Tf).$$
%
%
% I'm commenting the following out because I moved the mention of the
% natural transformation before the mention of the naturality of the 
% bijection.  This allows me to give the functors F_1 and F_2 the more 
% natural names Hom_D(T(-),-) and Hom_C(-,S(-)).
%
%
%The word ``natural'' in this definition needs some explanation.  We can construct a functor 
%\begin{align*}
%\mathcal{F}_1\colon\mathcal{C}\times\mathcal{D}&\to\mathbf{Set} \\
% (C,D)&\mapsto \Hom_{\mathcal{D}}(T(C),D)
%\end{align*}
%and a second functor
%\begin{align*}
%\mathcal{F}_2\colon\mathcal{C}\times\mathcal{D}&\to\mathbf{Set} \\
% (C,D)&\mapsto \Hom_{\mathcal{C}}(C,S(D)).
%\end{align*}
%Then the family of bijections $\nu_{C,D}$ should form a natural transformation from $\mathcal{F}_1$ to $\mathcal{F}_2$.
%

If $T:\mathcal{C}\to\mathcal{D}$ is a left adjoint of $S:\mathcal{D}\to \mathcal{C}$, then we say that the ordered pair $(T,S)$ is an \emph{adjoint pair}, and the ordered triple $(T,S,\nu)$ an \emph{adjunction} from $\mathcal{C}$ to $\mathcal{D}$, written $$(T,S,\nu):\mathcal{C}\to \mathcal{D},$$ where $\nu$ is the natural equivalence defined above.  

An adjoint to a functor is in some ways like an inverse (as in the case of an adjoint matrix); often formal properties about a functor lead to formal properties of its adjoint (for example the right adjoint to a left-exact functor takes \PMlinkescapetext{injectives} to \PMlinkescapetext{injectives}).  An adjoint to any functor is unique up to natural isomorphism.

\textbf{Examples}:
\begin{enumerate}

\item Let $R$ be a commutative ring, and fix an $R$-module $N$.  Let 
\[
{-\otimes N}\colon {R\! -\!\mathbf{mod}}\to {R\! -\!\mathbf{mod}}
\] 
be the functor 
\[
M\mapsto N\otimes M,
\] 
and let
\[
{\Hom(N,-)}:{R\! -\!\mathbf{mod}}\to {R\! -\!\mathbf{mod}}
\] 
given by 
\[
L\mapsto\mathrm{Hom}_R(N,L).
\]   
Then one can show that ${-\otimes N}$ is the left adjoint to ${\Hom(N,-)}$.  This pair of adjoint functors is the most commonly used and studied, and astonishingly deep facts spring from this adjoint relationship. 

\item Let $U:\mathbf{Top}\to \mathbf{Set}$ be the forgetful functor (i.e. $U$ takes topological spaces to their underlying sets, and continuous maps to set functions). Then $U$ is right adjoint to the functor $F:\mathbf{Set} \to \mathbf{Top}$ which gives each set the discrete topology.

\item If $U:\mathbf{Grp} \to \mathbf{Set}$ is again the forgetful functor, this time on the category of groups, the functor $F: \mathbf{Set} \to \mathbf{Grp}$ which takes a set $A$ to the free group generated by $A$ is left adjoint to $U$.
\end{enumerate}

\textbf{Remarks on Adjointness:}
\begin{enumerate}
\item There are several theorems that link limit and colimit preserving properties
of functors to adjointness (e.g., ref. \cite{NP75}). Thus, a left adjoint functor preserves colimits or acts naturally 
on the colimit functor (if the latter exists); dually, a right adjoint preserves limits.  

\item According to William F. Lawvere, Adjointness is closely involved with the Foundation of Mathematics.

\item Adjoint functors define dynamic similarities between general systems in categorical dynamics.

\end{enumerate}


\begin{thebibliography}{9}
\bibitem{K}
Daniel~M.~Kan. Adjoint functors.  {\it Transactions of the American Mathematical Society}, Vol. 87, No. 2, (1958), 294--329.
\bibitem{Ma}
S. Mac Lane, \emph{Categories for the Working Mathematician} (2nd edition), Springer-Verlag, 1997.

\bibitem{NP75}
N. Popescu.1975., \emph{Abelian Categories with Applications to Rings and Modules.}
Academic Press: New York and London.

\end{thebibliography}
%%%%%
%%%%%
\end{document}
