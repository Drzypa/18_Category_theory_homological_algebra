\documentclass[12pt]{article}
\usepackage{pmmeta}
\pmcanonicalname{PropertiesOfOrthogonalityOnMorphisms}
\pmcreated{2013-03-22 18:29:42}
\pmmodified{2013-03-22 18:29:42}
\pmowner{CWoo}{3771}
\pmmodifier{CWoo}{3771}
\pmtitle{properties of orthogonality on morphisms}
\pmrecord{5}{41179}
\pmprivacy{1}
\pmauthor{CWoo}{3771}
\pmtype{Derivation}
\pmcomment{trigger rebuild}
\pmclassification{msc}{18A32}

\endmetadata

\usepackage{amssymb,amscd}
\usepackage{amsmath}
\usepackage{amsfonts}
\usepackage{mathrsfs}

% used for TeXing text within eps files
%\usepackage{psfrag}
% need this for including graphics (\includegraphics)
%\usepackage{graphicx}
% for neatly defining theorems and propositions
\usepackage{amsthm}
% making logically defined graphics
%%\usepackage{xypic}
\usepackage{pst-plot}

% define commands here
\newcommand*{\abs}[1]{\left\lvert #1\right\rvert}
\newtheorem{prop}{Proposition}
\newtheorem{thm}{Theorem}
\newtheorem{ex}{Example}
\newcommand{\real}{\mathbb{R}}
\newcommand{\pdiff}[2]{\frac{\partial #1}{\partial #2}}
\newcommand{\mpdiff}[3]{\frac{\partial^#1 #2}{\partial #3^#1}}
\begin{document}
This entry lists and proves some of the basic properties of the orthogonality relations on morphisms.

\begin{prop} For an arbitrary $f$, $f\perp g$ for any isomorphism $g$.  Dually, if $g$ is arbitrary, then $f\perp g$ for any isomorphism $f$.  \end{prop}

\begin{proof}
Suppose $x\circ f = g\circ y$, and $z\circ g=1_C$ and $g\circ z=1_D$.  Then by defining $h:=z \circ x$, we get $h\circ f= z\circ x \circ f= z \circ g\circ y = 1_C \circ y = y$, and $g\circ h= g\circ z\circ x=1_D \circ x = x$.  This shows the existence of $h$.

If $h':B\to C$ is another morphism such that $h'\circ f = y$ and $g\circ h'=x$.  Then $h = z\circ x = z \circ g\circ h' = 1_C \circ h' = h'$, showing that $h$ is unique.

The proof of the dual statement is similar.
\end{proof}

\begin{prop} If $f\perp f$, then $f$ is an isomorphism. \end{prop}
\begin{proof}
This follows from the commutative diagram
$$\xymatrix@+=1.5cm{A \ar[r]^f \ar[d]_{1_A} & B \ar[d]^{1_B} \ar@{.>}[dl]|g \\ A \ar[r]_f & B}$$
So we get a ``diagonal'' morphism $g:B\to A$ making the resulting diagram commutative again.  So, $f\circ g=1_B$ and $g\circ f=1_A$, or $f$ is an isomorphism.
\end{proof}

In addition, $\perp$ preserves morphism composition, in the following sense:

\begin{prop} Suppose $(g,h)$ is a composable pair of morphisms ($h\circ g$ exists).  If $f\perp g$ and $f\perp h$, then $f\perp (h\circ g)$.  Similarly, if $g\perp f$ and $h\perp f$, then $(h\circ g)\perp f$. \end{prop}
\begin{proof}
Suppose $f:A\to B$, $g:C\to D$ and $h:D\to E$ are morphisms as described above, and have a commutative diagram
$$\xymatrix@+=1.5cm{A \ar[r]^f \ar[d]_x & B \ar[d]^y="1" &
A \ar[rr]^f \ar[d]_x="2" \ar[dr]^{g\circ x} & & B \ar[d]^y 
\\ C \ar[r]_{h\circ g} & E & 
C \ar[r]_g & D \ar[r]_h & E
\ar@{}"1";"2"|{=}
}$$
Because $f\perp h$, we get a unique morphism $s:B\to D$ and a commutative diagram
$$\xymatrix@+=1.5cm{A \ar[r]^f \ar[d]_{g\circ x} & B \ar[d]^y="1" \ar@{.>}[dl]|s &
A \ar[r]^f \ar[d]_x="2" & B \ar@{.>}[d]^s="3" \ar[dr]^y & 
\\ D \ar[r]_h & E &
C \ar[r]_g & D \ar[r]_h & E
\ar@{}"1";"2"|{=}
\ar@{}"2";"3"|{\mbox{I}}
}$$
Because $f\perp g$, we have a unique morphism $t:B\to C$, so the commutative square I above becomes
$$\xymatrix@+=1.5cm{A \ar[r]^f \ar[d]_x & B \ar[d]^s \ar@{.>}[dl]|t \\ C \ar[r]_g & D }$$
Since $(h\circ g)\circ t = h\circ (g\circ t) = h\circ s = y$, we get a commutative diagram
$$\xymatrix@+=1.5cm{A \ar[r]^f \ar[d]_x & B \ar[d]^y \ar@{.>}[dl]|t \\ C \ar[r]_{h\circ g} & E }$$
showing that $f\perp (h\circ g)$.  The dual statement of this is proved similarly.
\end{proof}
%%%%%
%%%%%
\end{document}
