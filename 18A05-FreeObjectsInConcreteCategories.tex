\documentclass[12pt]{article}
\usepackage{pmmeta}
\pmcanonicalname{FreeObjectsInConcreteCategories}
\pmcreated{2013-03-22 18:47:29}
\pmmodified{2013-03-22 18:47:29}
\pmowner{joking}{16130}
\pmmodifier{joking}{16130}
\pmtitle{free objects in concrete categories}
\pmrecord{9}{41589}
\pmprivacy{1}
\pmauthor{joking}{16130}
\pmtype{Definition}
\pmcomment{trigger rebuild}
\pmclassification{msc}{18A05}

% this is the default PlanetMath preamble.  as your knowledge
% of TeX increases, you will probably want to edit this, but
% it should be fine as is for beginners.

% almost certainly you want these
\usepackage{amssymb}
\usepackage{amsmath}
\usepackage{amsfonts}

% used for TeXing text within eps files
%\usepackage{psfrag}
% need this for including graphics (\includegraphics)
%\usepackage{graphicx}
% for neatly defining theorems and propositions
%\usepackage{amsthm}
% making logically defined graphics
%%%\usepackage{xypic}

% there are many more packages, add them here as you need them

% define commands here

\begin{document}
By concrete category we will understand pair $(\mathcal{C},U)$, where $C$ is a category and $U:\mathcal{C}\to\mathcal{SET}$ is a faithful (covariant) functor. Assume that $(\mathcal{C},U)$ is a concrete category.

\textbf{Definition 1.} Let $X$ be an object in $\mathcal{C}$. Subset $B\subseteq U(X)$ (possibly empty) is called a \textit{basis of $X$} if for any object $Y$ in $\mathcal{C}$ and any function $g:B\to U(Y)$ there exists exactly one morphism $\alpha:X\to Y$ such that $U(\alpha)(x)=g(x)$ for any $x\in B$. In this case we will say that $g$ \textit{lifts} to $\alpha$.

\textbf{Definition 2.} Object $X$ will be called \textit{free} if there exists basis of $X$.

Free objects generalize the notion of free modules over a ring. Some of the properties of free modules can be easily generalized to free objects in arbitrary concrete category. For example:

\textbf{Proposition.} Let $X$ and $Y$ be free objects with bases $B$ and $B'$ respectively and let $f:B\to B'$ be a function. The following statements hold:\\
$\mathrm{i)}$ If $f$ is an injection, then there exists a section $\alpha:X\to Y$ in $\mathcal{C}$;\\
$\mathrm{ii)}$ If $f$ is a surjection, then there exists a retraction $\beta:X\to Y$ in $\mathcal{C}$;\\
$\mathrm{iii)}$ If $f$ is a bijection, then $X$ and $Y$ are isomorphic.\\

\textit{Proof.} $\mathrm{i)}$ Assume that $f:B\to B'$ is an injection. Let $f_1:B\to U(Y)$ be defined as $$f_1(x)=f(x)$$
for all $x\in B$. Now, since $f:B\to B'$ is an injection, then there exists a surjection $f':B'\to B$ such that $$f'(f(x))=x$$
for all $x\in B$. Let $f_2:B'\to U(X)$ be defined by
$$f_2(y)=f'(y)$$
for all $y\in B'$. Now both $X$ and $Y$ are free and thus there are morphism $\alpha: X\to Y$ and $\beta:Y\to X$ such that
$$U(\alpha)(x)=f_1(x)\mbox{ and }U(\beta)(y)=f_2(y)$$
for all $x\in B$ and $y\in B'$. It is easy to check, that this implies that
$$U(\beta\circ\alpha)(x)=x$$
for all $x\in B$. But $U(\mathrm{id}_{X})(x)=x$ for all $x\in B$ and thus canonical injection $i:B\to U(X)$ lifts to both $\beta\circ\alpha$ and $\mathrm{id}_{X}$. Since lift is unique, then $\beta\circ\alpha=\mathrm{id}_{X}$, so $\alpha$ is a section.

$\mathrm{ii)}$ Note that if $f:B\to B'$ is a surjection, then there exists an injection $g:B'\to B$ such that $f(g(y))=y$ for all $y\in B'$. Thus, from $\mathrm{i)}$ we obtain that $\beta\circ\alpha=\mathrm{id}_{Y}$ for $\alpha:Y\to X$ and $\beta:X\to Y$ constructed as in $\mathrm{i)}$. Therefore $\beta:X\to Y$ is a retraction.

$\mathrm{iii)}$ If $f$ is a bijection, then proof of $\mathrm{i)}$ and $\mathrm{ii)}$ shows that there are two morphisms $\alpha:X\to Y$ and $\beta:Y\to X$ such that $\beta\circ\alpha=\mathrm{id}_{X}$ and $\alpha\circ\beta=\mathrm{id}_{Y}$. Thus $X$ and $Y$ are isomorphic. $\square$


\textbf{Remark 1.} Free objects does not have to exist. For example, the category of finite groups (without the trivial group) and group homomorphisms (where $U$ is a forgetful functor) does not have free objects (this is because there are no nontrivial group homomorphisms between groups with relatively prime orders).

\textbf{Remark 2.} Note that, if there is a free object $X$ in a concrete category $(\mathcal{C},U)$ such that $\emptyset$ is a basis of $X$, then $X$ is an initial object. This follows directly from the definition, since any morphism $\alpha:X\to Y$ is a lift of $f:\emptyset\to U(Y)$, thus it has to be unique. Conversly one can easily show, that initial object is always free with $\emptyset$ as a basis.
%%%%%
%%%%%
\end{document}
