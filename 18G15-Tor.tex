\documentclass[12pt]{article}
\usepackage{pmmeta}
\pmcanonicalname{Tor}
\pmcreated{2013-03-22 14:32:36}
\pmmodified{2013-03-22 14:32:36}
\pmowner{whm22}{2009}
\pmmodifier{whm22}{2009}
\pmtitle{Tor}
\pmrecord{12}{36090}
\pmprivacy{1}
\pmauthor{whm22}{2009}
\pmtype{Definition}
\pmcomment{trigger rebuild}
\pmclassification{msc}{18G15}
\pmclassification{msc}{16E30}
%\pmkeywords{homology}
%\pmkeywords{homological algebra}
\pmrelated{HomologyChainComplex}
\pmrelated{CohomologyOfACochainComplex}
\pmdefines{Tor}
\pmdefines{Ext}

\endmetadata

% this is the default PlanetMath preamble.  as your knowledge
% of TeX increases, you will probably want to edit this, but
% it should be fine as is for beginners.

% almost certainly you want these
\usepackage{amssymb}
\usepackage{amsmath}
\usepackage{amsfonts}

% used for TeXing text within eps files
%\usepackage{psfrag}
% need this for including graphics (\includegraphics)
%\usepackage{graphicx}
% for neatly defining theorems and propositions
%\usepackage{amsthm}
% making logically defined graphics
%%%\usepackage{xypic}

% there are many more packages, add them here as you need them

% define commands here
\begin{document}
Let $R$ be a ring with multiplicative identity.  Let $M$ be a (\PMlinkescapetext{right}) module over $R$.  We may assume there exists an exact sequence $P_*$:

$$
\dots\dots\rightarrow P_2\rightarrow P_1\rightarrow P_0
$$

with the $P_n$ projective and the cokernel of the last map $M$.  Given $M$, this sequence is unique up to chain homotopy.  Hence we may make the following definitions.

For a (\PMlinkescapetext{right}) $R$- module $A$ we may define

$$
Ext_R^n(M,A)=H^n(P_*; A)
$$

For a (left) $R$- module $A$ we may define

$$
Tor_R^n(M,A)=H_n(P_*; A)
$$
%%%%%
%%%%%
\end{document}
