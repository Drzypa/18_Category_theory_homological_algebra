\documentclass[12pt]{article}
\usepackage{pmmeta}
\pmcanonicalname{Epi}
\pmcreated{2013-03-22 14:50:19}
\pmmodified{2013-03-22 14:50:19}
\pmowner{CWoo}{3771}
\pmmodifier{CWoo}{3771}
\pmtitle{epi}
\pmrecord{14}{36507}
\pmprivacy{1}
\pmauthor{CWoo}{3771}
\pmtype{Definition}
\pmcomment{trigger rebuild}
\pmclassification{msc}{18A20}
\pmclassification{msc}{18A05}
\pmsynonym{epimorphism}{Epi}
\pmsynonym{epimorphic}{Epi}
\pmrelated{Monic}
\pmdefines{epic}

\endmetadata

% this is the default PlanetMath preamble.  as your knowledge
% of TeX increases, you will probably want to edit this, but
% it should be fine as is for beginners.

% almost certainly you want these
\usepackage{amssymb,amscd}
\usepackage{amsmath}
\usepackage{amsfonts}
%%\usepackage{xypic}

% used for TeXing text within eps files
\usepackage{psfrag}
% need this for including graphics (\includegraphics)
%\usepackage{graphicx}
% for neatly defining theorems and propositions
%\usepackage{amsthm}
% making logically defined graphics

% there are many more packages, add them here as you need them

% define commands here
\begin{document}
A morphism $f : A\to B$ in a category $\mathcal{C}$ is called {\em epi} if for any object $C$ and any morphisms $g_1,g_2 : B\to C$, if $g_1 f = g_2 f$ then $g_1 = g_2$.  In other words, any diagram 
\begin{center}
$\xymatrix{A \ar[r]^f & B \ar[r]^{g_1} & C}=\xymatrix{A \ar[r]^f & B \ar[r]^{g_2} & C}$
\end{center}
reduces to the diagram $$\xymatrix{B \ar[r]^{g_1} & C}=\xymatrix{B \ar[r]^{g_2} & C}.$$

An \emph{epimorphism} is just an epi morphism, and epi is also known as \emph{right cancellable}, \emph{epimorphic}, or simply \emph{epic}.

\textbf{Remarks.}
\begin{enumerate}
\item If $\mathcal{C}$ is an abelian category, then an epi has the property that $gf=0$ implies $g=0$ (surely, since $gf=0=0f$, and the result follows).
\item Epi is the generalization of a function being onto.  In some categories where surjections are well-defined (such as sets and groups), epi is the same as being onto.
\item The dual notion of epi is that of \PMlinkname{monic}{Monic}.
\end{enumerate}
%%%%%
%%%%%
\end{document}
