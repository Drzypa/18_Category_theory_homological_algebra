\documentclass[12pt]{article}
\usepackage{pmmeta}
\pmcanonicalname{AbelianGroupsFormAnAbelianCategoryProofThat}
\pmcreated{2013-03-22 16:42:55}
\pmmodified{2013-03-22 16:42:55}
\pmowner{mps}{409}
\pmmodifier{mps}{409}
\pmtitle{abelian groups form an abelian category, proof that}
\pmrecord{12}{38932}
\pmprivacy{1}
\pmauthor{mps}{409}
\pmtype{Proof}
\pmcomment{trigger rebuild}
\pmclassification{msc}{18E10}
\pmsynonym{category $Ab$ of Abelian groups}{AbelianGroupsFormAnAbelianCategoryProofThat}
%\pmkeywords{proof}
%\pmkeywords{Abelian category of Abelian groups}
\pmrelated{AlternativeDefinitionOfAnAbelianCategory}
\pmrelated{IndexOfCategories}
\pmrelated{ExamplesOfAbelianCategory}
\pmdefines{category  Ab of abelian groups}

\endmetadata

% this is the default PlanetMath preamble.  as your knowledge
% of TeX increases, you will probably want to edit this, but
% it should be fine as is for beginners.

% almost certainly you want these
\usepackage{amssymb}
\usepackage{amsmath}
\usepackage{amsfonts}

% used for TeXing text within eps files
%\usepackage{psfrag}
% need this for including graphics (\includegraphics)
%\usepackage{graphicx}
% for neatly defining theorems and propositions
\usepackage{amsthm}
% making logically defined graphics
%%\usepackage{xypic}
\xyoption{all}

% there are many more packages, add them here as you need them

% define commands here
\newcommand{\Ab}{\mathbf{Ab}}
\DeclareMathOperator{\Hom}{Hom}
\DeclareMathOperator{\coker}{coker}
\newcommand{\Mod}[1]{{}_{#1}\mathbf{Mod}}
\newcommand{\rMod}[1]{\mathbf{Mod}_{#1}}

\newtheorem*{claim*}{Claim}
\begin{document}
\PMlinkescapeword{properties}
\PMlinkescapeword{structure}
\PMlinkescapeword{sum}
\PMlinkescapeword{satisfies}
\PMlinkescapeword{similar}
\PMlinkescapeword{terminal}

\begin{claim*}
The category $\Ab$ of abelian groups is an abelian category.
\end{claim*}

\begin{proof}
We will justify all the axioms.

(Axiom 1.)  Suppose $A$ and $B$ are abelian groups.  We need to show that $\Hom(A,B)$ has the structure of an abelian group.  Suppose $f\colon A\to B$ and $g\colon A\to B$ are elements of $\Hom(A,B)$, and define their sum $f+g\colon A\to B$ by the rule
\[
(f+g)(x) = f(x) + g(x)
\]
for any $x\in A$.  This operation inherits the commutativity and associativity of the addition in $B$.  Moreover, the function $f+g$ is a homomorphism.  To see this, suppose $x$ and $y$ are in $A$.  Then
\begin{align*}
(f+g)(x+y) &= f(x+y) + g(x+y) = f(x) + g(x) + f(y) + g(y) \\
           &= (f+g)(x) + (f+g)(y).
\end{align*}
The identity in $\Hom(A,B)$ is the constant zero function $0\colon A\to B$, since for any $x\in A$, 
\[
(f+0)(x) = f(x) + 0(x) = f(x) + 0 = f(x).
\]
Thus $\Hom(A,B)$ is an abelian group.  

Now we show that composition of morphisms distributes over addition in $\Hom(\cdot,\cdot)$.  Suppose we are given a diagram
\[\xymatrix{
A \ar[r]^f & B\ar@/^/[r]^g\ar@/_/[r]_h & C\ar[r]^k & D
}\]
of abelian groups.  We claim that $(g+h)f = gf+hf$ and $k(g+h) = kg+kh$.  Since the proofs are similar, we prove only the first identity.  Let $x\in A$.  Then
\begin{align*}
((g+h)f)(x) &= (g+h)(f(x)) = g(f(x)) + h(f(x)) \\
            &= (gf)(x) + (hf)(x) = (gf+hf)(x).
\end{align*}
Thus $\Ab$ satisfies Axiom 1.

(Axiom 2.)  The trivial group $0$ is a zero object in $\Ab$.  It is initial because there exists a unique morphism $0\to A$ for any abelian group $A$, and it is terminal because there exists a unique morphism $A\to 0$ for any abelian group $A$.  In both cases the morphism is the constant zero function.

(Axiom 3.)  The \PMlinkname{Cartesian product}{DirectProductAndRestrictedDirectProductOfGroups} of two abelian groups is a categorical direct product.  Since the Cartesian product of two abelian groups is again abelian, it follows that $\Ab$ has finite products.

(Axiom 4.)  Now we show that $\Ab$ has kernels and cokernels.  Let $f\colon A\to B$ be a morphism.  Define a subset $K\subset A$ by $K = \{ x\in A\colon f(x) = 0\}$, and let $i\colon K\to A$ be inclusion.  The set $K$ is the \PMlinkname{group-theoretic kernel}{KernelOfAGroupHomomorphism} of $f$, hence a normal subgroup of $A$.  Since $A$ is abelian, so is $K$.  The inclusion $i\colon K\to A$ is a morphism in $\Ab$.  Moreover, by construction we have that $fi=0$.  

So suppose that $j\colon L\to A$ is a morphism in $\Ab$ such that $fj=0$.  We need to show that the diagram
\[\xymatrix{
          & L\ar[d]^j\ar@{.>}[dl]_{\tilde{\j}} &   \\
K\ar[r]_i & A\ar[r]_f                              & B
}\]
has a unique filler $\tilde{\j}$.  Let $x\in L$.  Since $fj(x)=0$, it follows that $j(x)\in K$.  So define $\tilde{\j}(x)=j(x)$.  Since the inclusion $i\colon K\to A$ is injective, any alternative choice for $\tilde{\j}$ fails to make the diagram commute.  So $\Ab$ has kernels.

Now we construct a cokernel for $f$.  Define a subset $I\subset B$ by 
\[
I = \{ f(x) \colon x\in A\}.
\]
Then $I$ is a subgroup of the abelian subgroup $B$, so we may form the quotient group $C=B/I$.  Thus $C$ is the group-theoretic cokernel of $f$.  Define a group homomorphism $p\colon B\to C$ by $p(x) = x + I$.  Since $C$ is a quotient of an abelian group, it is abelian, so $p\colon B\to C$ is a morphism in $\Ab$.  Moreover, by construction we have that $pf = 0$.

So suppose that $q\colon A\to D$ is a morphism in $\Ab$ such that $qf=0$.  We need to show that the diagram
\[\xymatrix{
A\ar[r]^f & B\ar[r]^p\ar[d]_q & C\ar@{.>}[dl]^{\tilde{q}} \\
          & D
}\]
has a unique filler.  Let $x+I\in C$.  Suppose $y$ is another representative in $B$ for $x+I$, that is, that $p(x) = p(y)$.  Then $x-y\in I$, that is, there is some $w\in A$ such that $f(w) = x - y$.  By assumption, $qf = 0$, so $q(x) = q(y)$.  So there is a well-defined function $\tilde{q}\colon C\to D$ defined by $\tilde{q}(x+I) = q(x)$.  Moreover, for any $x+I$ and $y+I$ in $C$,
\[
\tilde{q}(x+y+I) = q(x+y) = q(x) + q(y) = \tilde{q}(x+I) +\tilde{q}(y+I),
\]
so $\tilde{q}$ is a morphism in $\Ab$.  But any filler for the diagram must be defined in exactly the way we have defined it in order for the diagram to commute.  Hence $\Ab$ has cokernels.

(Axiom 5.) Suppose $f\colon A\to B$ is a monomorphism, and let $p\colon B\to\coker f$ be  a cokernel of $f$.  We must show that $f$ is a kernel of $p$.  Since $p$ is a cokernel of $f$, we know that $pf = 0$.  Now we must show that if $j\colon L\to B$ is any morphism in $\Ab$ such that $pj = 0$, then the diagram
\[\xymatrix{
          & L\ar[d]^j\ar@{.>}[dl]_{\tilde{\j}} &   \\
A\ar[r]_f & B\ar[r]_<<<p                       & \coker f
}\]
has a unique filler $\tilde{\j}$.  So suppose $x\in L$.  Since $pj(x) = 0$, it follows by the construction of the cokernel given above that there is some $y\in A$ such that $f(y) = j(x)$.  Since $f$ is monomorphism, $f$ is injective, so this $y$ is unique.  We may therefore define $\tilde{\j}$ by the formula $\tilde{\j}(x) = f^{-1}(j(x))$.  Hence $f$ is a kernel for the cokernel of $f$.

(Axiom 6.) Suppose $f\colon A\to B$ is an epimorphism, and let $i\colon \ker f\to A$ be a kernel of $f$.  We must show that $f$ is a cokernel of $i$.  Since $i$ is a kernel of $f$, we know that $fi = 0$.  Now we must show that if $q\colon A\to D$ is any morphism in $\Ab$ such that $qi=0$, then the diagram
\[\xymatrix{
\ker f\ar[r]^i & A\ar[r]^f\ar[d]_q & B\ar@{.>}[dl]^{\tilde{q}} \\
               & D                 &
}\]
has a unique filler $\tilde{q}$.

To prove the existence of $\tilde{q}$, first recall that epimorphisms in $\Ab$ are surjections.  Let $z\in B$.  Suppose $f(x)=f(y)=0$.  Then $x-y$ is in $\ker f$.  By assumption, $qi=0$, so this implies that $q(x)=q(y)$.  So we may define a morphism $\tilde{q}\colon B\to D$ in $\Ab$ by the formula $\tilde{q}(z) = q(x)$, where $x$ is any element of $f^{-1}(z)$.  Moreover, $\tilde{q}f = q$ by construction.

To prove the uniqueness of $\tilde{q}$, suppose that $\hat{q}$ is an alternative filler for the diagram.  Since $\tilde{q}f = \hat{q}f$, it follows that $(\tilde{q}-\hat{q})f=0$.  Since $f$ is an epimorphism, this implies that $\tilde{q}-\hat{q}=0$.  Hence we have shown that $f$ is a cokernel for $i$.
\end{proof}

\textbf{Remark:}
A straightforward, more concise proof is readily obtained using the alternative definition of an Abelian category 
reported by Barry Mitchell (1965). 
%%%%%
%%%%%
\end{document}
