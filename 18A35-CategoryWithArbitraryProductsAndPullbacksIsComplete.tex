\documentclass[12pt]{article}
\usepackage{pmmeta}
\pmcanonicalname{CategoryWithArbitraryProductsAndPullbacksIsComplete}
\pmcreated{2013-03-22 18:42:44}
\pmmodified{2013-03-22 18:42:44}
\pmowner{CWoo}{3771}
\pmmodifier{CWoo}{3771}
\pmtitle{category with arbitrary products and pullbacks is complete}
\pmrecord{4}{41478}
\pmprivacy{1}
\pmauthor{CWoo}{3771}
\pmtype{Corollary}
\pmcomment{trigger rebuild}
\pmclassification{msc}{18A35}
\pmrelated{RelationBetweenPullbacksAndOtherCategoricalLimits}

\usepackage{amssymb,amscd}
\usepackage{amsmath}
\usepackage{amsfonts}
\usepackage{mathrsfs}

% used for TeXing text within eps files
%\usepackage{psfrag}
% need this for including graphics (\includegraphics)
%\usepackage{graphicx}
% for neatly defining theorems and propositions
\usepackage{amsthm}
% making logically defined graphics
\usepackage[curve]{xypic}
\usepackage{pst-plot}

% define commands here
\newcommand*{\abs}[1]{\left\lvert #1\right\rvert}
\newtheorem{prop}{Proposition}
\newtheorem{cor}{Corollary}
\newtheorem{lem}{Lemma}
\newtheorem{thm}{Theorem}
\newtheorem{ex}{Example}
\newcommand{\real}{\mathbb{R}}
\newcommand{\pdiff}[2]{\frac{\partial #1}{\partial #2}}
\newcommand{\mpdiff}[3]{\frac{\partial^#1 #2}{\partial #3^#1}}
\begin{document}
In the parent entry, it is stated that a complete category can be characterized as being a category with arbitrary products and equalizers.  In this entry, we show, as a corollary, that every category with arbitrary products and pullbacks is complete.  We begin with the following observation:

\begin{lem} If a category has finite products and pullbacks, it has equalizers. \end{lem}
\begin{proof}
Suppose we have a pair of morphisms $f,g: A\to B$.  Given the product $A\times B$, there are unique morphisms $f', g': A\to A\times B$ with the following commutative diagrams
\[
\xymatrix@+=1.5cm{
& A \ar[dr]^f \ar[d]_{f'} \ar[dl]_{1_A} & \\
A & A\times B \ar[l]^-{\pi_A} \ar[r]_-{\pi_B} & B
}
\qquad\quad
\xymatrix@+=1.5cm{
& A \ar[dr]^g \ar[d]_{g'} \ar[dl]_{1_A} & \\
A & A\times B \ar[l]^-{\pi_A} \ar[r]_-{\pi_B} & B
}
\]
For the pair $f',g': A\to A\times B$, let
\[
\xymatrix@+=2cm{
P\ar[d]_p \ar[r]^q & A\ar[d]^{g'} \\
A\ar[r]_-{f'}       & A\times B
}
\]
be the pullback diagram, which, after combining with the two small commutative triangles containing the edge $\pi_A$ above, produces the following commutative diagram
\[
\xymatrix@+=1.5cm{
P\ar[d]_p \ar[r]^q & A\ar[d]_-{g'} \ar@/^1ex/[ddr]^{1_A} & \\
A\ar[r]^-{f'} \ar@/_1ex/[drr]_{1_A} & A\times B \ar[dr]|-{\pi_A} & \\
& & A
}
\]
This implies that $p=q$.  This result, together with the pullback diagram combined with the remaining commutative triangles (containing the edge $\pi_B$)
\[
\xymatrix@+=1.5cm{
P\ar[d]_p \ar[r]^p & A\ar[d]_-{g'} \ar@/^1ex/[ddr]^{g} & \\
A\ar[r]^-{f'} \ar@/_1ex/[drr]_{f} & A\times B \ar[dr]|-{\pi_B} & \\
& & B
}
\]
we see that $p$ equalizes $f$ and $g$.  Suppose now that $r:R\to A$ also equalizes $f$ and $g$: $f\circ r = g\circ r$.  Then we get two commutative diagrams
\[
\xymatrix@+=1.5cm{
R\ar[d]_r \ar[r]^r & A\ar[d]_-{g'} \ar@/^1ex/[ddr]^{g} & \\
A\ar[r]^-{f'} \ar@/_1ex/[drr]_{f} & A\times B \ar[dr]|-{\pi_B} & \\
& & B
}
\qquad
\xymatrix@+=1.5cm{
R\ar[d]_r \ar[r]^r & A\ar[d]_-{g'} \ar@/^1ex/[ddr]^{1_A} & \\
A\ar[r]^-{f'} \ar@/_1ex/[drr]_{1_A} & A\times B \ar[dr]|-{\pi_A} & \\
& & A
}
\]
first of which comes from the equation $f\circ r = g\circ r$ and the second one is obvious.  By the universality of the product $A\times B$, we have the commutative diagram
\[
\xymatrix@+=2cm{
R\ar[d]_r \ar[r]^r & A\ar[d]^{g'} \\
A\ar[r]_-{f'}       & A\times B
}
\]
By the universality of the pullback diagram, there is a unique morphism $s: R\to P$ so that $r = p\circ s$, which implies that $p$ is the equalizer of $f$ and $g$.
\end{proof}

The following corollary is now immediate:

\begin{cor} A category $\mathcal{C}$ with arbitrary products and pullbacks is a complete category.  \end{cor}


%%%%%
%%%%%
\end{document}
