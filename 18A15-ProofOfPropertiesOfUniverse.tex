\documentclass[12pt]{article}
\usepackage{pmmeta}
\pmcanonicalname{ProofOfPropertiesOfUniverse}
\pmcreated{2013-03-22 15:37:11}
\pmmodified{2013-03-22 15:37:11}
\pmowner{rspuzio}{6075}
\pmmodifier{rspuzio}{6075}
\pmtitle{proof of properties of universe}
\pmrecord{8}{37544}
\pmprivacy{1}
\pmauthor{rspuzio}{6075}
\pmtype{Proof}
\pmcomment{trigger rebuild}
\pmclassification{msc}{18A15}
\pmclassification{msc}{03E30}

% this is the default PlanetMath preamble.  as your knowledge
% of TeX increases, you will probably want to edit this, but
% it should be fine as is for beginners.

% almost certainly you want these
\usepackage{amssymb}
\usepackage{amsmath}
\usepackage{amsfonts}

% used for TeXing text within eps files
%\usepackage{psfrag}
% need this for including graphics (\includegraphics)
%\usepackage{graphicx}
% for neatly defining theorems and propositions
%\usepackage{amsthm}
% making logically defined graphics
%%%\usepackage{xypic}

% there are many more packages, add them here as you need them

% define commands here
\begin{document}
\begin{enumerate}
\item This is the special case of axiom 2 with $x=y$ since $\{x,x\} = \{x\}$.  (In other words, in set theory, we do not count duplicate entries twice.)
\item By definition of power set, if $x \subset y$, then $x \in \mathcal{P} (y)$.  By axiom 3, $\mathcal{P} (y) \in \mathbf{U}$.  By axiom 1, it follows that $x \in \mathbf{U}$.
\item  By axiom 2, $\{x,y\} \in \mathbf{U}$.  By axiom 2 again, it follows that $\{\{x,y\},x\} \in \mathbf{U}$.
\item  By axiom 2, $\{x,y\} \in \mathbf{U}$.  If we set $z_x = x$ and $z_y = y$, then $x \cup y = \bigcup_{i \in \{x,y\}} z_i$, hence, by axiom 4, $x \cup y \in \mathbf{U}$. --- If $x \in U$ and $y \in U$ then, by axiom 1, $a \in \mathbf{U}$ for all $a \in x$ and $b \in \mathbf{U}$ for all $b \in y$.  By property 3, if $a \in \mathbf{U}$ and $b \in \mathbf{U}$, then $(a,b) \in \mathbf{U}$; further, by property 1, $\{(a,b)\} \in \mathbf{U}$.  Hence, by axiom 4, $\{(a,b) \mid b \in y \} = \bigcup_{b \in y} \{(a,b)\} \in \mathbf{U}$ for all $a \in x$.  Using axiom 4 again, we conclude that $x \times y = \{(a,b) \mid a \in x \wedge b \in y \} = \bigcup_{a \in y} \{(a,b) \mid b \in y \} \in \mathbf{U}$
\item  By axiom 4, $\bigcup_{i \in I} x_i \in \mathbf{U}$.  By property 4, $I \times \bigcup_{i \in I} x_i \in \mathbf{U}$.  Now, every function from $I$ to $\bigcup_{i \in I} x_i \in \mathbf{U}$ is a subset of $I \times \bigcup_{i \in I} x_i \in \mathbf{U}$.  Since $\prod_{i \in I} x_i$ is a set of functions from $I$ to $\bigcup_{i \in I} x_i \in \mathbf{U}$, we have, by defintion of power set, $\prod_{i \in I} x_i \subset \mathcal(P) (I \times \bigcup_{i \in I} x_i)$.  Hence, by axiom 3 and property 2, we conclude that $\prod_{i \in I} x_i \in \mathbf{U}$.
\item  Assume the contrary, namely that $x \in \mathbf{U}$ and $\#x \ge \#\mathbf{U}$.  By axiom 3, $\mathcal{P} (x) \in \mathbf{U}$ but $\#(\mathcal{P}(x)) = 2^{\#x} \ge 2^{\#\mathbf{U}}$.  Since, by axiom 1, every element of $\mathcal{P} (x)$ belongs to $\mathbf{U}$, this would mean that we would have at least $2^{\#\mathbf{U}}$ elements of $\mathbf{U}$, which contradicts the fact that $\#U < 2^{\#\mathbf{U}}$.  (This argument is a variation on Cantor's paradox.)
\end{enumerate}
%%%%%
%%%%%
\end{document}
