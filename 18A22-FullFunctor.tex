\documentclass[12pt]{article}
\usepackage{pmmeta}
\pmcanonicalname{FullFunctor}
\pmcreated{2013-03-22 14:21:52}
\pmmodified{2013-03-22 14:21:52}
\pmowner{CWoo}{3771}
\pmmodifier{CWoo}{3771}
\pmtitle{full functor}
\pmrecord{4}{35848}
\pmprivacy{1}
\pmauthor{CWoo}{3771}
\pmtype{Definition}
\pmcomment{trigger rebuild}
\pmclassification{msc}{18A22}
\pmrelated{FaithfulFunctor}

% this is the default PlanetMath preamble.  as your knowledge
% of TeX increases, you will probably want to edit this, but
% it should be fine as is for beginners.

% almost certainly you want these
\usepackage{amssymb,amscd}
\usepackage{amsmath}
\usepackage{amsfonts}

% used for TeXing text within eps files
%\usepackage{psfrag}
% need this for including graphics (\includegraphics)
%\usepackage{graphicx}
% for neatly defining theorems and propositions
%\usepackage{amsthm}
% making logically defined graphics
%%%\usepackage{xypic}

% there are many more packages, add them here as you need them

% define commands here
\begin{document}
\PMlinkescapeword{arrow}
\PMlinkescapeword{function}

A functor $T:\mathcal{C}\to\mathcal{D}$ is \emph{full} if the \emph{arrow function} of $T$ is surjective for every 
pair of objects in $\mathcal{C}$.  More precisely, for every pair $C_1, C_2\in \operatorname{Ob}(\mathcal{C})$, the 
arrow function $T_{(C_1,C_2)}$ of $T:$ 
$$T_{(C_1,C_2)}:\operatorname{hom_{\mathcal{C}}}(C_1,C_2)\to\operatorname{hom_{\mathcal{D}}}(T(C_1),T(C_2))$$
given by $T_{(C_1,C_2)}(f)=T(f)$ is a surjection.
%%%%%
%%%%%
\end{document}
