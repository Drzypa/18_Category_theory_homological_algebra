\documentclass[12pt]{article}
\usepackage{pmmeta}
\pmcanonicalname{RegularCategory}
\pmcreated{2013-03-22 18:27:41}
\pmmodified{2013-03-22 18:27:41}
\pmowner{CWoo}{3771}
\pmmodifier{CWoo}{3771}
\pmtitle{regular category}
\pmrecord{9}{41129}
\pmprivacy{1}
\pmauthor{CWoo}{3771}
\pmtype{Definition}
\pmcomment{trigger rebuild}
\pmclassification{msc}{18E10}
\pmsynonym{exact fork}{RegularCategory}
\pmrelated{QuotientCategory}
\pmdefines{exact sequence}

\usepackage{amssymb,amscd}
\usepackage{amsmath}
\usepackage{amsfonts}
\usepackage{mathrsfs}

% used for TeXing text within eps files
%\usepackage{psfrag}
% need this for including graphics (\includegraphics)
%\usepackage{graphicx}
% for neatly defining theorems and propositions
\usepackage{amsthm}
% making logically defined graphics
%%\usepackage{xypic}
\usepackage{pst-plot}

% define commands here
\newcommand*{\abs}[1]{\left\lvert #1\right\rvert}
\newtheorem{prop}{Proposition}
\newtheorem{thm}{Theorem}
\newtheorem{ex}{Example}
\newcommand{\real}{\mathbb{R}}
\newcommand{\pdiff}[2]{\frac{\partial #1}{\partial #2}}
\newcommand{\mpdiff}[3]{\frac{\partial^#1 #2}{\partial #3^#1}}
\begin{document}
A category $\mathcal{C}$ is called a \emph{regular category} if 
\begin{enumerate}
\item every morphism has a kernel pair,
\item every kernel pair has a coequalizer, and
\item the pullback of every regular epimorphism along any morphism exists and is again regular.  This means the following: if $f:A\to B$ is a regular epimorphism, and $g:C\to B$ is any morphism, then the pullback diagram below
$$
\xymatrix@+=1.5cm{D \ar[r]^h \ar[d] & C \ar[d]^g \\ A \ar[r]_f & B}
$$
exists, and $h$ is again a regular epimorphism.
\end{enumerate}

Some examples of regular categories are: any abelian category, the category of sets, and the category of groups.  On the other hand, the category of topological spaces and the category of small categories are not regular.

\textbf{Remarks}.  
\begin{itemize}
\item If a category $\mathcal{C}$ is finitely complete, it can be shown that $\mathcal{C}$ is regular iff the strong epimorphisms are stable under pullbacks, and every morphism has a mono-strong-epi factorization: for every morphisms $f$, we have $f=g\circ h$ where $g$ is a monomorphism and $h$ is a strong epimorphism.
\item
Regular categories are generalizations of abelian categories, so that the exactness conditions can be defined without the requirement that the categories be additive.  More precisely, in a regular category $\mathcal{C}$, we define an \emph{exact sequence}, or \emph{exact fork}, to be a 6-tuple $(A,B,C,f,g,h)$ where 
\begin{itemize}
\item $A,B,C$ are objects
\item $f,g:A\to B$ and $h:B\to C$ are morphisms: $\xymatrix@+=2cm{A \ar@<0.5ex>[r]^f \ar@<-0.5ex>[r]_g & B \ar[r]^h & C}$
\end{itemize}
such that $(f,g)$ is the kernel pair of $h$ and $h$ is the coequalizer of $f$ and $g$.  $h$ is the coequalizer portion of the exact sequence, and $(f,g)$ is the kernel pair portion of the exact sequence.

One of the first consequences of the above definition is: every regular epimorphism in a regular category is the coequalizer portion of an exact sequence.

The main result, however, is that, in an abelian category, the two notions of the exactness coincide in the following sense: $(A,B,C,f,g,h)$ is exact precisely when 
$$\xymatrix@+=2cm{0 \ar[r] & A \ar[r]^-{f \choose g} & B\oplus B \ar[r]^-{(h \enspace -h)} & C \ar[r] & 0}$$
is a short exact sequence.
\end{itemize}

\begin{thebibliography}{9}
\bibitem{fb} F. Borceux \emph{Categories and Structures, Handbook of Categorical Algebra II}, Cambridge University Press, Cambridge (1994)
\end{thebibliography}
%%%%%
%%%%%
\end{document}
