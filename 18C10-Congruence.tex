\documentclass[12pt]{article}
\usepackage{pmmeta}
\pmcanonicalname{Congruence}
\pmcreated{2013-03-22 11:43:15}
\pmmodified{2013-03-22 11:43:15}
\pmowner{rspuzio}{6075}
\pmmodifier{rspuzio}{6075}
\pmtitle{congruence}
\pmrecord{27}{30101}
\pmprivacy{1}
\pmauthor{rspuzio}{6075}
\pmtype{Definition}
\pmcomment{trigger rebuild}
\pmclassification{msc}{18C10}
\pmclassification{msc}{11A07}
\pmclassification{msc}{11A05}
\pmclassification{msc}{92B20}
\pmclassification{msc}{92B05}
\pmclassification{msc}{55M05}
\pmclassification{msc}{18E05}
\pmclassification{msc}{18-00}
\pmsynonym{congruent}{Congruence}
%\pmkeywords{number theory}
%\pmkeywords{integers}
%\pmkeywords{divisibility}
\pmrelated{Congruence2}
\pmrelated{GCD}
\pmrelated{ExampleOfGcd}
\pmrelated{MathbbZ_n}
\pmrelated{CongruenceModuloAnIdeal}
\pmrelated{CongruenceInAlgebraicNumberField}
\pmrelated{NumberTheory}
\pmdefines{number congruence}
\pmdefines{residue class}
\pmdefines{residue class ring}

\usepackage{amssymb}
\usepackage{amsmath}
\usepackage{amsfonts}
\usepackage{graphicx}
%%%%%%%%%%\usepackage{xypic}
\begin{document}
Let $a$, $b$ be integers and $m$ a non-zero integer.\, We say that \emph{$a$ is congruent to $b$ modulo $m$}, if $m$ divides $b-a$ (the word {\em modulo} is the dative case of the Latin noun \PMlinkescapetext{{\em modulus}} meaning the 'module').\, We write this {\em number congruence} or shortly {\em congruence} as
                         $$a\equiv b\pmod{m}.$$

If $a$ and $b$ are congruent modulo $m$, it means that both numbers leave the same residue when divided by $m$.

Congruence with a fixed module is an equivalence relation on $\mathbb{Z}$.\, The set of equivalence classes, the so-called {\em residue classes}, is a cyclic group of order $m$ (assuming it positive) with respect to addition and a ring if we consider also the multiplication modulo $m$.\, This ring is usually denoted as 
$$\frac{\mathbb{Z}}{m\mathbb{Z}}$$
and called the {\em residue class ring} modulo $m$.
This ring is also commonly denoted as\, $\mathbb{Z}_m$,\, $\mathbb{Z}/(m)$.\, However, when\, $m = p$\, is a prime number, notation $\mathbb{Z}_p$ is also used to denote $p$-adic numbers.\\

\textbf{\PMlinkescapetext{Properties} of congruences}
\begin{enumerate}
\item If\; $a\equiv b\pmod{m}$,\, then\, $a\!+\!c\equiv b\!+\!c\pmod{m}$\, and 
$ac\equiv bc\pmod{m}$.
\item If\; $a\equiv b\pmod{m}$\, and\, $c\equiv d\pmod{m}$,\, then\,
$a\!\pm\!c\equiv b\!\pm\!d\pmod{m}$\, and\, $ac\equiv bd\pmod{m}$.
\item If\; $a\equiv b\pmod{m}$\, and $f$ is a polynomial with integer coefficients, then\, $f(a)\equiv f(b)\pmod{m}$.
\item If\; $ac\equiv bc\pmod{m}$\, and\, $\gcd(c,\,m) = 1$,\, then\, 
$a\equiv b\pmod{m}$.
\item If\; $ac \equiv bc \pmod{m}$,\; then\; $a \equiv b \pmod{\frac{m}{\gcd(c,\,m)}}$.\\
\end{enumerate}


{\em Proof of} 5.\, Let\; $\gcd(c,\,m) := d,\;\; c := c'd,\;\; m := m'd$,\, where\; $\gcd(c',\,m') = 1$.\; The given congruence means that\, $m \mid (a\!-\!b)c$, whence\; $m' \mid (a\!-\!b)c'$.\; Since $c'$ and $m'$ are coprime, we infer that\; 
$m' \mid a\!-\!b$,\; i.e.\; $a \equiv b \pmod{m'}$.\, Q.E.D.\\

\textbf{Remark.}\, For justifying the latter asserted congruence of 2, one forms the sum $(a-b)c+(c-d)b$ in which the both differences are supposed divisible by $m$.\, Since the sum is simply $ac-bd$ and divisible by $m$, one obtains the asserted congruence.\, By induction, it is generalised to the case with any number $n$ of factors on both \PMlinkescapetext{sides}; hence one infers also the result\; $a^n \equiv b^n \pmod{m}$.

%%%%%
%%%%%
%%%%%
%%%%%
%%%%%
%%%%%
%%%%%
%%%%%
%%%%%
%%%%%
\end{document}
