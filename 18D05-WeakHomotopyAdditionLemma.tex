\documentclass[12pt]{article}
\usepackage{pmmeta}
\pmcanonicalname{WeakHomotopyAdditionLemma}
\pmcreated{2013-03-22 18:15:09}
\pmmodified{2013-03-22 18:15:09}
\pmowner{bci1}{20947}
\pmmodifier{bci1}{20947}
\pmtitle{weak homotopy addition lemma}
\pmrecord{42}{40847}
\pmprivacy{1}
\pmauthor{bci1}{20947}
\pmtype{Corollary}
\pmcomment{trigger rebuild}
\pmclassification{msc}{18D05}
\pmclassification{msc}{55N33}
\pmclassification{msc}{55N20}
\pmclassification{msc}{55P10}
\pmclassification{msc}{22A22}
\pmclassification{msc}{55U40}
\pmsynonym{homotopy addition lemma}{WeakHomotopyAdditionLemma}
%\pmkeywords{weak homotopy equivalence lemma}
\pmrelated{WeakHomotopyEquivalence}
\pmrelated{CubicallyThinHomotopy}
\pmrelated{ThinDoubleTracks}
\pmrelated{ThinEquivalenceRelation}
\pmrelated{GeneralizedVanKampenTheoremsHigherDimensional}
\pmrelated{HigherDimensionalAlgebraHDA}
\pmrelated{WeakHomotopyEquivalence}
\pmrelated{WeakHomotopyDoubleGroupoid}
\pmrelated{HomotopyAdditionLemma}
\pmrelated{FEquivalenceI}
\pmdefines{weak homotopy addition}
\pmdefines{path groupoid}
\pmdefines{weak homotopy double groupoid}

\endmetadata

% this is the default PlanetMath preamble.  as your knowledge
% of TeX increases, you will probably want to edit this, but
% it should be fine as is for beginners.

% almost certainly you want these
\usepackage{amssymb}
\usepackage{amsmath}
\usepackage{amsfonts}

% used for TeXing text within eps files
%\usepackage{psfrag}
% need this for including graphics (\includegraphics)
%\usepackage{graphicx}
% for neatly defining theorems and propositions
%\usepackage{amsthm}
% making logically defined graphics
%%%\usepackage{xypic}

% there are many more packages, add them here as you need them

% define commands here
\usepackage{amsmath, amssymb, amsfonts, amsthm, amscd, latexsym}
%%\usepackage{xypic}
\usepackage[mathscr]{eucal}

\setlength{\textwidth}{6.5in}
%\setlength{\textwidth}{16cm}
\setlength{\textheight}{9.0in}
%\setlength{\textheight}{24cm}

\hoffset=-.75in     %%ps format
%\hoffset=-1.0in     %%hp format
\voffset=-.4in

\theoremstyle{plain}
\newtheorem{lemma}{Lemma}[section]
\newtheorem{proposition}{Proposition}[section]
\newtheorem{theorem}{Theorem}[section]
\newtheorem{corollary}{Corollary}[section]

\theoremstyle{definition}
\newtheorem{definition}{Definition}[section]
\newtheorem{example}{Example}[section]
%\theoremstyle{remark}
\newtheorem{remark}{Remark}[section]
\newtheorem*{notation}{Notation}
\newtheorem*{claim}{Claim}

\renewcommand{\thefootnote}{\ensuremath{\fnsymbol{footnote%%@
}}}
\numberwithin{equation}{section}

\newcommand{\Ad}{{\rm Ad}}
\newcommand{\Aut}{{\rm Aut}}
\newcommand{\Cl}{{\rm Cl}}
\newcommand{\Co}{{\rm Co}}
\newcommand{\DES}{{\rm DES}}
\newcommand{\Diff}{{\rm Diff}}
\newcommand{\Dom}{{\rm Dom}}
\newcommand{\Hol}{{\rm Hol}}
\newcommand{\Mon}{{\rm Mon}}
\newcommand{\Hom}{{\rm Hom}}
\newcommand{\Ker}{{\rm Ker}}
\newcommand{\Ind}{{\rm Ind}}
\newcommand{\IM}{{\rm Im}}
\newcommand{\Is}{{\rm Is}}
\newcommand{\ID}{{\rm id}}
\newcommand{\GL}{{\rm GL}}
\newcommand{\Iso}{{\rm Iso}}
\newcommand{\Sem}{{\rm Sem}}
\newcommand{\St}{{\rm St}}
\newcommand{\Sym}{{\rm Sym}}
\newcommand{\SU}{{\rm SU}}
\newcommand{\Tor}{{\rm Tor}}
\newcommand{\U}{{\rm U}}

\newcommand{\A}{\mathcal A}
\newcommand{\Ce}{\mathcal C}
\newcommand{\D}{\mathcal D}
\newcommand{\E}{\mathcal E}
\newcommand{\F}{\mathcal F}
\newcommand{\G}{\mathcal G}
\newcommand{\Q}{\mathcal Q}
\newcommand{\R}{\mathcal R}
\newcommand{\cS}{\mathcal S}
\newcommand{\cU}{\mathcal U}
\newcommand{\W}{\mathcal W}

\newcommand{\bA}{\mathbb{A}}
\newcommand{\bB}{\mathbb{B}}
\newcommand{\bC}{\mathbb{C}}
\newcommand{\bD}{\mathbb{D}}
\newcommand{\bE}{\mathbb{E}}
\newcommand{\bF}{\mathbb{F}}
\newcommand{\bG}{\mathbb{G}}
\newcommand{\bK}{\mathbb{K}}
\newcommand{\bM}{\mathbb{M}}
\newcommand{\bN}{\mathbb{N}}
\newcommand{\bO}{\mathbb{O}}
\newcommand{\bP}{\mathbb{P}}
\newcommand{\bR}{\mathbb{R}}
\newcommand{\bV}{\mathbb{V}}
\newcommand{\bZ}{\mathbb{Z}}

\newcommand{\bfE}{\mathbf{E}}
\newcommand{\bfX}{\mathbf{X}}
\newcommand{\bfY}{\mathbf{Y}}
\newcommand{\bfZ}{\mathbf{Z}}

\renewcommand{\O}{\Omega}
\renewcommand{\o}{\omega}
\newcommand{\vp}{\varphi}
\newcommand{\vep}{\varepsilon}

\newcommand{\diag}{{\rm diag}}
\newcommand{\grp}{{\mathbb G}}
\newcommand{\dgrp}{{\mathbb D}}
\newcommand{\desp}{{\mathbb D^{\rm{es}}}}
\newcommand{\Geod}{{\rm Geod}}
\newcommand{\geod}{{\rm geod}}
\newcommand{\hgr}{{\mathbb H}}
\newcommand{\mgr}{{\mathbb M}}
\newcommand{\ob}{{\rm Ob}}
\newcommand{\obg}{{\rm Ob(\mathbb G)}}
\newcommand{\obgp}{{\rm Ob(\mathbb G')}}
\newcommand{\obh}{{\rm Ob(\mathbb H)}}
\newcommand{\Osmooth}{{\Omega^{\infty}(X,*)}}
\newcommand{\ghomotop}{{\rho_2^{\square}}}
\newcommand{\gcalp}{{\mathbb G(\mathcal P)}}

\newcommand{\rf}{{R_{\mathcal F}}}
\newcommand{\glob}{{\rm glob}}
\newcommand{\loc}{{\rm loc}}
\newcommand{\TOP}{{\rm TOP}}

\newcommand{\wti}{\widetilde}
\newcommand{\what}{\widehat}

\renewcommand{\a}{\alpha}
\newcommand{\be}{\beta}
\newcommand{\ga}{\gamma}
\newcommand{\Ga}{\Gamma}
\newcommand{\de}{\delta}
\newcommand{\del}{\partial}
\newcommand{\ka}{\kappa}
\newcommand{\si}{\sigma}
\newcommand{\ta}{\tau}
\newcommand{\med}{\medbreak}
\newcommand{\medn}{\medbreak \noindent}
\newcommand{\bign}{\bigbreak \noindent}
\newcommand{\lra}{{\longrightarrow}}
\newcommand{\ra}{{\rightarrow}}
\newcommand{\rat}{{\rightarrowtail}}
\newcommand{\oset}[1]{\overset {#1}{\ra}}
\newcommand{\osetl}[1]{\overset {#1}{\lra}}
\newcommand{\hr}{{\hookrightarrow}}
\begin{document}
\textbf{Weak homotopy addition lemma.} \\
\emph{Consider $\textbf{u} : I^3 \to X _{cg}$ to be a singular cube in a compactly-generated 
space $X _{cg}$. Then by restricting \textbf{$u$} to the faces of $I^3$ and taking the 
corresponding elements in $\boldsymbol{\rho}^{\square}_2 (X_{cg})$, one obtains a cube in 
$\boldsymbol{\rho}^{\square} (X _{cg})$ which is commutative by an extension of the homotopy
addition lemma for $\boldsymbol{\rho}^{\square} (X_{cg})$ (\cite{BHKP}, Proposition 5.5) to 
\emph{weak homotopy} in $X _{cg}$. Consequently, if $f : \boldsymbol{\rho}^{\square} (X_{cg}) \to \mathsf{D}$ is a morphism of double groupoids with connections, any singular cube in $X_{cg}$ determines a commutative {3-shell} in $\mathsf{D}$.}

\subsection{Related concepts} 

\textbf{Definition 0.1} \emph{\textbf{Weak homotopy double groupoid}}

Let us first define the \emph{weak homotopy double groupoid (WHDG)} of a 
\emph{compactly--generated space} $X _{cg}$, (weak Hausdorff space). We utilize here the construction method developed by R. Brown (ref. \cite{BHKP}) for the \emph{homotopy double groupoid of a Hausdorff space}, with the key change in this construction that involves replacing the regular homotopy equivalence relation with the 
\PMlinkname{weak homotopy equivalence relation}{WeakHomotopyEquivalence} in the definition of the fundamental groupoid, as well as the replacement of the Hausdorff space by the compactly-generated space $X_{cg}$. Therefore, the weak homotopy data for the \emph{weak homotopy double groupoid} of $X_{cg}$, $\boldsymbol{\rho}^{\square} (X_{cg})$, will now be: 

 \[
\begin{array}{c}
(\boldsymbol{\rho}^{{\square}_{2} (X _{cg})} , \boldsymbol{\rho}_1^{\square (X _{cg}) ,
\partial^{-}_{1} , \partial^{+}_{1} , +_{1} , \varepsilon _{1}}) ,
\ (\boldsymbol{\rho}^{{\square}_{2} (X _{cg}), \boldsymbol{\rho}^\square_1 (X _{cg}) ,
\partial^{-}_{2} , \partial^{+}_{2} , +_{2} , \varepsilon _{2}} )\\[3mm]
(\boldsymbol{\rho}^{\square_1 (X _{cg}) , X _{cg} , \partial^{-} ,
\partial^{+} , + , \varepsilon }),
\end{array}\]
\bigbreak
where the data following the square symbol define how the construction is carried around the square as specifically 
explained by the concepts of \emph{thin double track, thin equivalence relation and cubically thin homotopy}. 
\subsection{Path Groupoid}

 \textbf{Definition 0.2}  

We have also introduced above the notation $\boldsymbol{\rho}_1 (X _{cg})$ which denotes the \emph{path 
groupoid} of $X _{cg}$, analogous to the definition in ref. \cite{HKK} for a Hausdorff space. 
The objects of $\boldsymbol{\rho}_1 (X _{cg}) $ are therefore the points of $X _{cg}$. 
The morphisms of $\boldsymbol{\rho}^\square_1 (X _{cg})$ are the \emph{weak} homotopy equivalence classes of paths in 
$X _{cg}$ with respect to the \emph{weak} homotopy equivalence, $\sim _W$, defined on $X _{cg}$.

    With these definitions one can now proceed as suggested next to prove the weak homotopy addition lemma with the weak homotopy equivalence defined for compactly--generated topological spaces, $X _{cg}$.

\subsection{Remarks: a brief outline of lemma's consequences}

The weak homotopy double groupoid can be expressed--at least in principle-- with the help of the 
weak homotopy addition lemma (WHAL), and the general, categorical form of the higher dimensional, generalized Van Kampen Theorems (HDA-GVKT), by determining (\emph{assuming that it exists!}) the categorical \emph{colimit} of a sequence of fundamental groupoids of the spaces that define a filtering sequence; for example in the case of quantum spaces (QSS), or algebraic QFT (AQFT), of the extended Gel'fand triple $(\Phi_{cg}, \textbf{H}, [\Phi^*]_{cg})$ representation of the generalized QSS of AQFT and/or \emph{non-Abelian Quantum Gravity} (NAQG) theories, with $\Phi_{cg}$ and $[\Phi^*]_{cg}$ being compactly--generated topological spaces.  


\begin{thebibliography}{9}

\bibitem{HKK}
K.A. Hardie, K.H. Kamps and R.W. Kieboom., A homotopy 2--groupoid of a Hausdorff space,
\emph{Applied Cat. Structures}, \textbf{8} (2000): 209-234.

\bibitem{BHKP}
R. Brown, K.A. Hardie, K.H. Kamps  and T. Porter., A homotopy double groupoid of a Hausdorff %%@
space., {\it Theory and Applications of Categories.} \textbf{10},(2002): 71-93.

\end{thebibliography}
%%%%%
%%%%%
\end{document}
