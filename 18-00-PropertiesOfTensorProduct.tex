\documentclass[12pt]{article}
\usepackage{pmmeta}
\pmcanonicalname{PropertiesOfTensorProduct}
\pmcreated{2013-03-22 17:21:06}
\pmmodified{2013-03-22 17:21:06}
\pmowner{polarbear}{3475}
\pmmodifier{polarbear}{3475}
\pmtitle{properties of tensor product}
\pmrecord{10}{39709}
\pmprivacy{1}
\pmauthor{polarbear}{3475}
\pmtype{Derivation}
\pmcomment{trigger rebuild}
\pmclassification{msc}{18-00}
\pmclassification{msc}{13-00}

\endmetadata

% this is the default PlanetMath preamble.  as your knowledge
% of TeX increases, you will probably want to edit this, but
% it should be fine as is for beginners.

% almost certainly you want these
\usepackage{amssymb}
\usepackage{amsmath}
\usepackage{amsfonts}

% used for TeXing text within eps files
%\usepackage{psfrag}
% need this for including graphics (\includegraphics)
%\usepackage{graphicx}
% for neatly defining theorems and propositions
\usepackage{amsthm}
% making logically defined graphics
%%%\usepackage{xypic}

% there are many more packages, add them here as you need them

% define commands here
\newtheorem{theorem}{Theorem}


\begin{document}
\begin{theorem}
 Let $V_1,V_2$ be two vector spaces over a field $F$.
 If $x_1,\ldots,x_n\in V_1$ are linearly independent and $y_1,\ldots,y_n\in V_2$ then
$$
\sum x_i \otimes y_i =0 \Rightarrow y_i=0,\mbox{ for all } i$$\end{theorem}
\begin{proof}
 Take the dual vectors $x_i^{*}$ to the vectors $x_i$, i.e. $x_i^{*}(x_j)=\delta_{i,j}$. Given arbitrary linear functionals $f_i:V_2 \rightarrow F$, define a bilinear form $f:V_1\times V_2\rightarrow F$ by
$$ f(x,y)=\sum_{j=1}^n x_{j}^{*}(x)f_j(y)$$
 By the definition of tensor product there exists a unique linear functional $\phi:V_1\otimes V_2\rightarrow F$ such that $\phi \circ \iota =f$. Therefore
\begin{eqnarray*} 
0&=&\phi\left(\sum_{i} x_i\otimes y_i\right) \\
&=&\sum_{i} \phi\circ \iota (x_i,y_i) \\
 &=&\sum_{i} f(x_i,y_i) \\
&=&\sum_i \sum_j x_{j}^{*}(x_i)f_j(y_i)\\
& =&\sum_{i} f_i(y_i).
\end{eqnarray*}

 Since the $f_i$ are arbitrary, it follows that $y_i=0$ for all $i$.


\end{proof}
%%%%%
%%%%%
\end{document}
