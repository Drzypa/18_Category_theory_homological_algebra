\documentclass[12pt]{article}
\usepackage{pmmeta}
\pmcanonicalname{SimplicialCategory}
\pmcreated{2013-03-22 12:59:33}
\pmmodified{2013-03-22 12:59:33}
\pmowner{mhale}{572}
\pmmodifier{mhale}{572}
\pmtitle{simplicial category}
\pmrecord{8}{33367}
\pmprivacy{1}
\pmauthor{mhale}{572}
\pmtype{Definition}
\pmcomment{trigger rebuild}
\pmclassification{msc}{18G30}
\pmrelated{SimplicialObject}
\pmrelated{Nerve}

\usepackage{amssymb}
\usepackage{amsmath}
\usepackage{amsfonts}
\usepackage{amsthm}

% used for TeXing text within eps files
%\usepackage{psfrag}
% need this for including graphics (\includegraphics)
%\usepackage{graphicx}
% making logically defined graphics
%%%\usepackage{xypic}

% my maths package

\newcommand*{\Nset}{\mathbb{N}}
\newcommand*{\Zset}{\mathbb{Z}}
\newcommand*{\Qset}{\mathbb{Q}}
\newcommand*{\Rset}{\mathbb{R}}
\newcommand*{\Cset}{\mathbb{C}}
\newcommand*{\Hset}{\mathbb{H}}
\newcommand*{\Oset}{\mathbb{O}}
\newcommand*{\Bset}{\mathbb{B}}
\newcommand*{\Kset}{\mathbb{K}}
\newcommand*{\Sset}{\mathbb{S}}
\newcommand*{\Tset}{\mathbb{T}}
\newcommand*{\GLgrp}{\mathrm{GL}}
\newcommand*{\SLgrp}{\mathrm{SL}}
\newcommand*{\Ogrp}{\mathrm{O}}
\newcommand*{\SOgrp}{\mathrm{SO}}
\newcommand*{\Ugrp}{\mathrm{U}}
\newcommand*{\SUgrp}{\mathrm{SU}}
\newcommand*{\e}{\mathop{\mathrm{e}}\nolimits}
\newcommand*{\im}{\mathord{\mathrm{i}}}
\newcommand*{\identity}{\mathord{\mathrm{1\!\!\!\:I}}}
\newcommand*{\tr}{\mathop{\mathrm{tr}}}
\newcommand*{\Tr}{\mathop{\mathrm{Tr}}}
\renewcommand*{\d}{\mathrm{d}}
\newcommand*{\deriv}[2]{\frac{\d #1}{\d #2}}
\newcommand*{\pderiv}[2]{\frac{\partial #1}{\partial #2}}
\newcommand*{\fderiv}[2]{\frac{\delta #1}{\delta #2}}

% my noncommutative geometry package

\newcommand*{\algebra}[1][A]{\mathord{\mathcal{#1}}}
\newcommand*{\hilbert}[1][H]{\mathord{\mathcal{#1}}}
\newcommand*{\hilbmod}[1][E]{\mathord{\mathcal{#1}}}
\newcommand*{\Matrix}[2]{\mathord{\mathrm{M}_{#1}(#2)}}
\newcommand*{\dixmier}{\mathop{\mathrm{Tr}_\omega}}
\newcommand*{\Res}{\mathop{\mathrm{Res}}}
\newcommand*{\Wres}{\mathop{\mathrm{Wres}}}
\newcommand*{\Aut}{\mathop{\mathrm{Aut}}\nolimits}
\newcommand*{\Inn}{\mathop{\mathrm{Inn}}\nolimits}
\newcommand*{\Out}{\mathop{\mathrm{Out}}\nolimits}
\newcommand*{\Diff}{\mathop{\mathrm{Diff}}\nolimits}
\newcommand*{\Ker}{\mathop{\mathrm{Ker}}\nolimits}
\newcommand*{\Coker}{\mathop{\mathrm{Coker}}\nolimits}
\newcommand*{\Img}{\mathop{\mathrm{Im}}\nolimits}
\newcommand*{\End}{\mathop{\mathrm{End}}\nolimits}
\newcommand*{\spin}{\mathop{\mathrm{spin}}\nolimits}
\newcommand*{\Ind}{\mathop{\mathrm{Ind}}\nolimits}
\newcommand*{\KK}{\mathit{KK}}
\newcommand*{\HH}{\mathit{HH}}
\newcommand*{\HC}{\mathit{HC}}
\newcommand*{\ch}{\mathop{\mathrm{ch}}\nolimits}

% my category theory package

\newcommand*{\mathcat}[1]{\mathord{\mathbf{#1}}}
\newcommand*{\id}{\mathrm{id}}
\newcommand*{\op}{\mathrm{op}}
\newcommand*{\boxprod}{\mathbin{\square}}

% my environments

\newtheoremstyle{inlinedefn}{}{0pt}{}{}{\bfseries}{.}{0.5em}{}
\theoremstyle{inlinedefn}
\newtheorem{definition}{Definition}

\newtheoremstyle{break}{\baselineskip}{\baselineskip}{\itshape}{}{\bfseries}{}{\newline}{}
\theoremstyle{break}
\newtheorem{example}{Example}

% misc commands

\newcommand*{\defn}[1]{\textbf{#1}}
\begin{document}
The \textbf{simplicial category} $\Delta$ is defined as the small category
whose objects are the totally ordered finite sets
\begin{equation}
[n] = \{0<1<2<\ldots<n\}, \quad n\geq0,
\end{equation}
and whose morphisms are monotonic non-decreasing (order-preserving) maps.
It is generated by two families of morphisms:
\begin{eqnarray*}
\delta^n_i & \colon & [n-1] \to [n] \quad\mbox{is the injection missing\ } i\in[n], \\
\sigma^n_i & \colon & [n+1] \to [n] \quad\mbox{is the surjection such that\ } \sigma^n_i(i)=\sigma^n_i(i+1)=i\in[n].
\end{eqnarray*}
The $\delta^n_i$ morphisms are called \defn{face maps},
and the $\sigma^n_i$ morphisms are called \defn{degeneracy maps}.
They satisfy the following relations,
\begin{eqnarray}
\delta^{n+1}_j\,\delta^n_i & = & \delta^{n+1}_i\,\delta^n_{j-1}
\quad\mbox{for\ } i<j, \\
\sigma^{n-1}_j\,\sigma^n_i & = & \sigma^{n-1}_i\,\sigma^n_{j+1}
\quad\mbox{for\ } i\leq j, \\
\sigma^n_j\,\delta^{n+1}_i & = & \left\{
\begin{array}{ll}
\delta^n_i\,\sigma^{n-1}_{j-1} & \mbox{if\ } i<j, \\
\id_n & \mbox{if\ } i=j \mbox{\ or\ } i=j+1, \\
\delta^n_{i-1}\,\sigma^{n-1}_j & \mbox{if\ }i>j+1.
\end{array}\right.
\end{eqnarray}
All morphisms $[n] \to [0]$ factor through $\sigma^0_0$,
so [0] is terminal.

There is a bifunctor $+\colon \Delta\times\Delta \to \Delta$ defined by
\begin{eqnarray}
[m]+[n] & = & [m+n+1], \\
(f+g)(i) & = & \left\{
\begin{array}{ll}
f(i) & \mbox{if\ } 0 \leq i \leq m, \\
g(i-m-1)+m'+1 & \mbox{if\ } m < i \leq (m+n+1),
\end{array}\right.
\end{eqnarray}
where $f\colon [m] \to [m']$ and $g\colon [n] \to [n']$.
Sometimes, the simplicial category is defined to include the
empty set $[-1] = \emptyset$, which provides an initial object for the category.
This makes $\Delta$ a strict monoidal category as $\emptyset$
is a unit for the bifunctor: $\emptyset+[n] = [n] = [n]+\emptyset$
and $\id_\emptyset+f = f = f+\id_\emptyset$.
Further, $\Delta$ is then the free monoidal category on a monoid object
(the monoid object being [0], with product $\sigma^0_0\colon [0]+[0] \to [0]$).

There is a fully faithful functor from $\Delta$ to $\mathcat{Top}$,
which sends each object $[n]$ to an oriented $n$-simplex.
The face maps then embed an $(n-1)$-simplex in an $n$-simplex, and the degeneracy maps collapse an $(n+1)$-simplex to an $n$-simplex.
The bifunctor forms a simplex from the disjoint union of two simplicies by joining their vertices together in a way compatible with their orientations.

There is also a fully faithful functor from $\Delta$ to $\mathcat{Cat}$,
which sends each object $[n]$ to a pre-order $\mathcat{n+1}$.
The pre-order $\mathcat{n}$ is the category consisting of $n$ partially-ordered objects, with one morphism $a \to b$ if and only if $a \leq b$.
%%%%%
%%%%%
\end{document}
