\documentclass[12pt]{article}
\usepackage{pmmeta}
\pmcanonicalname{SupplementalAxiomsForAnAbelianCategory}
\pmcreated{2013-03-22 12:02:53}
\pmmodified{2013-03-22 12:02:53}
\pmowner{archibal}{4430}
\pmmodifier{archibal}{4430}
\pmtitle{supplemental axioms for an Abelian category}
\pmrecord{11}{31094}
\pmprivacy{1}
\pmauthor{archibal}{4430}
\pmtype{Axiom}
\pmcomment{trigger rebuild}
\pmclassification{msc}{18-00}
%\pmkeywords{Abelian category}
%\pmkeywords{monic}
%\pmkeywords{kernel}
%\pmkeywords{cokernel}
%\pmkeywords{coproduct}
%\pmkeywords{product}
\pmrelated{AbelianCategory}
\pmrelated{NonAbelianTheories}
\pmrelated{NonAbelianStructures}
\pmrelated{CommutativeVsNonCommutativeDynamicModelingDiagrams}
\pmrelated{GeneralizedToposesTopoiWithManyValuedLogicSubobjectClassifiers}
\pmrelated{CategoricalAlgebras}
\pmrelated{TopicEntryOnTheAlgebraicFoundationsOfMathematics}
\pmrelated{JordanBan}
\pmdefines{complete}
\pmdefines{cocomplete}

\usepackage{amssymb}
\usepackage{amsmath}
\usepackage{amsfonts}
\usepackage{graphicx}
%%%\usepackage{xypic}
\begin{document}
These are axioms introduced by Alexandre Grothendieck for an Abelian category.  The first two are satisfied by definition in an Abelian category, and others may or may not be.

\begin{itemize}
  \item[(Ab1)] Every morphism has a kernel and a cokernel.
  \item[(Ab2)] Every monic is the kernel of its cokernel.
  \item[(Ab3)] Coproducts exist.  (Coproducts are also called direct sums.)  If this axiom is satisfied the category is often just called cocomplete.
  \item[(Ab3*)] Products exist.  If this axiom is satisfied the category is often just called complete.
  \item[(Ab4)] Coproducts exist and the coproduct of monics is a monic.
  \item[(Ab4*)] Products exist and the product of epics is an epic.
  \item[(Ab5)] Coproducts exist and filtered colimits of exact sequences are exact.
  \item[(Ab5*)] Products exist and filtered inverse limits of exact sequences are exact.
\end{itemize}

Grothendieck introduced these in his homological algebra paper \emph{Sur quelques points d'alg\`ebre homologique} in the T\^ohoku Math Journal (number 2, volume 9, 1957).  They can also be found in Weibel's excellent book \emph{An introduction to homological algebra}, Cambridge Studies in Advanced Mathematics (Cambridge University Press, 1994).
%%%%%
%%%%%
%%%%%
\end{document}
