\documentclass[12pt]{article}
\usepackage{pmmeta}
\pmcanonicalname{HomotopyGroupoidsAndCrossedComplexesNoncommutativeStructuresInHigherDimensionalAlgebraHDA}
\pmcreated{2013-03-22 18:14:25}
\pmmodified{2013-03-22 18:14:25}
\pmowner{bci1}{20947}
\pmmodifier{bci1}{20947}
\pmtitle{homotopy groupoids and crossed complexes: non-commutative structures in higher dimensional algebra (HDA)}
\pmrecord{39}{40831}
\pmprivacy{1}
\pmauthor{bci1}{20947}
\pmtype{Feature}
\pmcomment{trigger rebuild}
\pmclassification{msc}{18D25}
\pmclassification{msc}{18D20}
\pmclassification{msc}{18D15}
\pmclassification{msc}{18C10}
\pmclassification{msc}{18A40}
\pmclassification{msc}{18A30}
\pmclassification{msc}{18A25}
\pmclassification{msc}{18-00}
\pmclassification{msc}{18A20}
\pmclassification{msc}{18A15}
\pmclassification{msc}{55U35}
\pmclassification{msc}{18D05}
\pmsynonym{Fields Institute Workshop 2004: categorical structures for Descent and Galois theory}{HomotopyGroupoidsAndCrossedComplexesNoncommutativeStructuresInHigherDimensionalAlgebraHDA}
\pmsynonym{Hopf algebras and semiabelian categories}{HomotopyGroupoidsAndCrossedComplexesNoncommutativeStructuresInHigherDimensionalAlgebraHDA}
\pmsynonym{higher homotopy groupoid}{HomotopyGroupoidsAndCrossedComplexesNoncommutativeStructuresInHigherDimensionalAlgebraHDA}
%\pmkeywords{cross complexes and homotopy groupoids in Algebraic Topology}
\pmrelated{BibliographyForTopology}
\pmrelated{CategoricalAlgebras}
\pmrelated{CategoricalOntologyABibliographyOfCategoryTheory}
\pmrelated{GeometricallyAndorAlgebraicallyThinSquares}
\pmrelated{HomotopyAdditionLemma}
\pmrelated{NaturalEquivalenceOfC_GAndC_MCategories}
\pmdefines{homotopy groupoid}
\pmdefines{cross complex}
\pmdefines{non-commutative cross complex}

\endmetadata

% this is the default PlanetMath preamble.  as your knowledge
% of TeX increases, you will probably want to edit this, but
% it should be fine as is for beginners.

% almost certainly you want these
\usepackage{amssymb}
\usepackage{amsmath}
\usepackage{amsfonts}

% used for TeXing text within eps files
%\usepackage{psfrag}
% need this for including graphics (\includegraphics)
%\usepackage{graphicx}
% for neatly defining theorems and propositions
%\usepackage{amsthm}
% making logically defined graphics
%%%\usepackage{xypic}

% there are many more packages, add them here as you need them

% define commands here

\begin{document}
\subsection{Fields Institute 2004 Workshop on Categorical Structures for Descent--Galois Theory, Hopf Algebras and Semiabelian Categories}

This is the topic of a series of papers that were published in 2004 on ``\PMlinkname{Categorical Structures}{CategoricalAlgebras} for Descent and Galois Theory, Hopf Algebras and Semiabelian Categories.'' that appeared
as part of the \emph{Proceedings of the Fields Institute Workshop on \PMlinkname{Categorical Structures}{Categoricalalgebras} for Descent and Galois Theory, Hopf Algebras and Semiabelian Categories}, \cite{FIC2004}.

\subsubsection{Homotopy Groupoids and Crossed Complexes provide tools for Solving Local-to-Global Problems: 
 Non-commutative Structures in Higher Dimensional Algebra(HDA)}

Among these remarkable mathematical contributions is an interesting paper on crossed complexes and homotopy groupoids as non-commutative tools for higher dimensional local-to-global problems. In this paper it was pointed out that
\emph{``the structures which enable the full use of crossed complexes as a tool in algebraic topology are substantial, intricate and interrelated''}. These applications of crossed complexes are also closely connected with the concept of
\PMlinkname{double groupoid}{HomotopyDoubleGroupoidOfAHausdorffSpace}.


\begin{thebibliography}{9}

\bibitem{FIC2004}
PFIWCS-2004. \emph{Proceedings of the Fields Institute Workshop on \PMlinkname{Categorical Structures}{Categoricalalgebras} for Descent and Galois Theory, Hopf Algebras and Semiabelian Categories}., September 23-28, 2004, published in the \emph{Fields Institute Communications \textbf{43}, (2004)}.

\bibitem{RBetal2k4}
R. Brown et al. ``Crossed complexes and homotopy groupoids as non-commutative tools for higher dimensional local-to-global problems'', in \emph{Fields Institute Communications \textbf{43}:101-130 (2004)},
(PDF and ps documents at arXiv/ math.AT/0212274).

\end{thebibliography} 
%%%%%
%%%%%
\end{document}
