\documentclass[12pt]{article}
\usepackage{pmmeta}
\pmcanonicalname{Endomorphism}
\pmcreated{2013-03-22 15:33:34}
\pmmodified{2013-03-22 15:33:34}
\pmowner{porton}{9363}
\pmmodifier{porton}{9363}
\pmtitle{endomorphism}
\pmrecord{11}{37462}
\pmprivacy{1}
\pmauthor{porton}{9363}
\pmtype{Definition}
\pmcomment{trigger rebuild}
\pmclassification{msc}{18A20}
\pmclassification{msc}{18A05}
%\pmkeywords{morphism}
%\pmkeywords{homomorphism}
%\pmkeywords{types of morphisms}
\pmrelated{TypesOfHomomorphisms}
\pmrelated{Morphism}
\pmrelated{Category}
\pmrelated{CategoryTheory}
\pmrelated{Automorphism}
\pmrelated{GroupHomomorphism}
\pmdefines{endomorphism}
\pmdefines{automorphism}

% this is the default PlanetMath preamble.  as your knowledge
% of TeX increases, you will probably want to edit this, but
% it should be fine as is for beginners.

% almost certainly you want these
\usepackage{amssymb}
\usepackage{amsmath}
\usepackage{amsfonts}

% used for TeXing text within eps files
%\usepackage{psfrag}
% need this for including graphics (\includegraphics)
%\usepackage{graphicx}
% for neatly defining theorems and propositions
%\usepackage{amsthm}
% making logically defined graphics
%%%\usepackage{xypic}

% there are many more packages, add them here as you need them

% define commands here
\begin{document}
\emph{Endomorphism} is such morphism (morphism is another \PMlinkescapetext{term} for homomorphism) whose source and destination are the same object.

That is a morphism $f$ is \emph{endomorphism}, when $\mathrm{Src}f=\mathrm{Dst}f=A$ where $A$ is some object (e.g. $A$ may be an abstract algebra). Then one can say, the object of endomorphism $f$ is $A$.

In the most general case endomorphisms are encountered in category theory. As a special case of this endomorphisms are also encountered in abstract algebra.

A morphism which is both an endomorphism and an isomorphism is called \emph{automorphism}.

The sets of endomorphisms and automorphisms for an object $A$ of a category are often denoted correspondingly as $\mathrm{End}(A)$ and $\mathrm{Aut}(A)$ or sometimes as $\mathrm{end}(A)$ and $\mathrm{aut}(A)$.

\emph{Endomorphisms} also can be considered as objects of \PMlinkname{category of intermorphisms}{PseudomorphismsAndIntermorphisms} and (if the set of morphisms of our category is preordered) also of \PMlinkname{category of pseudomorphisms}{PseudomorphismsAndIntermorphisms}.
%%%%%
%%%%%
\end{document}
