\documentclass[12pt]{article}
\usepackage{pmmeta}
\pmcanonicalname{PowerObject}
\pmcreated{2013-03-22 15:41:16}
\pmmodified{2013-03-22 15:41:16}
\pmowner{rmilson}{146}
\pmmodifier{rmilson}{146}
\pmtitle{power object}
\pmrecord{29}{37630}
\pmprivacy{1}
\pmauthor{rmilson}{146}
\pmtype{Definition}
\pmcomment{trigger rebuild}
\pmclassification{msc}{18A20}
%\pmkeywords{PowerSet}
%\pmkeywords{Subobject}
%\pmkeywords{SubobjectClassifier}
%\pmkeywords{topoi}
%\pmkeywords{topos}
\pmrelated{PowerSet}
\pmrelated{Subobject}
\pmrelated{SubobjectClassifier}
\pmrelated{Topos}

\usepackage{amsmath}
\usepackage{amsfonts}
\usepackage{amssymb}
\usepackage{graphicx}

\newcommand{\cC}{{\mathcal{C}}}
\newcommand{\Set}{\mathit{Set}}
\newcommand{\Map}{\mathrm{Map}}
\newcommand{\Hom}{\mathrm{Hom}}
\newcommand{\Sub}{\mathrm{Sub}}
\newcommand{\hf}{\hat{f}}
\newcommand{\hm}{\hat{m}}
\begin{document}
\section{Preliminary remarks.}
Let $A$ be a set. The \emph{power set} $P(A)$ is the set of all subsets of
$A$; as such, this idea belongs to the realm of set theory.  However,  properly formulated, the power set notion can be extended to
categories more general than $\Set$.

Every subset $S\subset A$ determines, and is in turn determined by the \PMlinkname{characteristic map}{CharacteristicFunction}\, 
$\chi_S:A \to \{0,\,1\}$\,  defined by
\[ \chi_S(a)=
\begin{cases}
  1 & a\in S,\\
  0 & a\notin S;
\end{cases}
\quad a\in A.
\]
In other words, elements of the power set $P(A)$ correspond in a
one-to-one fashion to maps from $A$ to $\{0,1\}$.  Suppose now that
$A$ is an object in a category $\cC$; this will be specified as $ A \in Ob \cC$,
with $Ob \cC$ being the class of objects of category $\cC$.  

It is tempting to generalize the power set construction by considering general homomorphisms into
some special object $\Omega \in Ob \cC$ that can somehow serve as a
category-theoretical analogue of the two element set.  This, however,
will give us $P(A)$ as a Hom-set, whereas we want $P(A)$ to be another
object of $\cC$.  What to do?

A guiding insight\cite{BW} is to regard morphisms with codomain $A$ as
``generalized elements'' (gelements) of $A$.  Let's return, for the
moment, to the category of sets. Henceforth we use $\Map(X,Y)$ to
denote the set of maps from $X$ to $Y$, and use the symbol $2$ to
denote the two element set $\{0,1\}$.  Recall that the set
$\Map(A,\Map(X,Y))$ is naturally isomorphic to the set $\Map(A\times
X,Y)$.  In other words, the gelements of $\Map(X,Y)$ are those special
gelements of $Y$ whose domains are products $A\times X$.  It follows
that a gelement of $P(A)$, which is to say a map from $X$ to
$\Map(A,2)$, is naturally the same thing as a map from $X\times A$ to
$2$. The latter corresponds to a subset of $X\times A$.  We thereby
arrive at the key insight:
\begin{quote}
  Subsets of $X\times A$ are naturally isomorphic to  maps from $X$ to
  $P(A)$.  
\end{quote}
The rest, as they say, is details.

\section{Definition of power object.}
Let $\cC$ be a category with finite limits (products and pullbacks
exist).  For a morphism $m$ of the category, let $[m]$ denote the
equivalence class consisting of all morphisms $mf$ with $f$ an
isomorphism.  Recall that a subobject of an object $X \in Ob \cC$ is an
equivalence class $[m]$ where $m$ is a monomorphism.

Let $\Sub(-)$ denote the contravariant functor from $\cC$ to $\Set$
that sends an object $X\in Ob \cC$ to $\Sub(X)$, the set of all
subobjects of $X$.  For a morphism $f:X\to Y$ we define $\Sub(f):
\Sub(Y)\to \Sub(X)$ to be the set map that acts by sending a
monomorphism $m:S\to Y$ along $f$ to a monomorphism
$\hm:S\times_f X\to X$ obtained as the pullback of $m$ along $f$ (see the
figure below).
Note: monomorphisms are pullback stable.
\begin{figure}[h]
  \centering
  \includegraphics[width=4cm]{powerobject-fig1}
  \caption{functoriality of $\Sub$}
  \label{fig:1}
\end{figure}
%\usepackage{pstricks}
%\usepackage{pst-node}

%\begin{document}
%\pagestyle{empty}
%\noindent
%$
%\psmatrix[colsep=2cm,rowsep=2cm]
%  S\times_f X  & S \\
%  X & Y
%\endpsmatrix
%psset{nodesep=5pt,arrows=->}
% \ncline{1,1}{1,2}
% \ncline{1,1}{2,1}\tlput{\hat{m}}
% \ncline{1,2}{2,2}\trput{m}
% \ncline{2,1}{2,2}\taput{f}
%$
%\end{document}


Next, let us fix an object $A\in Ob \cC$ and define what it means to be a
power object of $A$.  Let $F_A= \Sub(-\times A)$ be the functor from
$\cC$ to $\Set$ that sends $X\in \cC$ to $\Sub(X\times A)$.  To
establish the functoriality of $F_A$ we note that it is the
composition of the functor $-\times A: \cC \to \cC$ and the functor
$\Sub:\cC \to \Set$.  We say that an object $B\in \cC$ is a power
object of $A$ if $B$ represents the functor $F_A$.  If such a $B$
exists, then by the Yoneda lemma, it is unique up to isomorphism.  In
other words, $B$ is a power object of $A$ if there every monomorphism
$m:S\to X\times A$ corresponds in a natural fashion to a morphism
$\hm \in \Hom_\cC(X,B)$, with equivalent monomorphisms corresponding to
the same $\hm$.

\textbf{Note} that categories $\cC_L$ that have finite limits and are endowed \emph{with the power objects} defined above satisfy the two axioms of topoi and are therefore called \emph{topoi} (or toposes).


\begin{thebibliography}{99}
\bibitem{BW} Michael Barr and Charles Wells, ``Toposes, Triples, and
  Theories'', \PMlinkexternal{\texttt{http://www.cwru.edu/artsci/math/wells/pub/ttt.html}}{http://www.cwru.edu/artsci/math/wells/pub/ttt.html}
\end{thebibliography}
%%%%%
%%%%%
\end{document}
