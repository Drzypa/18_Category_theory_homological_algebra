\documentclass[12pt]{article}
\usepackage{pmmeta}
\pmcanonicalname{QuantumRiemannianGeometry}
\pmcreated{2013-03-22 18:17:17}
\pmmodified{2013-03-22 18:17:17}
\pmowner{bci1}{20947}
\pmmodifier{bci1}{20947}
\pmtitle{quantum Riemannian geometry}
\pmrecord{30}{40898}
\pmprivacy{1}
\pmauthor{bci1}{20947}
\pmtype{Topic}
\pmcomment{trigger rebuild}
\pmclassification{msc}{18D25}
\pmclassification{msc}{18-00}
\pmclassification{msc}{55U99}
\pmclassification{msc}{81-00}
\pmclassification{msc}{81P05}
\pmclassification{msc}{81Q05}
\pmsynonym{non-commutative geometry}{QuantumRiemannianGeometry}
\pmsynonym{non-Abelian geometry}{QuantumRiemannianGeometry}
\pmsynonym{Non-Abelian Topology}{QuantumRiemannianGeometry}
%\pmkeywords{quantum gravity theories}
%\pmkeywords{non-commutative geometry}
%\pmkeywords{non-Abelian geometry}
%\pmkeywords{non-Abelian topology}
\pmrelated{SpinNetworksAndSpinFoams}
\pmrelated{QuantumGeometry}
\pmrelated{NoncommutativeGeometry}
\pmrelated{QuantumGravityTheories}
\pmrelated{RiemannianMetric}
\pmrelated{EinsteinFieldEquations}

% this is the default PlanetMath preamble.  as your knowledge
% of TeX increases, you will probably want to edit this, but
% it should be fine as is for beginners.

% almost certainly you want these
\usepackage{amssymb}
\usepackage{amsmath}
\usepackage{amsfonts}

% used for TeXing text within eps files
%\usepackage{psfrag}
% need this for including graphics (\includegraphics)
%\usepackage{graphicx}
% for neatly defining theorems and propositions
%\usepackage{amsthm}
% making logically defined graphics
%%%\usepackage{xypic}

% there are many more packages, add them here as you need them

% define commands here
\usepackage{amsmath, amssymb, amsfonts, amsthm, amscd, latexsym}
%%\usepackage{xypic}
\usepackage[mathscr]{eucal}

\setlength{\textwidth}{6.5in}
%\setlength{\textwidth}{16cm}
\setlength{\textheight}{9.0in}
%\setlength{\textheight}{24cm}

\hoffset=-.75in     %%ps format
%\hoffset=-1.0in     %%hp format
\voffset=-.4in

\theoremstyle{plain}
\newtheorem{lemma}{Lemma}[section]
\newtheorem{proposition}{Proposition}[section]
\newtheorem{theorem}{Theorem}[section]
\newtheorem{corollary}{Corollary}[section]

\theoremstyle{definition}
\newtheorem{definition}{Definition}[section]
\newtheorem{example}{Example}[section]
%\theoremstyle{remark}
\newtheorem{remark}{Remark}[section]
\newtheorem*{notation}{Notation}
\newtheorem*{claim}{Claim}

\renewcommand{\thefootnote}{\ensuremath{\fnsymbol{footnote%%@
}}}
\numberwithin{equation}{section}

\newcommand{\Ad}{{\rm Ad}}
\newcommand{\Aut}{{\rm Aut}}
\newcommand{\Cl}{{\rm Cl}}
\newcommand{\Co}{{\rm Co}}
\newcommand{\DES}{{\rm DES}}
\newcommand{\Diff}{{\rm Diff}}
\newcommand{\Dom}{{\rm Dom}}
\newcommand{\Hol}{{\rm Hol}}
\newcommand{\Mon}{{\rm Mon}}
\newcommand{\Hom}{{\rm Hom}}
\newcommand{\Ker}{{\rm Ker}}
\newcommand{\Ind}{{\rm Ind}}
\newcommand{\IM}{{\rm Im}}
\newcommand{\Is}{{\rm Is}}
\newcommand{\ID}{{\rm id}}
\newcommand{\GL}{{\rm GL}}
\newcommand{\Iso}{{\rm Iso}}
\newcommand{\Sem}{{\rm Sem}}
\newcommand{\St}{{\rm St}}
\newcommand{\Sym}{{\rm Sym}}
\newcommand{\SU}{{\rm SU}}
\newcommand{\Tor}{{\rm Tor}}
\newcommand{\U}{{\rm U}}

\newcommand{\A}{\mathcal A}
\newcommand{\Ce}{\mathcal C}
\newcommand{\D}{\mathcal D}
\newcommand{\E}{\mathcal E}
\newcommand{\F}{\mathcal F}
\newcommand{\G}{\mathcal G}
\newcommand{\Q}{\mathcal Q}
\newcommand{\R}{\mathcal R}
\newcommand{\cS}{\mathcal S}
\newcommand{\cU}{\mathcal U}
\newcommand{\W}{\mathcal W}

\newcommand{\bA}{\mathbb{A}}
\newcommand{\bB}{\mathbb{B}}
\newcommand{\bC}{\mathbb{C}}
\newcommand{\bD}{\mathbb{D}}
\newcommand{\bE}{\mathbb{E}}
\newcommand{\bF}{\mathbb{F}}
\newcommand{\bG}{\mathbb{G}}
\newcommand{\bK}{\mathbb{K}}
\newcommand{\bM}{\mathbb{M}}
\newcommand{\bN}{\mathbb{N}}
\newcommand{\bO}{\mathbb{O}}
\newcommand{\bP}{\mathbb{P}}
\newcommand{\bR}{\mathbb{R}}
\newcommand{\bV}{\mathbb{V}}
\newcommand{\bZ}{\mathbb{Z}}

\newcommand{\bfE}{\mathbf{E}}
\newcommand{\bfX}{\mathbf{X}}
\newcommand{\bfY}{\mathbf{Y}}
\newcommand{\bfZ}{\mathbf{Z}}

\renewcommand{\O}{\Omega}
\renewcommand{\o}{\omega}
\newcommand{\vp}{\varphi}
\newcommand{\vep}{\varepsilon}

\newcommand{\diag}{{\rm diag}}
\newcommand{\grp}{{\mathbb G}}
\newcommand{\dgrp}{{\mathbb D}}
\newcommand{\desp}{{\mathbb D^{\rm{es}}}}
\newcommand{\Geod}{{\rm Geod}}
\newcommand{\geod}{{\rm geod}}
\newcommand{\hgr}{{\mathbb H}}
\newcommand{\mgr}{{\mathbb M}}
\newcommand{\ob}{{\rm Ob}}
\newcommand{\obg}{{\rm Ob(\mathbb G)}}
\newcommand{\obgp}{{\rm Ob(\mathbb G')}}
\newcommand{\obh}{{\rm Ob(\mathbb H)}}
\newcommand{\Osmooth}{{\Omega^{\infty}(X,*)}}
\newcommand{\ghomotop}{{\rho_2^{\square}}}
\newcommand{\gcalp}{{\mathbb G(\mathcal P)}}

\newcommand{\rf}{{R_{\mathcal F}}}
\newcommand{\glob}{{\rm glob}}
\newcommand{\loc}{{\rm loc}}
\newcommand{\TOP}{{\rm TOP}}

\newcommand{\wti}{\widetilde}
\newcommand{\what}{\widehat}

\renewcommand{\a}{\alpha}
\newcommand{\be}{\beta}
\newcommand{\ga}{\gamma}
\newcommand{\Ga}{\Gamma}
\newcommand{\de}{\delta}
\newcommand{\del}{\partial}
\newcommand{\ka}{\kappa}
\newcommand{\si}{\sigma}
\newcommand{\ta}{\tau}
\newcommand{\med}{\medbreak}
\newcommand{\medn}{\medbreak \noindent}
\newcommand{\bign}{\bigbreak \noindent}
\newcommand{\lra}{{\longrightarrow}}
\newcommand{\ra}{{\rightarrow}}
\newcommand{\rat}{{\rightarrowtail}}
\newcommand{\oset}[1]{\overset {#1}{\ra}}
\newcommand{\osetl}[1]{\overset {#1}{\lra}}
\newcommand{\hr}{{\hookrightarrow}}
\begin{document}
\section{Quantum Riemannian geometry}

An interesting, but perhaps limiting approach to Quantum Gravity (QG), involves defining a \emph{quantum Riemannian geometry} \cite{AL2k5} in place of the classical Riemannian manifold that is employed in the well-known, Einstein's classical approach to General Relativity (GR). Whereas a classical Riemannian manifold has a
metric defined by a special, \emph{Riemannian tensor}, the \emph{quantum Riemannian geometry} may be defined in different theoretical approaches to QG by either quantum loops (or perhaps `strings'), or  \emph{spin networks and spin foams} (in locally covariant GR quantized space-times). The latter two concepts are related to the `standard' quantum spin observables and thus have the advantage of precise mathematical definitions. As spin foams can be defined as functors of spin network categories, \emph{quantized space-times (QST)s} can be represented by, or defined in terms of, \emph{natural transformations of `spin foam' functors}.  The latter definition is not however the usual one adopted for quantum Riemannian geometry, and other (for example, noncommutative geometry) approaches attempt to define a QST metric not by a \emph{Riemannian tensor} --as in the classical GR case-- but in relation to a generalized, quantum `Dirac' operator in a spectral triplet. 

\textbf{Remarks.}
Other approaches to Quantum Gravity include: Loop Quantum Gravity (LQG), AQFT approaches,
Topological Quantum Field Theory (TQFT)/ Homotopy Quantum Field Theories (HQFT; Tureaev and Porter, 2005),
Quantum Theories on a Lattice (QTL), string theories and spin network models. 

\begin{definition}
\emph{Quantum Geometry} is defined as a \emph{field of Mathematical or Theoretical Physics based on geometrical and Algebraic Topology approaches to Quantum Gravity}- one such approach is based on Noncommutative Geometry and SUSY (the `Standard' Model in current Physics).
\end{definition}

\emph{A Result for Quantum Spin Foam Representations of Quantum Space-Times (QST)s:}
There exists an $n$-connected CW model $(Z,QSF)$ for the pair $(QST,QSF)$ such that: 
$f_*: \pi_i(Z) \rightarrow \pi_i (QST)$, is an isomorphism for $i>n$, and it is a monomorphism for $i=n$. 
The $n$-connected CW model is unique up to homotopy equivalence. (The $CW$ complex, $Z$, considered here is a homotopic `hybrid' between QSF and QST).

\begin{thebibliography}{9}
\bibitem{AC94}
A. Connes. 1994. \emph{Noncommutative Geometry}. Academic Press: New York and London.  
\bibitem{AL2k5}
Abhay Ashtekar and Jerzy Lewandowski.2005. Quantum Geometry and Its Applications. 
\PMlinkexternal{PDF file download}{http://cgpg.gravity.psu.edu/people/Ashtekar/articles/qgfinal.pdf}.
\end{thebibliography}
%%%%%
%%%%%
\end{document}
