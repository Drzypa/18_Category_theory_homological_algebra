\documentclass[12pt]{article}
\usepackage{pmmeta}
\pmcanonicalname{GrothendieckCategoryLemma}
\pmcreated{2013-03-22 18:27:55}
\pmmodified{2013-03-22 18:27:55}
\pmowner{bci1}{20947}
\pmmodifier{bci1}{20947}
\pmtitle{Grothendieck category lemma}
\pmrecord{26}{41134}
\pmprivacy{1}
\pmauthor{bci1}{20947}
\pmtype{Corollary}
\pmcomment{trigger rebuild}
\pmclassification{msc}{18-00}
\pmclassification{msc}{18E05}
%\pmkeywords{Grothendieck category Lemma}
\pmrelated{GrothendieckCategory}
\pmdefines{proper generator}

\endmetadata

% this is the default PlanetMath preamble.  as your 

% almost certainly you want these
\usepackage{amssymb}
\usepackage{amsmath}
\usepackage{amsfonts}
% there are many more packages, add them here as you need them

% define commands here
\usepackage{amsmath, amssymb, amsfonts, amsthm, amscd, latexsym}
%%\usepackage{xypic}
\usepackage[mathscr]{eucal}
\theoremstyle{plain}
\newtheorem{lemma}{Lemma}[section]
\newtheorem{proposition}{Proposition}[section]
\newtheorem{theorem}{Theorem}[section]
\newtheorem{corollary}{Corollary}[section]
\theoremstyle{definition}
\newtheorem{definition}{Definition}[section]
\newtheorem{example}{Example}[section]
%\theoremstyle{remark}
\newtheorem{remark}{Remark}[section]
\newtheorem*{notation}{Notation}
\newtheorem*{claim}{Claim}
\renewcommand{\thefootnote}{\ensuremath{\fnsymbol{footnote%%@
}}}
\numberwithin{equation}{section}
\newcommand{\Ad}{{\rm Ad}}
\newcommand{\Aut}{{\rm Aut}}
\newcommand{\Cl}{{\rm Cl}}
\newcommand{\Co}{{\rm Co}}
\newcommand{\DES}{{\rm DES}}
\newcommand{\Diff}{{\rm Diff}}
\newcommand{\Dom}{{\rm Dom}}
\newcommand{\Hol}{{\rm Hol}}
\newcommand{\Mon}{{\rm Mon}}
\newcommand{\Hom}{{\rm Hom}}
\newcommand{\Ker}{{\rm Ker}}
\newcommand{\Ind}{{\rm Ind}}
\newcommand{\IM}{{\rm Im}}
\newcommand{\Is}{{\rm Is}}
\newcommand{\ID}{{\rm id}}
\newcommand{\GL}{{\rm GL}}
\newcommand{\Iso}{{\rm Iso}}
\newcommand{\Sem}{{\rm Sem}}
\newcommand{\St}{{\rm St}}
\newcommand{\Sym}{{\rm Sym}}
\newcommand{\SU}{{\rm SU}}
\newcommand{\Tor}{{\rm Tor}}
\newcommand{\U}{{\rm U}}
\newcommand{\A}{\mathcal A}
\newcommand{\Ce}{\mathcal C}
\newcommand{\D}{\mathcal D}
\newcommand{\E}{\mathcal E}
\newcommand{\F}{\mathcal F}
\newcommand{\G}{\mathcal G}
\newcommand{\Q}{\mathcal Q}
\newcommand{\R}{\mathcal R}
\newcommand{\cS}{\mathcal S}
\newcommand{\cU}{\mathcal U}
\newcommand{\W}{\mathcal W}
\newcommand{\bA}{\mathbb{A}}
\newcommand{\bB}{\mathbb{B}}
\newcommand{\bC}{\mathbb{C}}
\newcommand{\bD}{\mathbb{D}}
\newcommand{\bE}{\mathbb{E}}
\newcommand{\bF}{\mathbb{F}}
\newcommand{\bG}{\mathbb{G}}
\newcommand{\bK}{\mathbb{K}}
\newcommand{\bM}{\mathbb{M}}
\newcommand{\bN}{\mathbb{N}}
\newcommand{\bO}{\mathbb{O}}
\newcommand{\bP}{\mathbb{P}}
\newcommand{\bR}{\mathbb{R}}
\newcommand{\bV}{\mathbb{V}}
\newcommand{\bZ}{\mathbb{Z}}
\newcommand{\bfE}{\mathbf{E}}
\newcommand{\bfX}{\mathbf{X}}
\newcommand{\bfY}{\mathbf{Y}}
\newcommand{\bfZ}{\mathbf{Z}}
\renewcommand{\O}{\Omega}
\renewcommand{\o}{\omega}
\newcommand{\vp}{\varphi}
\newcommand{\vep}{\varepsilon}
\newcommand{\diag}{{\rm diag}}
\newcommand{\grp}{{\mathbb G}}
\newcommand{\dgrp}{{\mathbb D}}
\newcommand{\desp}{{\mathbb D^{\rm{es}}}}
\newcommand{\Geod}{{\rm Geod}}
\newcommand{\geod}{{\rm geod}}
\newcommand{\hgr}{{\mathbb H}}
\newcommand{\mgr}{{\mathbb M}}
\newcommand{\ob}{{\rm Ob}}
\newcommand{\obg}{{\rm Ob(\mathbb G)}}
\newcommand{\obgp}{{\rm Ob(\mathbb G')}}
\newcommand{\obh}{{\rm Ob(\mathbb H)}}
\newcommand{\Osmooth}{{\Omega^{\infty}(X,*)}}
\newcommand{\ghomotop}{{\rho_2^{\square}}}
\newcommand{\gcalp}{{\mathbb G(\mathcal P)}}
\newcommand{\rf}{{R_{\mathcal F}}}
\newcommand{\glob}{{\rm glob}}
\newcommand{\loc}{{\rm loc}}
\newcommand{\TOP}{{\rm TOP}}
\newcommand{\wti}{\widetilde}
\newcommand{\what}{\widehat}
\renewcommand{\a}{\alpha}
\newcommand{\be}{\beta}
\newcommand{\ga}{\gamma}
\newcommand{\Ga}{\Gamma}
\newcommand{\de}{\delta}
\newcommand{\del}{\partial}
\newcommand{\ka}{\kappa}
\newcommand{\si}{\sigma}
\newcommand{\ta}{\tau}
\newcommand{\lra}{{\longrightarrow}}
\newcommand{\ra}{{\rightarrow}}
\newcommand{\rat}{{\rightarrowtail}}
\newcommand{\oset}[1]{\overset {#1}{\ra}}
\newcommand{\osetl}[1]{\overset {#1}{\lra}}
\newcommand{\hr}{{\hookrightarrow}}
\begin{document}
\subsection{Introduction: proper generator}
\begin{definition}

 Let us recall that a \emph{generator} of a Grothendieck category $\mathcal{G}$ is called \emph{proper} if $U$  has the property that a monomorphism  $i: U' \to U$ induces an isomorphism 
$$Hom_{\mathcal{G}}(U,U) \cong Hom_{\mathcal{G}}(U',U)$$ if and only if $i$ is an isomorphism (viz. p. 251 in ref.
\cite{NP65}).
\end{definition}


\subsection{Grothendieck category lemma}

\begin{lemma}
 Any Grothendieck category $\mathcal{G}$ has a proper generator.
\end{lemma}

\begin{thebibliography}{9}

\bibitem[AG4]{sga}
Alexander Grothendieck et al. \emph{S\'eminaires en G\'eometrie Alg\`ebrique- 4}, Tome 1, Expos\'e 1
(or the Appendix to Expos\'ee 1, by `N. Bourbaki' for more detail and a large number of results.),
AG4 is \PMlinkexternal{freely available}{http://modular.fas.harvard.edu/sga/sga/pdf/index.html} in French;
also available here is an extensive
\PMlinkexternal{Abstract in English}{http://planetmath.org/?op=getobj&from=books&id=158}.

\bibitem{NP65}
Nicolae Popescu. {\em Abelian Categories with Applications to Rings and Modules.},
Academic Press: New York and London, 1973 and 1976 edns., ({\em English translation by I. C. Baianu}.)

\bibitem{LS94}
Leila Schneps. 1994.
\PMlinkexternal{The Grothendieck Theory of Dessins d'Enfants}{http://planetmath.org/?op=getobj&from=books&id=163}.
(London Mathematical Society Lecture Note Series), Cambridge University Press, 376 pp.

\bibitem{DHSL2k}
David Harbater and Leila Schneps. 2000.
\PMlinkexternal{Fundamental groups of moduli and the Grothendieck-Teichm\"uller group}{http://www.ams.org/tran/2000-352-07/S0002-9947-00-02347-3/home.html}, \emph{Trans. Amer. Math. Soc}. 352 (2000), 3117-3148.
MSC: Primary 11R32, 14E20, 14H10; Secondary 20F29, 20F34, 32G15.

\end{thebibliography}

%%%%%
%%%%%
\end{document}
