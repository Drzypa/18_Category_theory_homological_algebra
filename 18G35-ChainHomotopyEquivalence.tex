\documentclass[12pt]{article}
\usepackage{pmmeta}
\pmcanonicalname{ChainHomotopyEquivalence}
\pmcreated{2013-03-22 14:51:11}
\pmmodified{2013-03-22 14:51:11}
\pmowner{CWoo}{3771}
\pmmodifier{CWoo}{3771}
\pmtitle{chain homotopy equivalence}
\pmrecord{6}{36525}
\pmprivacy{1}
\pmauthor{CWoo}{3771}
\pmtype{Definition}
\pmcomment{trigger rebuild}
\pmclassification{msc}{18G35}
\pmrelated{HomotopyEquivalence}
\pmdefines{chain homotopic equivalent}

% this is the default PlanetMath preamble.  as your knowledge
% of TeX increases, you will probably want to edit this, but
% it should be fine as is for beginners.

% almost certainly you want these
\usepackage{amssymb,amscd}
\usepackage{amsmath}
\usepackage{amsfonts}

% used for TeXing text within eps files
%\usepackage{psfrag}
% need this for including graphics (\includegraphics)
%\usepackage{graphicx}
% for neatly defining theorems and propositions
%\usepackage{amsthm}
% making logically defined graphics
%%%\usepackage{xypic}

% there are many more packages, add them here as you need them

% define commands here
\begin{document}
\PMlinkescapeword{equivalent}

Let $C$ and $D$ be two objects from the abelian category of chain complexes.  A morphism (or chain map) $f\colon C\to D$ is said to be a \emph{chain homotopy equivalence} if there is a morphism $g\colon D\to C$ such that

\begin{enumerate}
\item there is a chain homotopy between $fg$ and $1\colon D\to D$; and 
\item there is a chain homotopy between $gf$ and $1\colon C\to C$.
\end{enumerate}

If a chain homotopy equivalence from a chain complex $C$ to $D$ exists, then $C$ is said to be \emph{chain homotopy equivalent} to $D$.  Chain homotopy equivalence is an equivalence relation among chain complexes.
%%%%%
%%%%%
\end{document}
