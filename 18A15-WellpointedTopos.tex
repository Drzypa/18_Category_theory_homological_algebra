\documentclass[12pt]{article}
\usepackage{pmmeta}
\pmcanonicalname{WellpointedTopos}
\pmcreated{2013-03-22 16:36:54}
\pmmodified{2013-03-22 16:36:54}
\pmowner{mps}{409}
\pmmodifier{mps}{409}
\pmtitle{well-pointed topos}
\pmrecord{4}{38812}
\pmprivacy{1}
\pmauthor{mps}{409}
\pmtype{Definition}
\pmcomment{trigger rebuild}
\pmclassification{msc}{18A15}
\pmsynonym{well-pointed topoi}{WellpointedTopos}
\pmsynonym{well-pointed}{WellpointedTopos}
\pmdefines{complemented}
\pmdefines{supports split}

% this is the default PlanetMath preamble.  as your knowledge
% of TeX increases, you will probably want to edit this, but
% it should be fine as is for beginners.

% almost certainly you want these
\usepackage{amssymb}
\usepackage{amsmath}
\usepackage{amsfonts}

% used for TeXing text within eps files
%\usepackage{psfrag}
% need this for including graphics (\includegraphics)
%\usepackage{graphicx}
% for neatly defining theorems and propositions
%\usepackage{amsthm}
% making logically defined graphics
%%\usepackage{xypic}

% there are many more packages, add them here as you need them

% define commands here

\begin{document}
\PMlinkescapeword{satsifies}

The concept of well-pointed topoi was introduced by Freyd in~\cite{Fr}.  A topos is \emph{well-pointed} if it satisfies either the following equivalent conditions:
\begin{enumerate}
\item
The terminal object $1$ distinguishes morphisms in the sense that if the diagram
\[\xymatrix{
1\ar[r]^x & A\ar@<1ex>[r]^f\ar@<-1ex>[r]_g & B
}\]
commutes for every morphism $x\colon 1\to A$, then in fact 
\[\xymatrix{
A\ar@<1ex>[r]^f\ar@<-1ex>[r]_g & B
}\]
commutes, that is, $f=g$.  Moreover, $1$ is not isomorphic to the initial object.

\item
The topos $\mathcal{T}$ is complemented and supports split, and the truth object $\Omega$ of $\mathcal{T}$ has exactly two elements, $\top\colon 1\to\Omega$, and $\bot\colon 1\to\Omega$.  To say that $\mathcal{T}$ is \emph{complemented} means that if $m\colon X\to Y$ is a monomorphism, then there exists a monomorphism $m'\colon X'\to Y$ such that $m\sqcup m'\colon X\sqcup X'\to Y$ is an isomorphism.  To say that $\mathcal{T}$ \emph{supports split} means that every subobject of $1$ is projective.
\end{enumerate}

Every well-pointed topos is a Boolean topos.

\begin{thebibliography}{99}
\bibitem{Fr}
P.~Freyd.  Aspects of topoi.  {\it Bull. Austral. Math. Soc.} {\bf 7} (1972), 1--76.

\bibitem{Jo}
P.~T.~Johnstone.  {\it Topos theory}.  Academic Press, 1977.

\bibitem{MaMo}
S.~Mac~Lane and I.~Moerdijk. {\it Sheaves and Geometry in Logic: A First Introduction to Topos Theory}, Springer-Verlag, 1992.
\end{thebibliography}


%%%%%
%%%%%
\end{document}
