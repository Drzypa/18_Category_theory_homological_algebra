\documentclass[12pt]{article}
\usepackage{pmmeta}
\pmcanonicalname{ZeroObject}
\pmcreated{2013-03-22 12:36:28}
\pmmodified{2013-03-22 12:36:28}
\pmowner{djao}{24}
\pmmodifier{djao}{24}
\pmtitle{zero object}
\pmrecord{5}{32864}
\pmprivacy{1}
\pmauthor{djao}{24}
\pmtype{Definition}
\pmcomment{trigger rebuild}
\pmclassification{msc}{18A05}
\pmdefines{initial object}
\pmdefines{terminal object}

% this is the default PlanetMath preamble.  as your knowledge
% of TeX increases, you will probably want to edit this, but
% it should be fine as is for beginners.

% almost certainly you want these
\usepackage{amssymb}
\usepackage{amsmath}
\usepackage{amsfonts}

% used for TeXing text within eps files
%\usepackage{psfrag}
% need this for including graphics (\includegraphics)
%\usepackage{graphicx}
% for neatly defining theorems and propositions
%\usepackage{amsthm}
% making logically defined graphics
%%%\usepackage{xypic} 

% there are many more packages, add them here as you need them

% define commands here
\begin{document}
An {\em initial object} in a category $\mathcal{C}$ is an object $A$ in $\mathcal{C}$ such that, for every object $X$ in $\mathcal{C}$, there is exactly one morphism $A \longrightarrow X$.

A {\em terminal object} in a category $\mathcal{C}$ is an object $B$ in $\mathcal{C}$ such that, for every object $X$ in $\mathcal{C}$, there is exactly one morphism $X \longrightarrow B$.

A {\em zero object} in a category $\mathcal{C}$ is an object $0$ that is both an initial object and a terminal object.

All initial objects (respectively, terminal objects, and zero objects), if they exist, are isomorphic in $\mathcal{C}$.
%%%%%
%%%%%
\end{document}
