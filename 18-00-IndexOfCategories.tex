\documentclass[12pt]{article}
\usepackage{pmmeta}
\pmcanonicalname{IndexOfCategories}
\pmcreated{2013-03-22 18:30:41}
\pmmodified{2013-03-22 18:30:41}
\pmowner{bci1}{20947}
\pmmodifier{bci1}{20947}
\pmtitle{index of categories}
\pmrecord{63}{41196}
\pmprivacy{1}
\pmauthor{bci1}{20947}
\pmtype{Topic}
\pmcomment{trigger rebuild}
\pmclassification{msc}{18-00}
%\pmkeywords{`all' types of categories with specific examples}
\pmrelated{IndexOfCategoryTheory}
\pmrelated{Category}
\pmrelated{CategoryTheory}
\pmrelated{AlgebraicTopology}
\pmrelated{C_1Category}
\pmrelated{C_2Category}
\pmrelated{C_3Category}
\pmrelated{GrothendieckCategory}
\pmrelated{ClosedMonoidalCategory}
\pmrelated{ProofThatAbelianGroupsFormAnAbelianCategory}
\pmrelated{ExamplesOfAbelianCategory}
\pmrelated{CartesianClosedCategory}
\pmrelated{GroupoidCDyn}
\pmdefines{$Cc$-category}

% this is the default PlanetMath preamble.  as your knowledge
% of TeX increases, you will probably want to edit this, but
% it should be fine as is for beginners.

% almost certainly you want these
\usepackage{amssymb}
\usepackage{amsmath}
\usepackage{amsfonts}

% used for TeXing text within eps files
%\usepackage{psfrag}
% need this for including graphics (\includegraphics)
%\usepackage{graphicx}
% for neatly defining theorems and propositions
%\usepackage{amsthm}
% making logically defined graphics
%%%\usepackage{xypic}

% there are many more packages, add them here as you need them

% define commands here
\usepackage{amsmath, amssymb, amsfonts, amsthm, amscd, latexsym}
%%\usepackage{xypic}
\usepackage[mathscr]{eucal}

\setlength{\textwidth}{6.5in}
%\setlength{\textwidth}{16cm}
\setlength{\textheight}{9.0in}
%\setlength{\textheight}{24cm}

\hoffset=-.75in     %%ps format
%\hoffset=-1.0in     %%hp format
\voffset=-.4in

\theoremstyle{plain}
\newtheorem{lemma}{Lemma}[section]
\newtheorem{proposition}{Proposition}[section]
\newtheorem{theorem}{Theorem}[section]
\newtheorem{corollary}{Corollary}[section]

\theoremstyle{definition}
\newtheorem{definition}{Definition}[section]
\newtheorem{example}{Example}[section]
%\theoremstyle{remark}
\newtheorem{remark}{Remark}[section]
\newtheorem*{notation}{Notation}
\newtheorem*{claim}{Claim}

\renewcommand{\thefootnote}{\ensuremath{\fnsymbol{footnote%%@
}}}
\numberwithin{equation}{section}

\newcommand{\Ad}{{\rm Ad}}
\newcommand{\Aut}{{\rm Aut}}
\newcommand{\Cl}{{\rm Cl}}
\newcommand{\Co}{{\rm Co}}
\newcommand{\DES}{{\rm DES}}
\newcommand{\Diff}{{\rm Diff}}
\newcommand{\Dom}{{\rm Dom}}
\newcommand{\Hol}{{\rm Hol}}
\newcommand{\Mon}{{\rm Mon}}
\newcommand{\Hom}{{\rm Hom}}
\newcommand{\Ker}{{\rm Ker}}
\newcommand{\Ind}{{\rm Ind}}
\newcommand{\IM}{{\rm Im}}
\newcommand{\Is}{{\rm Is}}
\newcommand{\ID}{{\rm id}}
\newcommand{\GL}{{\rm GL}}
\newcommand{\Iso}{{\rm Iso}}
\newcommand{\Sem}{{\rm Sem}}
\newcommand{\St}{{\rm St}}
\newcommand{\Sym}{{\rm Sym}}
\newcommand{\SU}{{\rm SU}}
\newcommand{\Tor}{{\rm Tor}}
\newcommand{\U}{{\rm U}}

\newcommand{\A}{\mathcal A}
\newcommand{\Ce}{\mathcal C}
\newcommand{\D}{\mathcal D}
\newcommand{\E}{\mathcal E}
\newcommand{\F}{\mathcal F}
\newcommand{\G}{\mathcal G}
\newcommand{\Q}{\mathcal Q}
\newcommand{\R}{\mathcal R}
\newcommand{\cS}{\mathcal S}
\newcommand{\cU}{\mathcal U}
\newcommand{\W}{\mathcal W}

\newcommand{\bA}{\mathbb{A}}
\newcommand{\bB}{\mathbb{B}}
\newcommand{\bC}{\mathbb{C}}
\newcommand{\bD}{\mathbb{D}}
\newcommand{\bE}{\mathbb{E}}
\newcommand{\bF}{\mathbb{F}}
\newcommand{\bG}{\mathbb{G}}
\newcommand{\bK}{\mathbb{K}}
\newcommand{\bM}{\mathbb{M}}
\newcommand{\bN}{\mathbb{N}}
\newcommand{\bO}{\mathbb{O}}
\newcommand{\bP}{\mathbb{P}}
\newcommand{\bR}{\mathbb{R}}
\newcommand{\bV}{\mathbb{V}}
\newcommand{\bZ}{\mathbb{Z}}

\newcommand{\bfE}{\mathbf{E}}
\newcommand{\bfX}{\mathbf{X}}
\newcommand{\bfY}{\mathbf{Y}}
\newcommand{\bfZ}{\mathbf{Z}}

\renewcommand{\O}{\Omega}
\renewcommand{\o}{\omega}
\newcommand{\vp}{\varphi}
\newcommand{\vep}{\varepsilon}

\newcommand{\diag}{{\rm diag}}
\newcommand{\grp}{{\mathbb G}}
\newcommand{\dgrp}{{\mathbb D}}
\newcommand{\desp}{{\mathbb D^{\rm{es}}}}
\newcommand{\Geod}{{\rm Geod}}
\newcommand{\geod}{{\rm geod}}
\newcommand{\hgr}{{\mathbb H}}
\newcommand{\mgr}{{\mathbb M}}
\newcommand{\ob}{{\rm Ob}}
\newcommand{\obg}{{\rm Ob(\mathbb G)}}
\newcommand{\obgp}{{\rm Ob(\mathbb G')}}
\newcommand{\obh}{{\rm Ob(\mathbb H)}}
\newcommand{\Osmooth}{{\Omega^{\infty}(X,*)}}
\newcommand{\ghomotop}{{\rho_2^{\square}}}
\newcommand{\gcalp}{{\mathbb G(\mathcal P)}}

\newcommand{\rf}{{R_{\mathcal F}}}
\newcommand{\glob}{{\rm glob}}
\newcommand{\loc}{{\rm loc}}
\newcommand{\TOP}{{\rm TOP}}

\newcommand{\wti}{\widetilde}
\newcommand{\what}{\widehat}

\renewcommand{\a}{\alpha}
\newcommand{\be}{\beta}
\newcommand{\ga}{\gamma}
\newcommand{\Ga}{\Gamma}
\newcommand{\de}{\delta}
\newcommand{\del}{\partial}
\newcommand{\ka}{\kappa}
\newcommand{\si}{\sigma}
\newcommand{\ta}{\tau}
\newcommand{\med}{\medbreak}
\newcommand{\medn}{\medbreak \noindent}
\newcommand{\bign}{\bigbreak \noindent}
\newcommand{\lra}{{\longrightarrow}}
\newcommand{\ra}{{\rightarrow}}
\newcommand{\rat}{{\rightarrowtail}}
\newcommand{\oset}[1]{\overset {#1}{\ra}}
\newcommand{\osetl}[1]{\overset {#1}{\lra}}
\newcommand{\hr}{{\hookrightarrow}}

\begin{document}
\subsection{Index of categories}
\emph{The following is a contributed listing, or Index of Categories:}

\begin{enumerate}
\item Category of sets, $Set$ or $Ens$
\item \PMlinkname{Cartesian closed category}{CartesianClosedCategory}, or $Cc$-category; an example is
$Set$
\item Category of molecular sets: a $Cc$-category
\item Category of organismic sets: a model subcategory of $Set$
\item \PMlinkname{Category of $(M,R)$--systems}{CategoryOfMRSystems3}: a $Cc$-category
\item Dual category
\item Double dual category, $V^{**}$
\item Category of dynamical systems
\item Abelian categories: Grothendieck category, \PMlinkname{category of Abelian groups}{ProofThatAbelianGroupsFormAnAbelianCategory}, category of sheaves of abelian groups, category of reversible automata
$\mathcal{A}_R$
\item Complete and cocomplete categories
\item Comma category
\item \PMlinkname{$C_1$-category}{C_1Category}
\item \PMlinkname{$C_2$-category}{C_2Category}
\item \PMlinkname{$C_3$-category}{C_3Category}
\item \PMlinkname{$Ab5$-category}{GrothendieckCategory}
\item Category of additive fractions
\item Category of fractions
\item Quotient category
\item Grassmann category
\item Category of groups
\item Category of crossed modules of groups
\item Category of crossed modules of algebroids
\item Category of matrices
\item Category of Abelian (or commutative) groups
\item Category of Polish groups
\item Category of Lie 2-groups, (also an example of internal category)
\item Categorical groups
\item Internal categories
\item Monoidal category: a. {\em symmetric}, b. {\em braided}, c.\PMlinkname{\em closed}{ClosedMonoidalCategory}
\item Category of semigroups
\item Category of rings
\item Category of modules
\item Category of crossed modules
\item Category of crossed complexes
\item Groupoid category
\item Category of topological spaces
\item Homotopy category
\item \PMlinkname{Category of sheaves of Abelian groups}{ExamplesOfAbelianCategory} 
\item Category of Riemannian manifolds
\item Borel category
\item Category of graphs: category of reflexive (or directed) graphs
\item \PMlinkname{Category of paths on a graph}{CategoryOfPathsOnAGraph}
\item Category of hypergraphs
\item Category of groupoids, Groupoid category
\item Category of topological groupoids
\item Category of Borel groupoids
\item Cohomology of small categories
\item Category of C*-algebras
\item \PMlinkname{$2-C^*$ -category,${\mathcal{C}^*}_2$}{2CCategory}
\item Category of H $*$ -algebras
\item Category of Hilbert spaces
\item Category of automata
\item Category of quantum automata
\item Algebraic categories
\item Category of logic algebras
\item Category of lattices
\item Category of Boolean algebras
\item \PMlinkname{Algebraic category of \L{}ukasiewicz-Moisil, $LM_n$--algebras}{AlgebraicCategoryOfLMnLogicAlgebras}
\item Category of Heyting algebras
\item Category of topoi
\item Category of genetic nets
\item Category of quantum logic algebras
\item R-category
\item Category of algebroids
\item Category of double algebroids
\item Grassmann-Hopf algebroid category
\item Topological category
\item Galois category 
\item Subcategory
\item Internal category
\item Groupoid groups
\item Groupoid category
\item Fibered categories
\item Non-Abelian categories
\item Category of diagrams
\item Functor category, or category of functors and functorial morphisms (natural transformations)
\item Category of small categories
\item Category of categories, super-category
\item Supercategory
\item Meta-category
\item Category of functors, (or Functor category)
\item 2-category; specific examples are: a Lie 2-group, Abelian 2-category, 
2-category of commutative (or Abelian) groupoids.
\item Double category
\item n-category
\item Double groupoid category
\item s-category 
\end{enumerate}


%%%%%
%%%%%
\end{document}
