\documentclass[12pt]{article}
\usepackage{pmmeta}
\pmcanonicalname{QuantumCategories}
\pmcreated{2013-03-22 19:32:06}
\pmmodified{2013-03-22 19:32:06}
\pmowner{bci1}{20947}
\pmmodifier{bci1}{20947}
\pmtitle{quantum categories}
\pmrecord{11}{42512}
\pmprivacy{1}
\pmauthor{bci1}{20947}
\pmtype{Topic}
\pmcomment{trigger rebuild}
\pmclassification{msc}{18-00}
\pmdefines{bialgebras}
\pmdefines{bialgebroids}
\pmdefines{quantum categories}
\pmdefines{weak Hopf bialgebroid}
\pmdefines{quantum bi-groupoid}

\endmetadata

% this is the default PlanetMath preamble. as your knowledge
% of TeX increases, you will probably want to edit this, but
\usepackage{amsmath, amssymb, amsfonts, amsthm, amscd, latexsym}
%%\usepackage{xypic}
\usepackage[mathscr]{eucal}
% define commands here
\theoremstyle{plain}
\newtheorem{lemma}{Lemma}[section]
\newtheorem{proposition}{Proposition}[section]
\newtheorem{theorem}{Theorem}[section]
\newtheorem{corollary}{Corollary}[section]
\theoremstyle{definition}
\newtheorem{definition}{Definition}[section]
\newtheorem{example}{Example}[section]
%\theoremstyle{remark}
\newtheorem{remark}{Remark}[section]
\newtheorem*{notation}{Notation}
\newtheorem*{claim}{Claim}
\renewcommand{\thefootnote}{\ensuremath{\fnsymbol{footnote%%@
}}}
\numberwithin{equation}{section}
\newcommand{\Ad}{{\rm Ad}}
\newcommand{\Aut}{{\rm Aut}}
\newcommand{\Cl}{{\rm Cl}}
\newcommand{\Co}{{\rm Co}}
\newcommand{\DES}{{\rm DES}}
\newcommand{\Diff}{{\rm Diff}}
\newcommand{\Dom}{{\rm Dom}}
\newcommand{\Hol}{{\rm Hol}}
\newcommand{\Mon}{{\rm Mon}}
\newcommand{\Hom}{{\rm Hom}}
\newcommand{\Ker}{{\rm Ker}}
\newcommand{\Ind}{{\rm Ind}}
\newcommand{\IM}{{\rm Im}}
\newcommand{\Is}{{\rm Is}}
\newcommand{\ID}{{\rm id}}
\newcommand{\GL}{{\rm GL}}
\newcommand{\Iso}{{\rm Iso}}
\newcommand{\Sem}{{\rm Sem}}
\newcommand{\St}{{\rm St}}
\newcommand{\Sym}{{\rm Sym}}
\newcommand{\SU}{{\rm SU}}
\newcommand{\Tor}{{\rm Tor}}
\newcommand{\U}{{\rm U}}
\newcommand{\A}{\mathcal A}
\newcommand{\Ce}{\mathcal C}
\newcommand{\D}{\mathcal D}
\newcommand{\E}{\mathcal E}
\newcommand{\F}{\mathcal F}
\newcommand{\G}{\mathcal G}
\newcommand{\Q}{\mathcal Q}
\newcommand{\R}{\mathcal R}
\newcommand{\cS}{\mathcal S}
\newcommand{\cU}{\mathcal U}
\newcommand{\W}{\mathcal W}
\newcommand{\bA}{\mathbb{A}}
\newcommand{\bB}{\mathbb{B}}
\newcommand{\bC}{\mathbb{C}}
\newcommand{\bD}{\mathbb{D}}
\newcommand{\bE}{\mathbb{E}}
\newcommand{\bF}{\mathbb{F}}
\newcommand{\bG}{\mathbb{G}}
\newcommand{\bK}{\mathbb{K}}
\newcommand{\bM}{\mathbb{M}}
\newcommand{\bN}{\mathbb{N}}
\newcommand{\bO}{\mathbb{O}}
\newcommand{\bP}{\mathbb{P}}
\newcommand{\bR}{\mathbb{R}}
\newcommand{\bV}{\mathbb{V}}
\newcommand{\bZ}{\mathbb{Z}}
\newcommand{\bfE}{\mathbf{E}}
\newcommand{\bfX}{\mathbf{X}}
\newcommand{\bfY}{\mathbf{Y}}
\newcommand{\bfZ}{\mathbf{Z}}
\renewcommand{\O}{\Omega}
\renewcommand{\o}{\omega}
\newcommand{\vp}{\varphi}
\newcommand{\vep}{\varepsilon}
\newcommand{\diag}{{\rm diag}}
\newcommand{\grp}{{\mathbb G}}
\newcommand{\dgrp}{{\mathbb D}}
\newcommand{\desp}{{\mathbb D^{\rm{es}}}}
\newcommand{\Geod}{{\rm Geod}}
\newcommand{\geod}{{\rm geod}}
\newcommand{\hgr}{{\mathbb H}}
\newcommand{\mgr}{{\mathbb M}}
\newcommand{\ob}{{\rm Ob}}
\newcommand{\obg}{{\rm Ob(\mathbb G)}}
\newcommand{\obgp}{{\rm Ob(\mathbb G')}}
\newcommand{\obh}{{\rm Ob(\mathbb H)}}
\newcommand{\Osmooth}{{\Omega^{\infty}(X,*)}}
\newcommand{\ghomotop}{{\rho_2^{\square}}}
\newcommand{\gcalp}{{\mathbb G(\mathcal P)}}
\newcommand{\rf}{{R_{\mathcal F}}}
\newcommand{\glob}{{\rm glob}}
\newcommand{\loc}{{\rm loc}}
\newcommand{\TOP}{{\rm TOP}}
\newcommand{\wti}{\widetilde}
\newcommand{\what}{\widehat}
\renewcommand{\a}{\alpha}
\newcommand{\be}{\beta}
\newcommand{\ga}{\gamma}
\newcommand{\Ga}{\Gamma}
\newcommand{\de}{\delta}
\newcommand{\del}{\partial}
\newcommand{\ka}{\kappa}
\newcommand{\si}{\sigma}
\newcommand{\ta}{\tau}
\newcommand{\lra}{{\longrightarrow}}
\newcommand{\ra}{{\rightarrow}}
\newcommand{\rat}{{\rightarrowtail}}
\newcommand{\oset}[1]{\overset {#1}{\ra}}
\newcommand{\osetl}[1]{\overset {#1}{\lra}}
\newcommand{\hr}{{\hookrightarrow}}

\begin{document}
\section{On Quantum Categories}

Quantum categories were introduced in [1] as generalizations of both bi(co)algebroids and small categories. 

The monadic definition of a quantum category leads to a set of axioms that are very close to the definitions of a bialgebroid in the literature on Hopf algebras and quantum groups. hence the qualifier quantum for such categories that generalize quantum groups. The specific notions of functor and natural transformation were also introduced for quantum categories in [2].

Quantum categories were defined in [1] within a monoidal category $\mathbb{V}$. Thus,  when $\mathbb{V}$ is the opposite
category of modules over a commutative ring, a {\em quantum category} coincides  with a bialgebroid; intuitively,
[\em bialgebroids} can be thought of as ``several object'' generalizations of bialgebras, or ``bialgebras with many objects''.

Moreover, a {\em quantum category}  was defined by B. Day and R. Street in [1]  in a general monoidal category $\mathbb{V}$  by incorporating both bialgebroids, in the manner specified here, and also ordinary categories, by taking the monoidal category $\mathbb{V}$ to be the category of sets, {\bf Set}.

\subsection{Remarks}

This leads to an interesting concept of a quantum groupoid as a bialgebroid, or quantum category with all invertible morphisms, perhaps also with the underlying topological structure being locally compact. 

An alternative definition of a {\em quantum bi-groupoid} based on [2]would be as a {\em weak Hopf bialgebroid}. 

\section{References}


[1] B. Day and R. Street. Quantum categories, star autonomy, and quantum groupoids, Fields Institute Communications (American Math. Soc.) 43 (2004), 187--226.

[2] G. a B$\"o$hm and K. Szlach$\'a$nyi. Hopf algebroids with bijective antipodes: axioms, integrals and duals, J. Algebra 274 no. 2 (2004): 708-750.

[3] B. Day, R. Street, Monoidal bicategories and Hopf algebroids, Advances in Math. 129 (1997) 99--157.

[4] J. B$\'e$nabou, Introduction to bicategories, Lecture Notes in Mathematics (Springer--Verlag, Berlin) 47 (1967), 1--77.

[5] J.-H. Lu, Hopf algebroids and quantum groupoids, Int. J. Math. 7 (1996), 47--70.
{\bf ... more to come}

%%%%%
%%%%%
\end{document}
