\documentclass[12pt]{article}
\usepackage{pmmeta}
\pmcanonicalname{FundamentalComplexityDiagrams}
\pmcreated{2013-03-22 18:13:20}
\pmmodified{2013-03-22 18:13:20}
\pmowner{bci1}{20947}
\pmmodifier{bci1}{20947}
\pmtitle{fundamental complexity diagrams}
\pmrecord{12}{40806}
\pmprivacy{1}
\pmauthor{bci1}{20947}
\pmtype{Topic}
\pmcomment{trigger rebuild}
\pmclassification{msc}{18B20}
\pmclassification{msc}{18C50}
\pmclassification{msc}{18B10}
\pmclassification{msc}{93A99}
\pmclassification{msc}{93A15}
\pmclassification{msc}{18A15}
\pmclassification{msc}{93A05}
\pmclassification{msc}{93A13}
\pmsynonym{the theory of levels in ontology}{FundamentalComplexityDiagrams}
%\pmkeywords{super- and ultra-complexity diagrams of highly complex systems}
%\pmkeywords{higher complexity levels in categorical ontology}
\pmdefines{higher complexity levels in categorical ontology}

% this is the default PlanetMath preamble.  as your knowledge
% of TeX increases, you will probably want to edit this, but
% it should be fine as is for beginners.

% almost certainly you want these
\usepackage{amssymb}
\usepackage{amsmath}
\usepackage{amsfonts}

% used for TeXing text within eps files
%\usepackage{psfrag}
% need this for including graphics (\includegraphics)
%\usepackage{graphicx}
% for neatly defining theorems and propositions
%\usepackage{amsthm}
% making logically defined graphics
%%%\usepackage{xypic}

% there are many more packages, add them here as you need them

% define commands here
\usepackage{amsmath, amssymb, amsfonts, amsthm, amscd, latexsym,color,enumerate}
%%\usepackage{xypic}
\xyoption{curve}
\usepackage[mathscr]{eucal}

\setlength{\textwidth}{7in}
%\setlength{\textwidth}{16cm}
\setlength{\textheight}{10.0in}
%\setlength{\textheight}{24cm}

\hoffset=-.75in     %%ps format
%\hoffset=-1.0in     %%hp format
\voffset=-.4in

%the next gives two direction arrows at the top of a 2 x 2 matrix

\newcommand{\directs}[2]{\def\objectstyle{\scriptstyle}  \objectmargin={0pt}
\xy
(0,4)*+{}="a",(0,-2)*+{\rule{0em}{1.5ex}#2}="b",(7,4)*+{\;#1}="c"
\ar@{->} "a";"b" \ar @{->}"a";"c" \endxy }

\theoremstyle{plain}
\newtheorem{lemma}{Lemma}[section]
\newtheorem{proposition}{Proposition}[section]
\newtheorem{theorem}{Theorem}[section]
\newtheorem{corollary}{Corollary}[section]
\newtheorem{conjecture}{Conjecture}[section]

\theoremstyle{definition}
\newtheorem{definition}{Definition}[section]
\newtheorem{example}{Example}[section]
%\theoremstyle{remark}
\newtheorem{remark}{Remark}[section]
\newtheorem*{notation}{Notation}
\newtheorem*{claim}{Claim}


\theoremstyle{plain}
\renewcommand{\thefootnote}{\ensuremath{\fnsymbol{footnote}}}
\numberwithin{equation}{section}
\newcommand{\Ad}{{\rm Ad}}
\newcommand{\Aut}{{\rm Aut}}
\newcommand{\Cl}{{\rm Cl}}
\newcommand{\Co}{{\rm Co}}
\newcommand{\DES}{{\rm DES}}
\newcommand{\Diff}{{\rm Diff}}
\newcommand{\Dom}{{\rm Dom}}
\newcommand{\Hol}{{\rm Hol}}
\newcommand{\Mon}{{\rm Mon}}
\newcommand{\Hom}{{\rm Hom}}
\newcommand{\Ker}{{\rm Ker}}
\newcommand{\Ind}{{\rm Ind}}
\newcommand{\IM}{{\rm Im}}
\newcommand{\Is}{{\rm Is}}
\newcommand{\ID}{{\rm id}}
\newcommand{\GL}{{\rm GL}}
\newcommand{\Iso}{{\rm Iso}}
\newcommand{\Sem}{{\rm Sem}}
\newcommand{\St}{{\rm St}}
\newcommand{\Sym}{{\rm Sym}}
\newcommand{\SU}{{\rm SU}}
\newcommand{\Tor}{{\rm Tor}}
\newcommand{\U}{{\rm U}}

\newcommand{\A}{\mathcal A}
\newcommand{\D}{\mathcal D}
\newcommand{\E}{\mathcal E}
\newcommand{\F}{\mathcal F}
\newcommand{\G}{\mathcal G}
\newcommand{\R}{\mathcal R}
\newcommand{\cS}{\mathcal S}
\newcommand{\cU}{\mathcal U}
\newcommand{\W}{\mathcal W}

\newcommand{\Ce}{\mathsf{C}}
\newcommand{\Q}{\mathsf{Q}}
\newcommand{\grp}{\mathsf{G}}
\newcommand{\dgrp}{\mathsf{D}}

\newcommand{\bA}{\mathbb{A}}
\newcommand{\bB}{\mathbb{B}}
\newcommand{\bC}{\mathbb{C}}
\newcommand{\bD}{\mathbb{D}}
\newcommand{\bE}{\mathbb{E}}
\newcommand{\bF}{\mathbb{F}}
\newcommand{\bG}{\mathbb{G}}
\newcommand{\bK}{\mathbb{K}}
\newcommand{\bM}{\mathbb{M}}
\newcommand{\bN}{\mathbb{N}}
\newcommand{\bO}{\mathbb{O}}
\newcommand{\bP}{\mathbb{P}}
\newcommand{\bR}{\mathbb{R}}
\newcommand{\bV}{\mathbb{V}}
\newcommand{\bZ}{\mathbb{Z}}

\newcommand{\bfE}{\mathbf{E}}
\newcommand{\bfX}{\mathbf{X}}
\newcommand{\bfY}{\mathbf{Y}}
\newcommand{\bfZ}{\mathbf{Z}}

\renewcommand{\O}{\Omega}
\renewcommand{\o}{\omega}
\newcommand{\vp}{\varphi}
\newcommand{\vep}{\varepsilon}

\newcommand{\diag}{{\rm diag}}
\newcommand{\desp}{{\mathbb D^{\rm{es}}}}
\newcommand{\Geod}{{\rm Geod}}
\newcommand{\geod}{{\rm geod}}
\newcommand{\hgr}{{\mathbb H}}
\newcommand{\mgr}{{\mathbb M}}
\newcommand{\ob}{\operatorname{Ob}}
\newcommand{\obg}{{\rm Ob(\mathbb G)}}
\newcommand{\obgp}{{\rm Ob(\mathbb G')}}
\newcommand{\obh}{{\rm Ob(\mathbb H)}}
\newcommand{\Osmooth}{{\Omega^{\infty}(X,*)}}
\newcommand{\ghomotop}{{\rho_2^{\square}}}
\newcommand{\gcalp}{{\mathbb G(\mathcal P)}}

\newcommand{\rf}{{R_{\mathcal F}}}
\newcommand{\glob}{{\rm glob}}
\newcommand{\loc}{{\rm loc}}
\newcommand{\TOP}{{\rm TOP}}

\newcommand{\wti}{\widetilde}
\newcommand{\what}{\widehat}

\renewcommand{\a}{\alpha}
\newcommand{\be}{\beta}
\newcommand{\ga}{\gamma}
\newcommand{\Ga}{\Gamma}
\newcommand{\de}{\delta}
\newcommand{\del}{\partial}
\newcommand{\ka}{\kappa}
\newcommand{\si}{\sigma}
\newcommand{\ta}{\tau}


\newcommand{\lra}{{\longrightarrow}}
\newcommand{\ra}{{\rightarrow}}
\newcommand{\rat}{{\rightarrowtail}}
\newcommand{\oset}[1]{\overset {#1}{\ra}}
\newcommand{\osetl}[1]{\overset {#1}{\lra}}
\newcommand{\hr}{{\hookrightarrow}}


\newcommand{\hdgb}{\boldsymbol{\rho}^\square}
\newcommand{\hdg}{\rho^\square_2}

\newcommand{\med}{\medbreak}
\newcommand{\medn}{\medbreak \noindent}
\newcommand{\bign}{\bigbreak \noindent}

\renewcommand{\leq}{{\leqslant}}
\renewcommand{\geq}{{\geqslant}}

\def\red{\textcolor{red}}
\def\magenta{\textcolor{magenta}}
\def\blue{\textcolor{blue}}
\def\<{\langle}
\def\>{\rangle}
\begin{document}
\subsection{Fundamental complexity diagrams} 

 Categorical comparisons of different types of dynamical systems in diagrams provide useful means for both their classification and understanding the relations between them. Such powerful mathematical tools may also be considered as a further, practically useful elaboration of Spencer's philosophical principle ideas in biology and sociology in terms of the emergence of higher complexity levels in living systems and societies. 
As explained by Barry Mitchell (1965) and other category theoreticians, diagrams can be defined as functors, and functor categories involve `meta-diagrams of diagrams', or functors and natural transformations, meta-categories of categories, and so on to higher dimensional algebras (HDA). Therefore, category theory in higher dimensions/HDA
appears as the natural setting for considering the emergence of higher complexity levels--such as living organisms,
the human mind and societies in relation to the simpler, physicochemical/molecular/quantum systems. 

\subsection{Diagrams of complexity levels}

 When viewed from a formal perspective of Poli's theory of levels (Baianu and Poli, 2008), the two levels of super-- and ultra-- complex systems are quite \emph{distinct} in many of their defining properties, and therefore, categorical diagrams that `mix' such distinct levels \emph{do not commute}. 
Considering dynamic similarity, Rosen (1968) introduced the concept of \emph{`analogous'} (classical) dynamical systems in terms of categorical, dynamic isomorphisms between their isomorphic state-spaces that commute with their transition (state) function, or dynamic laws.  However, the extension of this concept to either complex or super-complex systems has not yet been investigated, and may be similar in importance to the introduction of the Lorentz-Poincar\'e group of transformations for reference frames in Relativity theory. Furthermore, one
is always seeking the underlying \emph{relational invariance} or \emph{similarities in functionality} among different organisms or between different stages of development during ontogeny (the development of an organism from a fertilized egg), as well as during phylogeny, the evolution of organisms and species, (encompassing also biomolecular evolution
that may however be often `neutral' with respect to the the emergence of new phenotypes). In this context,
the categorical concept of `\emph{dynamically adjoint systems}' was introduced in relation to the data obtained through nuclear transplant experiments (Baianu and Scripcariu, 1974). Thus, extending the latter concept to super-- and ultra-- complex systems , one has in  general, that two complex or supercomplex systems with `state spaces' being defined respectively as A and A*, are dynamically adjoint if they can be represented naturally by the following (functorial) diagram:

\begin{equation}
\def\labelstyle{\textstyle}
 \xymatrix@M=0.1pc @=4pc{A \ar[r]^{F}
 \ar[d]_{F'} & A^* \ar[d]^{G} \\{A^*}  \ar[r]_{G'} & {A}}
\end{equation}

with $F \approx F'$  and $G \approx G'$ being isomorphic (that is,  $\approx$ representing natural equivalences between adjoint functors of the same kind, either left or right), and as above in diagram (2.5), the two diagonals are, respectively, the state-space transition functions $\Delta: A  \rightarrow  A$   and $\Delta^*: A^* \rightarrow   A^*$  of the two adjoint dynamical systems. (It would also be interesting to investigate dynamic adjointness in the context of quantum dynamical systems and quantum automata, as defined in Baianu, 1971a).


 A \emph{left-adjoint} functor, such as the functor F in the above commutative diagram between categories representing state spaces of equivalent cell nuclei \emph{preserves inductive limits}, whereas the \emph{right-adjoint} (or coadjoint) functor, such as G above, \emph{preserves projective limits (colimits)}. (For precise definitions of adjoint functors the reader is
referred to Brown, Galzebrook and Baianu, 2007, as well as to Popescu, 1973,
Baianu and Scripcariu, 1974, and the initial paper by Kan, 1958). 

\subsubsection{Applications to mathematical models of: nuclear transplants, cloning, embryogenesis, development and related mathematical biology problems}

 A \emph{left-adjoint} functor, such as the functor F in the above commutative diagram between categories representing state spaces of equivalent cell nuclei \emph{preserves limits}, whereas the \emph{right-adjoint} (or coadjoint) functor, such as G above, \emph{preserves colimits}. (For precise definitions of adjoint functors the reader is
referred to Brown, Galzebrook and Baianu, 2007, as well as to Popescu, 1973,
Baianu and Scripcariu, 1974, and the initial paper by Kan, 1958).
Thus, {\em dynamic attractors and genericity of states} are preserved for differentiating cells up to the blastula stage of organismic development. Subsequent stages of ontogenetic development can be considered only `weakly adjoint' or partially analogous. Similar dynamic controls may operate for controlling division cycles in the cells of different organisms; therefore, such instances are also good example of the dynamic adjointness relation between cells of different organisms that may be very far apart phylogenetically, even on different `branches of the tree of life.'  A more elaborate dynamic concept of `homology' between the genomes of different
species during evolution was also proposed (Baianu, 1971a), suggesting that an entire phylogenetic series can be characterized by a topologically--rather than biologically--\emph{homologous sequence} of genomes which preserves certain genes encoding \emph{the essential} biological functions. A striking example was recently suggested involving the differentiation of the nervous system in the fruit fly and mice (and perhaps also man) which leads to the formation of the back, middle and front parts of the neural tube. A related, topological generalization of such a dynamic similarity between systems was previously introduced as \emph{topological conjugacy }(Baianu, 1986-1987a; Baianu and Lin, 2004), which replaces recursive, digital simulation with symbolic, topological modelling for both super-- and ultra-- complex systems (Baianu and Lin., 2004; Baianu, 2004c; Baianu et al., 2004, 2006b). This approach stems logically from the introduction of topological/symbolic computation and topological computers Baianu, 1971b), as well as their natural extensions to quantum nano-automata (Baianu, 2004a), quantum automata and quantum computers (Baianu, 1971a, and 1971b, respectively); the latter may allow us to make a `quantum leap' in our understanding life and the higher complexity levels in general. Quantum logics and $LMn$-logic algebra are also essential to understanding the `immanent' operational logics of the human brain and the associated mind meta--level. Husserl might have also considered that the transcedental logic of the human mind is very different from the Boolean logic of digital computers. 
%%%%%
%%%%%
\end{document}
