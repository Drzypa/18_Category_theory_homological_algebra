\documentclass[12pt]{article}
\usepackage{pmmeta}
\pmcanonicalname{BibliographyForMathematicalPhysicsFoundationsAxiomaticsAndCategories}
\pmcreated{2013-03-22 18:17:58}
\pmmodified{2013-03-22 18:17:58}
\pmowner{bci1}{20947}
\pmmodifier{bci1}{20947}
\pmtitle{bibliography for mathematical physics foundations: axiomatics and categories}
\pmrecord{26}{40917}
\pmprivacy{1}
\pmauthor{bci1}{20947}
\pmtype{Bibliography}
\pmcomment{trigger rebuild}
\pmclassification{msc}{18-00}
\pmclassification{msc}{55-00}
\pmclassification{msc}{78-00}
\pmclassification{msc}{82-00}
\pmclassification{msc}{83-00}
\pmclassification{msc}{81-00}
\pmsynonym{bibliography of foundations in mathematical physics}{BibliographyForMathematicalPhysicsFoundationsAxiomaticsAndCategories}
%\pmkeywords{ETACS}
%\pmkeywords{axiomatics and categorical foundations of mathematical physics}

\endmetadata

% this is the default PlanetMath preamble.  as your knowledge
% of TeX increases, you will probably want to edit this, but
% it should be fine as is for beginners.

% almost certainly you want these
\usepackage{amssymb}
\usepackage{amsmath}
\usepackage{amsfonts}

% used for TeXing text within eps files
%\usepackage{psfrag}
% need this for including graphics (\includegraphics)
%\usepackage{graphicx}
% for neatly defining theorems and propositions
%\usepackage{amsthm}
% making logically defined graphics
%%%\usepackage{xypic}

% there are many more packages, add them here as you need them

% define commands here
\usepackage{amsmath, amssymb, amsfonts, amsthm, amscd, latexsym}
%%\usepackage{xypic}
\usepackage[mathscr]{eucal}

\setlength{\textwidth}{6.5in}
%\setlength{\textwidth}{16cm}
\setlength{\textheight}{9.0in}
%\setlength{\textheight}{24cm}

\hoffset=-.75in     %%ps format
%\hoffset=-1.0in     %%hp format
\voffset=-.4in

\theoremstyle{plain}
\newtheorem{lemma}{Lemma}[section]
\newtheorem{proposition}{Proposition}[section]
\newtheorem{theorem}{Theorem}[section]
\newtheorem{corollary}{Corollary}[section]

\theoremstyle{definition}
\newtheorem{definition}{Definition}[section]
\newtheorem{example}{Example}[section]
%\theoremstyle{remark}
\newtheorem{remark}{Remark}[section]
\newtheorem*{notation}{Notation}
\newtheorem*{claim}{Claim}

\renewcommand{\thefootnote}{\ensuremath{\fnsymbol{footnote%%@
}}}
\numberwithin{equation}{section}

\newcommand{\Ad}{{\rm Ad}}
\newcommand{\Aut}{{\rm Aut}}
\newcommand{\Cl}{{\rm Cl}}
\newcommand{\Co}{{\rm Co}}
\newcommand{\DES}{{\rm DES}}
\newcommand{\Diff}{{\rm Diff}}
\newcommand{\Dom}{{\rm Dom}}
\newcommand{\Hol}{{\rm Hol}}
\newcommand{\Mon}{{\rm Mon}}
\newcommand{\Hom}{{\rm Hom}}
\newcommand{\Ker}{{\rm Ker}}
\newcommand{\Ind}{{\rm Ind}}
\newcommand{\IM}{{\rm Im}}
\newcommand{\Is}{{\rm Is}}
\newcommand{\ID}{{\rm id}}
\newcommand{\GL}{{\rm GL}}
\newcommand{\Iso}{{\rm Iso}}
\newcommand{\Sem}{{\rm Sem}}
\newcommand{\St}{{\rm St}}
\newcommand{\Sym}{{\rm Sym}}
\newcommand{\SU}{{\rm SU}}
\newcommand{\Tor}{{\rm Tor}}
\newcommand{\U}{{\rm U}}

\newcommand{\A}{\mathcal A}
\newcommand{\Ce}{\mathcal C}
\newcommand{\D}{\mathcal D}
\newcommand{\E}{\mathcal E}
\newcommand{\F}{\mathcal F}
\newcommand{\G}{\mathcal G}
\newcommand{\Q}{\mathcal Q}
\newcommand{\R}{\mathcal R}
\newcommand{\cS}{\mathcal S}
\newcommand{\cU}{\mathcal U}
\newcommand{\W}{\mathcal W}

\newcommand{\bA}{\mathbb{A}}
\newcommand{\bB}{\mathbb{B}}
\newcommand{\bC}{\mathbb{C}}
\newcommand{\bD}{\mathbb{D}}
\newcommand{\bE}{\mathbb{E}}
\newcommand{\bF}{\mathbb{F}}
\newcommand{\bG}{\mathbb{G}}
\newcommand{\bK}{\mathbb{K}}
\newcommand{\bM}{\mathbb{M}}
\newcommand{\bN}{\mathbb{N}}
\newcommand{\bO}{\mathbb{O}}
\newcommand{\bP}{\mathbb{P}}
\newcommand{\bR}{\mathbb{R}}
\newcommand{\bV}{\mathbb{V}}
\newcommand{\bZ}{\mathbb{Z}}

\newcommand{\bfE}{\mathbf{E}}
\newcommand{\bfX}{\mathbf{X}}
\newcommand{\bfY}{\mathbf{Y}}
\newcommand{\bfZ}{\mathbf{Z}}

\renewcommand{\O}{\Omega}
\renewcommand{\o}{\omega}
\newcommand{\vp}{\varphi}
\newcommand{\vep}{\varepsilon}

\newcommand{\diag}{{\rm diag}}
\newcommand{\grp}{{\mathbb G}}
\newcommand{\dgrp}{{\mathbb D}}
\newcommand{\desp}{{\mathbb D^{\rm{es}}}}
\newcommand{\Geod}{{\rm Geod}}
\newcommand{\geod}{{\rm geod}}
\newcommand{\hgr}{{\mathbb H}}
\newcommand{\mgr}{{\mathbb M}}
\newcommand{\ob}{{\rm Ob}}
\newcommand{\obg}{{\rm Ob(\mathbb G)}}
\newcommand{\obgp}{{\rm Ob(\mathbb G')}}
\newcommand{\obh}{{\rm Ob(\mathbb H)}}
\newcommand{\Osmooth}{{\Omega^{\infty}(X,*)}}
\newcommand{\ghomotop}{{\rho_2^{\square}}}
\newcommand{\gcalp}{{\mathbb G(\mathcal P)}}

\newcommand{\rf}{{R_{\mathcal F}}}
\newcommand{\glob}{{\rm glob}}
\newcommand{\loc}{{\rm loc}}
\newcommand{\TOP}{{\rm TOP}}

\newcommand{\wti}{\widetilde}
\newcommand{\what}{\widehat}

\renewcommand{\a}{\alpha}
\newcommand{\be}{\beta}
\newcommand{\ga}{\gamma}
\newcommand{\Ga}{\Gamma}
\newcommand{\de}{\delta}
\newcommand{\del}{\partial}
\newcommand{\ka}{\kappa}
\newcommand{\si}{\sigma}
\newcommand{\ta}{\tau}
\newcommand{\med}{\medbreak}
\newcommand{\medn}{\medbreak \noindent}
\newcommand{\bign}{\bigbreak \noindent}
\newcommand{\lra}{{\longrightarrow}}
\newcommand{\ra}{{\rightarrow}}
\newcommand{\rat}{{\rightarrowtail}}
\newcommand{\oset}[1]{\overset {#1}{\ra}}
\newcommand{\osetl}[1]{\overset {#1}{\lra}}
\newcommand{\hr}{{\hookrightarrow}}
\begin{document}
This is an extensive, but not exhaustive, bibliography for axiomatics, categories, supercategories and mathematical physics foundations 


\begin{thebibliography}{99}

\bibitem{AS}
Alfsen, E.M. and F. W. Schultz: \emph{Geometry of State Spaces of
Operator Algebras}, Birkh\"auser, Boston--Basel--Berlin (2003).

\bibitem{AMF56}
Atiyah, M.F. 1956. ``On the Krull-Schmidt theorem with applications to sheaves.'',
\emph{Bull. Soc. Math. France}, \textbf{84}: 307--317.

\bibitem{AMF56}
Auslander, M. 1965. ``Coherent Functors.'', in: \emph{Proc. Conf. Cat. Algebra, La Jolla},
189--231.
  
\bibitem{AS-BC2k}
Awodey, S. \& Butz, C. 2000. ``Topological Completeness for Higher Order Logic.'', Journal of Symbolic Logic, 65, 3, 1168--1182. 

\bibitem{AS-RER2k2}
Awodey, S. \& Reck, E. R. 2002. ``Completeness and Categoricity I. 
Nineteen-Century Axiomatics to Twentieth-Century Metalogic.'', History and Philosophy of Logic, 23, 1, 1--30.
  
\bibitem{AS-RER2k2}
Awodey, S. \& Reck, E. R. 2002. ``Completeness and Categoricity II. Twentieth-Century Metalogic to Twenty-first-Century Semantics.'', \emph{History and Philosophy of Logic}, 23, (2): 77--94.  

\bibitem{AS96}
Awodey, S., 1996, ``Structure in Mathematics and Logic: A Categorical Perspective.'', 
\emph{Philosophia Mathematica}, {\bf 3}: 209--237. 

\bibitem{AS2k4}
Awodey, S., 2004, ``An Answer to Hellman's Question: Does Category Theory Provide a Framework for Mathematical Structuralism.'', \emph{Philosophia Mathematica}, 12: 54--64. 

\bibitem{AS2k6}
Awodey, S., 2006, {\em Category Theory}, Oxford: Clarendon Press. 

\bibitem{BAJ-DJ98a}
Baez, J. \& Dolan, J. 1998a. ``Higher-Dimensional Algebra: III. n-Categories and the Algebra of Opetopes.'', 
in: \emph{Advances in Mathematics}, {\bf 135}: 145--206.  

\bibitem{BAJ-DJ98B}
Baez, J. \& Dolan, J. 1998b. ``Categorification'', Higher Category Theory, Contemporary Mathematics, 230, Providence: AMS, 1--36. 

\bibitem{BAJ-DJ2k1}
Baez, J. \& Dolan, J. 2001 ``From Finite Sets to Feynman Diagrams.'', 
in \emph{Mathematics Unlimited -- 2001 and Beyond}, Berlin: Springer, 29--50.  

\bibitem{BAJ-DJ97}
Baez, J. 1997. ``An Introduction to n-Categories.'', in \emph{Category Theory and Computer Science, Lecture Notes in Computer Science}, 1290, Berlin: Springer-Verlag, 1--33. 

\bibitem{ICB4}
Baianu, I.C.: 1971a, ``Organismic Supercategories and Qualitative Dynamics of Systems.'', \emph{Ibid.}, \textbf{33} (3), 339--354.
 
\bibitem{ICB4}
Baianu, I.C.: 1971b, ``Categories, Functors and Quantum Algebraic Computations'', in P. Suppes (ed.), \emph{Proceed. Fourth Intl. Congress Logic-Mathematics-Philosophy of Science}, September 1--4, 1971, Bucharest.

\bibitem{ICB-HG-EO84}
Baianu, I.C., H. S. Gutowsky, and E. Oldfield: 1984, {\em Proc. Natl. Acad. Sci. USA}, \textbf{81}(12):
3713-3717.
 
\bibitem{Bgg2}
Baianu, I. C., Glazebrook, J. F. and G. Georgescu: 2004, ``Categories of Quantum Automata and N-Valued \L ukasiewicz Algebras in Relation to Dynamic Bionetworks, \textbf{(M,R)}--Systems and Their Higher Dimensional Algebra.'', 
\PMlinkexternal{PDF's of Abstract and Preprint of Report}{\\http://www.ag.uiuc.edu/fs401/QAuto.pdf}.

\bibitem{ICB8}
Baianu, I.C.: 2004a, ``Quantum Nano--Automata (QNA): Microphysical Measurements with Microphysical QNA Instruments.'', \emph{CERN Preprint EXT--2004--125}.

\bibitem{Bgb2}
Baianu, I. C., Brown, R. and J. F. Glazebrook: 2006a, {\em Quantum Algebraic Topology and Field Theories}.  
\PMlinkexternal{Preprint subm.}{http://www.ag.uiuc.edu/fs40l/QAT.pdf}. 

\bibitem{BBGG1}
Baianu I. C., Brown R., Georgescu G. and J. F. Glazebrook: 2006b, ``Complex Nonlinear Biodynamics in Categories, Higher Dimensional Algebra and \L ukasiewicz--Moisil Topos: Transformations of Neuronal, Genetic and Neoplastic Networks.'', \emph{Axiomathes}, \textbf{16} Nos. 1--2: 65--122.

\bibitem{Bggb4}
Baianu, I.C.,  R. Brown and J. F. Glazebrook: 2007b, ``A Non-Abelian, Categorical Ontology of Spacetimes and Quantum Gravity.'', Axiomathes, 17: 169-225.

\bibitem{Ba-We2k}
M.~Barr and C.~Wells. {\em Toposes, Triples and Theories}. Montreal: McGill University, 2000.

\bibitem{Ba-We85}
Barr, M. \& Wells, C., 1985, ``Toposes, Triples and Theories.'', New York: Springer-Verlag.
 
\bibitem{BM-CW99}
Barr, M. \& Wells, C., 1999, ``Category Theory for Computing Science.'', Montreal: CRM. 
 
\bibitem{BaM98}
Batanin, M. 1998. ``Monoidal Globular Categories as a Natural Environment for the Theory of Weak n-Categories.'', 
\emph{Advances in Mathematics}, 136: 39--103.   

\bibitem{BJL81}
Bell, J. L., 1981, ``Category Theory and the Foundations of Mathematics.'', 
\emph{British Journal for the Philosophy of Science}, 32, 349--358. 
 
\bibitem{BJL82}
Bell, J. L., 1982, ``Categories, Toposes and Sets.'', \emph{Synthese}, 51, 3, 293--337. 
 
\bibitem{BJL86}
Bell, J. L., 1986, ``From Absolute to Local Mathematics.'', \emph{Synthese}, 69, 3, 409--426. 

\bibitem{BJL88} 
Bell, J. L., 1988, {\em Toposes and Local Set Theories: An Introduction}, Oxford: Oxford University Press. 

\bibitem{BG-MCLS99}
Birkoff, G. \& Mac Lane, S., 1999, {\em Algebra}, 3rd ed., Providence: AMS.  

\bibitem{BDK2k3}
Biss, D.K., 2003, ``Which Functor is the Projective Line?'', \emph{American Mathematical Monthly}, 110, 7, 574--592. 

\bibitem{BA-SA83}
Blass, A. \& Scedrov, A., 1983, ``Classifying Topoi and Finite Forcing.'' , Journal of Pure and Applied Algebra, 28, 111--140. 

\bibitem{BA-SA89}
Blass, A. \& Scedrov, A., 1989, ``Freyd's Model for the Independence of the Axiom of Choice.'', Providence: AMS.  

\bibitem{BASA92}
Blass, A. \& Scedrov, A., 1992, ``Complete Topoi Representing Models of Set Theory.'', 
\emph{Annals of Pure and Applied Logic}, 57, no. 1, 1--26.  

\bibitem{BA84}
Blass, A., 1984, ``The Interaction Between Category Theory and Set Theory.'', Mathematical Applications of Category Theory, 30, Providence: AMS, 5--29. 

\bibitem{BR-SP2k4}
Blute, R. \& Scott, P., 2004, ``Category Theory for Linear Logicians.'', 
in: {\em Linear Logic in Computer Science.}

\bibitem{Borceux94}
Borceux, F.: 1994, \emph{Handbook of Categorical Algebra}, vols: 1--3, 
in {\em Encyclopedia of Mathematics and its Applications} \textbf{50} to \textbf{52}, Cambridge University Press.

\bibitem{Bourbaki1}
Bourbaki, N. 1961 and 1964: \emph{Alg\`{e}bre commutative.},
in \'{E}l\'{e}ments de Math\'{e}matique., Chs. 1--6., Hermann: Paris.

\bibitem{BrownBook1}
R. Brown: \emph{Topology and Groupoids}, BookSurge LLC (2006).

\bibitem{BJk4}
Brown, R. and G. Janelidze: 2004, ``Galois theory and a new homotopy
double groupoid of a map of spaces.'', \emph{Applied Categorical
Structures} \textbf{12}: 63-80.

\bibitem{BHR2}
Brown, R., Higgins, P. J. and R. Sivera,: 2007a, \emph{Non-Abelian
Algebraic Topology},\PMlinkexternal{Vol.I PDF}{http://www.bangor.ac.uk/~mas010/nonab-t/partI010604.pdf}.

\bibitem{BGB2k7b}
Brown, R., Glazebrook, J. F. and I.C. Baianu.: 2007b, ``A Conceptual, Categorical and Higher Dimensional Algebra Framework of Universal Ontology and the Theory of Levels for Highly Complex Structures and Dynamics.'', \emph{Axiomathes} (17): 321--379.

\bibitem{BPP2k4}
Brown, R., Paton, R. and T. Porter.: 2004, ``Categorical language and
hierarchical models for cell systems.'', in \emph{Computation in
Cells and Tissues - Perspectives and Tools of Thought}, Paton, R.;
Bolouri, H.; Holcombe, M.; Parish, J.H.; Tateson, R. (Eds.)
Natural Computing Series, Springer Verlag, 289-303.

\bibitem{BP2k3}
Brown R. and T. Porter: 2003, ``Category theory and higher
dimensional algebra: potential descriptive tools in neuroscience.'', In:
{\em Proceedings of the International Conference on Theoretical
Neurobiology}, Delhi, February 2003, edited by Nandini Singh,
National Brain Research Centre, Conference Proceedings 1, 80-92.

\bibitem{Br-Har-Ka-Po2k2}
Brown, R., Hardie, K., Kamps, H. and T. Porter: 2002, ``The homotopy
double groupoid of a Hausdorff space.'', \emph{Theory and
Applications of Categories} \textbf{10}, 71-93.

\bibitem{Br-Hardy76}
Brown, R., and Hardy, J.P.L.:1976, ``Topological groupoids I:
universal constructions.'', \emph{Math. Nachr.}, 71: 273-286.

\bibitem{Br-Sp76}
Brown, R. and Spencer, C.B.: 1976, ``Double groupoids and crossed
modules.'', \emph{Cah.  Top. G\'{e}om. Diff.} \textbf{17}, 343-362.

\bibitem{BRTPT2k6}
Brown R, and Porter T (2006) ``Category theory: an abstract setting for analogy and comparison. In: What is
category theory?'', in: {\em  Advanced studies in mathematics and logic}. Polimetrica Publisher, Italy, pp.
257-274.

\bibitem{BR-SCB76}
Brown R, Razak Salleh A (1999) ``Free crossed resolutions of groups and presentations of modules of
identities among relations.'' {\em LMS J. Comput. Math.}, \textbf{2}: 25--61.

\bibitem{RBPGV} 
Ronald Brown et al.: Non-Abelian Algebraic Topology, vols. I and II. 2010, 620 pages with Index. (March 4, 2010-in press: Springer): \PMlinkexternal{Nonabelian Algebraic Topology:filtered spaces, crossed complexes, cubical higher homotopy groupoids}{http://www.bangor.ac.uk/~mas010/rbrsbookb-e040310.pdf}

\bibitem{BDA55}
Buchsbaum, D. A.: 1955, ``Exact categories and duality.'', Trans. Amer. Math. Soc. \textbf{80}: 1-34.

\bibitem{BDA55}
Buchsbaum, D. A.: 1969, A note on homology in categories., Ann. of Math. \textbf{69}: 66-74.

\bibitem{BI65}
Bucur, I. (1965). {\em Homological Algebra}. (orig. title: ``Algebra Omologica'')
Ed. Didactica si Pedagogica: Bucharest.

\bibitem{BI-DA68}
Bucur, I., and Deleanu A. (1968). {\em  Introduction to the Theory of Categories and Functors}. J.Wiley and Sons: London

\bibitem{BL2k3}
Bunge, M. and S. Lack: 2003, ``Van Kampen theorems for toposes.'', \emph{Adv. in Math.} \textbf{179}, 291-317.

\bibitem{BM74} 
Bunge, M., 1974, ``Topos Theory and Souslin's Hypothesis'', Journal of Pure and Applied Algebra, 4, 159-187.  

\bibitem{BM84}
Bunge, M., 1984, ``Toposes in Logic and Logic in Toposes'', Topoi, 3, no. 1, 13-22. 

\bibitem{BM-LS2k3}
Bunge M, Lack S (2003) Van Kampen theorems for toposes. {\em Adv Math}, \textbf {179}: 291-317.

\bibitem{BJ-ICJ2k1}
Butterfield J., Isham C.J. (2001) Spacetime and the philosophical challenges of quantum gravity. In:
Callender C, Hugget N (eds) Physics meets philosophy at the Planck scale. Cambridge University
Press, pp 33-89.

\bibitem{BJ-ICJ98-2k2}
Butterfield J., Isham C.J. 1998, 1999, 2000-2002, A topos perspective on the Kochen-Specker theorem
I-IV, Int J Theor Phys 37(11):2669-2733; 38(3):827-859; 39(6):1413-1436; 41(4): 613-639.

\bibitem{CH-ES56}
Cartan, H. and Eilenberg, S. 1956. {\em Homological Algebra}, Princeton Univ. Press: Pinceton. 

\bibitem{Chaician}
M. Chaician and A. Demichev. 1996. {\em Introduction to Quantum Groups}, World Scientific .

\bibitem{CC46}
Chevalley, C. 1946. {\em The theory of Lie groups}. Princeton University Press, Princeton NJ

\bibitem{CPM65}
Cohen, P.M. 1965. {\em Universal Algebra}, Harper and Row: New York, london and Tokyo.

\bibitem{CF}
M. Crainic and R. Fernandes.2003. Integrability of Lie brackets, {\em Ann.of Math}. \textbf{157}: 575-620.

\bibitem{CA94}
Connes A 1994. \emph{Noncommutative geometry}. Academic Press: New York.

\bibitem{CR-LL63}
Croisot, R. and Lesieur, L. 1963. \emph{Alg\`ebre noeth\'erienne non-commutative.},
Gauthier-Villard: Paris.

\bibitem{CRL94}
Crole, R.L., 1994, {\em Categories for Types}, Cambridge: Cambridge University Press.  

\bibitem{CJ-LJ91}
Couture, J. \& Lambek, J., 1991, {\em Philosophical Reflections on the Foundations of Mathematics}, Erkenntnis, 34, 2, 187--209. 

\bibitem{DJ-ALEX60-71}
Dieudonn\'e, J. \& Grothendieck, A., 1960, [1971], \'El\'ements de G\'eom\'etrie Alg\'ebrique, Berlin: Springer-Verlag.  

\bibitem{Dirac30}
Dirac, P. A. M., 1930, {\em The Principles of Quantum Mechanics}, Oxford: Clarendon 
Press. 

\bibitem{Dirac33}
Dirac, P. A. M., 1933, {\em The Lagrangian in Quantum Mechanics}, Physikalische 
Zeitschrift der Sowietunion, \textbf{3}: 64-72. 

\bibitem{Dirac43}
Dirac, P. A. M.,, 1943, {\em Quantum Electrodynamics}, Communications of the Dublin 
Institute for Advanced Studies, \textbf{A1}: 1-36. 

\bibitem{Dixmier}
Dixmier, J., 1981, {\em Von Neumann Algebras}, Amsterdam: North-Holland Publishing 
Company. [First published in French in 1957: Les Algebres d'Operateurs dans 
l'Espace Hilbertien, Paris: Gauthier--Villars.]

\bibitem{Durdevich1}
M. Durdevich : ``Geometry of quantum principal bundles: I.'', Commun.
Math. Phys. \textbf{175} (3) (1996), 457--521.

\bibitem{Durdevich2}
M. Durdevich : ``Geometry of quantum principal bundles: II'', Rev.
Math. Phys. \textbf{9} (5) (1997), 531--607.

\bibitem{EC}
Ehresmann, C.: 1965, \emph{Cat\'egories et Structures}, Dunod, Paris.

\bibitem{EC}
Ehresmann, C.: 1966, ``Trends Toward Unity in Mathematics.'',
\emph{Cahiers de Topologie et Geometrie Differentielle}
\textbf{8}: 1-7.

\bibitem{Eh-pseudo}
Ehresmann, C.: 1952, ``Structures locales et structures infinit\'esimales.'',
\emph{C.R.A.S.} Paris \textbf{274}: 587-589.

\bibitem{Eh}
Ehresmann, C.: 1959, ``Cat\'egories topologiques et cat\'egories
diff\'erentiables.'', \emph{Coll. G\'eom. Diff. Glob.} Bruxelles, pp.137-150.

\bibitem{Eh-quintettes}
Ehresmann, C.:1963, Cat\'egories doubles des quintettes: applications covariantes
, \emph{C.R.A.S. Paris}, \textbf{256}: 1891--1894.

\bibitem{Eh-Oe}
Ehresmann, C.: 1984, \emph{Oeuvres compl\`etes et  comment\'ees:
Amiens, 1980-84}, edited and commented by Andr\'ee Ehresmann.

\bibitem{EACV1}
Ehresmann, A. C. and J.-P. Vanbremersch: 1987, ``Hierarchical
Evolutive Systems: A mathematical model for complex systems.'',
\emph{Bull. of Math. Biol.} \textbf{49} (1): 13-50.

\bibitem{EACV2}
Ehresmann, A. C. and J.-P. Vanbremersch: 2006, ``The Memory Evolutive Systems as
a model of Rosen's Organisms'', \emph{Axiomathes} \textbf{16} (1--2): 13-50.

\bibitem{EML1}
Eilenberg, S. and S. Mac Lane.: 1942, ``Natural Isomorphisms in Group Theory.'', \emph{American Mathematical Society 43}: 757-831.

\bibitem{EL}
Eilenberg, S. and S. Mac Lane: 1945, ``The General Theory of Natural Equivalences.'', \emph{Transactions of the American Mathematical Society} \textbf{58}: 231-294.

\bibitem{ES-CH56}
Eilenberg, S. \& Cartan, H., 1956, {\em Homological Algebra}, Princeton: Princeton University Press. 

\bibitem{ES-MCLS42}
Eilenberg, S. \& MacLane, S., 1942, ``Group Extensions and Homology'', Annals of Mathematics, 43, 757--831. 

\bibitem{ES-SN52}
Eilenberg, S. \& Steenrod, N., 1952, {\em Foundations of Algebraic Topology}, Princeton: Princeton University Press. 

\bibitem{ES60}
Eilenberg, S.: 1960. ``Abstract description of some basic functors.'', J. Indian Math.Soc., \textbf{24} :221-234.

\bibitem{S.Eilenberg}
S.Eilenberg. ``Relations between Homology and Homotopy Groups''.
{\em Proc.Natl.Acad.Sci.USA} (1966),v:10--14.  

\bibitem{ED88}
Ellerman, D., 1988, ``Category Theory and Concrete Universals'', Synthese, 28, 409--429. 

\bibitem{ETH}
Z. F. Ezawa, G. Tsitsishvilli and K. Hasebe : ``Noncommutative
geometry, extended $W_{\infty}$ algebra and Grassmannian solitons
in multicomponent Hall systems.'', arXiv:hep--th/0209198.

\bibitem{FS77}
Feferman, S., 1977. ``Categorical Foundations and Foundations of Category Theory'', in \emph{Logic, Foundations of Mathematics and Computability}, R. Butts (ed.), Reidel, 149--169.

\bibitem{Fell}
Fell, J. M. G., 1960. ``The Dual Spaces of  C*-Algebras'', 
\emph{Transactions of the American Mathematical Society}, 94: 365-403. 

\bibitem{Feynman}
Feynman, R. P., 1948, ``Space--Time Approach to Non--Relativistic Quantum 
Mechanics.'', \emph{Reviews of Modern Physics}, 20: 367--387. [It is reprinted in (Schwinger 1958).] 

\bibitem{FP60}
Freyd, P., 1960. {\em Functor Theory (Dissertation)}. Princeton University, Princeton, New Jersey.
 
\bibitem{FP63}
Freyd, P., 1963, ``Relative homological algebra made absolute.'' , {\em Proc. Natl. Acad. USA}, \textbf{49}:19-20.
  
\bibitem{FP64}
Freyd, P., 1964, {\em Abelian Categories. An Introduction to the Theory of Functors}, New York and London: Harper and Row.  

\bibitem{FP65}
Freyd, P., 1965, ``The Theories of Functors and Models.'', in: {\em Theories of Models}, Amsterdam: North Holland, 107--120. 

\bibitem{FP66}
Freyd, P., 1966, ``Algebra-valued Functors in general categories and tensor product in particular.'', {\em Colloq. Mat}. 
{14}: 89--105.

\bibitem{FP72}
Freyd, P., 1972, ``Aspects of Topoi'',{\em Bulletin of the Australian Mathematical Society}, \textbf{7}: 1--76.  

\bibitem{FP80}
Freyd, P., 1980, ``The Axiom of Choice'', Journal of Pure and Applied Algebra, 19, 103--125. 

\bibitem{FP87}
Freyd, P., 1987, ``Choice and Well-Ordering'', Annals of Pure and Applied Logic, 35, 2, 149--166.  

\bibitem{FP90}
Freyd, P., 1990, Categories, Allegories, Amsterdam: North Holland. 

\bibitem{FP2k2}
Freyd, P., 2002, ``Cartesian Logic'', Theoretical Computer Science, 278, no. 1--2, 3--21.  

\bibitem{FP-FH-SA87}
Freyd, P., Friedman, H. \& Scedrov, A., 1987, ``Lindembaum Algebras of Intuitionistic Theories and Free Categories'', Annals of Pure and Applied Logic, 35, 2, 167--172.

\bibitem{Gablot}
Gablot, R. 1971. ``Sur deux classes de cat\'{e}gories de Grothendieck''. Thesis..  Univ. de Lille.

\bibitem{Gabriel1}
Gabriel, P.: 1962, ``Des cat\'egories ab\'eliennes'', \emph{Bull. Soc.
Math. France} \textbf{90}: 323-448.

\bibitem{Gabriel2}
Gabriel, P. and M.Zisman:. 1967: \emph{Category of fractions and homotopy theory}, \emph{Ergebnesse der math.} Springer: Berlin.

\bibitem{GabrielNP}
Gabriel, P. and N. Popescu: 1964, Caract\'{e}risation des cat\'egories ab\'eliennes
avec g\'{e}n\'{e}rateurs et limites inductives. , \emph{CRAS Paris} \textbf{258}: 4188-4191.

\bibitem{GA-RG-SM2k}
Galli, A. \& Reyes, G. \& Sagastume, M., 2000, ``Completeness Theorems via the Double Dual Functor.'', Studia Logical, 64, no. 1, 61--81. 

\bibitem{GN}
Gelfan'd, I. and Naimark, M., 1943. ``On the Imbedding of Normed Rings into the 
Ring of Operators in Hilbert Space.'' ,Recueil Math\'ematique [Matematicheskii 
Sbornik] Nouvelle S\'erie, 12 [54]: 197-213. [Reprinted in C*--algebras: 
1943--1993, in the series Contemporary Mathematics, 167,  Providence, R.I. : 
American Mathematical Society, 1994.] 

\bibitem{GV70}
Georgescu, G. and C. Vraciu 1970. ``On the Characterization of \L{}ukasiewicz 
Algebras.'' \emph{J Algebra}, \textbf{16} (4), 486-495.

\bibitem{GS-ZM2K2}
Ghilardi, S. \& Zawadowski, M., 2002, Sheaves, Games \& Model Completions: A Categorical Approach to Nonclassical Porpositional Logics, Dordrecht: Kluwer.  

\bibitem{gs89}
Ghilardi, S., 1989, ``Presheaf Semantics and Independence Results for some Non-classical first-order logics'', Archive for Mathematical Logic., 29, no. 2, 125--136. 

\bibitem{Gob68}
Goblot, R., 1968, Cat\'egories modulaires , {\em C. R. Acad. Sci. Paris, S\'erie A.}, \textbf{267}: 381--383.

\bibitem{Gob71}
Goblot, R., 1971, Sur deux classes de cat\'egories de Grothendieck, {\em Th\`ese.}, Univ. Lille, 1971.

\bibitem{GR79}
Goldblatt, R., 1979, Topoi: The Categorical Analysis of Logic, Studies in logic and the foundations of mathematics, Amsterdam: Elsevier North-Holland Publ. Comp. 

\bibitem{Goldie}
Goldie, A. W., 1964, Localization in non-commutative noetherian rings, {\em J.Algebra}, \textbf{1}: 286-297.

\bibitem{Godement}
Godement,R. 1958. Th\'{e}orie des faisceaux. Hermann: Paris.

\bibitem{GRAY65}
Gray, C. W.: 1965. Sheaves with values in a category.,\emph {Topology}, 3: 1-18.

\bibitem{Alex1}
Grothendieck, A.: 1971, Rev\^{e}tements \'Etales et Groupe Fondamental (SGA1),
chapter VI: Cat\'egories fibr\'ees et descente, \emph{Lecture Notes in Math.}
\textbf{224}, Springer--Verlag: Berlin.

\bibitem{Alex2}
Grothendieck, A.: 1957, Sur quelque point d-alg\'{e}bre homologique. , \emph{Tohoku Math. J.}, \textbf{9:} 119-121.

\bibitem{Alex3}
Grothendieck, A. and J. Dieudon\'{e}.: 1960, El\'{e}ments de geometrie alg\'{e}brique., \emph{Publ. Inst. des Hautes Etudes de Science}, \textbf{4}.

\bibitem{ALEXsem}
Grothendieck, A. et al., ``S\'eminaire de G\'eom\'etrie Alg\'ebrique'', Vol. 1--7, Berlin: Springer-Verlag.
  
\bibitem{TMMFJ84}
Groups Authors: J. Faria Martins, Timothy Porter.,
On Yetter's Invariant and an Extension of the Dijkgraaf-Witten Invariant to Categorical
$math.QA/0608484 [abs, ps, pdf, other]$.

\bibitem{GL66}
Gruson, L, 1966, Compl\'etion ab\'elienne. {\em Bull. Math.Soc. France}, \textbf{90}: 17-40.
 

\bibitem{HKK}
K.A. Hardie, K.H. Kamps and R.W. Kieboom. 2000.  A homotopy 2-groupoid of a Hausdorff 
space, \emph{Applied Cat. Structures} 8: 209--234.

\bibitem{HWS82}
Hatcher, W. S. 1982. {\em The Logical Foundations of Mathematics}, Oxford: Pergamon Press. 
  
\bibitem{Heller58}
Heller, A. :1958, Homological algebra in Abelian categories., \emph{Ann. of Math.}
\textbf{68}: 484-525.

\bibitem{HellerRowe62}
Heller, A.  and K. A. Rowe.:1962, On the category of sheaves., \emph{Amer J. Math.}
\textbf{84}: 205-216.

\bibitem{HG2k3}
Hellman, G., 2003, "Does Category Theory Provide a Framework for Mathematical Structuralism?", Philosophia Mathematica, 11, 2, 129--157. 

\bibitem{HC-MM-PJ2K}
Hermida, C. \& Makkai, M. \& Power, J., 2000, On Weak Higher-dimensional Categories. I, Journal of Pure and Applied Algebra, 154, no. 1-3, 221--246. 

\bibitem{HC-MM-PI2K1}
Hermida, C. \& Makkai, M. \& Power, J., 2001, On Weak Higher-dimensional Categories. II, Journal of Pure and Applied Algebra, 157, no. 2-3, 247--277.  

\bibitem{HC-MM-PI2K2}
Hermida, C. \& Makkai, M. \& Power, J., 2002, On Weak Higher-dimensional Categories. III, Journal of Pure and Applied Algebra, 166, no. 1-2, 83--104.  

\bibitem{HPJbook}
Higgins, P. J.: 2005, \emph{Categories and groupoids}, Van
Nostrand Mathematical Studies: 32, (1971); \emph{Reprints in
Theory and Applications of Categories}, No. 7: 1-195.

\bibitem{HPJ2k5}
Higgins, Philip J. Thin elements and commutative shells in cubical
$\omega$-categories. Theory Appl. Categ. 14 (2005), No. 4, 60--74
(electronic). (Reviewer: Timothy Porter) 18D05.

\bibitem{HJ-RE-RG90}
Hyland,  J.M.E. \& Robinson,  E.P. \& Rosolini, G., 1990, "The Discrete Objects in the Effective Topos", Proceedings of the London Mathematical Society (3), 60, no. 1, 1--36. 

\bibitem{HJME82}
Hyland,  J.M.E., 1982, "The Effective Topos", Studies in Logic and the Foundations of Mathematics, 110, Amsterdam: North Holland, 165--216.  

\bibitem{HJME88}
Hyland, J. M..E., 1988, "A Small Complete Category", Annals of Pure and Applied Logic, 40, no. 2, 135--165. 

\bibitem{HJME91}
Hyland,  J. M .E., 1991, "First Steps in Synthetic Domain Theory", Category Theory (Como 1990), Lecture Notes in Mathematics, 1488, Berlin: Springer, 131-156.  

\bibitem{HJME2K2}
Hyland, J. M.E., 2002, "Proof Theory in the Abstract", Annals of Pure and Applied Logic, 114, no. 1--3, 43--78. 

\bibitem{E.Hurewicz}
E.Hurewicz. CW Complexes.Trans AMS.1955.

\bibitem{IT-PR-IB70}
Ionescu, Th., R. Parvan and I. Baianu, 1970, {\em C. R. Acad. Sci. Paris, S\'erie A.}, \textbf{269}:
112-116, {\em communiqu\'ee par Louis N\'eel}.  

\bibitem{Isham1}
C. J. Isham : A new approach to quantising space--time: I.
quantising on a general category, \emph{Adv. Theor. Math. Phys.}
\textbf{7} (2003), 331--367.

\bibitem{JB99}
Jacobs, B., 1999, Categorical Logic and Type Theory, Amsterdam: North Holland.  

\bibitem{JPT77}
Johnstone, P. T., 1977, Topos Theory, New York: Academic Press. 

\bibitem{JPT79A}
Johnstone, P. T., 1979a, "Conditions Related to De Morgan's Law", Applications of Sheaves, Lecture Notes in Mathematics, 753, Berlin: Springer, 479--491. 

\bibitem{JPT79B}
Johnstone, P.T., 1979b, "Another Condition Equivalent to De Morgan's Law", Communications in Algebra, 7, no. 12, 1309--1312.  

\bibitem{JPT81}
Johnstone, P. T., 1981, "Tychonoff's Theorem without the Axiom of Choice", Fundamenta Mathematicae, 113, no. 1, 21--35. 

\bibitem{JPT52}
Johnstone, P. T., 1982, Stone Spaces, Cambridge:Cambridge University Press.  

\bibitem{JPT85}
Johnstone, P. T., 1985, "How General is a Generalized Space?", Aspects of Topology, Cambridge: Cambridge University Press, 77--111. 

\bibitem{JPT2K2A}
Johnstone, P. T., 2002a, Sketches of an Elephant: a Topos Theory Compendium. Vol. 1, Oxford Logic Guides, 43, Oxford: Oxford University Press.  

\bibitem{JAMI95}
Joyal, A. \& Moerdijk, I., 1995, Algebraic Set Theory, Cambridge: Cambridge University Press.  

\bibitem{kampen1-1933}
Van Kampen, E. H.: 1933, On the Connection Between the Fundamental
Groups of some Related Spaces, \emph{Amer. J. Math.} \textbf{55}: 261-267

\bibitem{KDM58}
Kan, D. M., 1958, ``Adjoint Functors.'', Transactions of the American Mathematical Society, 87, 294-329.  

\bibitem{Kleisli62}
Kleisli, H.: 1962, ``Homotopy theory in Abelian categories.'',{\em Can. J. Math.}, \textbf{14}: 139-169.

\bibitem{KJT70}
Knight, J.T., 1970, ``On epimorphisms of non-commutative rings.'', {\em Proc. Cambridge Phil. Soc.},
\textbf{25}: 266-271.

\bibitem{KA81}
Kock, A., 1981, Synthetic Differential Geometry, London Mathematical Society Lecture Note Series, 51, Cambridge: Cambridge University Press. 

\bibitem{KN1}
S. Kobayashi and K. Nomizu : Foundations of Differential Geometry
Vol I., Wiley Interscience, New York--London 1963.

\bibitem{Krips}
H. Krips : Measurement in Quantum Theory, \emph{The Stanford
Encyclopedia of Philosophy } ({Winter 1999 Edition}), Edward N.
Zalta (ed.), $URL=<http://plato.stanford.edu/archives/win1999/entries/qt--measurement/>$

\bibitem{LTY}
Lam, T. Y., 1966, The category of noetherian modules, {\em Proc. Natl. Acad. Sci. USA}, \textbf{55}: 1038-104.

\bibitem{LJ-SPJ81}
Lambek, J. \& Scott, P. J., 1981, ``Intuitionistic Type Theory and Foundations", Journal of Philosophical Logic, 10, 1, 101--115. 

\bibitem{LJ-SPJ86}
Lambek, J. \& Scott, P.J. 1986. {\em Introduction to Higher Order Categorical Logic}, Cambridge: Cambridge University Press. 

\bibitem{LJ68}
Lambek, J., 1968, ``Deductive Systems and Categories I. Syntactic Calculus and Residuated Categories.'', Mathematical Systems Theory.'', 2, 287--318. 

\bibitem{LJ69}
Lambek, J., 1969, ``Deductive Systems and Categories II. Standard Constructions and Closed Categories.'', Category Theory, Homology Theory and their Applications I'', Berlin: Springer, 76--122. 

\bibitem{LJ72}
Lambek, J., 1972, ``Deductive Systems and Categories III. Cartesian Closed Categories, Intuitionistic Propositional Calculus, and Combinatory Logic.'', Toposes, Algebraic Geometry and Logic, Lecture Notes in Mathematics, 274, Berlin: Springer, 57--82.  

\bibitem{LJ82} 
Lambek, J., 1982, ``The Influence of Heraclitus on Modern Mathematics.'', Scientific Philosophy Today, J. Agassi and R.S. Cohen, eds., Dordrecht, Reidel, 111--122.  

\bibitem{LJ86}
Lambek, J. 1986. ``Cartesian Closed Categories and Typed lambda calculi.'', Combinators and Functional Programming Languages, Lecture Notes in Computer Science, 242, Berlin: Springer, 136--175.   

\bibitem{LT89A}
Lambek, J. 1989A. ``On Some Connections Between Logic and Category Theory.'', Studia Logica, 48, 3, 269--278. 

\bibitem{LJ89B}
Lambek, J., 1989B, ``On the Sheaf of Possible Worlds'', in: {\em Categorical Topology and its relation to Analysis, Algebra and Combinatorics}, Teaneck: World Scientific Publishing, 36--53. 

\bibitem{LJ94a}
Lambek, J. 1994a, ``Some Aspects of Categorical Logic'', Logic, {\em Methodology and Philosophy of Science IX, Studies in Logic and the Foundations of Mathematics}, 134, Amsterdam: North Holland, 69--89. 

\bibitem{LJ94b}
Lambek, J. 1994b, ``What is a Deductive System?; What is a Logical System?'', Studies in Logic and Computation, 4, Oxford: Oxford University Press, 141-159.  
 
\bibitem{LJ2k4}
Lambek, J., 2004, ``What is the world of Mathematics? Provinces of Logic Determined.'', {\em Annals of Pure and Applied Logic}, 126(1-3), 149--158. 

\bibitem{LaSc}
Lambek, J. and P.~J.~Scott. {\em Introduction to higher order categorical logic}. Cambridge University Press, 1986.

\bibitem{Lance}
E. C. Lance.1995.  ``Hilbert C*--Modules''. \emph{London Math. Soc. Lect.
Notes} \textbf{210}, \emph{Cambridge Univ. Press.}, 1995.

\bibitem{LE-MJP2k5}
Landry, E. \& Marquis, J.-P., 2005, ``Categories in Context: Historical, Foundational and philosophical.'', Philosophia Mathematica, 13, 1--43.  

\bibitem{LE99}
Landry, E., 1999, ``Category Theory: the Language of Mathematics'', Philosophy of Science, 66, 3: supplement, S14--S27. 

\bibitem{LE99}
Landry, E., 2001, ``Logicism, Structuralism and Objectivity'', Topoi, 20, 1, 79--95. 

\bibitem{LandNP98}
Landsman, N. P.: 1998, \emph{Mathematical Topics between Classical and Quantum Mechanics}, Springer Verlag: New York.

\bibitem{Land}
N. P.  Landsman : {\em Mathematical topics between classical and
quantum mechanics.}, \emph{Springer Verlag}, New York, 1998.

\bibitem{Land1}
N. P. Landsman : ``Compact quantum groupoids'', arXiv:math-ph/9912006

\bibitem{LPRM94}
La Palme Reyes, M., et. al., 1994, ``The non-Boolean Logic of Natural Language Negation.'', Philosophia Mathematica, 2, no. 1, 45--68.

\bibitem{LPRM99} 
La Palme Reyes, M., et. al., 1999, ``Count Nouns, Mass Nouns, and their Transformations: a Unified Category-theoretic Semantics'', in: {\em Language, Logic and Concepts.}, Cambridge: MIT Press, 427--452.  
 

\bibitem{LFW64}
Lawvere, F. W., 1964, ``An Elementary Theory of the Category of Sets'' (ETACS), Proceedings of the National Academy of Sciences U.S.A., 52, 1506--1511. 

\bibitem{LFW65}
Lawvere, F. W., 1965, ``Algebraic Theories, Algebraic Categories, and Algebraic Functors'', Theory of Models, Amsterdam: North Holland, 413--418.  

\bibitem{LFW66}
Lawvere, F. W., 1966, ``The Category of Categories as a Foundation for Mathematics.'', Proceedings of the Conference on Categorical Algebra, La Jolla, New York: Springer-Verlag, 1--21. 

\bibitem{LFW69a}
Lawvere, F. W., 1969a, ``Diagonal Arguments and Cartesian Closed Categories.'', Category Theory, Homology Theory, and their Applications II, Berlin: Springer, 134--145.  

\bibitem{LFW69b}
Lawvere, F. W., 1969b, ``Adjointness in Foundations'', {\em Dialectica}, 23, 281--295.  

\bibitem{LFW70}
Lawvere, F. W., 1970, ``Equality in Hyper doctrines and Comprehension Schema as an Adjoint Functor'',in {\em Applications of Categorical Algebra}, Providence: AMS, 1-14.  

\bibitem{LT271}
Lawvere, F. W., 1971, ``Quantifiers and Sheaves'', Actes du Congr\'es International des Math\'ematiciens, Tome 1, Paris: Gauthier-Villars, 329--334. 

\bibitem{LFW72}
Lawvere, F. W., 1972, ``Introduction'', in{\em Toposes, Algebraic Geometry and Logic, Lecture Notes in Mathematics}, 274, Springer-Verlag, 1--12.  

\bibitem{LFW75}
Lawvere, F. W., 1975, ``Continuously Variable Sets: Algebraic Geometry = Geometric Logic.'', Proceedings of the Logic Colloquium Bristol 1973, Amsterdam: North Holland, 135--153. 

\bibitem{LFW76}
Lawvere, F. W., 1976, ``Variable Quantities and Variable Structures in Topoi'', Algebra, Topology, and Category Theory, New York: Academic Press, 101--131. 

\bibitem{LFW97}
Lawvere, F. W. \& Schanuel, S., 1997, Conceptual Mathematics: A First Introduction to Categories, Cambridge: Cambridge University Press. 

\bibitem{LFW66}
Lawvere, F. W.: 1966, ``The Category of Categories as a Foundation for Mathematics.'', in
\emph{Proc. Conf. Categorical Algebra- La Jolla}., Eilenberg, S. et al., eds. Springer--Verlag:
Berlin, Heidelberg and New York., pp. 1-20.

\bibitem{LFW63}
Lawvere, F. W.: 1963, Functorial Semantics of Algebraic Theories,
\emph{Proc. Natl. Acad. Sci. USA, Mathematics}, \textbf{50}: 869-872.

\bibitem{LFW69}
Lawvere, F. W.: 1969, \emph{Closed Cartesian Categories}., Lecture held as a guest of the
Romanian Academy of Sciences, Bucharest.

\bibitem{LFW92}
Lawvere, F. W., 1992, ``Categories of Space and of Quantity.'', {\em The Space of Mathematics, Foundations of Communication and Cognition.}, Berlin: De Gruyter, 14--30.  

\bibitem{LFW94a}
Lawvere, F. W., 1994a, ``Cohesive Toposes and Cantor's lauter Ensein.'', Philosophia Mathematica, 2, 1, 5--15. 

\bibitem{LFW94b}
Lawvere, F. W., 1994b, ``Tools for the Advancement of Objective Logic: Closed Categories and Toposes", The Logical Foundations of Cognition, Vancouver Studies in Cognitive Science, 4, Oxford: Oxford University Press, 43--56.  

\bibitem{LFW95}
Lawvere, H. W (ed.), 1995.{\em Springer Lecture Notes in Mathematics.} 274,:13-42. 

\bibitem{LFW2k}
Lawvere, F. W., 2000, ``Comments on the Development of Topos Theory.'', Development of Mathematics 1950-2000, Basel: Birkh\'user, 715--734. 

\bibitem{LFW2k2}
Lawvere, F. W., 2002, ``Categorical Algebra for Continuum Micro Physics'', Journal of Pure and Applied Algebra, 175, nos. 1-3, 267-287. 

\bibitem{LFW-RR2k3}
Lawvere, F. W. \& Rosebrugh, R., 2003, {\em Sets for Mathematics}, Cambridge: Cambridge University Press.  

\bibitem{LFWk3}
Lawvere, F. W., 2003, ``Foundations and Applications: Axiomatization and Education. New Programs and Open Problems in the Foundation of Mathematics.'', {\em Bulletin of Symbolic Logic}, 9, 2, 213--224. 

\bibitem{LFW63}
Lawvere, F.W., 1963, ``Functorial Semantics of Algebraic Theories.'', Proceedings of the National Academy of Sciences U.S.A., 50, 869--872. 

\bibitem{LT2k2}
Leinster, T., 2002, ``A Survey of Definitions of n-categories.'', Theory and Applications of Categories, (electronic), 10, 1--70. 

\bibitem{LiM-PV97}
Li, M. and P. Vitanyi: 1997, \emph{An introduction to Kolmogorov Complexity and its Applications}, Springer Verlag: New York.

\bibitem{Lofgren68}
L\"{o}fgren,  L.: 1968, ``An Axiomatic Explanation of Complete Self-Reproduction.'', \emph{Bulletin of Mathematical Biophysics}, \textbf{30}: 317-348

\bibitem{LS60}
Lubkin, S., 1960. Imbedding of abelian categories.,  {\em Trans. Amer. Math. Soc.}, \textbf{97}: 410-417.

\bibitem{LP-LFJV88}
Luisi, P. L. and F. J. Varela: 1988, Self-replicating micelles a chemical version of a minimal autopoietic system. Origins of Life and Evolution of Biospheres \textbf{19}(6): 633-643.

\bibitem{Mack1}
K. C. H. Mackenzie : Lie Groupoids and Lie Algebroids in
Differential Geometry, LMS Lect. Notes \textbf{124}, Cambridge
University Press, 1987

\bibitem{MCLSS48}
Mac Lane, S.: 1948. ``Groups, categories, and duality.'', {\em Proc. Natl. Acad. Sci.U.S.A},
\textbf{34}: 263-267.

\bibitem{MCLSS69}
Mac Lane, S., 1969, ``Foundations for Categories and Sets'', Category Theory, Homology Theory and their Applications II, Berlin: Springer, 146--164. 

\bibitem{MCLS69}
Mac Lane, S., 1969, ``One Universe as a Foundation for Category Theory'', Reports of the Midwest Category Seminar III, Berlin: Springer, 192--200. 

\bibitem{MCLS71}MacLane, S., 1971, ``Categorical algebra and Set-Theoretic Foundations'', Axiomatic Set Theory, Providence: AMS, 231--240. 

\bibitem{MCLS75}
Mac Lane, S., 1975, ``Sets, Topoi, and Internal Logic in Categories.'', in {Studies in Logic and the Foundations of Mathematics}, 80, Amsterdam: North Holland, 119--134. 

\bibitem{MCLS81}
Mac Lane, S., 1981, ``Mathematical Models: a Sketch for the Philosophy of Mathematics.'', \emph{American Mathematical Monthly}, 88, 7, 462--472.
 
\bibitem{MCLS86}
Mac Lane, S., 1986, \emph{Mathematics, Form and Function}, New York: Springer. 

\bibitem{MCLS88}
MacLane, S., 1988, Concepts and Categories in Perspective, in \emph{A Century of Mathematics in America}, Part I, Providence: AMS, 323--365. 

\bibitem{MCLS89}
Mac Lane, S., 1989, The Development of Mathematical Ideas by Collision: the Case of Categories and Topos Theory, in 
\emph{Categorical Topology and its Relation to Analysis, Algebra and Combinatorics}, Teaneck: World Scientific, 1--9.

\bibitem{MS-IM92}
S. Mac Lane and I. Moerdijk : Sheaves in Geometry and Logic- A first Introduction to Topos Theory, Springer Verlag, New York, 1992. 

\bibitem{MLS50}
MacLane, S., 1950, Dualities for Groups, \emph{Bulletin of the American Mathematical Society}, 56, 485--516. 

\bibitem{MCLS96} 
MacLane, S., 1996, Structure in Mathematics. Mathematical Structuralism., Philosophia Mathematica, 4, 2, 174-183. 

\bibitem{MCLS98}
MacLane, S., 1997, Categories for the Working Mathematician, 2nd edition, New York: Springer-Verlag. 

\bibitem{MCLS97}
MacLane, S., 1997, Categorical Foundations of the Protean Character of Mathematics., Philosophy of Mathematics Today, Dordrecht: Kluwer, 117--122. 

\bibitem{MaMo}
MacLane, S., and I.~Moerdijk. {\em Sheaves and Geometry in Logic: A First Introduction to Topos Theory}, Springer-Verlag, 1992.

\bibitem{Majid1}
Majid, S.: 1995, \emph{Foundations of Quantum Group Theory}, Cambridge Univ. Press: Cambridge, UK.

\bibitem{Majid2}
Majid, S.: 2002, \emph{A Quantum Groups Primer}, Cambridge Univ.Press: Cambridge, UK.

\bibitem{MM-RG95} 
Makkai, M. and Par\'e, R., 1989, Accessible Categories: the Foundations of Categorical Model Theory, Contemporary Mathematics 104, Providence: AMS. 

\bibitem{MM-RG77}
Makkai, M. and Reyes, G., 1977, \emph{First-Order Categorical Logic}, Springer Lecture Notes in Mathematics 611, New York: Springer. 

\bibitem{MM98}
Makkai, M., 1998, Towards a Categorical Foundation of Mathematics, in \emph{Lecture Notes in Logic}, 11, Berlin: Springer, 153--190. 

\bibitem{MM99}
Makkai, M., 1999, On Structuralism in Mathematics, in \emph{Language, Logic and Concepts}, 
Cambridge: MIT Press, 43--66. 

\bibitem{MM-RG95}
Makkei, M. \& Reyes, G., 1995, Completeness Results for Intuitionistic and Modal Logic in a Categorical Setting, 
\emph{Annals of Pure and Applied Logic}, 72, 1, 25--101. 

\bibitem{Mallios1}
Mallios, A. and I. Raptis: 2003, Finitary, Causal and Quantal Vacuum Einstein Gravity, \emph{Int. J. Theor. Phys.} \textbf{42}:
1479.

\bibitem{Manders}
Manders, K.L.: 1982, On the space-time ontology of physical theories, \emph{Philosophy of Science} \textbf{49} no. 4: 575--590.

\bibitem{MJP93}
Marquis, J.-P., 1993, Russell's Logicism and Categorical Logicisms, in \emph{Russell and Analytic Philosophy}, A. D. Irvine \& G. A. Wedekind, (eds.), Toronto, University of Toronto Press, 293--324.

\bibitem{MJP95}
Marquis, J.-P., 1995, Category Theory and the Foundations of Mathematics: Philosophical Excavations., \emph{Synthese}, 103, 421--447. 

\bibitem{MJP2k}
Marquis, J.-P., 2000, Three Kinds of Universals in Mathematics?, in {Logical Consequence: Rival Approaches and New Studies in Exact Philosophy: Logic, Mathematics and Science}, Vol. II, B. Brown and J. Woods, eds., Oxford: Hermes, 191-212, 2000 ,

\bibitem{MJP2k6} 
Marquis, J.-P., 2006, Categories, Sets and the Nature of Mathematical Entities, in: \emph{The Age of Alternative Logics. Assessing philosophy of logic and mathematics today}, J. van Benthem, G. Heinzmann, Ph. Nabonnand, M. Rebuschi, H.Visser, eds., Springer,181-192. 

\bibitem{Martins}
Martins, J. F and T. Porter: 2004, On Yetter's Invariant and an Extension of the Dijkgraaf-Witten Invariant to Categorical Groups, math.QA/0608484

\bibitem{MJP1999}
May, J.P. 1999, \emph{A Concise Course in Algebraic Topology}, The University of Chicago Press: Chicago.

\bibitem{MLC91}
Mc Larty, C., 1991, Axiomatizing a Category of Categories, \emph{Journal of Symbolic Logic}, 56, no. 4, 1243-1260. 

\bibitem{MLC92} 
Mc Larty, C., 1992, Elementary Categories, Elementary Toposes, Oxford: Oxford University Press.

\bibitem{MLC94}
Mc Larty, C., 1994, Category Theory in Real Time, \emph{Philosophia Mathematica}, 2, no. 1, 36-44.

Misra, B.,  I. Prigogine and M. Courbage.: 1979, Lyaponouv variables: Entropy and measurement in quantum mechanics,
\emph{Proc. Natl. Acad. Sci. USA} \textbf{78} (10): 4768--4772.

\bibitem{Mitchell1}
Mitchell, B.: 1965, \emph{Theory of Categories}, Academic Press:London.

\bibitem{Mitchell2}
Mitchell, B.: 1964, The full imbedding theorem. \emph{Amer. J. Math}. \textbf{86}: 619-637.

\bibitem{MI-P2k2}
Moerdijk, I. \& Palmgren, E., 2002, Type Theories, Toposes and Constructive Set Theory: Predicative Aspects of AST., Annals of Pure and Applied Logic, 114, no. 1--3, 155--201. 

\bibitem{MO98}
Moerdijk, I., 1998, Sets, Topoi and Intuitionism., \emph{Philosophia Mathematica}, 6, no. 2, 169-177.

\bibitem{Moer1}
I. Moerdijk : Classifying toposes and foliations, {\it Ann. Inst. Fourier, Grenoble} \textbf{41}, 1 (1991) 189-209.

\bibitem{Moer2}
I. Moerdijk : Introduction to the language of stacks and gerbes, $arXiv:math.AT/0212266$ (2002).

\bibitem{MK62}
Morita, K. 1962. Category isomorphism and endomorphism rings of modules,
{\em Trans. Amer. Math. Soc.}, \textbf{103}: 451-469.

\bibitem{MK70}
Morita, K. , 1970. Localization in categories of modules. I., {\em Math. Z.}, 
\textbf{114}: 121-144.

\bibitem{Mostow}
M. A. Mostow : The differentiable space structure of Milnor classifying spaces, simplicial complexes, 
and geometric realizations, \emph{J. Diff. Geom.} \textbf{14} (1979) 255-293.

\bibitem{OB69}
Oberst, U.: 1969, Duality theory for Grothendieck categories., \emph{Bull. Amer. Math. Soc.} \textbf{75}: 1401-1408.

\bibitem{ORT70}
Oort, F.: 1970. On the definition of an abelian category. \emph{Proc. Roy. Neth. Acad. Sci}. \textbf{70}: 13-02.

\bibitem{OO31}
Ore, O., 1931, Linear equations on non-commutative fields, {\em Ann. Math.}
\textbf{32}: 463-477.

\bibitem{Penrose}
Penrose, R.: 1994, \emph{Shadows of the Mind}, Oxford University
Press: Oxford.

\bibitem{PLR}
Plymen, R.J. and P. L. Robinson: 1994,  \emph{Spinors in Hilbert Space}, Cambridge Tracts in Math. 
\textbf{114}, Cambridge Univ. Press, Cambridge.

\bibitem{PB70}
Pareigis, B., 1970, Categories and Functors, New York: Academic Press. 

\bibitem{PMC 2k4}
Pedicchio, M. C. \& Tholen, W., 2004, Categorical Foundations, Cambridge: Cambridge University Press. 

\bibitem{PAM2k}
Pitts, A. M., 2000, Categorical Logic, in \textbf{Handbook of Logic in Computer Science}, Vol.5, Oxford: Oxford Unversity Press, 39--128.

\bibitem{PB2k} 
Plotkin, B., 2000, ``Algebra, Categories and Databases.'', Handbook of Algebra, Vol. 2, Amsterdam: Elsevier, 79--148. 

\bibitem{Popescu}
Popescu, N.: 1973, \emph{Abelian Categories with Applications to Rings and Modules.} New York and London: Academic Press., 2nd edn. 1975. \emph{(English translation by I.C. Baianu)}.

\bibitem{Pradines1966}
Pradines, J.: 1966, ``Th\'eorie de Lie pour les groupoides diff\'erentiable, relation entre propri\'etes locales et globales.'', \emph{C. R. Acad Sci. Paris S\'er. A} \textbf{268}: 907-910.

\bibitem{Prib1-1991}
Pribram, K. H. 1991. \emph{Brain and Perception: Holonomy and Structure in Figural processing}, Lawrence Erlbaum Assoc.: Hillsdale.

\bibitem{Prib2k}
Pribram, K. H. 2000. ``Proposal for a quantum physical basis for selective learning.'', in (Farre, ed.)
\emph{Proceedings ECHO IV} 1-4.

\bibitem{Prigogine}
Prigogine, I.: 1980, \emph{From Being to Becoming : Time and Complexity in the Physical Sciences}, 
W. H. Freeman and Co.: San Francisco.

\bibitem{RZ2k}
Raptis, I. and R. R. Zapatrin. 2000. Quantisation of discretized spacetimes and the correspondence principle, \emph{Int. Jour. Theor. Phys.} \textbf{39}: 1.

\bibitem{Raptis1-2k3}
Raptis, I. 2003. Algebraic quantisation of causal sets, \emph{Int. Jour. Theor. Phys.} \textbf{39}: 1233.

\bibitem{Raptis2}
I. Raptis.2004. ``Quantum space--time as a quantum causal set.'', $arXiv:gr--qc/0201004$.

\bibitem{RGZH91}
Reyes, G. and Zolfaghari, H., 1991, Topos-theoretic Approaches to Modality, 
\emph{Category Theory (Como 1990), Lecture Notes in Mathematics}, 1488, Berlin: Springer, 359--378. 

\bibitem{RGZH96}
Reyes, G. andZolfaghari, H., 1996, Bi-Heyting Algebras, Toposes and Modalities, \emph{Journal of Philosophical Logic}, 25, no. 1, 25--43. 

\bibitem{RG74}
Reyes, G., 1974, From Sheaves to Logic, in \emph{Studies in Algebraic Logic}, A. Daigneault, ed., Providence: AMS. 

\bibitem{RG91}
Reyes, G., 1991, A Topos-theoretic Approach to Reference and Modality., \emph{Notre Dame Journal of Formal Logic}, 32, no. 3, 359-391.

\bibitem{Rieffel}
M. A. Rieffel : Group C*--algebras as compact quantum metric spaces, \emph{Documenta Math.} \textbf{7} (2002), 605-651.

\bibitem{Roberts}
Roberts, J. E.: 2004, More lectures on algebraic quantum field theory, in A. Connes, et al. \emph{Noncommutative Geometry}, Springer: Berlin and New York.

\bibitem{RSE-KEP94} 
Rodabaugh, S. E. \& Klement, E. P., eds., Topological and Algebraic Structures in Fuzzy Sets: A Handbook of Recent Developments in the Mathematics of Fuzzy Sets, Trends in Logic, 20, Dordrecht: Kluwer. 

\bibitem{Rota} 
G. C. Rota. On the foundation of combinatorial theory, I. The theory of M\"obius functions, \emph{Zetschrif f\"ur
Wahrscheinlichkeitstheorie} \textbf{2} (1968), 340.

\bibitem{Rovelli1}
Rovelli, C. 1998. Loop Quantum Gravity, in N. Dadhich, et al. \emph{Living Reviews in Relativity} (refereed electronic journal) $http:www.livingreviews.org/Articles/Volume1/1998~1 rovelli$.

\bibitem{Schrod2}
Schr\"odinger E.: 1945, \emph{What is Life?}, Cambridge University Press:
Cambridge, UK.

\bibitem{SPJ2k} 
Scott, P. J., 2000, {\em Some Aspects of Categories in Computer Science}, Handbook of Algebra, Vol. 2, Amsterdam: North Holland, 3--77. 

\bibitem{SRAG84}
Seely, R. A. G., 1984, Locally Cartesian Closed Categories and Type Theory, \emph{Mathematical Proceedings of the Cambridge Mathematical Society}, 95, no. 1, 33-48. 

\bibitem{SHS2k5}
Shapiro, S., 2005, Categories, Structures and the Frege-Hilbert Controversy: the Status of Metamathematics, 
\emph{Philosophia Mathematica}, 13, 1, 61--77.

\bibitem{Sorkin}
Sorkin, R.D.: 1991, Finitary substitute for continuous topology,
\emph{Int. J. Theor. Phys.} \textbf{30} No. 7.: 923--947.

\bibitem{Smolin}
Smolin, L.: 2001, \emph{Three Roads to Quantum Gravity}, Basic Books: New York.

\bibitem{Spanier}
Spanier, E. H.: 1966, \emph{Algebraic Topology}, McGraw Hill: New York.

\bibitem{Stapp}
Stapp, H.: 1993, \emph{Mind, Matter and Quantum Mechanics},
Springer Verlag: Berlin--Heidelberg--New York.

\bibitem{Stewart- Golub}
Stewart, I. and Golubitsky, M. : 1993. \emph{Fearful Symmetry: Is God a Geometer?}, Blackwell: Oxford, UK.

\bibitem{Szabo}
Szabo, R. J.: 2003, Quantum field theory on non-commutative spaces,
\emph{Phys. Rep.} \textbf{378}: 207--209.

\bibitem{TP96} 
Taylor, P., 1996, Intuitionistic sets and Ordinals, \emph{Journal of Symbolic Logic}, 61 : 705-744.
 
\bibitem{TP99}
Taylor, P., 1999, \emph{Practical Foundations of Mathematics}, Cambridge: Cambridge University Press. 

\bibitem{Unruh}
Unruh, W.G.: 2001, Black holes, dumb holes, and entropy, in C. Callender and N. Hugget (eds. ) \emph{Physics Meets Philosophy at the Planck scale}, Cambridge University Press, pp. 152-173.

\bibitem{VdHG-MI84a}
Van der Hoeven, G. and Moerdijk, I., 1984a, Sheaf Models for Choice Sequences, Annals of Pure and Applied Logic, 27, no. 1, 63--107. 

\bibitem{Varilly}
V\'arilly, J. C.: 1997, An introduction to noncommutative geometry \\ arXiv:physics/9709045
London.

\bibitem{vonNeumann}
von Neumann, J.: 1932, \emph{Mathematische Grundlagen der Quantenmechanik}, Springer: Berlin.

\bibitem{Weinstein96}
Weinstein, A.: 1996, Groupoids : unifying internal and external symmetry, \emph{Notices of the Amer. Math. Soc.} \textbf{43}: 744--752.

\bibitem{WB83}
Wess J. and J. Bagger: 1983, \emph{Supersymmetry and Supergravity}, Princeton University Press: Princeton, NJ.

\bibitem{Weinberg95}
Weinberg, S.: 1995, \emph{The Quantum Theory of Fields}  vols. 1 to 3, Cambridge Univ. Press.

\bibitem{Wheeler83}
Wheeler, J. and W. Zurek: 1983, \emph{Quantum Theory and Measurement}, Princeton University Press: Princeton, NJ.

\bibitem{Whitehead1941}
Whitehead, J. H. C.: 1941, On adding relations to homotopy groups, \emph{Annals of Math.} \textbf{42} (2): 409--428.

\bibitem{Woit}
Woit, P.: 2006, \emph{Not Even Wrong: The Failure of String Theory and the Search for Unity in Physical Laws}, Jonathan Cape.

\bibitem{WRJ2k4}
Wood, R.J., 2004, Ordered Sets via Adjunctions, In: \emph{Categorical Foundations}, M. C. Pedicchio \& W. Tholen, eds., Cambridge: Cambridge University Press. 

\end{thebibliography}

\subsection{Literature references for supercategories, axiomatics and applications}

\begin{thebibliography}{99}

\bibitem{AR2k5}
Assadollahi, R. and Brigitte Rockstroh. 2005. Neuromagnetic brain responses to words from semantic sub- and supercategories., {\em BMC Neurosci.},{6}: 57. 
\PMlinkexternal{PDF download}{http://www.pubmedcentral.nih.gov/articlerender.fcgi?artid=1236933}

\bibitem{Urs2k7P}
Urs Schreiber. 2007. Supercategories., 
\PMlinkexternal{(8 pp Preprint)}{http://www.math.uni-hamburg.de/home/schreiber/scat.pdf}.

\bibitem{}
E. Lowen-Colebunders and R. Lowen.: 1997. 
\PMlinkexternal{Supercategories of Top and the Inevitable Emergence of Topological Constructs.}{http://books.google.com/books?id=dV6WtepcZLkC&pg=PA969&lpg=PA969&dq=supercategories}.
 pp. 969-1027 in {\em Handbook of the History of General Topology},
Charles E. Aull, R. Lowen, Eds. Springer: Berlin, ISBN 079236970X, 9780792369707.

\bibitem{ICBM}
I.C.Baianu and M. Marinescu: 1968, Organismic Supercategories: Towards a Unitary Theory of Systems. \emph{Bulletin of Mathematical Biophysics} \textbf{30}, 148-159.

\bibitem{ICB3}
I.C. Baianu: 1970, Organismic Supercategories: II. On Multistable Systems. \emph{Bulletin of Mathematical Biophysics}, \textbf{32}: 539-561.

\bibitem{ICB1}
I.C. Baianu : 1971a, Organismic Supercategories and Qualitative Dynamics of Systems. \emph{Ibid.}, \textbf{33} (3), 339--354.

\bibitem{ICB71}
I.C. Baianu: 1971b, Categories, Functors and Quantum Algebraic Computations, in P. Suppes (ed.), \emph{Proceed. Fourth Intl. Congress Logic-Mathematics-Philosophy of Science}, September 1--4, 1971, University of Bucharest.

\bibitem{ICB04b}
I.C. Baianu: \L ukasiewicz-Topos Models of Neural Networks, Cell Genome and Interactome Nonlinear Dynamics). CERN Preprint EXT-2004-059. \textit{Health Physics and Radiation Effects} (June 29, 2004). 

\bibitem{BBGG1}
I.C. Baianu, Brown R., J. F. Glazebrook, and Georgescu G.: 2006, Complex Nonlinear Biodynamics in 
Categories, Higher Dimensional Algebra and \L ukasiewicz--Moisil Topos: Transformations of
Neuronal, Genetic and Neoplastic networks, \emph{Axiomathes} \textbf{16} Nos. 1--2, 65--122.

\bibitem{ICBs5}
I.C. Baianu and D. Scripcariu: 1973, On Adjoint Dynamical Systems. \emph{The Bulletin of Mathematical Biophysics}, \textbf{35}(4), 475--486.

\bibitem{ICB5}
I.C. Baianu: 1973, Some Algebraic Properties of \emph{\textbf{(M,R)}} -- Systems. \emph{Bulletin of Mathematical Biophysics} \textbf{35}, 213-217.

\bibitem{ICBm2}
I.C. Baianu and M. Marinescu: 1974, A Functorial Construction of \emph{\textbf{(M,R)}}-- Systems. \emph{Revue Roumaine de Mathematiques Pures et Appliquees} \textbf{19}: 388-391.

\bibitem{ICB6}
I.C. Baianu: 1977, A Logical Model of Genetic Activities in \L ukasiewicz Algebras: The Non-linear Theory. \emph{Bulletin of Mathematical Biophysics}, \textbf{39}: 249-258.

\bibitem{ICB7}
I.C. Baianu: 1980, Natural Transformations of Organismic Structures. \emph{Bulletin of Mathematical Biophysics}
\textbf{42}: 431-446.

\bibitem{ICB8}
I.C. Baianu: 1983, Natural Transformation Models in Molecular Biology., in \emph{Proceedings of the SIAM Natl. Meet}., Denver, CO.; Eprint at cogprints.org as No. 3675/0l (Naturaltransfmolbionu6.pdf).

\bibitem{ICB9}
I.C. Baianu: 1984, A Molecular-Set-Variable Model of Structural and Regulatory Activities in Metabolic and Genetic Networks., \emph{FASEB Proceedings} \textbf{43}, 917.

\bibitem{ICB2}
I.C. Baianu: 1987a, Computer Models and Automata Theory in Biology and Medicine.,  in M. Witten (ed.), 
\emph{Mathematical Models in Medicine}, vol. 7., Pergamon Press, New York, 1513--1577; \emph{CERN Preprint No. EXT-2004-072}. 

\bibitem{ICB9b}
I.C. Baianu: 1987b, Molecular Models of Genetic and Organismic Structures, in \emph{Proceed. Relational Biology Symp.} Argentina; \emph{CERN Preprint No.EXT-2004-067}. 

\bibitem{BGG2}
I.C. Baianu, Glazebrook, J. F. and G. Georgescu: 2004, Categories of Quantum Automata and 
N-Valued \L ukasiewicz Algebras in Relation to Dynamic Bionetworks, \textbf{(M,R)}--Systems and
Their Higher Dimensional Algebra, \emph{Abstract and Preprint of Report}. 

\bibitem{BHS2}
R. Brown R, P.J. Higgins, and R. Sivera.: \textit{``Non--Abelian Algebraic Topology''}. (\textit{vol.2 in preparation}
(2008)

\bibitem{BGB2}
R. Brown, J. F. Glazebrook and I. C. Baianu: A categorical and higher dimensional algebra framework for complex systems and spacetime structures, \emph{Axiomathes} \textbf{17}:409--493, (2007).

\bibitem{BM}
R. Brown and G. H. Mosa: Double algebroids and crossed modules of algebroids, University of Wales--Bangor, Maths Preprint, 1986.

\bibitem{BS}
R. Brown  and C.B. Spencer: Double groupoids and crossed modules,
\emph{Cahiers Top. G\'eom.Diff.} \textbf{17} (1976), 343--362.

\bibitem{LW1}
W.F. Lawvere: 1963. Functorial Semantics of Algebraic Theories. \emph{Proc. Natl. Acad. Sci. USA}, 50: 869--872

\bibitem{LW2}
W. F. Lawvere: 1966. The Category of Categories as a Foundation for Mathematics. , In {\em Proc. Conf. Categorical Algebra--La Jolla}, 1965, Eilenberg, S et al., eds. Springer --Verlag: Berlin, Heidelberg and New York, pp. 1--20.

\bibitem{LO68}
L. L$\ddot{o}$fgren: 1968. On Axiomatic Explanation of Complete Self--Reproduction. \emph{Bull. Math. Biophysics}, 
\textbf{30}: 317--348. 

\end{thebibliography}
%%%%%
%%%%%
\end{document}
