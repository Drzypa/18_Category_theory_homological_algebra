\documentclass[12pt]{article}
\usepackage{pmmeta}
\pmcanonicalname{InterchangeLaw}
\pmcreated{2013-03-22 18:27:08}
\pmmodified{2013-03-22 18:27:08}
\pmowner{CWoo}{3771}
\pmmodifier{CWoo}{3771}
\pmtitle{interchange law}
\pmrecord{5}{41114}
\pmprivacy{1}
\pmauthor{CWoo}{3771}
\pmtype{Definition}
\pmcomment{trigger rebuild}
\pmclassification{msc}{18A99}
\pmclassification{msc}{08A99}

\usepackage{amssymb,amscd}
\usepackage{amsmath}
\usepackage{amsfonts}
\usepackage{mathrsfs}

% used for TeXing text within eps files
%\usepackage{psfrag}
% need this for including graphics (\includegraphics)
%\usepackage{graphicx}
% for neatly defining theorems and propositions
\usepackage{amsthm}
% making logically defined graphics
%%\usepackage{xypic}
\usepackage{pst-plot}

% define commands here
\newcommand*{\abs}[1]{\left\lvert #1\right\rvert}
\newtheorem{prop}{Proposition}
\newtheorem{thm}{Theorem}
\newtheorem{ex}{Example}
\newcommand{\real}{\mathbb{R}}
\newcommand{\pdiff}[2]{\frac{\partial #1}{\partial #2}}
\newcommand{\mpdiff}[3]{\frac{\partial^#1 #2}{\partial #3^#1}}
\begin{document}
Let $S$ be a set and $\circ$ and $\bullet$ two \PMlinkname{partial}{PartialFunction} binary operations on $S$.  Then $\circ$ and $\bullet$ are said to satisfy the \emph{interchange law} if
$$(a\circ b)\bullet (c\circ d)=(a\bullet c)\circ (b\bullet d),$$
provided that the operations are defined on both sides of the equation.

An element $e\in S$ is a $\circ$-identity, or an identity with respect to $\circ$, if $e\circ a=a\circ e=a$ provided the operations are defined.

\begin{prop} If $\circ$ is a total function (defined for all of $S\times S$), then there is at most one $\circ$-identity. \end{prop}
\begin{proof}  If $e$ and $f$ are both $\circ$-idenities, then $e=e\circ f=f$.
\end{proof}

\begin{prop}  If both $\circ$ and $\bullet$ are total functions, and identities exist and the same with respect to both operations, then $\circ=\bullet$ and is commutative.\end{prop}
\begin{proof}  Suppose that $e$ is both the $\circ$-identity and the $\bullet$-identity.  Then, according to the interchange law, $a\bullet d = (a\circ e)\bullet (e\circ d)=(a\bullet e)\circ (e\bullet d) = a \circ d$, showing that $\bullet = \circ$.  Again, using the interchange law, $a\bullet d=(e\circ a)\bullet (d\circ e)= (e\bullet d)\circ (a\bullet e) = d\circ a = d\bullet a$, showing that $\bullet$ is commutative.
\end{proof}
%%%%%
%%%%%
\end{document}
