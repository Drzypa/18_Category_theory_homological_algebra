\documentclass[12pt]{article}
\usepackage{pmmeta}
\pmcanonicalname{AdditiveFunctor}
\pmcreated{2013-03-22 18:08:36}
\pmmodified{2013-03-22 18:08:36}
\pmowner{CWoo}{3771}
\pmmodifier{CWoo}{3771}
\pmtitle{additive functor}
\pmrecord{5}{40698}
\pmprivacy{1}
\pmauthor{CWoo}{3771}
\pmtype{Definition}
\pmcomment{trigger rebuild}
\pmclassification{msc}{18E05}
\pmrelated{PreAdditiveFunctors}
\pmrelated{CategoryOfAdditiveFractions}

\usepackage{amssymb,amscd}
\usepackage{amsmath}
\usepackage{amsfonts}
\usepackage{mathrsfs}

% used for TeXing text within eps files
%\usepackage{psfrag}
% need this for including graphics (\includegraphics)
%\usepackage{graphicx}
% for neatly defining theorems and propositions
\usepackage{amsthm}
% making logically defined graphics
%%\usepackage{xypic}
\usepackage{pst-plot}

% define commands here
\newcommand*{\abs}[1]{\left\lvert #1\right\rvert}
\newtheorem{prop}{Proposition}
\newtheorem{thm}{Theorem}
\newtheorem{ex}{Example}
\newcommand{\real}{\mathbb{R}}
\newcommand{\pdiff}[2]{\frac{\partial #1}{\partial #2}}
\newcommand{\mpdiff}[3]{\frac{\partial^#1 #2}{\partial #3^#1}}
\begin{document}
Let $\mathcal{A}$ and $\mathcal{B}$ be ab-categories.  A functor $F:\mathcal{A}\to \mathcal{B}$ is called an \emph{additive functor} if, for any objects $A,B$ in $\mathcal{A}$, the function $$F_{(A,B)}: \hom(A,B)\to \hom(F(A),F(B))$$ given by $F_{(A,B)}(f)=F(f)$ is a group homomorphism.  In other words, if $f,g: A\to B$ are two morphisms with common domain $A$ and codomain $B$, then $$F(f+g)=F(f)+F(g).$$

For example, the hom functor $\hom(A,-)$ where $A$ is an object in an abelian category, is additive.

\textbf{Remark}.  It can be shown that any exact functor between abelian categories is additive.

More to come...
%%%%%
%%%%%
\end{document}
