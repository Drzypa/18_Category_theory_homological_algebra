\documentclass[12pt]{article}
\usepackage{pmmeta}
\pmcanonicalname{ProofOfLeraysTheoremviaSpectralSequences}
\pmcreated{2013-03-22 14:42:35}
\pmmodified{2013-03-22 14:42:35}
\pmowner{mathcam}{2727}
\pmmodifier{mathcam}{2727}
\pmtitle{proof of Leray's Theorem (via spectral sequences)}
\pmrecord{12}{36329}
\pmprivacy{1}
\pmauthor{mathcam}{2727}
\pmtype{Proof}
\pmcomment{trigger rebuild}
\pmclassification{msc}{18G60}

% this is the default PlanetMath preamble.  as your knowledge
% of TeX increases, you will probably want to edit this, but
% it should be fine as is for beginners.

% almost certainly you want these
\usepackage{amssymb}
\usepackage{amsmath}
\usepackage{amsfonts}

% used for TeXing text within eps files
%\usepackage{psfrag}
% need this for including graphics (\includegraphics)
%\usepackage{graphicx}
% for neatly defining theorems and propositions
%\usepackage{amsthm}
% making logically defined graphics
%%%\usepackage{xypic}

% there are many more packages, add them here as you need them

% define commands here
\newcommand{\F}{\mathcal F}
\begin{document}
Let's consider an \PMlinkescapetext{acyclic} resolution (i.e. a \PMlinkname{resolution}{ResolutionOfASheaf} by \PMlinkname{acyclic sheaves}{AcyclicSheaf}) $\F^\bullet$ of $\F$ and the double complex $K^{\bullet,\bullet}=
\check C^\bullet(\mathcal U,\F^\bullet)$ endowed with the differentials
$$
\begin{aligned}
& \partial\colon\check C^p(\mathcal U,\F^\bullet)\to
\check C^{p+1}(\mathcal U,\F^\bullet),\\
& \bar\partial\colon\check C^\bullet(\mathcal U,\F^{q})\to
\check C^\bullet(\mathcal U,\F^{q+1}),
\end{aligned}
$$
which are respectively the \PMlinkname{Cech}{CechCohomologyGroup2} one and the one \PMlinkescapetext{induced} by the resolution.

The first filtration of this complex gives rise to a spectral sequence whose \PMlinkescapetext{terms} of the first generation are:
$$
{^\prime\!E^{p,q}_r}=H^q_{\bar\partial}
(\check C^p(\mathcal U,\F^\bullet))=\prod_{i_0,\dots,i_p\in I}
H^q(U_{i_0\cdots i_p},\F),
$$
where $I$ is the set indexing the covering and the last equality is obtained in virtue of the De Rham-Weil theorem.
Thanks to the hypothesis on $\mathcal U$, we have
$$
{^\prime\!E^{p,q}_1}=
\begin{cases}
\check C^p(\mathcal U,\F) & \text{if $q=0$} \\
0 & \text{if $q\ge 1$}.
\end{cases}
$$
Then
$$
\begin{aligned}
{^\prime\!E^{p,q}_2}
&=H^p_\partial(H^q_{\bar\partial}
(\check C^\bullet(\mathcal U,\F^\bullet))) \\
&=
\begin{cases}
H^p_\partial(\check C^\bullet(\mathcal U,\F)) & \text{if $q=0$} \\
0 & \text{if $q\ge 1$}.
\end{cases}
\end{aligned}
$$
So ${^\prime\!E^{p,0}_2}=\check H^p(\mathcal U,\F)$ and 
${^\prime\!E^{p,q}_2}=0$ if $q\ne 0$. The general \PMlinkescapetext{theory} of spectral sequences of a double complex now yields:
$$
H^l(K^\bullet)={^\prime\!E^{l,0}_2}=\check H^l(\mathcal U,\F).
$$
Now we consider the second filtration to obtain a spectral sequence whose \PMlinkescapetext{terms} of the first generations are:
$$
{^{\prime\prime}\!E^{p,q}_1}=H^q_\partial(\check C^\bullet(\mathcal U,\F^{p}))
=\check H^q(\mathcal U,\F^{p}).
$$
The resolution being \PMlinkescapetext{acyclic}, we obtain:
$$
{^{\prime\prime}\!E^{p,q}_1}=
\begin{cases}
\F^p(X) & \text{if $q=0$} \\
0 & \text{if $q\ge 1$},
\end{cases}
$$
and so
$$
\begin{aligned}
{^{\prime\prime}\!E^{p,q}_2}
&=H^p_{\bar\partial}(H^q_\partial(\check C^\bullet(\mathcal U,\F^\bullet)))
&=
\begin{cases}
H^p(\F^\bullet(X)) & \text{if $q=0$} \\
0 & \text{if $q\ge 1$}.
\end{cases}
\end{aligned}
$$
But then ${^{\prime\prime}\!E^{p,0}_2}=H^p(X,\F)$ thanks to the De Rham-Weil theorem and ${^{\prime\prime}\!E^{p,q}_2}=0$ if $q\ne 0$ and so we get:
$$
H^l(K^\bullet)={^{\prime\prime}\!E^{l,0}_2}=H^l(X,\F),
$$
which leads to the claim.
%%%%%
%%%%%
\end{document}
