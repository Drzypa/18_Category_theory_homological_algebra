\documentclass[12pt]{article}
\usepackage{pmmeta}
\pmcanonicalname{DiscreteCategory}
\pmcreated{2013-03-22 16:15:09}
\pmmodified{2013-03-22 16:15:09}
\pmowner{CWoo}{3771}
\pmmodifier{CWoo}{3771}
\pmtitle{discrete category}
\pmrecord{8}{38357}
\pmprivacy{1}
\pmauthor{CWoo}{3771}
\pmtype{Definition}
\pmcomment{trigger rebuild}
\pmclassification{msc}{18A05}
\pmdefines{trivial category}
\pmdefines{empty category}
\pmdefines{empty functor}

\usepackage{amssymb,amscd}
\usepackage{amsmath}
\usepackage{amsfonts}

% used for TeXing text within eps files
%\usepackage{psfrag}
% need this for including graphics (\includegraphics)
%\usepackage{graphicx}
% for neatly defining theorems and propositions
%\usepackage{amsthm}
% making logically defined graphics
%%\usepackage{xypic}
\usepackage{pst-plot}
\usepackage{psfrag}

% define commands here

\begin{document}
A category $\mathcal{C}$ is said to be a \emph{discrete category} if the only morphisms in $\mathcal{C}$ are the identity morphisms associated with each of the objects in $\mathcal{C}$.

For example, every set can be regarded as a discrete category.  The objects are just the elements of the set.  Furthermore, $\hom(a,a)$ is identified with $\lbrace a\rbrace$, and $\hom(a,b)=\varnothing$ if $a\ne b$.

\textbf{Remarks}.  
\begin{itemize}
\item
A discrete category with one object is called a \emph{trivial category}.  For every category $\mathcal{C}$, there is only one functor from $\mathcal{C}$ to a trivial category.  Hence, any trivial category is a terminal object in \textbf{Cat}, the category of small categories.
\item
A discrete category with no objects is called the \emph{empty category}.  For every category $\mathcal{C}$, there is only one functor from the empty category to $\mathcal{C}$.  This functor is called the \emph{empty functor}, where both the object and morphism functions are the empty set $\varnothing$.  Thus, the empty category is the initial object in \textbf{Cat}.
\item
Given any category $\mathcal{C}$, the smallest subcategory consisting of all objects in $\mathcal{C}$ is discrete, which is also the largest discrete subcategory in $\mathcal{C}$ (largest in the sense that it contains all objects of $\mathcal{C}$).  For every object $X\in \mathcal{C}$, we can associate the trivial category $\mathcal{C}_X$ consisting of one object, $X$, and one morphism $1_X$.
\end{itemize}
%%%%%
%%%%%
\end{document}
