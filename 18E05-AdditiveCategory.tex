\documentclass[12pt]{article}
\usepackage{pmmeta}
\pmcanonicalname{AdditiveCategory}
\pmcreated{2013-03-22 15:55:03}
\pmmodified{2013-03-22 15:55:03}
\pmowner{CWoo}{3771}
\pmmodifier{CWoo}{3771}
\pmtitle{additive category}
\pmrecord{8}{37922}
\pmprivacy{1}
\pmauthor{CWoo}{3771}
\pmtype{Definition}
\pmcomment{trigger rebuild}
\pmclassification{msc}{18E05}
\pmrelated{AbelianCategory}
\pmrelated{AlternativeDefinitionOfAnAbelianCategory}

\usepackage{amssymb,amscd}
\usepackage{amsmath}
\usepackage{amsfonts}

% used for TeXing text within eps files
%\usepackage{psfrag}
% need this for including graphics (\includegraphics)
%\usepackage{graphicx}
% for neatly defining theorems and propositions
\usepackage{amsthm}
% making logically defined graphics
%%\usepackage{xypic}

% define commands here

\begin{document}
Let $\mathcal{C}$ be a category.  Then $\mathcal{C}$ is an \emph{additive category} if 
\begin{enumerate}
\item $\mathcal{C}$ is a preadditive category, and
\item for every pair of objects $A,B$ in $\mathcal{C}$, their \PMlinkname{product}{CategoricalDirectProduct} exists.
\end{enumerate}

\textbf{Proposition}.  In a preadditive category, coproduct of two objects exists iff their product exists.  Furthermore, they are isomorphic.
\begin{proof}We shall prove the fact if the product $D$ of objects $A$ and $B$ exists, then $D$ is also their coproduct.  The other direction is dual.

Suppose $D$ is the product of $A$ and $B$, with morphisms
$$\xymatrix@1{D\ar[r]^{\pi_A}&A}\qquad\mbox{ and }\qquad\xymatrix@1{D\ar[r]^{\pi_B}&B}.$$
From these two morphisms, we construct two commutative diagrams
$$\xymatrix{&A \\ A \ar[ur]^1 \ar[dr]_0 \ar@{-->}[r]^{\alpha} & D
\ar[u]_{\pi_A} \ar[d]^{\pi_B}\\ &B}
\qquad\mbox{ and }\qquad
\xymatrix{&A \\ B \ar[ur]^0 \ar[dr]_1 \ar@{-->}[r]^{\beta} & D
\ar[u]_{\pi_A} \ar[d]^{\pi_B}\\ &B}$$
where $0$ and $1$ are zero morphisms and identity morphisms on $A$ and $B$, and $\alpha$ and $\beta$ are morphisms based on the definition of the product $D$.

Then it's not hard to see that $D$ is a coproduct of $A$ and $B$ with morphisms $\alpha$ and $\beta$, for if $r:A\rightarrow C$ and $s:B\rightarrow C$ are two morphisms into an object $C$, we can form two morphisms $r\pi_A$ and $s\pi_B$, both from $D$ to $C$.  Since $\operatorname{hom}(D,C)$ is an abelian group, these two can then be added to form $f:=r\pi_A+s\pi_B$.  Then $f\alpha=(r\pi_A+s\pi_B)\alpha=r$, and similarly $f\beta=s$.  This shows that $D$ is also the coproduct of $A$ and $B$ with morphisms $\alpha$ and $\beta$.
\end{proof}
An easy way to remember the relationships among the various morphisms in the above proof are the following two matrix products:
\begin{center}
$
\begin{pmatrix}
\pi_A \\
\pi_B 
\end{pmatrix}
\begin{pmatrix}
\alpha & \beta 
\end{pmatrix}
=
\begin{pmatrix}
1 & 0 \\
0 & 1 
\end{pmatrix}
\qquad
\mbox{ and } 
\qquad
\begin{pmatrix}
r & s 
\end{pmatrix}
\begin{pmatrix}
\pi_A \\
\pi_B 
\end{pmatrix}
=f$.
\end{center}

As a result of the above proposition, in an additive category, finite products and finite coproducts are synonymous.  Given objects $A,B$, we denote $A\oplus B$ to be their product.  We also call it the \emph{direct sum} of $A$ and $B$.

Many preadditive categories are also examples of additive categories.  The category $\textbf{CyclGrp}$ of cyclic groups as the subcategory of the category of abelian groups is an example of a preadditive category that is not additive, for the product of two cyclic groups $\mathbb{Z}/p \mathbb{Z}$ and $\mathbb{Z}/q \mathbb{Z}$ exists in $\textbf{CyclGrp}$ only when $p$ and $q$ are coprime.

%%%%%
%%%%%
\end{document}
