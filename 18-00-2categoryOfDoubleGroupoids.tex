\documentclass[12pt]{article}
\usepackage{pmmeta}
\pmcanonicalname{2categoryOfDoubleGroupoids}
\pmcreated{2013-03-22 19:19:46}
\pmmodified{2013-03-22 19:19:46}
\pmowner{bci1}{20947}
\pmmodifier{bci1}{20947}
\pmtitle{2-category of double groupoids}
\pmrecord{9}{42274}
\pmprivacy{1}
\pmauthor{bci1}{20947}
\pmtype{Definition}
\pmcomment{trigger rebuild}
\pmclassification{msc}{18-00}
\pmclassification{msc}{18C10}
%\pmkeywords{category}
%\pmkeywords{double groupoids}
\pmrelated{Bijection}
\pmrelated{DoubleCategory}
\pmrelated{2Category}
\pmrelated{FundamentalGroupoidFunctor}
\pmrelated{ThinSquare}
\pmdefines{2-category of groupoids}

\endmetadata

% this is the default PlanetMath preamble. as your knowledge
% of TeX increases, you will probably want to edit this, but
\usepackage{amsmath, amssymb, amsfonts, amsthm, amscd, latexsym}
%%\usepackage{xypic}
\usepackage[mathscr]{eucal}
% define commands here
\theoremstyle{plain}
\newtheorem{lemma}{Lemma}[section]
\newtheorem{proposition}{Proposition}[section]
\newtheorem{theorem}{Theorem}[section]
\newtheorem{corollary}{Corollary}[section]
\theoremstyle{definition}
\newtheorem{definition}{Definition}[section]
\newtheorem{example}{Example}[section]
%\theoremstyle{remark}
\newtheorem{remark}{Remark}[section]
\newtheorem*{notation}{Notation}
\newtheorem*{claim}{Claim}
\renewcommand{\thefootnote}{\ensuremath{\fnsymbol{footnote%%@
}}}
\numberwithin{equation}{section}
\newcommand{\Ad}{{\rm Ad}}
\newcommand{\Aut}{{\rm Aut}}
\newcommand{\Cl}{{\rm Cl}}
\newcommand{\Co}{{\rm Co}}
\newcommand{\DES}{{\rm DES}}
\newcommand{\Diff}{{\rm Diff}}
\newcommand{\Dom}{{\rm Dom}}
\newcommand{\Hol}{{\rm Hol}}
\newcommand{\Mon}{{\rm Mon}}
\newcommand{\Hom}{{\rm Hom}}
\newcommand{\Ker}{{\rm Ker}}
\newcommand{\Ind}{{\rm Ind}}
\newcommand{\IM}{{\rm Im}}
\newcommand{\Is}{{\rm Is}}
\newcommand{\ID}{{\rm id}}
\newcommand{\GL}{{\rm GL}}
\newcommand{\Iso}{{\rm Iso}}
\newcommand{\Sem}{{\rm Sem}}
\newcommand{\St}{{\rm St}}
\newcommand{\Sym}{{\rm Sym}}
\newcommand{\SU}{{\rm SU}}
\newcommand{\Tor}{{\rm Tor}}
\newcommand{\U}{{\rm U}}
\newcommand{\A}{\mathcal A}
\newcommand{\Ce}{\mathcal C}
\newcommand{\D}{\mathcal D}
\newcommand{\E}{\mathcal E}
\newcommand{\F}{\mathcal F}
\newcommand{\G}{\mathcal G}
\newcommand{\Q}{\mathcal Q}
\newcommand{\R}{\mathcal R}
\newcommand{\cS}{\mathcal S}
\newcommand{\cU}{\mathcal U}
\newcommand{\W}{\mathcal W}
\newcommand{\bA}{\mathbb{A}}
\newcommand{\bB}{\mathbb{B}}
\newcommand{\bC}{\mathbb{C}}
\newcommand{\bD}{\mathbb{D}}
\newcommand{\bE}{\mathbb{E}}
\newcommand{\bF}{\mathbb{F}}
\newcommand{\bG}{\mathbb{G}}
\newcommand{\bK}{\mathbb{K}}
\newcommand{\bM}{\mathbb{M}}
\newcommand{\bN}{\mathbb{N}}
\newcommand{\bO}{\mathbb{O}}
\newcommand{\bP}{\mathbb{P}}
\newcommand{\bR}{\mathbb{R}}
\newcommand{\bV}{\mathbb{V}}
\newcommand{\bZ}{\mathbb{Z}}
\newcommand{\bfE}{\mathbf{E}}
\newcommand{\bfX}{\mathbf{X}}
\newcommand{\bfY}{\mathbf{Y}}
\newcommand{\bfZ}{\mathbf{Z}}
\renewcommand{\O}{\Omega}
\renewcommand{\o}{\omega}
\newcommand{\vp}{\varphi}
\newcommand{\vep}{\varepsilon}
\newcommand{\diag}{{\rm diag}}
\newcommand{\grp}{{\mathbb G}}
\newcommand{\dgrp}{{\mathbb D}}
\newcommand{\desp}{{\mathbb D^{\rm{es}}}}
\newcommand{\Geod}{{\rm Geod}}
\newcommand{\geod}{{\rm geod}}
\newcommand{\hgr}{{\mathbb H}}
\newcommand{\mgr}{{\mathbb M}}
\newcommand{\ob}{{\rm Ob}}
\newcommand{\obg}{{\rm Ob(\mathbb G)}}
\newcommand{\obgp}{{\rm Ob(\mathbb G')}}
\newcommand{\obh}{{\rm Ob(\mathbb H)}}
\newcommand{\Osmooth}{{\Omega^{\infty}(X,*)}}
\newcommand{\ghomotop}{{\rho_2^{\square}}}
\newcommand{\gcalp}{{\mathbb G(\mathcal P)}}
\newcommand{\rf}{{R_{\mathcal F}}}
\newcommand{\glob}{{\rm glob}}
\newcommand{\loc}{{\rm loc}}
\newcommand{\TOP}{{\rm TOP}}
\newcommand{\wti}{\widetilde}
\newcommand{\what}{\widehat}
\renewcommand{\a}{\alpha}
\newcommand{\be}{\beta}
\newcommand{\ga}{\gamma}
\newcommand{\Ga}{\Gamma}
\newcommand{\de}{\delta}
\newcommand{\del}{\partial}
\newcommand{\ka}{\kappa}
\newcommand{\si}{\sigma}
\newcommand{\ta}{\tau}
\newcommand{\lra}{{\longrightarrow}}
\newcommand{\ra}{{\rightarrow}}
\newcommand{\rat}{{\rightarrowtail}}
\newcommand{\oset}[1]{\overset {#1}{\ra}}
\newcommand{\osetl}[1]{\overset {#1}{\lra}}
\newcommand{\hr}{{\hookrightarrow}}

\begin{document}
\section{2-Category of Double Groupoids}

This is an introduction to the subject of 2-category of double groupoids, including the definition of this new concept.
It can also be further generalized to the 2-category of double categories by removing the constrained that all double groupoid homomorphisms be invertible and by replacing the double groupoid with a double category; naturally, double groupoid homomorphisms are then replaced by 2-functors (that need not be invertible) between double categories. 

\subsection{Introduction}

\begin{definition}
Let us recall that if $X$ is a topological space, then a \emph{double goupoid} $\D$
is defined by the following categorical diagram of linked groupoids and sets:

\begin{equation}
\label{squ} \D := \vcenter{\xymatrix @=3pc {S \ar @<1ex> [r] ^{s^1} \ar @<-1ex> [r]
_{t^1} \ar @<1ex> [d]^{\, t_2} \ar @<-1ex> [d]_{s_2} & H \ar[l]
\ar @<1ex> [d]^{\,t}
\ar @<-1ex> [d]_s \\
V \ar [u] \ar @<1ex> [r] ^s \ar @<-1ex> [r] _t & M \ar [l] \ar[u]}},
\end{equation}

where $M$ is a set of points, $H,V$ are two groupoids (called, respectively, ``horizontal'' and ``vertical'' groupoids)
, and $S$ is a set of \PMlinkname{squares with two composition laws, $\bullet$ and $\circ$}{ThinSquare} (as first defined and represented in ref. \cite{BHKP} by Brown et al.) . A simplified notion of a thin square is that of ``a continuous map from the unit square of the real plane into $X$ which factors through a tree'' (\cite{BHKP}).
\end{definition}

\subsection{Homotopy double groupoid and homotopy 2-groupoid}

The algebraic composition laws, $\bullet$ and $\circ$, employed above to define a double groupoid $\D$ allow one also to define $\D$ as a groupoid internal to the \PMlinkname{category of groupoids}{GroupoidCategory}. Thus, in the particular case of a Hausdorff space, $X_H$, a double groupoid called the \emph{homotopy double groupoid of $X_H$} can be denoted as follows

$$\boldsymbol{\rho}^{\square}_2 (X_H) := \D ,$$

where $\square$ is in this case a \PMlinkname{thin square}{ThinSquare}. Thus, the construction of a homotopy double groupoid is based upon the geometric notion of thin square that extends the notion of thin relative homotopy as discussed in ref. \cite{BHKP}. One notes however a significant distinction between a homotopy 2-groupoid and homotopy double groupoid construction; thus, the construction of the $2$-cells of the homotopy double groupoid is based upon a suitable cubical approach to the notion of thin $3$-cube, whereas the construction of the 2-cells of the homotopy $2$-groupoid can be interpreted by means of a globular notion of thin $3$-cube. ``The homotopy double groupoid of a space, and the related homotopy $2$-groupoid, are constructed directly from the cubical singular complex and so (they) remain close to geometric intuition in an almost classical way'' (viz. \cite{BHKP}).


\subsection{Defintion of 2-Category of Double Groupoids}

\begin{definition}
The \PMlinkname{2-category}{2Category}, $\G^2$-- whose objects (or $2$-cells) are the above diagrams $\D$ that define double groupoids, and whose $2$-morphisms are functors $\mathbb{F}$ between double groupoid $\D$ diagrams-- is called the \emph{double groupoid 2-category}, or the \emph{2-category of double groupoids}.
\end{definition}

\begin{remark}
$\G^2$ is a relatively simple example of a category of diagrams, or a 1-supercategory, $\S_1$.
\end{remark}

\begin{thebibliography}{9}

\bibitem{BHKP}
R. Brown, K.A. Hardie, K.H. Kamps and T. Porter.,
\PMlinkexternal{A homotopy double groupoid of a Hausdorff space}{http://www.tac.mta.ca/tac/volumes/10/2/10-02.pdf} ,
{\it Theory and Applications of Categories} \textbf{10},(2002): 71-93.

\bibitem{BS1}
R. Brown and C.B. Spencer: Double groupoids and crossed modules, \emph{Cahiers Top. G\'eom.Diff.},
\textbf{17} (1976), 343--362.

\bibitem{BMos}
R. Brown and G. H. Mosa: Double algebroids and crossed modules of algebroids, University of Wales--Bangor, Maths Preprint, 1986.

\bibitem{HKK}
K.A. Hardie, K.H. Kamps and R.W. Kieboom., A homotopy 2-groupoid of a Hausdorff
\emph{Applied Categorical Structures}, \textbf{8} (2000): 209-234.

\bibitem{Agl-Br-St2k2}
Al-Agl, F.A., Brown, R. and R. Steiner: 2002, Multiple categories: the equivalence of a globular and cubical approach, \emph{Adv. in Math}, \textbf{170}: 711-118.

\end{thebibliography}

%%%%%
%%%%%
\end{document}
