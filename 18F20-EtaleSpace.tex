\documentclass[12pt]{article}
\usepackage{pmmeta}
\pmcanonicalname{EtaleSpace}
\pmcreated{2013-03-22 15:40:09}
\pmmodified{2013-03-22 15:40:09}
\pmowner{guffin}{12505}
\pmmodifier{guffin}{12505}
\pmtitle{\'Etal\'e space}
\pmrecord{6}{37608}
\pmprivacy{1}
\pmauthor{guffin}{12505}
\pmtype{Definition}
\pmcomment{trigger rebuild}
\pmclassification{msc}{18F20}
\pmclassification{msc}{54B40}
\pmclassification{msc}{14F05}
\pmsynonym{Espace Etale}{EtaleSpace}
\pmsynonym{Etale space}{EtaleSpace}
\pmsynonym{Espace \'Etal\'e}{EtaleSpace}
%\pmkeywords{Sheaf}
%\pmkeywords{Stalk}
%\pmkeywords{Etale Space}
%\pmkeywords{Sheafification}
\pmrelated{Stalk}
\pmrelated{Sheaf}
\pmdefines{\'Etal\'e Space}
\pmdefines{Etale Space}

\endmetadata

% this is the default PlanetMath preamble.  as your knowledge
% of TeX increases, you will probably want to edit this, but
% it should be fine as is for beginners.

% almost certainly you want these
\usepackage{amssymb}
\usepackage{amsmath}
\usepackage{amsfonts}

\newcommand{\sheaf}[1]{\mathcal{#1}}
\newcommand{\Etale}[0]{\'Etal\'e}

% used for TeXing text within eps files
%\usepackage{psfrag}
% need this for including graphics (\includegraphics)
%\usepackage{graphicx}
% for neatly defining theorems and propositions
%\usepackage{amsthm}
% making logically defined graphics
%%%\usepackage{xypic}

% there are many more packages, add them here as you need them

% define commands here
\begin{document}
The \Etale~space (Espace \Etale) is a topological space associated to a presheaf $\sheaf F$ on a space $X$.  The \'Etal\'e space is defined to be the disjoint union of stalks of the sheaf $\sheaf F$. 

\[\sheaf E_{\sheaf F} \equiv \coprod_{x\in X} \sheaf F_x\]


Over each open set $U\subset X$, there is a set of sections $\Gamma(U,\sheaf F)$.  A basis for the topology on the \Etale~space is formed by taking the open sets to be of the form $\sheaf U_s = \{s_x, x\in U\}$, for $s\in \Gamma(U,\sheaf F)$ and $s_x$ the germ of $s$ at $x$. There is a natural map $\pi\!:\!\sheaf E_{\sheaf F} \rightarrow X$ which takes germs $s_x$ in the stalk $\sheaf F_x$ over $x$ to $x$.


Let $s\in \Gamma(U,\sheaf F)$ and $s^\prime \in \Gamma(U^\prime,\sheaf F)$ with $U\cap U^\prime \ne \emptyset$. At each point $x\in U \cap U^\prime$ where $s_x = s^\prime_x$, by the definition of germs there exists an open set $V\subset U\cap U^\prime$ containing $x$ such that $s$ and $s^\prime$ restrict to the same section on $V$ ($s|_V = s^\prime|_V$).  This verifies that $\{\sheaf U_s\}$ form a basis for $\sheaf E_{\sheaf F}$.

Then there is another presheaf, $\widetilde{\sheaf F}$, whose sections are the continuous functions from $X$ to $\sheaf E_{\sheaf F}$ assigning an element $s(x)\in \sheaf F_x$ to each point $x \in X$.  This presheaf forms a sheaf equivalent to the sheafification of the presheaf $\sheaf F$.
%%%%%
%%%%%
\end{document}
