\documentclass[12pt]{article}
\usepackage{pmmeta}
\pmcanonicalname{CategoricalSequence}
\pmcreated{2013-03-22 18:19:17}
\pmmodified{2013-03-22 18:19:17}
\pmowner{bci1}{20947}
\pmmodifier{bci1}{20947}
\pmtitle{categorical sequence}
\pmrecord{33}{40951}
\pmprivacy{1}
\pmauthor{bci1}{20947}
\pmtype{Definition}
\pmcomment{trigger rebuild}
\pmclassification{msc}{18E05}
\pmclassification{msc}{18-00}
\pmsynonym{linear diagrams}{CategoricalSequence}
\pmsynonym{sequence}{CategoricalSequence}
%\pmkeywords{categorical sequence}
%\pmkeywords{or linear diagram}
%\pmkeywords{of sets and set-theoretical mappings}
%\pmkeywords{exact functor}
%\pmkeywords{short exact sequence}
%\pmkeywords{commutative homological algebra}
\pmrelated{ChainComplex}
\pmrelated{ExactSequence2}
\pmrelated{CommutativeDiagram}
\pmrelated{AbelianCategory}
\pmrelated{ShortExactSequence}
\pmrelated{ExactFunctor}
\pmrelated{ExactSequence}
\pmrelated{TangentialCauchyRiemannComplexOfCinftySmoothForms}
\pmrelated{AlternativeDefinitionOfAnAbelianCategory}
\pmrelated{SuperdiagramsAsHeterofunctors}
\pmrelated{CategoryTheory}
\pmdefines{linear diagram}
\pmdefines{(linear) sequence of morphisms}
\pmdefines{exact functor}
\pmdefines{short exact sequence}

\endmetadata

% this is the default PlanetMath preamble.  as your knowledge
% of TeX increases, you will probably want to edit this, but
% it should be fine as is for beginners.

% almost certainly you want these
\usepackage{amssymb}
\usepackage{amsmath}
\usepackage{amsfonts}
\def\im{\operatorname{im}}
\def\ker{\operatorname{ker}}
% used for TeXing text within eps files
%\usepackage{psfrag}
% need this for including graphics (\includegraphics)
%\usepackage{graphicx}
% for neatly defining theorems and propositions
%\usepackage{amsthm}
% making logically defined graphics
%%%\usepackage{xypic}

% there are many more packages, add them here as you need them

% define commands here
\usepackage{amsmath, amssymb, amsfonts, amsthm, amscd, latexsym}
%%\usepackage{xypic}
\usepackage[mathscr]{eucal}

\setlength{\textwidth}{6.5in}
%\setlength{\textwidth}{16cm}
\setlength{\textheight}{9.0in}
%\setlength{\textheight}{24cm}

\hoffset=-.75in     %%ps format
%\hoffset=-1.0in     %%hp format
\voffset=-.4in

\theoremstyle{plain}
\newtheorem{lemma}{Lemma}[section]
\newtheorem{proposition}{Proposition}[section]
\newtheorem{theorem}{Theorem}[section]
\newtheorem{corollary}{Corollary}[section]

\theoremstyle{definition}
\newtheorem{definition}{Definition}[section]
\newtheorem{example}{Example}[section]
%\theoremstyle{remark}
\newtheorem{remark}{Remark}[section]
\newtheorem*{notation}{Notation}
\newtheorem*{claim}{Claim}

\renewcommand{\thefootnote}{\ensuremath{\fnsymbol{footnote%%@
}}}
\numberwithin{equation}{section}

\newcommand{\Ad}{{\rm Ad}}
\newcommand{\Aut}{{\rm Aut}}
\newcommand{\Cl}{{\rm Cl}}
\newcommand{\Co}{{\rm Co}}
\newcommand{\DES}{{\rm DES}}
\newcommand{\Diff}{{\rm Diff}}
\newcommand{\Dom}{{\rm Dom}}
\newcommand{\Hol}{{\rm Hol}}
\newcommand{\Mon}{{\rm Mon}}
\newcommand{\Hom}{{\rm Hom}}
\newcommand{\Ker}{{\rm Ker}}
\newcommand{\Ind}{{\rm Ind}}
\newcommand{\IM}{{\rm Im}}
\newcommand{\Is}{{\rm Is}}
\newcommand{\ID}{{\rm id}}
\newcommand{\GL}{{\rm GL}}
\newcommand{\Iso}{{\rm Iso}}
\newcommand{\Sem}{{\rm Sem}}
\newcommand{\St}{{\rm St}}
\newcommand{\Sym}{{\rm Sym}}
\newcommand{\SU}{{\rm SU}}
\newcommand{\Tor}{{\rm Tor}}
\newcommand{\U}{{\rm U}}

\newcommand{\A}{\mathcal A}
\newcommand{\Ce}{\mathcal C}
\newcommand{\D}{\mathcal D}
\newcommand{\E}{\mathcal E}
\newcommand{\F}{\mathcal F}
\newcommand{\G}{\mathcal G}
\newcommand{\Q}{\mathcal Q}
\newcommand{\R}{\mathcal R}
\newcommand{\cS}{\mathcal S}
\newcommand{\cU}{\mathcal U}
\newcommand{\W}{\mathcal W}

\newcommand{\bA}{\mathbb{A}}
\newcommand{\bB}{\mathbb{B}}
\newcommand{\bC}{\mathbb{C}}
\newcommand{\bD}{\mathbb{D}}
\newcommand{\bE}{\mathbb{E}}
\newcommand{\bF}{\mathbb{F}}
\newcommand{\bG}{\mathbb{G}}
\newcommand{\bK}{\mathbb{K}}
\newcommand{\bM}{\mathbb{M}}
\newcommand{\bN}{\mathbb{N}}
\newcommand{\bO}{\mathbb{O}}
\newcommand{\bP}{\mathbb{P}}
\newcommand{\bR}{\mathbb{R}}
\newcommand{\bV}{\mathbb{V}}
\newcommand{\bZ}{\mathbb{Z}}

\newcommand{\bfE}{\mathbf{E}}
\newcommand{\bfX}{\mathbf{X}}
\newcommand{\bfY}{\mathbf{Y}}
\newcommand{\bfZ}{\mathbf{Z}}

\renewcommand{\O}{\Omega}
\renewcommand{\o}{\omega}
\newcommand{\vp}{\varphi}
\newcommand{\vep}{\varepsilon}

\newcommand{\diag}{{\rm diag}}
\newcommand{\grp}{{\mathbb G}}
\newcommand{\dgrp}{{\mathbb D}}
\newcommand{\desp}{{\mathbb D^{\rm{es}}}}
\newcommand{\Geod}{{\rm Geod}}
\newcommand{\geod}{{\rm geod}}
\newcommand{\hgr}{{\mathbb H}}
\newcommand{\mgr}{{\mathbb M}}
\newcommand{\ob}{{\rm Ob}}
\newcommand{\obg}{{\rm Ob(\mathbb G)}}
\newcommand{\obgp}{{\rm Ob(\mathbb G')}}
\newcommand{\obh}{{\rm Ob(\mathbb H)}}
\newcommand{\Osmooth}{{\Omega^{\infty}(X,*)}}
\newcommand{\ghomotop}{{\rho_2^{\square}}}
\newcommand{\gcalp}{{\mathbb G(\mathcal P)}}

\newcommand{\rf}{{R_{\mathcal F}}}
\newcommand{\glob}{{\rm glob}}
\newcommand{\loc}{{\rm loc}}
\newcommand{\TOP}{{\rm TOP}}

\newcommand{\wti}{\widetilde}
\newcommand{\what}{\widehat}

\renewcommand{\a}{\alpha}
\newcommand{\be}{\beta}
\newcommand{\ga}{\gamma}
\newcommand{\Ga}{\Gamma}
\newcommand{\de}{\delta}
\newcommand{\del}{\partial}
\newcommand{\ka}{\kappa}
\newcommand{\si}{\sigma}
\newcommand{\ta}{\tau}
\newcommand{\med}{\medbreak}
\newcommand{\medn}{\medbreak \noindent}
\newcommand{\bign}{\bigbreak \noindent}
\newcommand{\lra}{{\longrightarrow}}
\newcommand{\ra}{{\rightarrow}}
\newcommand{\rat}{{\rightarrowtail}}
\newcommand{\oset}[1]{\overset {#1}{\ra}}
\newcommand{\osetl}[1]{\overset {#1}{\lra}}
\newcommand{\hr}{{\hookrightarrow}}
\newcommand{\cok}{\operatorname{cok}}

\begin{document}
\begin{definition}
A \emph{categorical sequence} is a \emph{linear `diagram' of morphisms, or arrows, in an abstract category}. 
In a concrete category, such as the category of sets, the categorical sequence consists of sets joined by set-theoretical mappings in linear fashion, such as: 
\[
\cdots \rightarrow
A\buildrel f \over \longrightarrow
B \buildrel \phi \over  \longrightarrow
Hom_{Set}(A,B),
\]
where $Hom_{Set}(A,B)$ is the set of functions from set $A$ to set $B$.
\end{definition}


\subsection{Examples}

\subsubsection{The chain complex is a categorical sequence example:}

 Consider a ring $R$ and the \emph{chain complex} consisting of
a sequence of \PMlinkname{$R$-modules}{Module} and homomorphisms:

\[
\cdots \rightarrow
A_{n+1} \buildrel {d_{n+1}} \over \longrightarrow
A_n \buildrel {d_n} \over \longrightarrow
A_{n-1} \rightarrow
\cdots
\]
(with the additional condition imposed by $d_n\circ d_{n+1} = 0$ for each pair of adjacent homomorphisms $(d_{n+1}, d_n)$; this is equivalent to the condition $\im d_{n+1} \subseteq \ker d_n$ that needs to be satisfied in order to define this categorical sequence completely as a \emph{chain complex}). Furthermore, a sequence of homomorphisms
$$
\cdots \rightarrow
A_{n+1} \buildrel {f_{n+1}} \over \longrightarrow
A_n \buildrel {f_n} \over \longrightarrow
A_{n-1} \rightarrow
\cdots
$$
is said to be {\it exact} if each pair of adjacent homomorphisms $(f_{n+1}, f_n)$ is \emph{exact}, that is, 
if ${\rm im} f_{n+1} = {\rm ker} f_n$ for all $n$. This concept can be then generalized to  
morphisms in a \PMlinkname{categorical exact sequence}{ExactSequence2}, thus leading to the corresponding 
definition of an \PMlinkname{exact sequence}{ExactSequence2} in an Abelian category. 

\begin{remark}
Inasmuch as categorical diagrams can be defined as functors, exact sequences of special types of morphisms
can also be regarded as the corresponding, special functors. Thus, exact sequences in Abelian categories
can be regarded as certain functors of Abelian categories; the details of such functorial (abelian) constructions
are left to the reader as an exercise. Moreover, in (commutative or Abelian) homological algebra, an 
\PMlinkname{exact functor}{ExactFunctor} is simply defined as a functor $F$ between two Abelian categories, $\mathcal{A}$ and $\mathcal{B}$, $F: \mathcal{A} \to \mathcal{B}$, which preserves categorical exact sequences, that is, if $F$ carries a short exact sequence  $0 \to C \to D \to E \to 0$ (with $0, C, D$ and $E$ objects in $\mathcal{A}$) into the corresponding sequence in the Abelian category $\mathcal{B}$,  ($0 \to F(C) \to F(D) \to F(E) \to 0$), which is also exact (in $\mathcal{B}$).
\end{remark}

%%%%%
%%%%%
\end{document}
