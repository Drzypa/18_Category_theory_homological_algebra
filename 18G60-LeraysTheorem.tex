\documentclass[12pt]{article}
\usepackage{pmmeta}
\pmcanonicalname{LeraysTheorem}
\pmcreated{2013-03-22 14:42:32}
\pmmodified{2013-03-22 14:42:32}
\pmowner{mathcam}{2727}
\pmmodifier{mathcam}{2727}
\pmtitle{Leray's theorem}
\pmrecord{9}{36328}
\pmprivacy{1}
\pmauthor{mathcam}{2727}
\pmtype{Theorem}
\pmcomment{trigger rebuild}
\pmclassification{msc}{18G60}
\pmrelated{sheaf}
\pmrelated{sheafcohomology}
\pmrelated{SheafCohomology}

\endmetadata

% this is the default PlanetMath preamble.  as your knowledge
% of TeX increases, you will probably want to edit this, but
% it should be fine as is for beginners.

% almost certainly you want these
\usepackage{amssymb}
\usepackage{amsmath}
\usepackage{amsfonts}

% used for TeXing text within eps files
%\usepackage{psfrag}
% need this for including graphics (\includegraphics)
%\usepackage{graphicx}
% for neatly defining theorems and propositions
%\usepackage{amsthm}
% making logically defined graphics
%%%\usepackage{xypic}

% there are many more packages, add them here as you need them

% define commands here
\begin{document}
Let $\mathcal F$ be a sheaf on a topological space $X$ and $\mathcal U=\{U_i\}$ an open cover of $X$. If $\mathcal F$ is \PMlinkname{acyclic}{AcyclicSheaf} on every \PMlinkescapetext{finite} \PMlinkescapetext{intersection} of elements of $\mathcal U$, then
$$
\check H^q(\mathcal U,\mathcal F)=\check H^q(X,\mathcal F),
$$
where $\check H^q(\mathcal U,\mathcal F)$ is the $q$-th \PMlinkname{Cech cohomology group}{CechCohomologyGroup2} of $\mathcal F$ with respect to the open cover $\mathcal U$.

\begin{thebibliography}{9}
\bibitem{Bon} Bonavero, Laurent.  \emph{Cohomology of Line Bundles on Toric Varieties, Vanishing Theorems.}  Lectures 16-17 from ``Summer School 2000:  Geometry of Toric Varieties.''
\end{thebibliography}
%%%%%
%%%%%
\end{document}
