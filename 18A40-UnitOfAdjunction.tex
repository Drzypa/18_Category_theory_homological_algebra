\documentclass[12pt]{article}
\usepackage{pmmeta}
\pmcanonicalname{UnitOfAdjunction}
\pmcreated{2013-03-22 17:25:50}
\pmmodified{2013-03-22 17:25:50}
\pmowner{CWoo}{3771}
\pmmodifier{CWoo}{3771}
\pmtitle{unit of adjunction}
\pmrecord{16}{39806}
\pmprivacy{1}
\pmauthor{CWoo}{3771}
\pmtype{Definition}
\pmcomment{trigger rebuild}
\pmclassification{msc}{18A40}
\pmsynonym{co-unit}{UnitOfAdjunction}
\pmrelated{UniversalProperty}
\pmdefines{unit}
\pmdefines{counit}

\usepackage{amssymb,amscd}
\usepackage{amsmath}
\usepackage{amsfonts}
\usepackage{mathrsfs}

% used for TeXing text within eps files
%\usepackage{psfrag}
% need this for including graphics (\includegraphics)
%\usepackage{graphicx}
% for neatly defining theorems and propositions
\usepackage{amsthm}
% making logically defined graphics
%%\usepackage{xypic}
\usepackage{pst-plot}
\usepackage{psfrag}

% define commands here
\newtheorem{prop}{Proposition}
\newtheorem{thm}{Theorem}
\newtheorem{ex}{Example}
\newcommand{\real}{\mathbb{R}}
\newcommand{\pdiff}[2]{\frac{\partial #1}{\partial #2}}
\newcommand{\mpdiff}[3]{\frac{\partial^#1 #2}{\partial #3^#1}}
\begin{document}
Let $\mathcal{C},\mathcal{D}$ be categories and $(T,S,\nu)$ be an adjunction from $\mathcal{C}$ to $\mathcal{D}$.  For every pair of objects $C\in\mathcal{C}$ and $D\in\mathcal{D}$, we have a bijection
\begin{equation}
\nu_{C,D}:\hom_{\mathcal{D}}(T(C),D) \longrightarrow \hom_{\mathcal{C}}(C,S(D))
\end{equation}
that is natural in each variable.

If we set $D=T(C)$, and write $\nu_C$ for $\nu_{C,T(C)}$, then we get a bijection 
$$\nu_C:\hom_{\mathcal{D}}(T(C),T(C)) \longrightarrow \hom_{\mathcal{C}}(C,ST(C))$$ where $ST$ is the abbreviation of $S\circ T$.

As $1_{T(C)}$ is the identity morphism in $\hom_{\mathcal{D}}(T(C),T(C))$, define 
\begin{equation}
\eta_C:=\nu_C(1_{T(C)}).
\end{equation}  
Note that $\eta_C$ is a morphism in $\mathcal{C}$ from $C$ to $ST(C)$.  Also, naturality in $C$ means that if $f:C'\to C$ and $g:T(C)\to T(C')$, then 
\begin{equation}
Sg\circ \eta_c \circ f=\nu_{C'}(g\circ Tf).
\end{equation}

\begin{thm} $(T(C),\eta_C)$ is a universal arrow from $C$ to $S$. \end{thm}
\begin{proof}  Let $Y$ be an object in $\mathcal{D}$ and $f:C\to S(Y)$ a morphism in $\mathcal{C}$.  We want to find a morphism $g:T(C)\to Y$ in $\mathcal{D}$ such that 
$$
\xymatrix{
& C \ar[dr]^{f} \ar[dl]_{\eta_C} & \\
 ST(C) \ar[rr]_{S(g)} && S(Y) }$$
is a commutative diagram.  The existence and uniqueness of $g$ is guaranteed by the bijection 
$$\nu_{C,Y}:\hom_{\mathcal{D}}(T(C),Y) \longrightarrow \hom_{\mathcal{C}}(C,S(Y)),$$ where $f=\nu_{C,Y}(g)$, and the commutativity of the triangle above is guaranteed by the naturality in the second variable
$$
\xymatrix{
\hom_{\mathcal{D}}(T(C),T(C)) \ar[d]_{\hat{g}} \ar[rr]^{\nu_C} && \hom_{\mathcal{C}}(C,ST(C)) \ar[d]^{\overline{g}} \\
\hom_{\mathcal{D}}(T(C),Y) \ar[rr]_{\nu_{C,Y}} && \hom_{\mathcal{C}}(C,S(Y)), }$$
where $\hat{g}:=\hom_{\mathcal{D}}(1_{T(C)},g)$ and $\overline{g}:=\hom_{\mathcal{C}}(1_C,S(g))$, as $$\overline{g}\circ \nu_C(1_{T(C)})=\hom_{\mathcal{C}}(C,S(g))\circ \eta_C=S(g)\circ \eta_C$$ on the one hand, and $$\nu_{C,Y}\circ \hat{g}(1_{T(C)})=\nu_{C,Y}\circ \hom(T(C),g)(1_{T(C)})=\nu_{C,Y}(g\circ 1_{T(C)})=\nu_{C,Y}(g)=f$$ on the other, and the two are equal.
\end{proof}
\begin{thm} $\eta: C \mapsto \eta_C$ is a natural transformation from the identity functor $I_{\mathcal{C}}$ to $ST$.  \end{thm}
\begin{proof}
Suppose $f:A\to B$ is a morphism in $\mathcal{C}$.  We want to show that
$$\xymatrix{
A \ar[d]_{\eta_A} \ar[rr]^f && B \ar[d]^{\mu_B} \\
ST(A) \ar[rr]_{ST(f)} && ST(B) }$$
is commutative.  To see this, write out the expressions 
\begin{alignat*}{2}
\eta_B\circ f &= 1_{ST(B)}\circ \eta_B\circ f & \mbox{property of identity morphism} \\ 
&= S(1_{T(B)})\circ \eta_B\circ f & \qquad\qquad\qquad \mbox{property of functor on identity morphism} \\ 
&= \nu_A(1_{T(B)}\circ T(f)) & \mbox{by equation (3) above}\\
&= \nu_A(T(f)\circ 1_{T(A)}) & T(f)\mbox{ commutes with identity morphisms}\\
&= \nu_A(T(f)\circ T(1_A)) & \mbox{property of functor on identity morphism}\\
&= ST(f)\circ \eta_A\circ 1_A & \mbox{by equation (3) above}\\
&= ST(f)\circ \eta_A & \mbox{property of identity morphisms}.
\end{alignat*}
\end{proof}

\textbf{Definition}.  The natural transformation $\eta:I_{\mathcal{C}}\dot{\to} ST$ defined above is called the \emph{unit} of the adjunction $(T,S,\nu)$ from $\mathcal{C}$ to $\mathcal{D}$.

Dually, one can find a natural transformation $\epsilon:TS \dot{\to} I_{\mathcal{D}}$ called the \emph{counit} of the adjunction $(T,S,\nu):\mathcal{C}\to\mathcal{D}$.  To do this, set $C=S(D)$ and use equation (1) to get a bijection $\nu_D:=\nu_{S(D),D}$ and subsequently define 
\begin{equation}
\epsilon_D:=\nu_D(1_{S(D)}).
\end{equation}
As in the previous theorems, one can, by reversing all the arrows, show that $(S(D),\epsilon_D)$ is a universal arrow from $D$ to $T$, and that $\epsilon$ is indeed a natural transformation from $TS$ to $I_{\mathcal{D}}$.

\begin{thebibliography}{9}
\bibitem{Ma}S. Mac Lane, \emph{Categories for the Working Mathematician} (2nd edition), Springer-Verlag, 1997.
\end{thebibliography}
%%%%%
%%%%%
\end{document}
