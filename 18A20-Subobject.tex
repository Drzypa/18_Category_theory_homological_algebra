\documentclass[12pt]{article}
\usepackage{pmmeta}
\pmcanonicalname{Subobject}
\pmcreated{2013-03-22 14:45:29}
\pmmodified{2013-03-22 14:45:29}
\pmowner{CWoo}{3771}
\pmmodifier{CWoo}{3771}
\pmtitle{subobject}
\pmrecord{20}{36399}
\pmprivacy{1}
\pmauthor{CWoo}{3771}
\pmtype{Definition}
\pmcomment{trigger rebuild}
\pmclassification{msc}{18A20}
\pmsynonym{monomorphism equivalence}{Subobject}
\pmsynonym{epimorphism equivalence}{Subobject}
\pmrelated{PowerObject}
\pmrelated{WellPoweredCategory}
\pmrelated{WellpoweredCategory}
\pmdefines{subobject}
\pmdefines{quotient object}
\pmdefines{equivalent monomorphisms}
\pmdefines{equivalent epimorphisms}
\pmdefines{subobject functor}
\pmdefines{quotient object functor}

% this is the default PlanetMath preamble.  as your knowledge
% of TeX increases, you will probably want to edit this, but
% it should be fine as is for beginners.

% almost certainly you want these
\usepackage{amssymb,amscd}
\usepackage{amsmath}
\usepackage{amsfonts}

% used for TeXing text within eps files
%\usepackage{psfrag}
% need this for including graphics (\includegraphics)
%\usepackage{graphicx}
% for neatly defining theorems and propositions
%\usepackage{amsthm}
% making logically defined graphics
%%\usepackage{xypic}

% there are many more packages, add them here as you need them

% define commands here

\newcommand{\Sub}{{\mathrm{Sub}}}
\newcommand{\Quo}{{\mathrm{Quo}}}
\begin{document}
Let $\mathcal{C}$ be a category.

\subsubsection*{Equivalent Monomorphisms and Subobjects}

Let $\operatorname{Mono}(-,B)$ be the class of all monomorphisms into object $B$ in $\mathcal{C}$.  Two elements $f\colon A_1\to B$ and $g\colon A_2\to B$ in $\operatorname{Mono}(-,B)$ are \PMlinkescapetext{equivalent} if there exist two morphisms $r:A_1\rightarrow A_2$ and $s:A_2\rightarrow A_1$ such that we have the following two commutative diagrams:
\begin{center}
$
\xymatrix@R-=2pt{
A_1\ar[dr]^f\ar[dd]_r\\
&B\\
A_2\ar[ur]_g
}
\xymatrix@R-=2pt{
&&\\
&and&\\
&&
}
\xymatrix@R-=2pt{
A_1\ar[dr]^f\\
&B\\
A_2\ar[ur]_g\ar[uu]^s
}
$
\end{center}
It is easy to see that both $r$ and $s$ are monomorphic.  In fact, $A_1$ and $A_2$ are isomorphic objects, as $f=g\circ r$ and $g=f\circ s$, so that $f=(f\circ s)\circ r$, or $1_{A_1}=s\circ r$, since $s\circ r$ is monomorphic.  Similarly $1_{A_2}=r\circ s$.  Monomorphism equivalence is an equivalence relation on $\operatorname{Mono}(-,B)$.


\textbf{Definition}.  A \emph{subobject} of an object $X$ in $\mathcal{C}$ is an equivalence class in $\operatorname{Mono}(-,X)$.  Let's write $$[A\to X]$$ for a subobject of $X$, and a monomorphism $f:A\to X$ a representative of $[A\to X]$.  If there is no danger of confusion, it is often easier to identify a subobject $[A\to X]$ by $A$, and simply write $$A\subseteq X,$$ as long as we keep in mind that, along with the object $A$, there is a monomorphism from $A$ to $X$.  The class of subobjects of $X$ shall be denoted by $$\Sub(X):=\lbrace A\mid A\subseteq X\rbrace.$$
\textbf{Definition}.  A category $\mathcal{C}$ is said to be \emph{well-powered} or \emph{locally small} if for every object $X$ in $\mathcal{C}$, the class $\Sub(X)$ is a set.  Most common categories are locally small.

Suppose now that $\mathcal{C}$ is well-powered and has pullbacks (a pullback exists for every pair of morphisms into the same object).  We shall turn $\Sub$ into a functor from $\mathcal{C}$ to \textbf{Set}.  For every morphism $\alpha: X\to Y$, define $\Sub(\alpha)$ as follows: 
\begin{quote} take a representative $g\in[B\to Y]$, consider the pullback of $\alpha$ and $g$ indicated in the commutative diagram below:
\begin{center}
$\xymatrix@C+=30pt@R+=40pt{
A\ar[d]_f \ar[r]^{\beta} & B\ar[d]^g \\
X\ar[r]^{\alpha} & Y.
}
$
\end{center}
Since $g$ is monomorphism, so is $f:A\to X$, and hence $A$ is a subobject of $X$.  We set $\Sub(\alpha)([g]):=[f]$.  
\end{quote}
Because the diagram is a pullback, $\Sub(\alpha)([g])$ gives a unique value, and thus $$\Sub(\alpha):\Sub(Y)\to\Sub(X)$$ is a well-defined morphism.  Furthermore, it is easily verified that $\Sub(1_X)=1_{\Sub(X)}$ and $\Sub(\alpha\circ\beta)= \Sub(\beta)\circ\Sub(\alpha)$.  Thus, $\Sub$ is a contravariant functor from $\mathcal{C}$ to $\textbf{Set}$ (or a covariant functor $\mathcal{C}^{op}\to \textbf{Set}$), and is called the \emph{subobject functor} of $\mathcal{C}$. 

\subsubsection*{Equivalent Epimorphisms and Quotient Objects}

\par
Dually, given an object $A$ in a category $\mathcal{C}$, we can define an equivalence relation on $\operatorname{Epi}(A,-)$, the class of all epimorphisms from $A$, by reversing all arrows in the previous paragraph.  Specifically, two elements $f\colon A \to B_1$ and $g\colon A \to B_2$ in $\operatorname{Epi}(A,-)$ are \PMlinkescapetext{equivalent} if there exist two morphisms $B_1\rightarrow B_2$ and $B_2\rightarrow B_1$ such that the following two diagrams commute:
\begin{center}
$
\xymatrix@R-=2pt{
&B_1\ar[dd]\\
A\ar[ur]\ar[dr]\\
&B_2
}
\xymatrix@R-=2pt{
&&\\
&and&\\
&&
}
\xymatrix@R-=2pt{
&B_1\\
A\ar[ur]\ar[dr]\\
&B_2\ar[uu]
}
$
\end{center}
\par

\textbf{Definition}.  A \emph{quotient object} of $X$ is an equivalence class in $\operatorname{Epi}(X,-)$.  A typical quotient object is denoted by $\lbrack X\to B \rbrack$.  If $\mathcal{C}$ is a small category and has pushouts, then there is a covariant functor $\Quo:\mathcal{C}\to \textbf{Set}$ taking each object of $\mathcal{C}$ to its set of quotient objects and each morphism between two objects to a morphism between the sets of their quotient objects.  $\Quo$ is called the \emph{quotient object functor} of $\mathcal{C}$.
%%%%%
%%%%%
\end{document}
