\documentclass[12pt]{article}
\usepackage{pmmeta}
\pmcanonicalname{RingedSpace}
\pmcreated{2013-03-22 16:18:15}
\pmmodified{2013-03-22 16:18:15}
\pmowner{Mathprof}{13753}
\pmmodifier{Mathprof}{13753}
\pmtitle{ringed space}
\pmrecord{5}{38426}
\pmprivacy{1}
\pmauthor{Mathprof}{13753}
\pmtype{Definition}
\pmcomment{trigger rebuild}
\pmclassification{msc}{18F20}

\endmetadata

% this is the default PlanetMath preamble.  as your knowledge
% of TeX increases, you will probably want to edit this, but
% it should be fine as is for beginners.

% almost certainly you want these
\usepackage{amssymb}
\usepackage{amsmath}
\usepackage{amsfonts}

% used for TeXing text within eps files
%\usepackage{psfrag}
% need this for including graphics (\includegraphics)
%\usepackage{graphicx}
% for neatly defining theorems and propositions
%\usepackage{amsthm}
% making logically defined graphics
%%%\usepackage{xypic}

% there are many more packages, add them here as you need them

% define commands here

\begin{document}
A \emph{ringed space} is a topological space $X$ together with a sheaf of commutative rings $O_X$ on $X$.

Examples.
\begin{enumerate} 
\item
Let $X$ be a topological space and let $O_X$ be the sheaf of continuous functions.
\item
Let $X$ be a differentiable manifold and let $O_X$ be the sheaf of differentiable
functions.
\end{enumerate}

\begin{thebibliography}{99}
\bibitem{sha} I.R. Shafarevich, \emph{Basic Algebraic Geometry}, Springer-Verlag, 1977
\end{thebibliography}
%%%%%
%%%%%
\end{document}
