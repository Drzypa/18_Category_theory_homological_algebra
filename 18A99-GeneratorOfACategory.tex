\documentclass[12pt]{article}
\usepackage{pmmeta}
\pmcanonicalname{GeneratorOfACategory}
\pmcreated{2013-03-22 18:21:04}
\pmmodified{2013-03-22 18:21:04}
\pmowner{CWoo}{3771}
\pmmodifier{CWoo}{3771}
\pmtitle{generator of a category}
\pmrecord{10}{40987}
\pmprivacy{1}
\pmauthor{CWoo}{3771}
\pmtype{Definition}
\pmcomment{trigger rebuild}
\pmclassification{msc}{18A99}
\pmrelated{GrothendieckCategory}
\pmdefines{generator}
\pmdefines{generating set}
\pmdefines{progenerator}

\usepackage{amssymb,amscd}
\usepackage{amsmath}
\usepackage{amsfonts}
\usepackage{mathrsfs}

% used for TeXing text within eps files
%\usepackage{psfrag}
% need this for including graphics (\includegraphics)
%\usepackage{graphicx}
% for neatly defining theorems and propositions
\usepackage{amsthm}
% making logically defined graphics
%%\usepackage{xypic}
\usepackage{pst-plot}

% define commands here
\newcommand*{\abs}[1]{\left\lvert #1\right\rvert}
\newtheorem{prop}{Proposition}
\newtheorem{thm}{Theorem}
\newtheorem{ex}{Example}
\newcommand{\real}{\mathbb{R}}
\newcommand{\pdiff}[2]{\frac{\partial #1}{\partial #2}}
\newcommand{\mpdiff}[3]{\frac{\partial^#1 #2}{\partial #3^#1}}
\begin{document}
Let $\mathcal{C}$ be a category, and $f,g:A\to B$ a pair of distinct morphisms.  A morphism $h:X\to A$ is said to \emph{distinguish} or \emph{separate} $f$ and $g$ if $f\circ h\ne g\circ h$.  For example, if $f\ne g:A\to B$, then $1_A$ on $A$ distinguishes $f$ and $g$.

A set $S=\lbrace X_i\mid i\in I\rbrace$ of objects (indexed by a set $I$) is called a \emph{generating set} of $\mathcal{C}$ if any pair of distinct morphisms $f,g:A\to B$ can be distinguished by a morphism with domain in $S$ and codomain $A$.  In other words, there is $h:X_i\to A$ for some $i\in I$, such that $f\circ h\ne g\circ h$.  If $\lbrace X\rbrace$ is a generating family of $\mathcal{C}$, then $X$ is called a \emph{generator} of $\mathcal{C}$.  Any set of morphisms containing a generator is a generating set.

\textbf{Examples}
\begin{enumerate}
\item
In \textbf{Set}, the category of sets, any singleton is a generator.  Suppose $f,g:A\to B$ are distinct functions, so that $f(x)\ne g(x)$ for some $x\in A$.  Let $\lbrace y\rbrace$ be any singleton.  Then $h:\lbrace y\rbrace \to A$ defined by $h(y)=x$ is the function distinguishing $f$ and $g$: for $f\circ h(y)=f(x)\ne g(x)=g\circ h(y)$.
\item
In \textbf{Rng}, the category of rings, the ring $\mathbb{Z}$ is a generator.  If $f,g:R\to S$ are distinct ring homomorphisms, say, $f(r)\ne g(r)$ for some $r\in R$.  Then the ring homomorphism $h:\mathbb{Z}\to R$ given by $h(1)=r$ distinguishes $f$ and $g$.
\end{enumerate}

\textbf{Remark}.  A projective object that is also a generator is called a \emph{progenerator}.


\begin{thebibliography}{9}
\bibitem{fb} F. Borceux \emph{Basic Category Theory, Handbook of Categorical Algebra I}, Cambridge University Press, Cambridge (1994)
\end{thebibliography}
%%%%%
%%%%%
\end{document}
