\documentclass[12pt]{article}
\usepackage{pmmeta}
\pmcanonicalname{ChainMap}
\pmcreated{2013-03-22 12:13:11}
\pmmodified{2013-03-22 12:13:11}
\pmowner{RevBobo}{4}
\pmmodifier{RevBobo}{4}
\pmtitle{chain map}
\pmrecord{7}{31571}
\pmprivacy{1}
\pmauthor{RevBobo}{4}
\pmtype{Definition}
\pmcomment{trigger rebuild}
\pmclassification{msc}{18G35}

% this is the default PlanetMath preamble.  as your knowledge
% of TeX increases, you will probably want to edit this, but
% it should be fine as is for beginners.

% almost certainly you want these
\usepackage{amssymb}
\usepackage{amsmath}
\usepackage{amsfonts}

% used for TeXing text within eps files
%\usepackage{psfrag}
% need this for including graphics (\includegraphics)
%\usepackage{graphicx}
% for neatly defining theorems and propositions
%\usepackage{amsthm}
% making logically defined graphics
%%%\usepackage{xypic} 

% there are many more packages, add them here as you need them

% define commands here
\begin{document}
Let $(A,d)$ and $(A^{'},d^{'})$ be chain complexes. A \emph{chain map} $f:A \to A^{'}$ is a sequence of homomorphisms $\{f_n\}$ such that $d_{n}^{'} \circ f_{n} =  f_{n-1} \circ d_{n}$ for each $n$. Diagramatically, this says that the following diagram commutes: 
$$
\xymatrix{
& A_{n} \ar[d]^{f_n} \ar[r]^{d_{n}} & A_{n-1} \ar[d]^{f_{n-1}} \\
& A_{n}^{'} \ar[r]^{d_{n}^{'}} & A_{n-1}^{'}
}
$$
%%%%%
%%%%%
%%%%%
\end{document}
