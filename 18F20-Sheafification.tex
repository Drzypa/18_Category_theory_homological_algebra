\documentclass[12pt]{article}
\usepackage{pmmeta}
\pmcanonicalname{Sheafification}
\pmcreated{2013-03-22 12:37:38}
\pmmodified{2013-03-22 12:37:38}
\pmowner{djao}{24}
\pmmodifier{djao}{24}
\pmtitle{sheafification}
\pmrecord{8}{32889}
\pmprivacy{1}
\pmauthor{djao}{24}
\pmtype{Definition}
\pmcomment{trigger rebuild}
\pmclassification{msc}{18F20}
\pmclassification{msc}{54B40}
\pmclassification{msc}{14F05}
\pmsynonym{associated sheaf}{Sheafification}
\pmrelated{Sheafification2}

% this is the default PlanetMath preamble.  as your knowledge
% of TeX increases, you will probably want to edit this, but
% it should be fine as is for beginners.

% almost certainly you want these
\usepackage{amssymb}
\usepackage{amsmath}
\usepackage{amsfonts}

% used for TeXing text within eps files
%\usepackage{psfrag}
% need this for including graphics (\includegraphics)
%\usepackage{graphicx}
% for neatly defining theorems and propositions
%\usepackage{amsthm}
% making logically defined graphics
\usepackage[all]{xypic} 

% there are many more packages, add them here as you need them

% define commands here

\newcommand{\A}{\mathcal{A}}
\newcommand{\lra}{\longrightarrow}
\begin{document}
Let $F$ be a presheaf over a topological space $X$ with values in a category $\A$ for which sheaves are defined. The {\em sheafification} of $F$, if it exists, is a sheaf $F'$ over $X$ together with a morphism $\theta: F \lra F'$ satisfying the following universal property:

\begin{quotation}
For any sheaf $G$ over $X$ and any morphism of presheaves $\phi: F \lra G$ over $X$, there exists a unique morphism of sheaves $\psi: F' \lra G$ such that the diagram
$$
\xymatrix{
F \ar[r]^{\theta} \ar@/_1pc/[rr]_{\phi} & F' \ar[r]^{\psi} & G
}
$$
commutes.
\end{quotation}
In light of the universal property, the sheafification of $F$ is uniquely defined up to canonical isomorphism whenever it exists. In the case where $\A$ is a concrete category (one consisting of sets and set functions), the sheafification of any presheaf $F$ can be constructed by taking $F'(U)$ to be the set of all functions $s: U \lra \bigcup_{p \in U} F_p$ such that
\begin{enumerate}
\item $s(p) \in F_p$ for all $p \in U$
\item For all $p \in U$, there is a neighborhood $V \subset U$ of $p$ and a section $t \in F(V)$ such that, for all $q \in V$, the induced element $t_q \in F_q$ equals $s(q)$
\end{enumerate}
for all open sets $U \subset X$. Here $F_p$ denotes the stalk of the presheaf $F$ at the point $p$.

The following quote, taken from~\cite{mumford}, is perhaps the best explanation of sheafification to be found anywhere:
\begin{quotation}
$F'$ is ``the best possible sheaf you can get from $F$''. It is easy to imagine how to get it: first identify things which have the same restrictions, and then add in all the things which can be patched together.
\end{quotation}

\begin{thebibliography}{9}
\bibitem{mumford}{David Mumford, {\em The Red Book of Varieties and Schemes}, Second Expanded Edition, Springer--Verlag, 1999 (LNM {\bf 1358})}
\end{thebibliography}
%%%%%
%%%%%
\end{document}
