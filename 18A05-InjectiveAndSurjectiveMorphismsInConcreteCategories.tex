\documentclass[12pt]{article}
\usepackage{pmmeta}
\pmcanonicalname{InjectiveAndSurjectiveMorphismsInConcreteCategories}
\pmcreated{2013-03-22 18:47:32}
\pmmodified{2013-03-22 18:47:32}
\pmowner{joking}{16130}
\pmmodifier{joking}{16130}
\pmtitle{injective and surjective morphisms in concrete categories}
\pmrecord{4}{41590}
\pmprivacy{1}
\pmauthor{joking}{16130}
\pmtype{Definition}
\pmcomment{trigger rebuild}
\pmclassification{msc}{18A05}

\endmetadata

% this is the default PlanetMath preamble.  as your knowledge
% of TeX increases, you will probably want to edit this, but
% it should be fine as is for beginners.

% almost certainly you want these
\usepackage{amssymb}
\usepackage{amsmath}
\usepackage{amsfonts}

% used for TeXing text within eps files
%\usepackage{psfrag}
% need this for including graphics (\includegraphics)
%\usepackage{graphicx}
% for neatly defining theorems and propositions
%\usepackage{amsthm}
% making logically defined graphics
%%%\usepackage{xypic}

% there are many more packages, add them here as you need them

% define commands here

\begin{document}
Let $(\mathcal{C},U)$ be concrete category over $\mathcal{SET}$ and let $\alpha:X\to Y$ be a morphism in $\mathcal{C}$.

\textbf{Definition.} Morphism $\alpha$ is called \textit{injective} (resp. \textit{surjective}) if $U(\alpha)$ is an injection (resp. surjection).

Some properties of injective and surjective morphisms:

\textbf{Proposition.} Assume that $\alpha:X\to Y$ is a morphism in $\mathcal{C}$.\\
$\mathrm{i)}$ If $\alpha$ is injective (resp. surjective), then $\alpha$ is a monomorphism (resp. epimorphism);\\
$\mathrm{ii)}$ If $\alpha$ is a section (resp. retraction), then $\alpha$ is injective (resp. surjective).

\textit{Proof.} $\mathrm{i)}$ Assume that $\alpha$ is injective and let $\beta_1,\beta_2:Z\to X$ be morphisms in $\mathcal{C}$ such that $\alpha\circ\beta_1=\alpha\circ\beta_2$. Then $$U(\alpha\circ\beta_1)=U(\alpha\circ\beta_2)$$ and this implies that $$U(\alpha)\circ U(\beta_1)=U(\alpha)\circ U(\beta_1).$$
Since $U(\alpha)$ is injective, we obtain that $U(\beta_1)=U(\beta_2)$ and since $U$ is faithful, we get that $$\beta_1=\beta_2.$$
Analogously we prove, that surjective morphism is an epimorphism.\\

$\mathrm{ii)}$. Assume that $\alpha:X\to Y$ is a section. Then there exists $\beta:Y\to X$ such that $\beta\circ\alpha=\mathrm{id}_{X}$. Thus we have
$$U(\beta)\circ U(\alpha)=\mathrm{id}_{U(X)},$$
so $U(\alpha)$ is injective. Analogously retractions are surjective. $\square$
%%%%%
%%%%%
\end{document}
