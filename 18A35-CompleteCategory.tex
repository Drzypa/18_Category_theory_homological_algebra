\documentclass[12pt]{article}
\usepackage{pmmeta}
\pmcanonicalname{CompleteCategory}
\pmcreated{2013-03-22 16:15:50}
\pmmodified{2013-03-22 16:15:50}
\pmowner{CWoo}{3771}
\pmmodifier{CWoo}{3771}
\pmtitle{complete category}
\pmrecord{9}{38373}
\pmprivacy{1}
\pmauthor{CWoo}{3771}
\pmtype{Definition}
\pmcomment{trigger rebuild}
\pmclassification{msc}{18A35}
\pmsynonym{finitely complete}{CompleteCategory}
\pmsynonym{finitely cocomplete}{CompleteCategory}
\pmsynonym{cocomplete}{CompleteCategory}
\pmrelated{ExponentialObject}
\pmrelated{CartesianClosedCategory}
\pmdefines{finitely complete category}
\pmdefines{cocomplete category}
\pmdefines{finitely cocomplete category}

\endmetadata

\usepackage{amssymb,amscd}
\usepackage{amsmath}
\usepackage{amsfonts}

% used for TeXing text within eps files
%\usepackage{psfrag}
% need this for including graphics (\includegraphics)
%\usepackage{graphicx}
% for neatly defining theorems and propositions
%\usepackage{amsthm}
% making logically defined graphics
%%\usepackage{xypic}
\usepackage{pst-plot}
\usepackage{psfrag}

% define commands here

\begin{document}
A category $\mathcal{C}$ is said to be a \emph{complete category} if every small diagram has a limit, that is, a limiting cone exists over every small diagram (diagram such that collections of objects and morphisms are sets).

Of course, in a complete category, a product exists for any given set of objects.  Also, a set of morphisms with common domain and codomain has an equalizer.  Conversely, we have
\begin{quote}
in a category $\mathcal{C}$, if the product exists for an arbitrary set of objects, and the equalizer exists for any pair of morphisms with common domain and codomain, then $\mathcal{C}$ is complete.
\end{quote}

\textbf{Examples}
\begin{itemize}
\item \textbf{Set} is complete.
\item \textbf{Group} is complete.
\item \textbf{Vector Space} is complete
\item \textbf{R-module} is complete for a given unital ring $R$.
\item \textbf{Topological Space} is complete.
\end{itemize}

A category $\mathcal{C}$ is said to be \emph{finitely complete} if every finite diagram (sets of objects and morphisms are finite) has a limit.

A similar sufficient condition for a category $\mathcal{C}$ to be finitely complete is for $\mathcal{C}$ to possess a terminal object and that a pullback exists for every pair of morphisms with common codomain.

\textbf{Examples}
\begin{itemize}
\item Any complete category is clearly finitely complete.
\item The subcategories of the above examples consisting of all objects with finite cardinality are finitely complete (but not complete).
\end{itemize}

\textbf{Remark}.  The dual notion of a complete category is that of a \emph{cocomplete category}, and the dual of a finitely complete category is called a \emph{finitely cocomplete category}.
%%%%%
%%%%%
\end{document}
