\documentclass[12pt]{article}
\usepackage{pmmeta}
\pmcanonicalname{PreorderAsACategory}
\pmcreated{2013-03-22 16:44:48}
\pmmodified{2013-03-22 16:44:48}
\pmowner{CWoo}{3771}
\pmmodifier{CWoo}{3771}
\pmtitle{preorder as a category}
\pmrecord{8}{38971}
\pmprivacy{1}
\pmauthor{CWoo}{3771}
\pmtype{Example}
\pmcomment{trigger rebuild}
\pmclassification{msc}{18B35}
\pmdefines{preorder category}
\pmdefines{ordinal category}
\pmdefines{poset category}
\pmdefines{lattice category}

\endmetadata

\usepackage{amssymb,amscd}
\usepackage{amsmath}
\usepackage{amsfonts}

% used for TeXing text within eps files
%\usepackage{psfrag}
% need this for including graphics (\includegraphics)
%\usepackage{graphicx}
% for neatly defining theorems and propositions
\usepackage{amsthm}
% making logically defined graphics
%%\usepackage{xypic}
\usepackage{pst-plot}
\usepackage{psfrag}

% define commands here
\newtheorem{prop}{Proposition}
\newtheorem{thm}{Theorem}
\newtheorem{ex}{Example}
\newcommand{\real}{\mathbb{R}}
\begin{document}
Every preorder $P$ has an associated structure of a category.  Before describing what this category is, we first associate $P$ with a simpler structure, that of a precategory.  

Let's call this $\operatorname{PreCat}(P)$.  The objects of this precategory are elements of $P$ and for every $a,b\in P$, $\hom(a,b)$ is either a singleton if $a\le b$, or the empty set otherwise.  The category associated with $P$ is the category generated by enlarging $\operatorname{PreCat}(P)$.  For now, call this category $\operatorname{Cat}(P)$.  Then we see that the objects of $\operatorname{Cat}(P)$ are again elements of $P$, and for every $a,b\in P$, $\hom(a,b)$ is the set of all finite chains $f$ from $a$ to $b$.

With this association, we see the following constructs also have the structure of a category:

\begin{itemize}
\item a poset: here, a morphism in $\hom(a,b)$ is a finite chain from $a$ to $b$ where successive nodes are related such that the subsequent node covers the prior node
\item a partition of a (non-empty) set (a set with an equivalence relation): $\hom(a,b)$ is non-empty iff $a$ and $b$ belong to the same partition
\item a lattice: every pair of objects have a product and a coproduct
\item a well-ordered set, in particular an ordinal: if $\hom(a,b)$ is non-empty, it is a singleton.  For example, $\textbf{n}$ is the category consisting of objects $0,1,\ldots,n-1$, and if $a\le b$, a morphism in $\hom(a,b)$ is the chain $a\to a+1 \to \cdots \to b-1 \to b$.
\end{itemize}
%%%%%
%%%%%
\end{document}
