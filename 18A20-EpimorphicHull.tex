\documentclass[12pt]{article}
\usepackage{pmmeta}
\pmcanonicalname{EpimorphicHull}
\pmcreated{2013-03-22 18:28:33}
\pmmodified{2013-03-22 18:28:33}
\pmowner{jocaps}{12118}
\pmmodifier{jocaps}{12118}
\pmtitle{epimorphic hull}
\pmrecord{5}{41148}
\pmprivacy{1}
\pmauthor{jocaps}{12118}
\pmtype{Definition}
\pmcomment{trigger rebuild}
\pmclassification{msc}{18A20}
\pmrelated{epimorphicextension}
\pmrelated{essentialextension}
\pmrelated{completeringofquotients}
\pmdefines{epimorphic hull}

% this is the default PlanetMath preamble.  as your knowledge
% of TeX increases, you will probably want to edit this, but
% it should be fine as is for beginners.

% almost certainly you want these
\usepackage{amssymb}
\usepackage{amsmath}
\usepackage{amsfonts}


% used for TeXing text within eps files
%\usepackage{psfrag}
% need this for including graphics (\includegraphics)
%\usepackage{graphicx}
% for neatly defining theorems and propositions
%\usepackage{amsthm}
% making logically defined graphics
%%%\usepackage{xypic}

% there are many more packages, add them here as you need them

% define commands here
\newtheorem{theorem}{Theorem}
\newtheorem{remark}[theorem]{Remark}
\begin{document}
Let $\mathcal C$ be a category and $A$ be an object in this category. \underline{The} \emph{epimorphic hull} of $A$ is an 
object $E\in\mathrm{Ob}(C)$ such that there is an $f:A\rightarrow E$ that has the following property :

\begin{enumerate}
\item $f$ is an epimorphic extension
\item $f$ is an essential extension
\item $f$ is (roughly put "maximal epimorphic and essential extension") has the property that for any 
epimorphic and essential extension $g:A\rightarrow B$, there exists a morphism $g:B\rightarrow E$ such that
$$f=g\circ h$$
\end{enumerate}

\begin{remark}
We used "the epimorphic hull", because it can be proven that if an epimorphic hull exists for an object of the category,
it is unique upto isomorphism.
\end{remark}

In the category of semiprime commutative ring the epimorphic hull for every semiprime ring exist. The epimorphic hull of a reduced ring is the intersection of all the von Neumann regular rings that lie between the ring and its complete ring of quotient.
%%%%%
%%%%%
\end{document}
