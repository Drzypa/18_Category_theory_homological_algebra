\documentclass[12pt]{article}
\usepackage{pmmeta}
\pmcanonicalname{MathematicalProgramsInQuantumGravity}
\pmcreated{2013-03-22 18:15:15}
\pmmodified{2013-03-22 18:15:15}
\pmowner{bci1}{20947}
\pmmodifier{bci1}{20947}
\pmtitle{mathematical programs in quantum gravity}
\pmrecord{22}{40850}
\pmprivacy{1}
\pmauthor{bci1}{20947}
\pmtype{Topic}
\pmcomment{trigger rebuild}
\pmclassification{msc}{18D25}
\pmclassification{msc}{18-00}
\pmclassification{msc}{55U99}
\pmclassification{msc}{81-00}
\pmclassification{msc}{81P05}
\pmclassification{msc}{81Q05}
\pmsynonym{mathematical foundations of  quantum gravity theories}{MathematicalProgramsInQuantumGravity}
%\pmkeywords{Quantum Gravity Theories}
%\pmkeywords{Quantum Geometry}
\pmrelated{QuantumGravityTheories}
\pmrelated{NoncommutativeGeometry}
\pmrelated{SpacetimeQuantizationProblemsInQuantumGravityTheories}
\pmrelated{EinsteinFieldEquations}
\pmdefines{developing mathematics for quantum gravity theories}

\endmetadata

% this is the default PlanetMath preamble.  as your knowledge
% of TeX increases, you will probably want to edit this, but
% it should be fine as is for beginners.

% almost certainly you want these
\usepackage{amssymb}
\usepackage{amsmath}
\usepackage{amsfonts}

% used for TeXing text within eps files
%\usepackage{psfrag}
% need this for including graphics (\includegraphics)
%\usepackage{graphicx}
% for neatly defining theorems and propositions
%\usepackage{amsthm}
% making logically defined graphics
%%%\usepackage{xypic}

% there are many more packages, add them here as you need them

% define commands here
\usepackage{amsmath, amssymb, amsfonts, amsthm, amscd, latexsym}
%%\usepackage{xypic}
\usepackage[mathscr]{eucal}

\setlength{\textwidth}{6.5in}
%\setlength{\textwidth}{16cm}
\setlength{\textheight}{9.0in}
%\setlength{\textheight}{24cm}

\hoffset=-.75in     %%ps format
%\hoffset=-1.0in     %%hp format
\voffset=-.4in

\theoremstyle{plain}
\newtheorem{lemma}{Lemma}[section]
\newtheorem{proposition}{Proposition}[section]
\newtheorem{theorem}{Theorem}[section]
\newtheorem{corollary}{Corollary}[section]

\theoremstyle{definition}
\newtheorem{definition}{Definition}[section]
\newtheorem{example}{Example}[section]
%\theoremstyle{remark}
\newtheorem{remark}{Remark}[section]
\newtheorem*{notation}{Notation}
\newtheorem*{claim}{Claim}

\renewcommand{\thefootnote}{\ensuremath{\fnsymbol{footnote%%@
}}}
\numberwithin{equation}{section}

\newcommand{\Ad}{{\rm Ad}}
\newcommand{\Aut}{{\rm Aut}}
\newcommand{\Cl}{{\rm Cl}}
\newcommand{\Co}{{\rm Co}}
\newcommand{\DES}{{\rm DES}}
\newcommand{\Diff}{{\rm Diff}}
\newcommand{\Dom}{{\rm Dom}}
\newcommand{\Hol}{{\rm Hol}}
\newcommand{\Mon}{{\rm Mon}}
\newcommand{\Hom}{{\rm Hom}}
\newcommand{\Ker}{{\rm Ker}}
\newcommand{\Ind}{{\rm Ind}}
\newcommand{\IM}{{\rm Im}}
\newcommand{\Is}{{\rm Is}}
\newcommand{\ID}{{\rm id}}
\newcommand{\GL}{{\rm GL}}
\newcommand{\Iso}{{\rm Iso}}
\newcommand{\Sem}{{\rm Sem}}
\newcommand{\St}{{\rm St}}
\newcommand{\Sym}{{\rm Sym}}
\newcommand{\SU}{{\rm SU}}
\newcommand{\Tor}{{\rm Tor}}
\newcommand{\U}{{\rm U}}

\newcommand{\A}{\mathcal A}
\newcommand{\Ce}{\mathcal C}
\newcommand{\D}{\mathcal D}
\newcommand{\E}{\mathcal E}
\newcommand{\F}{\mathcal F}
\newcommand{\G}{\mathcal G}
\newcommand{\Q}{\mathcal Q}
\newcommand{\R}{\mathcal R}
\newcommand{\cS}{\mathcal S}
\newcommand{\cU}{\mathcal U}
\newcommand{\W}{\mathcal W}

\newcommand{\bA}{\mathbb{A}}
\newcommand{\bB}{\mathbb{B}}
\newcommand{\bC}{\mathbb{C}}
\newcommand{\bD}{\mathbb{D}}
\newcommand{\bE}{\mathbb{E}}
\newcommand{\bF}{\mathbb{F}}
\newcommand{\bG}{\mathbb{G}}
\newcommand{\bK}{\mathbb{K}}
\newcommand{\bM}{\mathbb{M}}
\newcommand{\bN}{\mathbb{N}}
\newcommand{\bO}{\mathbb{O}}
\newcommand{\bP}{\mathbb{P}}
\newcommand{\bR}{\mathbb{R}}
\newcommand{\bV}{\mathbb{V}}
\newcommand{\bZ}{\mathbb{Z}}

\newcommand{\bfE}{\mathbf{E}}
\newcommand{\bfX}{\mathbf{X}}
\newcommand{\bfY}{\mathbf{Y}}
\newcommand{\bfZ}{\mathbf{Z}}

\renewcommand{\O}{\Omega}
\renewcommand{\o}{\omega}
\newcommand{\vp}{\varphi}
\newcommand{\vep}{\varepsilon}

\newcommand{\diag}{{\rm diag}}
\newcommand{\grp}{{\mathbb G}}
\newcommand{\dgrp}{{\mathbb D}}
\newcommand{\desp}{{\mathbb D^{\rm{es}}}}
\newcommand{\Geod}{{\rm Geod}}
\newcommand{\geod}{{\rm geod}}
\newcommand{\hgr}{{\mathbb H}}
\newcommand{\mgr}{{\mathbb M}}
\newcommand{\ob}{{\rm Ob}}
\newcommand{\obg}{{\rm Ob(\mathbb G)}}
\newcommand{\obgp}{{\rm Ob(\mathbb G')}}
\newcommand{\obh}{{\rm Ob(\mathbb H)}}
\newcommand{\Osmooth}{{\Omega^{\infty}(X,*)}}
\newcommand{\ghomotop}{{\rho_2^{\square}}}
\newcommand{\gcalp}{{\mathbb G(\mathcal P)}}

\newcommand{\rf}{{R_{\mathcal F}}}
\newcommand{\glob}{{\rm glob}}
\newcommand{\loc}{{\rm loc}}
\newcommand{\TOP}{{\rm TOP}}

\newcommand{\wti}{\widetilde}
\newcommand{\what}{\widehat}

\renewcommand{\a}{\alpha}
\newcommand{\be}{\beta}
\newcommand{\ga}{\gamma}
\newcommand{\Ga}{\Gamma}
\newcommand{\de}{\delta}
\newcommand{\del}{\partial}
\newcommand{\ka}{\kappa}
\newcommand{\si}{\sigma}
\newcommand{\ta}{\tau}
\newcommand{\med}{\medbreak}
\newcommand{\medn}{\medbreak \noindent}
\newcommand{\bign}{\bigbreak \noindent}
\newcommand{\lra}{{\longrightarrow}}
\newcommand{\ra}{{\rightarrow}}
\newcommand{\rat}{{\rightarrowtail}}
\newcommand{\oset}[1]{\overset {#1}{\ra}}
\newcommand{\osetl}[1]{\overset {#1}{\lra}}
\newcommand{\hr}{{\hookrightarrow}}
\begin{document}
 There are several distinct research programs aimed at developing the mathematical foundations of
quantum gravity theories.  These include, but are not limited to, the following. 

\subsection{Mathematical programs developments in quantum gravity}
\begin{enumerate}
\item The twistors program applied to an open curved space-time (see refs. 
\cite{SH2k4, RP2k}), (which is presumably a globally hyperbolic, relativistic space-time). 
This may also include the idea of developing a \emph{`sheaf cohomology'} for twistors (see ref. 
\cite{RP2k}) but still needs to justify the assumption in this approach of a 
charged, fundamental fermion of spin-3/2 of undefined mass and unitary `homogeneity' (which 
has not been observed so far);
\item The \emph{supergravity} theory program, which is consistent with supersymmetry 
and superalgebra, and utilizes \emph{graded Lie algebras} and \emph{matter-coupled 
superfields} in the presence of \emph{weak} gravitational fields;
\item The no boundary (closed), \emph{continuous} space-time programme (ref. 
\cite{SH2k4}) in quantum cosmology, concerned with singularities, such as black 
and `white' holes; S. W. Hawking combines, joins, or glues an initially flat Euclidean 
metric with convex Lorentzian metrics in the expanding, and then contracting, space-times with 
a very small value of Einstein's cosmological `constant'. Such Hawking, double-pear shaped, 
space-times also have an initial Weyl tensor value close to zero and, ultimately, a largely 
fluctuating Weyl tensor during the `final crunch' of our universe, presumed to determine the 
irreversible arrow of time; furthermore, an observer will always be able to access through 
measurements only \emph{a limited part} of the global space-times in our universe; 
\item The TQFT/HQFT approach that aims at finding the topological invariants of a 
manifold embedded in an abstract vector space related to the statistical mechanics problem of 
defining extensions of the partition function for many-particle quantum systems; 
\item  The string and superstring theories/M-theory that `live' in higher dimensional 
spaces (e.g., $n\geq 6$, preferred $n-dim =11$), and can be considered to be topological 
representations of physical entities that vibrate, are quantized, interact, and that might also be able to predict fundamental masses relevant to quantum particles; 
\item The `categorification' and groupoidification programs (\cite{BAJ-DJ98b,BAJ-DJ2k1}) that aims to deal with quantum field and QG problems at the abstract level of categories and functors in what seems to be mostly a global approach; 
\item The `monoidal category' and valuation approach initiated by Isham to the quantum measurement problem and its possible solution through local-to-global, finite constructions in small categories. 
\end{enumerate}

\begin{thebibliography}{9}

\bibitem{SH2k4}
S.Hawkings. 2004. \emph{The beginning of time}. 

\bibitem{RP2k}
R. Penrose. 2000. {Shadows of the mind.}, Cambridge University Press: Cambridge, UK.

\bibitem{BAJ-DJ98b}
Baez, J. and Dolan, J., 1998b, \emph{``Categorification'', Higher Category Theory, Contemporary Mathematics}, 
\textbf{230}, Providence: \emph{AMS}, 1-36. 

\bibitem{BAJ-DJ2k1}
Baez, J. and Dolan, J., 2001, From Finite Sets to Feynman Diagrams, in \emph{Mathematics Unlimited -- 2001 and Beyond}, Berlin: Springer, pp. 29--50. 


\end{thebibliography}

%%%%%
%%%%%
\end{document}
