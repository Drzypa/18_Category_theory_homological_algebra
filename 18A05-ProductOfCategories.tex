\documentclass[12pt]{article}
\usepackage{pmmeta}
\pmcanonicalname{ProductOfCategories}
\pmcreated{2013-03-22 16:37:05}
\pmmodified{2013-03-22 16:37:05}
\pmowner{CWoo}{3771}
\pmmodifier{CWoo}{3771}
\pmtitle{product of categories}
\pmrecord{10}{38816}
\pmprivacy{1}
\pmauthor{CWoo}{3771}
\pmtype{Definition}
\pmcomment{trigger rebuild}
\pmclassification{msc}{18A05}
\pmrelated{CategoricalDirectProduct}
\pmrelated{DisjointUnionOfCategories}
\pmdefines{product category}
\pmdefines{projection functor}

\usepackage{amssymb,amscd}
\usepackage{amsmath}
\usepackage{amsfonts}

% used for TeXing text within eps files
%\usepackage{psfrag}
% need this for including graphics (\includegraphics)
%\usepackage{graphicx}
% for neatly defining theorems and propositions
%\usepackage{amsthm}
% making logically defined graphics
%%\usepackage{xypic}
\usepackage{pst-plot}
\usepackage{psfrag}

% define commands here

\begin{document}
There are occasions when we need to consider objects from different categories being paired up.  For example, if $\mathcal{C}$ is a category, then $\hom(A,B)$ where $A,B$ are objects of $\mathcal{C}$ is a set, and we can think of $\hom$ as a functor from $\mathcal{C}^{\operatorname{op}}\times \mathcal{C}$ to the category of sets.  But what is $\mathcal{C}^{\operatorname{op}}\times \mathcal{C}$ exactly?  We will give this a formal definition presently.

Let $\mathcal{C}$ and $\mathcal{D}$ be categories.  Define the \emph{Cartesian product} $\mathcal{C}\times \mathcal{D}$ of $\mathcal{C}$ and $\mathcal{D}$ as the following pair $(O,M)$, where
\begin{itemize}
\item $O$ is the class consisting of ordered pairs $(X,Y)$, where $X$ is an object in $\mathcal{C}$ and $Y$ is an object in $\mathcal{D}$
\item $M$ is the class consisting of ordered pairs $(f,g)$, where $f$ is a morphism in $\mathcal{C}$ and $g$ is a morphism in $\mathcal{D}$.
\end{itemize}
There is a category structure on $\mathcal{C}\times \mathcal{D}$.  But several things need to be defined first.
\begin{enumerate}
\item
Elements of $O$ are called the \emph{objects} of $\mathcal{C}\times \mathcal{D}$ and elements of $M$ are called the \emph{morphisms} of $\mathcal{C}\times\mathcal{D}$.  For each morphism $(f,g)\in M$, we define the domain and codomain operations $$\operatorname{dom}(f,g):=(\operatorname{dom}(f), \operatorname{dom}(g))\quad\mbox{ and }\quad \operatorname{cod}(f,g):=(\operatorname{cod}(f), \operatorname{cod}(g)).$$  Note that for simplicity, we have used the same symbol $\operatorname{dom}$ and $\operatorname{cod}$ for $\mathcal{C},\mathcal{D}$, and $\mathcal{C}\times \mathcal{D}$.
\item
Next, for each pair of objects $A,B\in \mathcal{C}\times\mathcal{D}$, we have a set $\hom(A,B)$ consisting of all morphisms in $\mathcal{C}\times \mathcal{D}$ whose domain is $A$ and codomain is $B$.  Note that $\hom(A,B)$ is a set because it is $\hom(X,Z)\times \hom(Y,T)$, where $A=(X,Y)$ and $B=(Z,T)$ and each component in the product is assumed to be a set.
\item
Finally, for objects $A,B,C$ in $\mathcal{C}\times\mathcal{D}$, we have a function $\circ$ called composition: 
$$\circ:\hom(A,B)\times \hom(B,C)\to \hom(A,C).$$  To define $\circ$, write each object $A,B,C$ as ordered pairs: $A=(X,Y)$, $B=(Z,T)$, $C=(U,V)$.  In addition, let $\alpha=(f,g)\in \hom(A,B)$ and $\beta=(p,q)\in \hom(B,C)$.  Then $$\circ(\alpha,\beta):=(\circ_1(f,p),\circ_2(g,q)),$$ where $\circ_1$ and $\circ_2$ are compositions defined in $\mathcal{C}$ and $\mathcal{D}$ respectively, such that 
$$\circ_1:\hom(X,Z)\times \hom(Z,U)\to \hom(X,U)\mbox{ and }\circ_2:\hom(Y,T)\times \hom(T,V)\to \hom(Y,V).$$
As usual, we write $\beta\circ \alpha$ for $\circ(\alpha,\beta)$.
\item
Now, it is not hard to see that $\mathcal{C}\times \mathcal{D}$ with $\circ$ is a category.  For example, let us verify that $(A,B)\ne (C,D)$ implies $\hom(A,B)\cap \hom(C,D)=\varnothing$.  Write $A=(X,Y)$, $B=(Z,S)$, $C=(T,U)$ and $D=(V,W)$.  Suppose $\alpha=(f,g)\in \hom(A,B)\cap \hom(C,D)$.  Then $f\in \hom(X,Z)\cap \hom(T,V)$ and $g\in \hom(Y,S)\cap \hom(U,W)$.  But this implies $X=T$, $Z=V$, $Y=U$, and $S=W$.  So $A=(X,Y)=(T,U)=C$ and $B=(Z,S)=(V,W)=D$.
\end{enumerate}

\textbf{Remarks}.
\begin{itemize}
\item
The above construction can be generalized to $n$-fold Cartesian products.  If $\mathcal{C}_1,\ldots,\mathcal{C}_n$ be categories.  Then $\mathcal{C}:=\mathcal{C}_1\times\cdots \mathcal{C}_n$ can be defined much the same way as in the case $n=2$.  $\mathcal{C}$ is a category and is sometimes written $\prod_{i=1}^n \mathcal{C}_i$.
\item
Associated with this product, we can form $n$ (covariant) functors called \emph{projection functors} $\Pi_i:\mathcal{C}\to \mathcal{C}_i$, given by $\Pi_i(A)=A_i$ and $\Pi_i(\alpha)=\alpha_i$, where $A=(A_1,\ldots,A_n)$ and $\alpha=(\alpha_1,\ldots,\alpha_n)$.
\item
The product $\mathcal{C}$ of $\mathcal{C}_i$ also enjoys the universal property that for every category $\mathcal{D}$ and functors $F_i: \mathcal{D}\to \mathcal{C}_i$, there is a unique functor $F:\mathcal{D}\to \mathcal{C}$ such that $\Pi_i\circ F=F_i$ (in other words, $F_i$ factors through $F$).
\item 
If fact, any category that enjoys the universal property described above is naturally equivalent to the product of $\mathcal{C}_i$.  We may actually define product category this way, and then prove its existence using the construction that is given as the definition at the beginning of this article.
\item
More generally, we can define arbitrary (direct) product of categories.  The definition is completely similar to the one above.  If $\lbrace \mathcal{C}_i\mid i\in I\rbrace$ is a family of categories indexed by a set $I$, we often write $\prod_{i\in I} \mathcal{C}_i$ as the product category.  Objects and morphisms are written $(A_i)_{i\in I}$ and $(\alpha_i)_{i\in I}$ respectively.  When all the $\mathcal{C}_i$ are identical, say, equal to $\mathcal{C}$, we also write the product as $\mathcal{C}^I$, and call it the \emph{$I$-fold direct product of} $\mathcal{C}$.
\item The existence of the product of categories indexed by an arbitrary set shows that the category of (small) categories \textbf{Cat} has products.
\item 
Let $\mathcal{C}=\mathcal{D}\times\mathcal{E}$.  Then we may identify $\mathcal{D}$ as a subcategory of $\mathcal{C}$: for each object $E$ in $\mathcal{E}$, define $F_E:\mathcal{D}\to\mathcal{C}$, by $F_E(A):=(A,E)$ and $F(\alpha)=(\alpha,1_E)$.  Then $F_E$ is a faithful functor.  The image $\mathcal{C}_E$ of $F_E$ (with objects $(A,E)$ and morphisms $(\alpha,1_E)$) is a subcategory of $\mathcal{C}$.  It is not hard to see that $\mathcal{D}$ and $\mathcal{C}_E$ are \PMlinkname{isomorphic as categories}{CategoryIsomorphism}.
\item 
The above also shows that for any objects $A,B$ in $\mathcal{E}$, $\mathcal{C}_A$ and $\mathcal{C}_B$ are isomorphic.
\end{itemize}
%%%%%
%%%%%
\end{document}
