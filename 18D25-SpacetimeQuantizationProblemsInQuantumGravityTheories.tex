\documentclass[12pt]{article}
\usepackage{pmmeta}
\pmcanonicalname{SpacetimeQuantizationProblemsInQuantumGravityTheories}
\pmcreated{2013-03-22 18:15:20}
\pmmodified{2013-03-22 18:15:20}
\pmowner{bci1}{20947}
\pmmodifier{bci1}{20947}
\pmtitle{space-time quantization problems in quantum gravity theories}
\pmrecord{21}{40851}
\pmprivacy{1}
\pmauthor{bci1}{20947}
\pmtype{Feature}
\pmcomment{trigger rebuild}
\pmclassification{msc}{18D25}
\pmclassification{msc}{18-00}
\pmclassification{msc}{55U99}
\pmclassification{msc}{81-00}
\pmclassification{msc}{81P05}
\pmclassification{msc}{81Q05}
\pmsynonym{mathematical foundations of  quantum gravity theories}{SpacetimeQuantizationProblemsInQuantumGravityTheories}
%\pmkeywords{quantizing spacetime problems in quantum gravity theories}
\pmrelated{MathematicalProgrammesForDevelopingQuantumGravityTheories}
\pmrelated{QuantumGravityTheories}
\pmdefines{quantizing spacetime problems in quantum gravity theories}

\endmetadata

% this is the default PlanetMath preamble.  as your knowledge
% of TeX increases, you will probably want to edit this, but
% it should be fine as is for beginners.

% almost certainly you want these
\usepackage{amssymb}
\usepackage{amsmath}
\usepackage{amsfonts}

% used for TeXing text within eps files
%\usepackage{psfrag}
% need this for including graphics (\includegraphics)
%\usepackage{graphicx}
% for neatly defining theorems and propositions
%\usepackage{amsthm}
% making logically defined graphics
%%%\usepackage{xypic}

% there are many more packages, add them here as you need them

% define commands here
\usepackage{amsmath, amssymb, amsfonts, amsthm, amscd, latexsym}
%%\usepackage{xypic}
\usepackage[mathscr]{eucal}

\setlength{\textwidth}{6.5in}
%\setlength{\textwidth}{16cm}
\setlength{\textheight}{9.0in}
%\setlength{\textheight}{24cm}

\hoffset=-.75in     %%ps format
%\hoffset=-1.0in     %%hp format
\voffset=-.4in

\theoremstyle{plain}
\newtheorem{lemma}{Lemma}[section]
\newtheorem{proposition}{Proposition}[section]
\newtheorem{theorem}{Theorem}[section]
\newtheorem{corollary}{Corollary}[section]

\theoremstyle{definition}
\newtheorem{definition}{Definition}[section]
\newtheorem{example}{Example}[section]
%\theoremstyle{remark}
\newtheorem{remark}{Remark}[section]
\newtheorem*{notation}{Notation}
\newtheorem*{claim}{Claim}

\renewcommand{\thefootnote}{\ensuremath{\fnsymbol{footnote%%@
}}}
\numberwithin{equation}{section}

\newcommand{\Ad}{{\rm Ad}}
\newcommand{\Aut}{{\rm Aut}}
\newcommand{\Cl}{{\rm Cl}}
\newcommand{\Co}{{\rm Co}}
\newcommand{\DES}{{\rm DES}}
\newcommand{\Diff}{{\rm Diff}}
\newcommand{\Dom}{{\rm Dom}}
\newcommand{\Hol}{{\rm Hol}}
\newcommand{\Mon}{{\rm Mon}}
\newcommand{\Hom}{{\rm Hom}}
\newcommand{\Ker}{{\rm Ker}}
\newcommand{\Ind}{{\rm Ind}}
\newcommand{\IM}{{\rm Im}}
\newcommand{\Is}{{\rm Is}}
\newcommand{\ID}{{\rm id}}
\newcommand{\GL}{{\rm GL}}
\newcommand{\Iso}{{\rm Iso}}
\newcommand{\Sem}{{\rm Sem}}
\newcommand{\St}{{\rm St}}
\newcommand{\Sym}{{\rm Sym}}
\newcommand{\SU}{{\rm SU}}
\newcommand{\Tor}{{\rm Tor}}
\newcommand{\U}{{\rm U}}

\newcommand{\A}{\mathcal A}
\newcommand{\Ce}{\mathcal C}
\newcommand{\D}{\mathcal D}
\newcommand{\E}{\mathcal E}
\newcommand{\F}{\mathcal F}
\newcommand{\G}{\mathcal G}
\newcommand{\Q}{\mathcal Q}
\newcommand{\R}{\mathcal R}
\newcommand{\cS}{\mathcal S}
\newcommand{\cU}{\mathcal U}
\newcommand{\W}{\mathcal W}

\newcommand{\bA}{\mathbb{A}}
\newcommand{\bB}{\mathbb{B}}
\newcommand{\bC}{\mathbb{C}}
\newcommand{\bD}{\mathbb{D}}
\newcommand{\bE}{\mathbb{E}}
\newcommand{\bF}{\mathbb{F}}
\newcommand{\bG}{\mathbb{G}}
\newcommand{\bK}{\mathbb{K}}
\newcommand{\bM}{\mathbb{M}}
\newcommand{\bN}{\mathbb{N}}
\newcommand{\bO}{\mathbb{O}}
\newcommand{\bP}{\mathbb{P}}
\newcommand{\bR}{\mathbb{R}}
\newcommand{\bV}{\mathbb{V}}
\newcommand{\bZ}{\mathbb{Z}}

\newcommand{\bfE}{\mathbf{E}}
\newcommand{\bfX}{\mathbf{X}}
\newcommand{\bfY}{\mathbf{Y}}
\newcommand{\bfZ}{\mathbf{Z}}

\renewcommand{\O}{\Omega}
\renewcommand{\o}{\omega}
\newcommand{\vp}{\varphi}
\newcommand{\vep}{\varepsilon}

\newcommand{\diag}{{\rm diag}}
\newcommand{\grp}{{\mathbb G}}
\newcommand{\dgrp}{{\mathbb D}}
\newcommand{\desp}{{\mathbb D^{\rm{es}}}}
\newcommand{\Geod}{{\rm Geod}}
\newcommand{\geod}{{\rm geod}}
\newcommand{\hgr}{{\mathbb H}}
\newcommand{\mgr}{{\mathbb M}}
\newcommand{\ob}{{\rm Ob}}
\newcommand{\obg}{{\rm Ob(\mathbb G)}}
\newcommand{\obgp}{{\rm Ob(\mathbb G')}}
\newcommand{\obh}{{\rm Ob(\mathbb H)}}
\newcommand{\Osmooth}{{\Omega^{\infty}(X,*)}}
\newcommand{\ghomotop}{{\rho_2^{\square}}}
\newcommand{\gcalp}{{\mathbb G(\mathcal P)}}

\newcommand{\rf}{{R_{\mathcal F}}}
\newcommand{\glob}{{\rm glob}}
\newcommand{\loc}{{\rm loc}}
\newcommand{\TOP}{{\rm TOP}}

\newcommand{\wti}{\widetilde}
\newcommand{\what}{\widehat}

\renewcommand{\a}{\alpha}
\newcommand{\be}{\beta}
\newcommand{\ga}{\gamma}
\newcommand{\Ga}{\Gamma}
\newcommand{\de}{\delta}
\newcommand{\del}{\partial}
\newcommand{\ka}{\kappa}
\newcommand{\si}{\sigma}
\newcommand{\ta}{\tau}
\newcommand{\med}{\medbreak}
\newcommand{\medn}{\medbreak \noindent}
\newcommand{\bign}{\bigbreak \noindent}
\newcommand{\lra}{{\longrightarrow}}
\newcommand{\ra}{{\rightarrow}}
\newcommand{\rat}{{\rightarrowtail}}
\newcommand{\oset}[1]{\overset {#1}{\ra}}
\newcommand{\osetl}[1]{\overset {#1}{\lra}}
\newcommand{\hr}{{\hookrightarrow}}
\begin{document}
\textbf{Mathematical and Physical Problems of Quantizing Spacetime in Quantum Gravity Theories:} \\


Beginning with Riemann there has been a prevailing tradition among mathematicians to leave space-time structure
problems for theoretical physicists to solve; this tradition has been however punctuated by 
mathematical contributions made to the space-time structure problem by major contributors such as 
Minkowski, Poincar\'e, Weyl, viewed in conjunction with, or separate from, those of Einstein and Lorentz.
This tradition is currently changing very rapidly with the major contributors being dedicated mathematicians.
As there are so many excellent contributing mathematicians it is very hard to provide a short list, and thus
only two will suffice here: \'E. Cartan (spinor theory) and A. Connes (noncommutative geometry in 
quantum gravity theories and SUSY). Arguably, such a trend will continue in favor of mathematicians contributing heavily to the foundation of unified physical theories of space and time as well as developing new mathematical concepts for their `own sake', or in their own right. Examples abound to the point that there is talk not only of mathematical, or theoretical physics, but also of physical mathematics (that is, just for the sake of Abstract mathematics); such is the reaction from many in the high-energy physics, `elementary particle' physics community. Negative attitudes aside, this is a relatively new, fertile and rather exciting borderline field at the very junction of mathematics and physics-- where the `intertwinner operators' act on both physical and mathematical spacetimes, or topoi, etc., in order to create new science that is neither `pure' mathematics nor `pure' physics (which has never existed anyway, as it is always expressed in a mathematical `language', theory, model, representation or formalism). \\

The new trend means however much more: it means to employ deeper mathematics and mathematical tools in order to either develop or derive deeper physics. On both sides of the fence many would argue against the other side to maintain a `status quo of pure' mathematics- that has never existed; this would be, of course, as counter-productive as it can be to sciences, in general, and especially to mathematics, computer science and informatics, for example.
\subsection {Quantum fields, symmetry, space-time and connections to general relativity} 

   As the experimental findings in high-energy physics--coupled with 
theoretical studies-- have revealed the presence of new fields and symmetries, 
there appeared the need in modern physics to develop systematic procedures for 
generalizing/generating space-times and quantum state space (QSS) 
representations that reflect the existence of such new fields and symmetries.
    In the general relativity (GR) formulation, the local structure of space-time -- which is characterized by its tensors and curvature -- incorporates the gravitational fields surrounding various masses.  In Einstein's own representation, the `physical space-time of GR' has the structure of a Riemann $R^4$ space over large distances, 
although the detailed local structure of space-time -- as Einstein suggested -- is likely to 
be significantly different. 

    On the other hand, there is a growing consensus in theoretical physics that 
a valid theory of quantum gravity requires a much deeper understanding of the 
small (est)--scale structure of quantum space-time (QST) than currently 
developed. In Einstein's GR theory and his subsequent attempts at developing an 
unified field theory (as in the space concept advocated by Leibnitz), space-time 
does \emph{not} have an \emph{independent existence} from objects, matter or 
fields, but is instead an entity generated by the \emph{continuous} 
transformations of fields \cite{Einstein} (Einstein, 1950, 1954). Hence, the 
continuous nature of space-time adopted in GR and Einstein's subsequent field theoretical 
developments. Furthermore, the quantum, or `quantized', versions of space-time, \emph{QST}, 
are operationally defined through local quantum measurements in general reference frames that 
are prescribed by GR theory. Such a definition is therefore subject to the postulates of both 
GR theory and the axioms of Local Quantum Physics (that are briefly summarized in Subsection 
3.3). We must empasize, however, that this is \emph{not} the usual definition of position and 
time observables in `standard' QM. Therefore, the general reference frame positioning in QST 
is itself subject to the Heisenberg uncertainty principle, and therefore it acquires through 
quantum measurements a certain `fuzziness' at the Planck scale which is intrinsic to  all 
microphysical quantum systems, as further explained in this section.  Whereas Newton, Riemann, 
Einstein, Weyl, Hawking, Weinberg and many other exceptional theoreticians regarded the 
physical space as being represented by a \emph{continuum}, there is an increasing number of 
proponents for a \emph{discrete, `quantized'} structure of space-time. The latter view is not 
without its problems and advantages. The biggest problem for any discrete, `point-set' (or 
discrete topology), view of physical spacetime is not only its immediate conflict with 
Einstein's General Relativity representation of spacetime as a \emph{continuous Riemann} space, but also the 
impossibility of carrying out quantum measurements to localize precisely either quantum events 
or masses at singular (in the sense of disconnected, or isolated), sharply defined, geometric 
points in space-time. One of the proposed resolutions of this problem is Non-commutative 
Geometry (NCG), or `Quantum Geometry', where QST has `no points' (or perhaps no point!), in 
the sense of visualization of such a geometrical space as some kind of a distributive and commutative 
lattice of space-time `points'.  The quantum `metric' of QST in NCG would be related to a 
certain, fundamental quantum field operator, or `fundamental triplet (or quintet)' 
construction (Connes, 2004).  Although quantization is standard in Quantum Mechanics (QM) for 
most of the quantum observables, it does run into major difficulties when applied to position 
and time. In standard QM, there are at least two implemented approaches to solve the problem, 
one of them designed 72 years ago by von Neumann (1933). 
\bigbreak

    Another potential concern is the inadequacy of the long-standing model of 
space-time as a $4$--dimensional manifold \emph{with a Lorentz metric}. The hope of some of 
the earlier approaches to quantum gravity (QG) was to cope with extremely small length scales 
where a manifold structure may be justifiably foresaken (for instance, at the Planck length 
$L_p = (\frac{G\hslash}{c^3})^{\frac{1}{2}} \approx 10^{-35}m$). On the other hand, one needs to 
reconcile the discreteness versus continuum approach in view of 
space--time diffeomorphisms and that space--time may be suitably modeled as some type of 
`combinatorial space' (such as a simplicial complex, a poset, or a spin network) The 
monumental difficulty is that to the present day, apart from a distinct lack of experimental 
evidence, there is no specific agreement on the kind of data, plus no agreement on the actual 
conceptual background to obtaining the data in the first place(!) This difficulty equates with 
how one can relate the approaches to QG to run the gauntlet of conceptual problems in QFT and 
(General Relativity) GR. To quote an example, the space--time metric tensor:
$\gamma = (\gamma_{ab})$ is less a fundamental field than perhaps once thought since it leads 
to describing an essentially classical gravitational field. A case study in ref. \cite 
{BIsham1} involves quantizing one side of Einstein's field equations by a quantum expectation value, so that a coupling of $\gamma$ to quantized matter is given by an expression such as:

$$ G_{\mu \nu} (\gamma) = \langle~ \psi ~\vert T_{\mu \nu} (g, \hat{\phi} \vert~ \psi 
~\rangle~, $$
\bigbreak
where $ \vert \psi \rangle$ denotes a state in the Hilbert space of quantized matter variables $\hat{\phi}$, and the subsequent source of the gravitational field is given by the expectation %%@
of the corresponding energy--momentum tensor $T_{\mu \nu}$~. Unfortunately, this expression is 
not without its ontological and `physical' problems sufficiently serious to prevent the 
development of a complete QG theory that includes this expression. Three possible approaches 
were suggested by Butterfield and Isham in ref. \cite {BIsham1} (cf. also an extensive survey 
article by Rovelli, 1997):



\begin{itemize}
\item[(1)] to develop and test a quantized form of classical relativity theory;
\item[(2)]to recover GR as the low energy limit of a QFT approach which is not a quantization 
of a classical theory (e.g., \emph{via quantum algebras/groups and their representations});
\item[(3)]to develop a new theory, such as a `quantization of topology' or 
`causal' structures where, for instance, microphysical states provide amplitudes to the values 
of quantities whose norms squared define probabilities of occurrence for \emph{physical}, 
quantum events.
\end{itemize}

    We turn now to another facet of quantum measurement. Note first that QFT 
pure states resist description in terms of field configurations since the former are not 
always physically either observable or interpretable. Algebraic Quantum Field Theory (AQFT) as 
expounded by Roberts (2004) points to various questions raised by considering theories of 
(unbounded) operator--valued distributions and \emph{quantum field nets} of von Neumann 
algebras. Using in part a gauge theoretic approach, the idea is to regard two field theories 
as equivalent when their associated nets of observables are isomorphic. More specifically, 
AQFT considers taking \emph{additive nets of quantum field algebras} over subsets of Minkowski 
space, which among other properties, enjoy Bose--Fermi commutation relations. There may be 
analogs with sheaf theory in this approach, even though these analogs appear to be limited. 
The typical AQFT net does not seem to give rise to a presheaf because the relevant morphism 
orientations are in reverse. Closer then is to regard a net as a precosheaf, but the 
additivity does not allow proceeding to  a cosheaf structure. This may be a reflection of some 
deeper incompatibility of AQFT with those aspects of quantum gravity (QG) where the sheaf--
theoretic/topos approaches are advocated (as, for example, in \cite {BIsham1} (Butterfield and 
Isham, 1999-2004).

\begin{thebibliography}{9}
\bibitem{BIsham1}
Butterfield, J. and C. J. Isham: 2001, Space-time and the
philosophical challenges of quantum gravity., in C. Callender and
N. Hugget (eds. ) \emph{Physics Meets Philosophy at the Planck
scale.}, Cambridge University Press,pp.33--89.

\bibitem{BIsham2}
Butterfield, J. and C. J. Isham: 1998, 1999, 2000--2002, A topos
perspective on the Kochen--Specker theorem I - IV, \emph{Int. J.
Theor. Phys}, \textbf{37}  No 11., 2669--2733 \textbf{38} No 3.,
827--859, \textbf{39} No 6., 1413--1436, \textbf{41} No 4.,
613--639.

\end{thebibliography}
%%%%%
%%%%%
\end{document}
