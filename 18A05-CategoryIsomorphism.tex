\documentclass[12pt]{article}
\usepackage{pmmeta}
\pmcanonicalname{CategoryIsomorphism}
\pmcreated{2013-03-22 14:22:04}
\pmmodified{2013-03-22 14:22:04}
\pmowner{CWoo}{3771}
\pmmodifier{CWoo}{3771}
\pmtitle{category isomorphism}
\pmrecord{9}{35852}
\pmprivacy{1}
\pmauthor{CWoo}{3771}
\pmtype{Definition}
\pmcomment{trigger rebuild}
\pmclassification{msc}{18A05}

\endmetadata

% this is the default PlanetMath preamble.  as your knowledge
% of TeX increases, you will probably want to edit this, but
% it should be fine as is for beginners.

% almost certainly you want these
\usepackage{amssymb,amscd}
\usepackage{amsmath}
\usepackage{amsfonts}

% used for TeXing text within eps files
%\usepackage{psfrag}
% need this for including graphics (\includegraphics)
%\usepackage{graphicx}
% for neatly defining theorems and propositions
%\usepackage{amsthm}
% making logically defined graphics
%%%\usepackage{xypic}

% there are many more packages, add them here as you need them

% define commands here
\begin{document}
Let $\mathcal{C}$ and $\mathcal{D}$ be categories.  An
\emph{isomorphism} $T:\mathcal{C}\to\mathcal{D}$ is a (covariant)
functor which has a \emph{two-sided inverse}.  In other words, there
is a (covariant) functor $S:\mathcal{D}\to\mathcal{C}$ such that
$T\circ S=I_{\mathcal{D}}$ and $S\circ T=I_{\mathcal{C}}$, where
$I_{\mathcal{D}}$ and $I_{\mathcal{C}}$ are the identity functors of
$\mathcal{D}$ and $\mathcal{C}$ respectively.  Two categories
$\mathcal{C}$ and $\mathcal{D}$ are \emph{isomorphic} if there exists
a functor $T:\mathcal{C}\to\mathcal{D}$ that is an isomorphism.

\textbf{Remarks}
\begin{enumerate}
\item
An isomorphism (functor) from $\mathcal{C}$ to $\mathcal{D}$ is just
an \PMlinkname{isomorphism}{Isomorphism2} (in the sense of morphism)
in the functor category $\mathcal{D}^{\mathcal{C}}$.

\item
Two isomorphic categories are
\PMlinkname{equivalent}{EquivalenceOfCategories}.  The converse is not
true.  For example, the category of all finite sets is
\PMlinkescapetext{equivalent} to its subcategory of all finite
ordinals.  But clearly these two categories are not isomorphic.
Isomorphism has a ``size'' restriction, whereas natural equivalence
does not.  
\end{enumerate}

\PMlinkescapeword{isomorphism}
\PMlinkescapeword{isomorphic}
\PMlinkescapeword{restriction}


%%%%%
%%%%%
\end{document}
