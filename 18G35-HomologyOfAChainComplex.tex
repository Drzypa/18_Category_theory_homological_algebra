\documentclass[12pt]{article}
\usepackage{pmmeta}
\pmcanonicalname{HomologyOfAChainComplex}
\pmcreated{2013-03-22 13:14:45}
\pmmodified{2013-03-22 13:14:45}
\pmowner{yark}{2760}
\pmmodifier{yark}{2760}
\pmtitle{homology of a chain complex}
\pmrecord{16}{33721}
\pmprivacy{1}
\pmauthor{yark}{2760}
\pmtype{Definition}
\pmcomment{trigger rebuild}
\pmclassification{msc}{18G35}
\pmsynonym{homology of a complex}{HomologyOfAChainComplex}
\pmsynonym{homology}{HomologyOfAChainComplex}
\pmrelated{ChainComplex}
\pmrelated{HomologyTopologicalSpace}
\pmrelated{Tor}
\pmdefines{homology group}
\pmdefines{homology module}

\endmetadata

\usepackage{amsmath}
\usepackage{amscd}

\newcommand{\im}{\operatorname{im}}
\begin{document}
\PMlinkescapeword{exact}
\PMlinkescapeword{objects}

If $(\mathbf{A},d)$ is a chain complex 
$$\begin{CD}
\cdots@<{d_{n-1}}<<A_{n-1} @<{d_{n}}<< A_n@<{d_{n+1}}<< A_{n+1} @<{d_{n+2}}<<\cdots
\end{CD}$$
then the $n$-th \emph{homology group} (or \emph{homology module})
$H_n(\mathbf{A},d)$ of $(\mathbf{A},d)$
is the quotient module
\[
H_n(\mathbf{A},d)=\frac{\ker d_n}{\im d_{n+1}}.
\]

The chain complex is an \PMlinkname{exact sequence}{ExactSequence} if and only if
all of the homology groups are trivial.
The homology groups can therefore be thought of
as measuring the extent to which the chain complex fails to be exact.

Homology groups of other objects are defined as the homology groups of an associated chain complex. (In particular, see the entry on the \PMlinkname{homology of topological spaces}{HomologyTopologicalSpace}.)

%%%%%
%%%%%
\end{document}
