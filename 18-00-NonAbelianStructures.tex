\documentclass[12pt]{article}
\usepackage{pmmeta}
\pmcanonicalname{NonAbelianStructures}
\pmcreated{2013-03-22 18:12:34}
\pmmodified{2013-03-22 18:12:34}
\pmowner{bci1}{20947}
\pmmodifier{bci1}{20947}
\pmtitle{non-Abelian structures}
\pmrecord{88}{40790}
\pmprivacy{1}
\pmauthor{bci1}{20947}
\pmtype{Topic}
\pmcomment{trigger rebuild}
\pmclassification{msc}{18-00}
\pmclassification{msc}{18A15}
\pmclassification{msc}{03G12}
\pmclassification{msc}{03G30}
\pmclassification{msc}{03G20}
\pmsynonym{nonabelian}{NonAbelianStructures}
\pmsynonym{non-commutative structure}{NonAbelianStructures}
\pmsynonym{non-abelian structure}{NonAbelianStructures}
%\pmkeywords{non-AbelianTheories}
%\pmkeywords{non-commutative structures}
%\pmkeywords{several examples of nonabelian structures}
\pmrelated{NonAbelianTheories}
\pmrelated{AbelianCategory}
\pmrelated{ExamplesOfAbelianCategory}
\pmrelated{AxiomsForAnAbelianCategory}
\pmrelated{GeneralizedVanKampenTheoremsHigherDimensional}
\pmrelated{AxiomaticTheoryOfSupercategories}
\pmrelated{AlgebraicCategoryOfLMnLogicAlgebras}
\pmrelated{CategoricalOntology}
\pmrelated{NonCommutingGraphOfAGrou}
\pmdefines{non-Abelian structure}

\endmetadata

% this is the default PlanetMath preamble. as your knowledge
% of TeX increases, you will probably want to edit this, but
% it should be fine as is for beginners.

% almost certainly you want these
\usepackage{amssymb}
\usepackage{amsmath}
\usepackage{amsfonts}

% used for TeXing text within eps files
%\usepackage{psfrag}
% need this for including graphics (\includegraphics)
%\usepackage{graphicx}
% for neatly defining theorems and propositions
%\usepackage{amsthm}
% making logically defined graphics
%%%\usepackage{xypic}

% there are many more packages, add them here as you need them

% define commands here
\usepackage{amsmath, amssymb, amsfonts, amsthm, amscd, latexsym}
%%\usepackage{xypic}
\usepackage[mathscr]{eucal}

\setlength{\textwidth}{6.5in}
%\setlength{\textwidth}{16cm}
\setlength{\textheight}{9.0in}
%\setlength{\textheight}{24cm}

\hoffset=-.75in %%ps format
%\hoffset=-1.0in %%hp format
\voffset=-.4in

\theoremstyle{plain}
\newtheorem{lemma}{Lemma}[section]
\newtheorem{proposition}{Proposition}[section]
\newtheorem{theorem}{Theorem}[section]
\newtheorem{corollary}{Corollary}[section]

\theoremstyle{definition}
\newtheorem{definition}{Definition}[section]
\newtheorem{example}{Example}[section]
%\theoremstyle{remark}
\newtheorem{remark}{Remark}[section]
\newtheorem*{notation}{Notation}
\newtheorem*{claim}{Claim}

\renewcommand{\thefootnote}{\ensuremath{\fnsymbol{footnote%%@
}}}
\numberwithin{equation}{section}

\newcommand{\Ad}{{\rm Ad}}
\newcommand{\Aut}{{\rm Aut}}
\newcommand{\Cl}{{\rm Cl}}
\newcommand{\Co}{{\rm Co}}
\newcommand{\DES}{{\rm DES}}
\newcommand{\Diff}{{\rm Diff}}
\newcommand{\Dom}{{\rm Dom}}
\newcommand{\Hol}{{\rm Hol}}
\newcommand{\Mon}{{\rm Mon}}
\newcommand{\Hom}{{\rm Hom}}
\newcommand{\Ker}{{\rm Ker}}
\newcommand{\Ind}{{\rm Ind}}
\newcommand{\IM}{{\rm Im}}
\newcommand{\Is}{{\rm Is}}
\newcommand{\ID}{{\rm id}}
\newcommand{\GL}{{\rm GL}}
\newcommand{\Iso}{{\rm Iso}}
\newcommand{\Sem}{{\rm Sem}}
\newcommand{\St}{{\rm St}}
\newcommand{\Sym}{{\rm Sym}}
\newcommand{\SU}{{\rm SU}}
\newcommand{\Tor}{{\rm Tor}}
\newcommand{\U}{{\rm U}}

\newcommand{\A}{\mathcal A}
\newcommand{\Ce}{\mathcal C}
\newcommand{\D}{\mathcal D}
\newcommand{\E}{\mathcal E}
\newcommand{\F}{\mathcal F}
\newcommand{\G}{\mathcal G}
\newcommand{\Q}{\mathcal Q}
\newcommand{\R}{\mathcal R}
\newcommand{\cS}{\mathcal S}
\newcommand{\cU}{\mathcal U}
\newcommand{\W}{\mathcal W}

\newcommand{\bA}{\mathbb{A}}
\newcommand{\bB}{\mathbb{B}}
\newcommand{\bC}{\mathbb{C}}
\newcommand{\bD}{\mathbb{D}}
\newcommand{\bE}{\mathbb{E}}
\newcommand{\bF}{\mathbb{F}}
\newcommand{\bG}{\mathbb{G}}
\newcommand{\bK}{\mathbb{K}}
\newcommand{\bM}{\mathbb{M}}
\newcommand{\bN}{\mathbb{N}}
\newcommand{\bO}{\mathbb{O}}
\newcommand{\bP}{\mathbb{P}}
\newcommand{\bR}{\mathbb{R}}
\newcommand{\bV}{\mathbb{V}}
\newcommand{\bZ}{\mathbb{Z}}

\newcommand{\bfE}{\mathbf{E}}
\newcommand{\bfX}{\mathbf{X}}
\newcommand{\bfY}{\mathbf{Y}}
\newcommand{\bfZ}{\mathbf{Z}}

\renewcommand{\O}{\Omega}
\renewcommand{\o}{\omega}
\newcommand{\vp}{\varphi}
\newcommand{\vep}{\varepsilon}

\newcommand{\diag}{{\rm diag}}
\newcommand{\grp}{{\mathbb G}}
\newcommand{\dgrp}{{\mathbb D}}
\newcommand{\desp}{{\mathbb D^{\rm{es}}}}
\newcommand{\Geod}{{\rm Geod}}
\newcommand{\geod}{{\rm geod}}
\newcommand{\hgr}{{\mathbb H}}
\newcommand{\mgr}{{\mathbb M}}
\newcommand{\ob}{{\rm Ob}}
\newcommand{\obg}{{\rm Ob(\mathbb G)}}
\newcommand{\obgp}{{\rm Ob(\mathbb G')}}
\newcommand{\obh}{{\rm Ob(\mathbb H)}}
\newcommand{\Osmooth}{{\Omega^{\infty}(X,*)}}
\newcommand{\ghomotop}{{\rho_2^{\square}}}
\newcommand{\gcalp}{{\mathbb G(\mathcal P)}}

\newcommand{\rf}{{R_{\mathcal F}}}
\newcommand{\glob}{{\rm glob}}
\newcommand{\loc}{{\rm loc}}
\newcommand{\TOP}{{\rm TOP}}

\newcommand{\wti}{\widetilde}
\newcommand{\what}{\widehat}

\renewcommand{\a}{\alpha}
\newcommand{\be}{\beta}
\newcommand{\ga}{\gamma}
\newcommand{\Ga}{\Gamma}
\newcommand{\de}{\delta}
\newcommand{\del}{\partial}
\newcommand{\ka}{\kappa}
\newcommand{\si}{\sigma}
\newcommand{\ta}{\tau}
\newcommand{\med}{\medbreak}
\newcommand{\medn}{\medbreak \noindent}
\newcommand{\bign}{\bigbreak \noindent}
\newcommand{\lra}{{\longrightarrow}}
\newcommand{\ra}{{\rightarrow}}
\newcommand{\rat}{{\rightarrowtail}}
\newcommand{\oset}[1]{\overset {#1}{\ra}}
\newcommand{\osetl}[1]{\overset {#1}{\lra}}
\newcommand{\hr}{{\hookrightarrow}} 
\begin{document}
\begin{definition} Any mathematical structure (algebraic or topological, etc.)--in the sense defined by either
C. Ehresmann \cite{EC65, EC66} or `N. Bourbaki' -- which is {\em not} commutative is usually called either {\em non-Abelian} ({\em non-abelian, nonabelian}) or {\em non-commutative}. 
\end{definition}

\subsection{Examples} 
 Every non-commutative ring, non-commutative group, non-commutative groupoid, non-commutative monoid, non-commutative algebra, and so on, has a {\em non-Abelian structure}; a few specific examples of non-Abelian algebras are : Clifford algebras, matrix algebras, non-commutative $C^*$-algebras, quantum `groups' (non-commuative Hopf algebras), quantum `groupoids' (non-commutative weak Hopf algebras).

\subsection{Remarks}  

 The term `non-Abelian'--instead of noncommutative-- is often the preferred qualifier for theories,
with the possible exeception of `noncommutative geometry', which is a non-Abelian theory.
On the other hand, the term `anabelian' (in French, ``anabelienne'') was employed by Alexander Grothendieck to 
describe a new field of research called ``Anabelian Geometry'' in his `Esquisse d'un Programme''

 The commutativity property in all Abelian structures, such as Abelian groups and Abelian categories is a {\em global} rather than partial, or {\em local}, property. Thus, many categories or toposes/topoi may exhibit
local but not global commutativity properties; for example, a category is still non-Abelian 
if $Hom_{Ab}(-, -)$ does not have the structure of a commutative (or {\em Abelian}) group;
alternatively, a category that does not satisfy one of the four $Ab$-axioms of Freyd, or one of the
$Ab1$ to $Ab6$ axioms in the current abelian category definition is non-Abelian.
\med

\subsection{Many-valued, algebraic logic examples}    
The structures of several, n-valued logic algebras are represented as non-commutative lattices and are, therefore, non-Abelian. Specific examples are provided by the generally non-commutative $LM_n$-logic algebras, categories of $LM_n$-logic algebras and lattice morphisms. On the other hand, the Heyting (intuitionistic logic) algebra subobject classifier of a standard topos is {\em commutative} and thus, it is {\em Abelian}; this does {\em not} mean however that all toposes/topoi have Abelian structure-in fact, this is not generally case. \\

\subsection{Examples of {\em} Abelian categories} 
\begin{enumerate} 
\item The category of Abelian (or commutative) groups is Abelian;
\item The Grothendieck category is a special case of $\mathcal{\A}b5$ category; 
\item Local Grothendieck categories are Abelian categories;
\item The Grassmann category is Abelian;
\item The category of (commutative) semi-noetherian rings is Abelian. (\cite{NP73});
\item The category of Heyting logic algebras, $Hy$ and the category of Boolean logic algebras, \textbf{Boole} are both categories of commutative lattices, and are Abelian categories;
\item If $\grp$ is a topological groupoid the {\em category of sheaves, $S_{hA}$} of Abelian groups over $\grp$ is an Abelian category.
\end{enumerate}

\textbf{Related results} 
If $\mathcal{\G}$ is a Grothendieck catgeory and $\mathcal{\A}$ is a localizing subcategory of $\mathcal{\G}$, then $\mathcal{\G} \slash \mathcal{\A}$ is
Abelian, as it is also a Grothendieck category (\textbf{COROLLARY 6.2} on p. 186 in \cite{NP73}) .

\begin{thebibliography}{9}

\bibitem{NP73}
Popescu, N. {\em Abelian Categories with Applications to Rings and Modules},
(Academic Press: New York and London, 1973 and 1976 edns., {\em English translation by I. C. Baianu}).

\bibitem{Bggb4}
Baianu, I.C.,  R. Brown and J. F. Glazebrook: 2007b, A Non-Abelian, Categorical Ontology of Spacetimes and Quantum Gravity, {\em Axiomathes}, \textbf{17}: 169-225.

\bibitem{Ba-We2k}
Barr, M. and C.~Wells. {\em Toposes, Triples and Theories}. Montreal: McGill University, 2000.
 
\bibitem{BaM98}
Batanin, M., 1998, Monoidal Globular Categories as a Natural Environment for the Theory of Weak n-Categories, Advances in Mathematics, \textbf{136}: 39--103.   

\bibitem{BJL81}
Bell, J. L., 1981, Category Theory and the Foundations of Mathematics, {\em British Journal for the Philosophy of Science}, \textbf{32}: 349-358. 
 
\bibitem{BJL86}
Bell, J. L., 1986, From Absolute to Local Mathematics, {\em Synthese}, \textbf{69} (3): 409-426. 

\bibitem{BJL88} 
Bell, J. L., 1988, {\em Toposes and Local Set Theories: An Introduction}, Oxford: Oxford University Press. 

\bibitem{BG-MCLS99}
Birkoff, G. \& Mac Lane, S., 1999, Algebra, 3rd ed., Providence: AMS.   


\bibitem{Borceux94}
Borceux, F.: 1994, \emph{Handbook of Categorical Algebra}, vols: 1--3, 
in {\em Encyclopedia of Mathematics and its Applications} \textbf{50} to \textbf{52}, Cambridge University Press.

\bibitem{Bourbaki1}
Bourbaki, N. 1961 and 1964: \emph{Alg\`{e}bre commutative.},
in \`{E}l\'{e}ments de Math\'{e}matique., Chs. 1--6., Hermann: Paris.

\bibitem (BJk4)
Brown, R. and G. Janelidze: 2004, Galois theory and a new homotopy double groupoid of a map of spaces, 
\emph{Applied Categorical Structures} \textbf{12}: 63-80.

\bibitem{BHR2}
Brown, R., Higgins, P. J. and R. Sivera,: 2007, \emph{Non-Abelian Algebraic Topology}, 
\PMlinkexternal{in preparation}{http://www.bangor.ac.uk/~mas010/nonab-t/partI010604.pdf}

\bibitem{BGB2k7b}
Brown, R., Glazebrook, J. F. and I.C. Baianu.: 2007b, A Conceptual, Categorical and Higher Dimensional Algebra Framework of Universal Ontology and the Theory of Levels for Highly Complex Structures and Dynamics., \emph{Axiomathes} (17): 321--379.

\bibitem{BP2k3}
Brown R. and T. Porter: 2003, Category theory and higher dimensional algebra: potential descriptive tools in neuroscience, In: {\em Proceedings of the International Conference on Theoretical
Neurobiology.}, Delhi, February 2003, edited by Nandini Singh, National Brain Research Centre, Conference Proceedings \textbf{1}: 80-92.

\bibitem{Br-Har-Ka-Po2k2}
Brown, R., Hardie, K., Kamps, H. and T. Porter: 2002, The homotopy double groupoid of a Hausdorff space., \emph{Theory and Applications of Categories} \textbf{10}, 71-93.

\bibitem{Br-Sp76}
Brown, R. and Spencer, C.B.: 1976, Double groupoids and crossed modules, \emph{Cah.  Top. G\'{e}om. Diff.} \textbf{17}, 343-362.

\bibitem{BDA55}
Buchsbaum, D. A.: 1955, Exact categories and duality., {\em Trans. Amer. Math. Soc.} \textbf{80}: 1-34.

\bibitem{BDA55}
Buchsbaum, D. A.: 1969, A note on homology in categories., {\em Ann. of Math}. \textbf{69}: 66-74.

\bibitem{BI65}
Bucur, I. (1965). {\em Homological Algebra}. (orig. title: ``Algebra Omologica'')
Ed. Didactica si Pedagogica: Bucharest.

\bibitem{BI-DA68}
Bucur, I., and Deleanu A. (1968). {\em  Introduction to the Theory of Categories and Functors}. J.Wiley and Sons: London

\bibitem{BL2k3}
Bunge, M. and S. Lack: 2003, Van Kampen theorems for toposes, \emph{Adv. in Math.} \textbf{179}, 291-317.

\bibitem{BM84}
Bunge, M., 1984, Toposes in Logic and Logic in Toposes, {\em Topoi}, 3, no. 1, 13-22. 

\bibitem{BM-LS2k3}
Bunge M, Lack S (2003) Van Kampen theorems for toposes. {\em Adv Math}, \textbf {179}: 291-317.

\bibitem{BJ-ICJ98-2k2}
Butterfield J., Isham C.J. 1998, 1999, 2000-2002, A topos perspective on the Kochen-Specker theorem
I-IV, Int J Theor Phys 37(11):2669-2733; 38(3):827-859; 39(6):1413-1436; 41(4): 613-639.

\bibitem{CH-ES56}
Cartan, H. and Eilenberg, S. 1956. {\em Homological Algebra}, Princeton Univ. Press: Pinceton. 

\bibitem{Chaician}
M. Chaician and A. Demichev. 1996. Introduction to Quantum Groups, World Scientific.

\bibitem{CC46}
Chevalley, C. 1946. The theory of Lie groups. Princeton University Press, Princeton NJ

\bibitem{CPM65}
Cohen, P.M. 1965. {\em Universal Algebra}, Harper and Row: New York, london and Tokyo.

\bibitem{CA94}
Connes A 1994. {\em Noncommutative geometry}. Academic Press: New York.

\bibitem{CR-LL63}
Croisot, R. and Lesieur, L. 1963. {\em Alg\`ebre noeth\'erienne non-commutative.},
Gauthier-Villard: Paris.

\bibitem{DJ-ALEX60-71}
Dieudonn\'e, J. and Grothendieck, A., 1960, [1971], \'Elements de G\'eom\'etrie Alg\'ebrique, Berlin: Springer-Verlag.  

\bibitem{Dirac30}
Dirac, P. A. M., 1930, {\em The Principles of Quantum Mechanics}, Oxford: Clarendon 
Press. {\em contains important examples of non-Abelian structures at the mathematical foundation of Quantum Mechanics.} 

\bibitem{Dixmier}
Dixmier, J., 1981, Von Neumann Algebras, Amsterdam: North-Holland Publishing 
Company. [First published in French in 1957: Les Alg\`ebres d'Operateurs dans 
l' \'Espace Hilbertien, Paris: Gauthier--Villars.]

\bibitem{Durdevich1}
M. Durdevich : Geometry of quantum principal bundles I, Commun. Math. Phys. \textbf{175} (3) (1996), 457--521.

\bibitem{Durdevich2}
M. Durdevich : Geometry of quantum principal bundles II, Rev. Math. Phys. \textbf{9} (5) (1997), 531--607.

\bibitem{EC65}
Ehresmann, C.: 1965, \emph{Cat\'egories et Structures}, Dunod, Paris.

\bibitem{EC66}
Ehresmann, C.: 1966, Trends Toward Unity in Mathematics., \emph{Cahiers de Topologie et Geometrie Differentielle}
\textbf{8}: 1-7.

\bibitem{Eh-pseudo}
Ehresmann, C.: 1952, Structures locales et structures infinit\'esimales,
\emph{C.R.A.S.} Paris \textbf{274}: 587-589.

\bibitem{Eh}
Ehresmann, C.: 1959, Cat\'egories topologiques et cat\'egories
diff\'erentiables, \emph{Coll. G\'eom. Diff. Glob.} Bruxelles, pp.137-150.

\bibitem{Eh-quintettes}
Ehresmann, C.:1963, Cat\'egories doubles des quintettes: applications covariantes
, \emph{C.R.A.S. Paris}, \textbf{256}: 1891--1894.

\bibitem{Eh-Oe}
Ehresmann, C.: 1984, \emph{Oeuvres compl\`etes et  comment\'ees:
Amiens, 1980-84}, edited and commented by Andr\'ee Ehresmann.

\bibitem{EML1}
Eilenberg, S. and S. Mac Lane.: 1942, Natural Isomorphisms in Group Theory., \emph{American Mathematical Society 43}: 757-831.

\bibitem{EL}
Eilenberg, S. and S. Mac Lane: 1945, The General Theory of Natural Equivalences, \emph{Transactions of the American Mathematical Society} \textbf{58}: 231-294.


\bibitem{EML45}
S. Eilenberg and S. MacLane.1945. Relations between homology 
and homotopy groups of spaces. {\em Ann. of Math.}, \textbf{46}:480--509.


\bibitem{EML50}
Eilenberg, S. and S. MacLane. 1950. Relations between homology and homotopy groups of spaces. II,  
{\em Annals of Mathematics} , \textbf{51}: 514--533.

\bibitem{EML66}
Eilenberg, S. and S. Mac Lane. 1966. Relations between Homology and Homotopy Groups
{\em Proceed. Natl. Acad. Sci. (USA)}, Volume 29, Issue 5, pp. 155--158.

\bibitem{ES-CH56}
Eilenberg, S. and Cartan, H., 1956, \emph{Homological Algebra}, Princeton: Princeton University Press. 

\bibitem{ES-MCLS42}
Eilenberg, S. and MacLane, S., 1942, Group Extensions and Homology, \emph{Annals of Mathematics}, 43, 757--831. 

\bibitem{ES-SN52}
Eilenberg, S. and Steenrod, N., 1952, \emph{Foundations of Algebraic Topology}, Princeton: Princeton University Press. 

\bibitem{ES60}
Eilenberg, S.: 1960. Abstract description of some basic functors., \emph{J. Indian Math.Soc.}, \textbf{24} :221-234.

\bibitem{S.Eilenberg}
S.Eilenberg. Relations between Homology and Homotopy Groups., \emph{Proc.Natl.Acad.Sci.USA} (1966),v:10--14.  

\bibitem{ED88}
Ellerman, D., 1988, Category Theory and Concrete Universals, {\em Synthese}, 28, 409--429. 

\bibitem{ETH}
Z. F. Ezawa, G. Tsitsishvilli and K. Hasebe : Noncommutative geometry, extended $W_{\infty}$ algebra and Grassmannian solitons in multicomponent Hall systems, arXiv:hep--th/0209198.

\bibitem{FS77}
Feferman, S., 1977, Categorical Foundations and Foundations of Category Theory, 
{\em Logic, Foundations of Mathematics and Computability}, R. Butts (ed.), Reidel, 149-169.

\bibitem{Fell}
Fell, J. M. G., 1960, The Dual Spaces of  C*-Algebras, {\em Transactions of the 
American Mathematical Society}, \textbf{94}: 365--403. 

\bibitem{FP60}
Freyd, P., 1960. \emph{Functor Theory} (Dissertation). Princeton University, Princeton, New Jersey.
 
\bibitem{FP63}
Freyd, P., 1963, Relative homological algebra made absolute. , {\em Proc. Natl. Acad. USA}, \textbf{49}:19-20.
  
\bibitem{FP64}
Freyd, P., 1964, {\em Abelian Categories. An Introduction to the Theory of Functors.}, New York and London: Harper and Row.  

\bibitem{FP65}
Freyd, P., 1965, The Theories of Functors and Models., {\em Theories of Models}, Amsterdam: North Holland, 107--120. 

\bibitem{FP66}
Freyd, P., 1966, Algebra-valued Functors in general categories and tensor product in particular., {\em Colloq. Mat}. 
{14}: 89--105.

\bibitem{FP72}
Freyd, P., 1972, Aspects of Topoi, {\em Bulletin of the Australian Mathematical Society}, \textbf{7}: 1--76.  
 
\bibitem{FP90}
Freyd, P., 1990, Categories, Allegories, Amsterdam: North Holland. 

\bibitem{FP2k2}
Freyd, P., 2002, Cartesian Logic, {\em Theoretical Computer Science}, 278, no. 1--2, 3--21.  

\bibitem{FP-FH-SA87}
Freyd, P., Friedman, H. \& Scedrov, A., 1987, Lindembaum Algebras of Intuitionistic Theories and Free Categories, Annals of Pure and Applied Logic, 35, 2, 167--172.

\bibitem{Gabriel1}
Gabriel, P.: 1962, Des cat\'egories ab\'eliennes, \emph{Bull. Soc. Math. France} \textbf{90}: 323-448.

\bibitem{Gabriel2}
Gabriel, P. and M.Zisman:. 1967: Category of fractions and homotopy theory, \emph{Ergebnesse der math.} Springer: Berlin.

\bibitem{GabrielNP}
Gabriel, P. and N. Popescu: 1964, Caract\'{e}risation des cat\'egories ab\'eliennes
avec g\'{e}n\'{e}rateurs et limites inductives. , \emph{CRAS Paris} \textbf{258}: 4188-4191.

\bibitem{GA-RG-SM2k}
Galli, A. \& Reyes, G. \& Sagastume, M., 2000, Completeness Theorems via the Double Dual Functor, {\em Studia Logica}, 64, no. 1, 61--81. 

\bibitem{GN}
Gelfan'd, I. and Naimark, M., 1943, On the Imbedding of Normed Rings into the Ring of Operators in Hilbert Space, {\em Recueil Math\'ematique [Matematicheskii Sbornik] Nouvelle S\'erie}, \textbf{12} [54]: 197--213. [Reprinted in C*--algebras: 
1943--1993, in the series Contemporary Mathematics, 167,  Providence, R.I. : 
American Mathematical Society, 1994.] 

\bibitem{GS-ZM2K2}
Ghilardi, S. \& Zawadowski, M., 2002, Sheaves, Games \& Model Completions: A Categorical Approach to Nonclassical Propositional Logics, Dordrecht: Kluwer.  

\bibitem{GS89}
Ghilardi, S., 1989, Presheaf Semantics and Independence Results for some Non-classical first-order logics, Archive for Mathematical Logic, 29, no. 2, 125--136. 

\bibitem{Gob68}
Goblot, R., 1968, Cat\'egories modulaires , {\em C. R. Acad. Sci. Paris, S\'erie A.}, \textbf{267}: 381--383.

\bibitem{Gob71}
Goblot, R., 1971, Sur deux classes de cat\'egories de Grothendieck, {\em Th\`ese.}, Univ. Lille, 1971.

\bibitem{GR79}
Goldblatt, R., 1979, Topoi: The Categorical Analysis of Logic, Studies in logic and the foundations of mathematics, Amsterdam: Elsevier North-Holland Publ. Comp. 

\bibitem{Goldie}
Goldie, A. W., 1964, \textbf{Localization in non-commutative noetherian rings}, {\em J.Algebra}, \textbf{1}: 286-297.

\bibitem{Godement}
Godement,R. 1958. Th\'{e}orie des faisceaux. Hermann: Paris.

\bibitem{GRAY65}
Gray, C. W.: 1965. Sheaves with values in a category.,\emph {Topology}, 3: 1-18.

\bibitem{Alex1}
Grothendieck, A.: 1971, Rev\^{e}tements \'Etales et Groupe Fondamental (SGA1),
chapter VI: Cat\'egories fibr\'ees et descente, \emph{Lecture Notes in Math.}
\textbf{224}, Springer--Verlag: Berlin.

\bibitem{Alex2}
Grothendieck, A.: 1957, Sur quelque point d-alg\'{e}bre homologique. , \emph{Tohoku Math. J.}, \textbf{9:} 119-121.

\bibitem{Alex3}
Grothendieck, A. and J. Dieudon\'{e}.: 1960, El\'{e}ments de geometrie alg\'{e}brique., \emph{Publ. Inst. des Hautes Etudes de Science}, \textbf{4}.

\bibitem{ALEXsem}
Grothendieck, A. et al., \emph{S\'eminaire de G\'eometrie Alg\'ebrique}, Berlin: Springer-Verlag.
(See also \PMlinkexternal{additional discussion at two websites}{http://en.wikipedia.org/wiki/Grothendieck}, and \PMlinkexternal{also English Abstracts for all volumes}{http://planetphysics.org/?op=getobj&from=books&id=201}).
  
\bibitem{ALEX57}
Grothendieck, A., 1957, Sur Quelques Points d'alg\'ebre homologique, {\em Tohoku Mathematics Journal}, 9, 119--221.
  
\bibitem{GL66}
Gruson, L, 1966, Compl\'etion ab\'elienne. {\em Bull. Math.Soc. France}, \textbf{90}: 17-40.
 
\bibitem{HKK}
K.A. Hardie, K.H. Kamps and R.W. Kieboom, A homotopy 2--groupoid of a Hausdorff 
space, {\em Applied Cat. Structures} 8 (2000), 209--234.

\bibitem{HWS82}
Hatcher, W. S., 1982, {\em The Logical Foundations of Mathematics}, Oxford: Pergamon Press. 
  
\bibitem{Heller58}
Heller, A. :1958, Homological algebra in Abelian categories., \emph{Ann. of Math.}
\textbf{68}: 484-525.

\bibitem{HellerRowe62}
Heller, A.  and K. A. Rowe.:1962, On the category of sheaves., \emph{Amer J. Math.}
\textbf{84}: 205-216.

\bibitem{HG2k3}
Hellman, G., 2003, Does Category Theory Provide a Framework for Mathematical Structuralism ?, Philosophia Mathematica, 11, 2, 129--157. 

\bibitem{HC-MM-PJ2K}
Hermida, C. \& Makkai, M. \& Power, J., 2000, On Weak Higher-dimensional Categories I, Journal of Pure and Applied Algebra, 154, no. 1-3, 221--246. 

\bibitem{HC-MM-PI2K1}
Hermida, C. \& Makkai, M. \& Power, J., 2001, On Weak Higher-dimensional Categories 2, Journal of Pure and Applied Algebra, 157, no. 2-3, 247--277.  

\bibitem{HC-MM-PI2K2}
Hermida, C. \& Makkai, M. \& Power, J., 2002, On Weak Higher-dimensional Categories 3, Journal of Pure and Applied Algebra, 166, no. 1-2, 83--104.  

\bibitem{HPJbook}
Higgins, P. J.: 2005, \emph{Categories and groupoids}, Van Nostrand Mathematical Studies: 32, (1971); \emph{Reprints in
Theory and Applications of Categories}, No. 7: 1-195.

\bibitem{HPJ2k5}
Higgins, Philip J. Thin elements and commutative shells in cubical $\omega$-categories. Theory Appl. Categ. 14 (2005), No. 4, 60--74 (electronic). 18D05.

\bibitem{E.Hurewicz}
E.Hurewicz. CW Complexes. {\em Trans AMS}., \textbf{55}: 737--755(1949), (cited in Spanier, E. H.: 1966, {\em Algebraic Topology}, McGraw Hill: New York. 

\bibitem{Isham1}
Isham, C. J., A new approach to quantising space--time: I. quantising on a general category, \emph{Adv. Theor. Math. Phys.} \textbf{7} (2003), 331--367.

\bibitem{JPT79A}
Johnstone, P. T., 1979a, Conditions Related to De Morgan's Law.,{\em Applications of Sheaves}, Lecture Notes in Mathematics, \textbf{753}, Berlin: Springer, 479--491. 

\bibitem{JPT79B}
Johnstone, P.T., 1979b, Another Condition Equivalent to De Morgan's Law, Communications in Algebra, 7, no. 12, 1309--1312.  

\bibitem{JPT85}
Johnstone, P. T., 1985, How General is a Generalized Space ?, {\em Aspects of Topology}, Cambridge: Cambridge University Press, 77--111. 

\bibitem{JAMI95}
Joyal, A. \& Moerdijk, I., 1995, \emph{Algebraic Set Theory}, Cambridge: Cambridge University Press.  

\bibitem{kampen1-1933}
Van Kampen, E. H.: 1933, On the Connection Between the Fundamental Groups of some Related Spaces, \emph{Amer. J. Math.} \textbf{55}: 261-267

\bibitem{KDM58}
Kan, D. M., 1958, Adjoint Functors, \emph{Transactions of the American Mathematical Society}, 87, 294-329.  

\bibitem{Kleisli62}
Kleisli, H.: 1962, Homotopy theory in Abelian categories.,{\em Can. J. Math.}, \textbf{14}: 139-169.

\bibitem{KJT70}
Knight, J.T., 1970, \emph{On epimorphisms of non-commutative rings.}, {\em Proc. Cambridge Phil. Soc.},
\textbf{25}: 266-271.

\bibitem{LTY}
Lam, T. Y., 1966, The category of noetherian modules, {\em Proc. Natl. Acad. Sci. USA}, \textbf{55}: 1038-104.

\bibitem{LJ-SPJ81}
Lambek, J. \& Scott, P. J., 1981, Intuitionistic Type Theory and Foundations, Journal of Philosophical Logic, 10, 1, 101--115. 

\bibitem{LJ-SPJ86}
Lambek, J. \& Scott, P.J., 1986, Introduction to Higher Order Categorical Logic, Cambridge: Cambridge University Press. 

\bibitem{LJ72}
Lambek, J., 1972, Deductive Systems and Categories III. Cartesian Closed Categories, Intuitionistic Propositional Calculus, and Combinatory Logic, in {\em Toposes, Algebraic Geometry and Logic}, Lecture Notes in Mathematics, 274, Berlin: Springer, 57--82.  

\bibitem{LT89A}
Lambek, J., 1989A, On Some Connections Between Logic and Category Theory, Studia Logica, 48, 3, 269--278. 

\bibitem{LJ89B}
Lambek, J., 1989B, On the Sheaf of Possible Worlds, in {\em Categorical Topology and its relation to Analysis, Algebra and Combinatorics}, Teaneck: World Scientific Publishing, 36--53. 
 
\bibitem{LaSc}
Lambek, J. and P.~J.~Scott. {\em Introduction to higher order categorical logic}. Cambridge University Press, 1986.

\bibitem{Lance}
E. C. Lance :  Hilbert C*--Modules. \emph{London Math. Soc. Lect. Notes} \textbf{210}, \emph{Cambridge Univ. Press.} 1995.

\bibitem{LE99}
Landry, E., 1999, Category Theory: the Language of Mathematics, {\em Philosophy of Science}, 66, 3: supplement, S14--S27. 

\bibitem{LE99}
Landry, E., 2001, Logicism, Structuralism and Objectivity, {\em Topoi}, 20, 1, 79--95. 

\bibitem{LandNP98}
Landsman, N. P.: 1998, \emph{Mathematical Topics between Classical and Quantum Mechanics}, Springer Verlag: New York.

\bibitem{Land1}
N. P. Landsman : Compact quantum groupoids, arXiv:math--ph/9912006

\bibitem{LPRM94}
La Palme Reyes, M., et. al., 1994, The non-Boolean Logic of Natural Language Negation, {\em Philosophia Mathematica}, 2, no. 1, 45--68.

\bibitem{LFW64}
Lawvere, F. W., 1964, An Elementary Theory of the Category of Sets. (ETAC), {\em Proceedings of the National Academy of Sciences U.S.A}., \textbf{52}: 1506-1511. 

\bibitem{LFW65}
Lawvere, F. W., 1965, Algebraic Theories, Algebraic Categories, and Algebraic Functors, {\em Theory of Models.}, 

\bibitem{LFW66}
Lawvere, F. W.: 1966, The Category of Categories as a Foundation for Mathematics., in
\emph{Proc. Conf. Categorical Algebra- La Jolla}., Eilenberg, S. et al., eds. Springer--Verlag:
Berlin, Heidelberg and New York., pp. 1-20.

\bibitem{LFW69a}
Lawvere, F. W., 1969a, Diagonal Arguments and Cartesian Closed Categories, {\em Category Theory, Homology Theory, and their Applications: II}, Berlin: Springer: 134--145.  

\bibitem{LFW69b}
Lawvere, F. W., 1969b, Adjointness in Foundations., {\em Dialectica}, \textbf{23}: 281-295.  

\bibitem{LFW72}
Lawvere, F. W., 1972, Introduction. , in {\em Toposes, Algebraic Geometry and Logic}, Lecture Notes in Mathematics, \textbf{274}, Springer-Verlag, 1-12.  

\bibitem{LFW75}
Lawvere, F. W., 1975, Continuously Variable Sets: Algebraic Geometry = Geometric Logic., {\em Proceedings of the Logic Colloquium,}, Bristol 1973, Amsterdam: North Holland, pp 135-153. 

\bibitem{LFW76}
Lawvere, F. W., 1976, Variable Quantities and Variable Structures in Topoi, {\em Algebra, Topology, and Category Theory}, New York: Academic Press,pp. 101-131. 

\bibitem{LFW63}
Lawvere, F. W.: 1963, Functorial Semantics of Algebraic Theories, \emph{Proc. Natl. Acad. Sci. USA, Mathematics}, \textbf{50}: 869-872.

\bibitem{LFW92}
Lawvere, F. W., 1992, Categories of Space and of Quantity, {\em The Space of Mathematics, Foundations of Communication and Cognition}, Berlin: De Gruyter, 14--30.  

\bibitem{LFW95}
Lawvere, H. W (ed.), 1995. Springer Lecture Notes in Mathematics, \textbf{274}:13-42. 

\bibitem{LFW2k}
Lawvere, F. W., 2000, Comments on the Development of Topos Theory., {\em Development of Mathematics 1950-2000}, Basel: Birkha\"user, pp. 715-734. 

\bibitem{LFW2k2}
Lawvere, F. W., 2002, Categorical Algebra for Continuum Micro-Physics, {\em Journal of Pure and Applied Algebra}, \textbf {175}, no. 1--3, 267--287. 

\bibitem{LFWk3}
Lawvere, F. W., 2003, Foundations and Applications: Axiomatization and Education. New Programs and Open Problems in the Foundation of Mathematics., {\em Bulletin of Symbolic Logic}, \textbf{9}, (2): 213--224. 

\bibitem{LT2k2}
Leinster, T., 2002, A Survey of Definitions of $n$-categories, {\em Theory and Applications of Categories}, (electronic), 10, 1-70. 

\bibitem{Lofgren68}
L\"{o}fgren, L.: 1968, An Axiomatic Explanation of Complete Self-Reproduction, \emph{Bulletin of Mathematical Biophysics}, \textbf{30}: 317-348

\bibitem{LS60}
Lubkin, S., 1960. Imbedding of abelian categories.,  {\em Trans. Amer. Math. Soc.}, \textbf{97}: 410-417.

\bibitem{Mack1}
K. C. H. Mackenzie : \emph{Lie Groupoids and Lie Algebroids in Differential Geometry}, LMS Lect. Notes \textbf{124}, Cambridge University Press, 1987.

\bibitem{MCLSS48}
MacLane, S.: 1948. Groups, categories, and duality., {\em Proc. Natl. Acad. Sci.U.S.A},
\textbf{34}: 263-267.

\bibitem{MCLSS69}
MacLane, S., 1969, Foundations for Categories and Sets., {\em Category Theory, Homology Theory and their Applications. II}, Berlin: Springer, pp. 146-164. 

\bibitem{MCLS69}
MacLane, S., 1969, One Universe as a Foundation for Category Theory., {\em Reports of the Midwest Category Seminar. III}, Berlin: Springer, 192--200. 

\bibitem{MCLS71}
MacLane, S., 1971, Categorical algebra and Set-Theoretic Foundations, {\em Axiomatic Set Theory}, Providence: AMS, 231--240. 

\bibitem{MCLS75}
MacLane, S., 1975, Sets, Topoi, and Internal Logic in Categories., {\em Studies in Logic and the Foundations of Mathematics}, \textbf{80}, Amsterdam: North Holland, pp. 119-134. 

\bibitem{MCLS89}
MacLane, S., 1989, The Development of Mathematical Ideas by Collision: the Case of Categories and Topos Theory, in 
{\em Categorical Topology and its Relation to Analysis, Algebra and Combinatorics}, Teaneck: World Scientific, pp. 1-9.

\bibitem{MS-IM92}
MacLane, S and I. Moerdijk., \emph{Sheaves in Geometry and Logic -- A first Introduction to Topos Theory}, Springer Verlag, New York. 1992. 

\bibitem{MCLS96} 
MacLane, S., 1996, Structure in Mathematics. Mathematical Structuralism., {\em Philosophia Mathematica},
 \textbf{4} (2): 174-183. 

\bibitem{MCLS98}
MacLane, S., 1997, {\em Categories for the Working Mathematician}, 2nd edition, New York: Springer-Verlag. 

\bibitem{Majid1}
Majid, S.: 1995, \emph{Foundations of Quantum Group Theory}, Cambridge Univ. Press: Cambridge, UK.

\bibitem{Majid2}
Majid, S.: 2002, \emph{A Quantum Groups Primer}, Cambridge Univ.Press: Cambridge, UK.

\bibitem{MM-RG95} 
Makkai, M. and Par, R., 1989, Accessible Categories: the Foundations of Categorical Model Theory, Contemporary Mathematics 104, Providence: AMS. 

\bibitem{MM98}
Makkai, M., 1998, Towards a Categorical Foundation of Mathematics, in {\em Lecture Notes in Logic}, \textbf{11}, Berlin: Springer, 153--190. 

\bibitem{MM99}
Makkai, M., 1999, On Structuralism in Mathematics, 
in {\em Language, Logic and Concepts}, Cambridge: MIT Press, 43--66. 

\bibitem{MM-RG77}
Makkai, M. and Reyes, G., 1977, {\em First-Order Categorical Logic}, Springer Lecture Notes in Mathematics 611, New York: Springer. 

\bibitem{MM-RG95}
Makkei, M. and Reyes, G., 1995, Completeness Results for Intuitionistic and Modal Logic in a Categorical Setting, 
{\em Annals of Pure and Applied Logic}, 72, 1, 25--101. 

\bibitem{Manders}
Manders, K.L.: 1982, On the space-time ontology of physical theories, \emph{Philosophy of Science} \textbf{49} no. 4: 575--590.

\bibitem{MJP95}
Marquis, J.-P., 1995, Category Theory and the Foundations of Mathematics: Philosophical Excavations., Synthese, 103, 421--447. 

\bibitem{MJP2k6} 
Marquis, J.-P., 2006, Categories, Sets and the Nature of Mathematical Entities, in {\em The Age of Alternative Logics. Assessing philosophy of logic and mathematics today.}, J. van Benthem, G. Heinzmann, Ph. Nabonnand, M. Rebuschi, H.Visser, eds., Springer, : 181-192. 

\bibitem{MJP1999}
May, J.P. 1999, \emph{A Concise Course in Algebraic Topology}, The University of Chicago Press: Chicago.

\bibitem{MCWP43}
McCulloch, W. and W. Pitt.: 1943, A logical Calculus of Ideas Immanent in Nervous Activity., \emph{Bull. Math. Biophysics,} \textbf{5}: 115-133.

\bibitem{MLC86}
Mc Larty, C., 1986, Left Exact Logic, {\em Journal of Pure and Applied Algebra}, {41}, no. 1" 63-66.

\bibitem{MLC91}
Mc Larty, C., 1991, {\em Axiomatizing a Category of Categories}, {\em Journal of Symbolic Logic}, 56, no. 4, 1243-1260. 

\bibitem{MLC92} 
Mc Larty, C., 1992, {\em Elementary Categories, Elementary Toposes}, Oxford: Oxford University Press.

\bibitem{MLC94}
Mc Larty, C., 1994, Category Theory in Real Time, {\em Philosophia Mathematica}, \textbf{2}, no. 1: 36-44.

\bibitem{2k4}
Mc Larty, C., 2004, Exploring Categorical Structuralism, {\em Philosophia Mathematica}, 12, 37-53.

\bibitem{Mitchell1}
Mitchell, B.: 1965, \emph{Theory of Categories}, Academic Press:London.

\bibitem{Mitchell2}
Mitchell, B.: 1964, The full imbedding theorem. \emph{Amer. J. Math}. \textbf{86}: 619-637.

\bibitem{MI-P2k2}
Moerdijk, I. and Palmgren, E., 2002, Type Theories, Toposes and Constructive Set Theory: Predicative Aspects of AST., Annals of Pure and Applied Logic, 114, no. 1--3, 155--201. 

\bibitem{MO98}
Moerdijk, I., 1998, Sets, Topoi and Intuitionism., {\em Philosophia Mathematica}, 6, no. 2, 169--177.

\bibitem{Moer1}
Moerdijk, I : Classifying toposes and foliations, {\it Ann. Inst. Fourier, Grenoble} \textbf{41}, 1 (1991) 189--209.

\bibitem{Moer2}
Moerdijk, I : Introduction to the language of stacks and gerbes, arXiv:math.AT/0212266 (2002).

\bibitem{MK70}
Morita, K. , 1970. Localization in categories of modules. I., {\em Math. Z.}, 
\textbf{114}: 121-144.

\bibitem{Mostow}
M. A. Mostow : The differentiable space structure of Milnor classifying spaces, 
simplicial complexes, and geometric realizations, \emph{J. Diff. Geom.} \textbf{14} (1979) 255-293.


\bibitem{NSA}
National Academy of Sciences. 2000. H. Bass, Henri Cartan, Peter Freyd, Alex Heller and Saunders Mac Lane.  \PMlinkexternal{Samuel Eilenberg: 1913-1998. A Biographical Memoir.}{http://www.nap.edu/html/biomems/seilenberg.pdf} 

\bibitem{OB69}
Oberst, U.: 1969, Duality theory for Grothendieck categories., \emph{Bull. Amer. Math. Soc.} \textbf{75}: 1401-1408.

\bibitem{ORT70}
Oort, F.: 1970. On the definition of an abelian category. \emph{Proc. Roy. Neth. Acad. Sci}. \textbf{70}: 13-02.

\bibitem{OO31}
Ore, O., 1931, Linear equations on non-commutative fields., {\em Ann. Math.},\textbf{32}: 463-477.

\bibitem{PB70}
Pareigis, B., 1970, {\em Categories and Functors}, New York: Academic Press. 

\bibitem{PMC 2k4}
Pedicchio, M. C. \& Tholen, W., 2004, {\em Categorical Foundations}, Cambridge: Cambridge University Press.  

\bibitem{PAM90}
Pitts, A. M., 1989, Conceptual Completeness for First-order Intuitionistic Logic: an Application 
of Categorical Logic, {\em Annals of Pure and Applied Logic}, 41, no. 1, 33--81. 

\bibitem{PAM2k}
Pitts, A. M., 2000, {\em Categorical Logic. Handbook of Logic in Computer Science}, Vol.5, Oxford: Oxford Unversity Press, 39--128.

\bibitem{Pradines1966}
Pradines, J.: 1966, Th\'eorie de Lie pour les groupo\"ides diff\'erentiable, relation entre propri\'etes locales et globales, \emph{C. R. Acad Sci. Paris S\'er. A} \textbf{268}: 907-910.

\bibitem{RGZH91}
Reyes, G. and Zolfaghari, H., 1991, Topos-theoretic Approaches to Modality, 
{\em Category Theory (Como 1990)}, Lecture Notes in Mathematics, 1488, Berlin: Springer, 359--378. 

\bibitem{RGZH96}
Reyes, G. and Zolfaghari, H., 1996, Bi-Heyting Algebras, Toposes and Modalities, 
{\em Journal of Philosophical Logic}, 25, no. 1, 25--43.

\bibitem{Seelik}
Paul Selick. 1997. Relations between Homology and Homotopy Groups., In {\em Introduction to homotopy theory}. 
Fields Institute Monograph.

\bibitem{SEP99}
SEP. 1999. {\em The Stanford Encyclopedia of Philosophy.} 
\PMlinkexternal{(Winter 1999 Edition)}{http://plato.stanford.edu/archives/win1999/entries/}. 

\bibitem{SEH66}
Spanier, E. H.: 1966, \emph{Algebraic Topology}, McGraw Hill: New York.

\bibitem{Szabo}
Szabo, R. J.: 2003, Quantum field theory on non-commutative spaces,
\emph{Phys. Rep.} \textbf{378}: 207--209.

\bibitem{Varilly97}
V\'arilly, J. C.: 1997, {\em An introduction to noncommutative geometry.},(preprint)
\emph{arXiv:physics/9709045}.

\end{thebibliography} 
%%%%%
%%%%%
\end{document}
