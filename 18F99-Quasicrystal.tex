\documentclass[12pt]{article}
\usepackage{pmmeta}
\pmcanonicalname{Quasicrystal}
\pmcreated{2013-03-22 19:21:41}
\pmmodified{2013-03-22 19:21:41}
\pmowner{bci1}{20947}
\pmmodifier{bci1}{20947}
\pmtitle{quasicrystal}
\pmrecord{34}{42314}
\pmprivacy{1}
\pmauthor{bci1}{20947}
\pmtype{Topic}
\pmcomment{trigger rebuild}
\pmclassification{msc}{18F99}
\pmclassification{msc}{58Z05}
\pmsynonym{paracrystal}{Quasicrystal}
\pmsynonym{liquid crystal}{Quasicrystal}
%\pmkeywords{Quasicrystals}
%\pmkeywords{Noncommutative Structures and Symmetry Groupoids}
\pmdefines{tiling}

\endmetadata

% this is the default PlanetMath preamble. as your knowledge
% of TeX increases, you will probably want to edit this, but
\usepackage{amsmath, amssymb, amsfonts, amsthm, amscd, latexsym}
%%\usepackage{xypic}
\usepackage[mathscr]{eucal}
% define commands here
\theoremstyle{plain}
\newtheorem{lemma}{Lemma}[section]
\newtheorem{proposition}{Proposition}[section]
\newtheorem{theorem}{Theorem}[section]
\newtheorem{corollary}{Corollary}[section]
\theoremstyle{definition}
\newtheorem{definition}{Definition}[section]
\newtheorem{example}{Example}[section]
%\theoremstyle{remark}
\newtheorem{remark}{Remark}[section]
\newtheorem*{notation}{Notation}
\newtheorem*{claim}{Claim}
\renewcommand{\thefootnote}{\ensuremath{\fnsymbol{footnote%%@
}}}
\numberwithin{equation}{section}
\newcommand{\Ad}{{\rm Ad}}
\newcommand{\Aut}{{\rm Aut}}
\newcommand{\Cl}{{\rm Cl}}
\newcommand{\Co}{{\rm Co}}
\newcommand{\DES}{{\rm DES}}
\newcommand{\Diff}{{\rm Diff}}
\newcommand{\Dom}{{\rm Dom}}
\newcommand{\Hol}{{\rm Hol}}
\newcommand{\Mon}{{\rm Mon}}
\newcommand{\Hom}{{\rm Hom}}
\newcommand{\Ker}{{\rm Ker}}
\newcommand{\Ind}{{\rm Ind}}
\newcommand{\IM}{{\rm Im}}
\newcommand{\Is}{{\rm Is}}
\newcommand{\ID}{{\rm id}}
\newcommand{\GL}{{\rm GL}}
\newcommand{\Iso}{{\rm Iso}}
\newcommand{\Sem}{{\rm Sem}}
\newcommand{\St}{{\rm St}}
\newcommand{\Sym}{{\rm Sym}}
\newcommand{\SU}{{\rm SU}}
\newcommand{\Tor}{{\rm Tor}}
\newcommand{\U}{{\rm U}}
\newcommand{\A}{\mathcal A}
\newcommand{\Ce}{\mathcal C}
\newcommand{\D}{\mathcal D}
\newcommand{\E}{\mathcal E}
\newcommand{\F}{\mathcal F}
\newcommand{\G}{\mathcal G}
\newcommand{\Q}{\mathcal Q}
\newcommand{\R}{\mathcal R}
\newcommand{\cS}{\mathcal S}
\newcommand{\cU}{\mathcal U}
\newcommand{\W}{\mathcal W}
\newcommand{\bA}{\mathbb{A}}
\newcommand{\bB}{\mathbb{B}}
\newcommand{\bC}{\mathbb{C}}
\newcommand{\bD}{\mathbb{D}}
\newcommand{\bE}{\mathbb{E}}
\newcommand{\bF}{\mathbb{F}}
\newcommand{\bG}{\mathbb{G}}
\newcommand{\bK}{\mathbb{K}}
\newcommand{\bM}{\mathbb{M}}
\newcommand{\bN}{\mathbb{N}}
\newcommand{\bO}{\mathbb{O}}
\newcommand{\bP}{\mathbb{P}}
\newcommand{\bR}{\mathbb{R}}
\newcommand{\bV}{\mathbb{V}}
\newcommand{\bZ}{\mathbb{Z}}
\newcommand{\bfE}{\mathbf{E}}
\newcommand{\bfX}{\mathbf{X}}
\newcommand{\bfY}{\mathbf{Y}}
\newcommand{\bfZ}{\mathbf{Z}}
\renewcommand{\O}{\Omega}
\renewcommand{\o}{\omega}
\newcommand{\vp}{\varphi}
\newcommand{\vep}{\varepsilon}
\newcommand{\diag}{{\rm diag}}
\newcommand{\grp}{{\mathbb G}}
\newcommand{\dgrp}{{\mathbb D}}
\newcommand{\desp}{{\mathbb D^{\rm{es}}}}
\newcommand{\Geod}{{\rm Geod}}
\newcommand{\geod}{{\rm geod}}
\newcommand{\hgr}{{\mathbb H}}
\newcommand{\mgr}{{\mathbb M}}
\newcommand{\ob}{{\rm Ob}}
\newcommand{\obg}{{\rm Ob(\mathbb G)}}
\newcommand{\obgp}{{\rm Ob(\mathbb G')}}
\newcommand{\obh}{{\rm Ob(\mathbb H)}}
\newcommand{\Osmooth}{{\Omega^{\infty}(X,*)}}
\newcommand{\ghomotop}{{\rho_2^{\square}}}
\newcommand{\gcalp}{{\mathbb G(\mathcal P)}}
\newcommand{\rf}{{R_{\mathcal F}}}
\newcommand{\glob}{{\rm glob}}
\newcommand{\loc}{{\rm loc}}
\newcommand{\TOP}{{\rm TOP}}
\newcommand{\wti}{\widetilde}
\newcommand{\what}{\widehat}
\renewcommand{\a}{\alpha}
\newcommand{\be}{\beta}
\newcommand{\ga}{\gamma}
\newcommand{\Ga}{\Gamma}
\newcommand{\de}{\delta}
\newcommand{\del}{\partial}
\newcommand{\ka}{\kappa}
\newcommand{\si}{\sigma}
\newcommand{\ta}{\tau}
\newcommand{\lra}{{\longrightarrow}}
\newcommand{\ra}{{\rightarrow}}
\newcommand{\rat}{{\rightarrowtail}}
\newcommand{\oset}[1]{\overset {#1}{\ra}}
\newcommand{\osetl}[1]{\overset {#1}{\lra}}
\newcommand{\hr}{{\hookrightarrow}}

\begin{document}
{\bf Note : this is a downwards compatibility test of the current PM NS 1.5 version and the PPhysics NS 1.0/1.1 current version -running under Ubuntu 10.04 OS}; minor incompatibilities are in the Bibliography section where foreign accents are missing in NS 1.5. 


{\em Notice that only the Page Image mode comes close to the original Bibliography listed and linked before the Bibliography Section; moreover the original Bibliography created with NS 1.0/1.1 on PlanetPhysics.org does include all the foreign accents required here by the references section at  http://planetphysics.us/encyclopedia/QuantumSymmetryBibliography.html.}

\section{Quasicrystals, Noncommutative Structures \\ and Symmetry Groupoids}


\subsection{Introduction}
In this highly-condensed review we are discussing several fundamental aspects of quantum symmetry, extended quantum symmetries, and  also their related quantum groupoid/categorical representations.  This is intended as an up-to-date review centered on quantum symmetry, invariance and representations.  We aim at an accessible presentation, as well as a wide field of view of quantum theories,  so that the hitherto `hidden' patterns of quantum relations, concepts and the  underlying, extended quantum symmetries become visible to the `mathematical-ready eyes' of the theoretical physicist. To this end, we are therefore focusing here on several promising developments related to extended quantum symmetries, such as `paragroups', `quasicrystals' and quantum groupoid/quantum algebroid representations whose roots  can be traced back to recent developments in solid state physics, crystallography, metal physics,  nanotechnology  laboratories, quantum chromodynamics theories, nuclear physics, nuclear fusion reactor engineering /designs, and so on, to other modern physics areas, including quantum gravity and supergravity theories. Then, we propose several unifying ideas such as:  general representations of abstract structures and relations relevant to the treatment of extended quantum symmetry, ultra-high energy physics, and topological order theories. 

\subsection{Symmetry Groups and Quantum Groupoids}

Symmetry groups have been the mainstay of Euclidean structures as have been envisaged in classical dynamical systems, relativity and particle physics, etc., where single, direct transformations are usually sufficient. In contrast, as is found to be the case in sub-atomic, extreme microscopic quantum mechanical and biomolecular systems, transformations are essentially simultaneous. A transformed state-configuration can be indistinguishable from its original and at the microscopic level a precise symmetry might neglect some anomalous behaviour of the underlying physical process particularly in the excitation spectra. Thus, the role played here by classical symmetry (Lie) groups is useful but limiting.  Extending the latter symmetry concepts, one looks towards the more abstract but necessary structures at this level, such as `paragroups' or `symmetry groupoids', and also encoding the ubiquitous concept of a `quantum group' into the architecture of a $C^*$-Hopf algebra. With this novel approach, a classical Lie algebra now evolves to a `Lie (bi)algebroid' as one of the means for capturing higher order and super- symmetries.  The convolution algebra of the {\em transition groupoid} of a bounded quantum system--which is related to its spectra--is however just {\em matrix algebra}, when viewed in terms of its representations.  The initial groupoid viewpoint-- which was initially embraced by Werner Heisenberg for a formal, quantum-theoretical treatment of spectroscopy--was thus replaced for expediency by a computable matrix formulation of Quantum Mechanics resulting from such group representations. This was happening at a time when the {\em groupoid} and category concepts have not yet entered the realm of either physics or the mainstream of pure mathematics. On the other hand, in Mathematics the groupoid theory and the groupoid mode of thought became established in algebraic topology \cite{RB1} towards the end of the twentieth century, and now  it is a fruitful, already rich in mathematical results; it is also a well-founded field of study in its own right, but also with potential, numerous applications in mathematical physics, and especially in developing {\em non-Abelian} physical theories. 

In the beginning, the algebraic foundation of Quantum Mechanics occurred along two different lines of approach- that of John von Neumann, published in 1932, and independently, Paul A.M. Dirac's approach in 1930; these two developments followed the first analytical formulation of Quantum Mechanics of the (electron in) Hydrogen Atom by Erwin Schr$\"{o}$dinger in 1921. The equivalence of Heisenberg's `matrix mechanics' and  Schr$\"{o}$dinger's analytical formulation is now universally accepted.  On the other hand, Von Neumann's formulation in terms of operators on Hilbert spaces and $W^*$-algebras has proven its fundamental role and  real value in providing a more general, algebraic framework for both quantum measurement theories and  the mathematical treatment of a very wide range of quantum systems. To this day, however, the underlying problem of the `right' quantum logic for von Neumann's algebraic formulation of quantum theories remains to be solved, but it would seem that a modified \L{}ukasiewicz, many-valued, {\em $LM$-- algebraic logic} is a very strong candidate \cite{BGB1}. Interestingly, such an $LM$-quantum algebraic logic is {\em noncommutative} by definition, and it would also have to be a {\em non-distributive} lattice ({\em loc.cit.,} and also the relevant references cited therein).  A topos theory based on a concept called a  'quantum topos' was also proposed for quantum gravity by rejecting the idea of a spacetime continuum \cite{Butterfield-Isham2k4}. The latter concept is based however on a Heyting (intuitionistic) logic algebra which is known however to be a {\em commutative lattice}, instead of the (noncommutative and non-distributive, multi-valued)  quantum logic expected of any quantum theory \cite{BGB1}.

The quantum operator algebra for various quantum systems then required also the introduction of: $C^*$-algebras, Hopf algebras/quantum groups, Clifford algebras (of observables), graded algebras or superalgebras, Weak Hopf algebras, quantum doubles, $6j$-symbols, Lie 2-algebras, Lie-2 groups, Lie 3-superalgebras \cite{Baez-Huerta2010}, and so on.  The current rapid expansion of the collection of such various types of `quantum algebras' suggests an eventual need for a {\em Categorical Ontology of quantum systems} which is steadily moving towards the framework of \emph{higher-dimensional algebra (HDA)} and the related, \emph{higher categorical, non-Abelian structures} underlying quantum field and higher gauge theories.  A survey of the basic mathematical approach of HDA,  also with several  examples of physical applications, can be found in an extensive, recent monograph \cite{BGB1}, complemented by a recent, introductory textbook on Quantum Algebraic Topology, Quantum Algebra and Symmetry \cite{Baianu2010}. The case of the $C^*$-algebras is particularly important as the von Neumann $W^*$- algebras can be considered as a special type of $C^*$-algebras. Moreover, Gelfand  and  Neumark \cite{Gelfand-Naimark43}  showed that any $C^*$- algebra can be given a concrete representation in a Hilbert space, that need not be separable; thus, there is an isomorphic mapping of the elements of a $C^*$- algebra into the set of bounded operators of a Hilbert space. Subsequently, Segal \cite{Segal47b} completed the work begun by Gelfand and Neumark by providing a procedure for the construction of concrete Hilbert space representations of an abstract $C^*$-algebra, called the {\em `GNS construction'} \cite{Segal47a}, from the initials of Gelfand, Neumark and Segal. Segal then proceeded to an algebraic formulation of quantum mechanics based on postulates that define a Segal $S$-algebra structure which is more general than a $C^*$-algebra \cite{Segal47b}.

We also introduce the reader to a series of novel concepts that are important for numerous applications of modern physics, such as: the generalised convolution algebras of functions, convolution product of groupoids/quantum groupoids, convolution of quantum algebroids and related crossed complexes, $C^*$--convolution quantum algebroids, graded quantum algebroids, the embeddings of quantum groups into \emph{groupoid $C^*$-algebras}, quantum (Lie) double groupoids with connection, the $R$-matrix of \emph{the Yang-Baxter equations}, the $6j$-symbol symmetry with their representations in relation to Clebsch-Gordan theories, Extended Topological Quantum Field Theories (ETQFTs), as well as their relationship with integrable systems, solutions of the generalised Yang-Baxter equation and $6j$-symbols. Other related structures, such as Clifford algebras, Grassman algebras, $R$--algebroids, quantum double algebroids were discussed in detail in a recent monograph \cite{BGB1}. Moreover, the new establishment of the dual concepts of quantum groupoid and quantum algebroid representations in a `Hopf'-algebra/bialgebroid setting are a natural consequence of their long-accepted use in the simpler guise of  finite, `quantum groups'. Many of the extended quantum symmetry concepts considered here need to be viewed in the light of fundamental theorems that already have a wide range of physical applications, such as the theorems of: Noether, Goldstone, Wigner, Stone-von Neumann and others,which we briefly recall. Further generalisations and important potential applications to theoretical physics of theorems such as the generalised Siefert-van Kampen theorem are then discussed.

Ultimately, we would like to see an {\em unified categorical framework} of the quantum symmetry fundamental concepts and the results based on them that are here encapsulated only as the sub-structures needed for a broader view of quantum symmetry theories than that traditionally emcountered. In this regard, our presentation also includes several novel approaches 

\subsection{Quasicrystals}

Penrose \cite{Penrose1} considered the problem of coverings of the whole plane by shifts of a finite number of non-overlapping polygons without gaps. These tilings, though being non-periodic, are \emph{quasi-periodic} in the sense that any portion of the tiling sequence, displayed as a non-periodic lattice, appears infinitely often and with extra symmetry (there are more general examples in 3-dimensions). In such tiling patterns there is a requirement for matching rules if the structure is to be interpreted as scheme of an energy ground state \cite{Radin}.
Remarkably, further examples arise from icosahedral symmetries as first observed in solid state physics by \cite{Schechtman} who described the creation of alloys ${Al}_6Mn$ with unusual icosahedral, 10-fold symmetries forbidden by normal crystallographic structures. These very unsual symmetries were discovered in the electron diffraction patterns of the latter solids which consisted of sharp Bragg peaks (true $\delta$-functions) that are typical of all crystalline structures that are highly ordered, and are thus in marked contrast to those of metallic glasses and other noncrystalline solids which exhibit only broad scattering bands instead of discrete Bragg diffraction peaks. Such unusual lattices were coined \emph{quasicrystals} because they contain relatively small amounts of structural disorder in such lattices of 10-fold symmetry, formed by closely packed icosahedra.

Further investigation of 10 - and higher- fold symmetries has suggested the use of noncommutative geometry to characterize the underlying electron distributions in such quasicrystals, as outlined for example in \cite{Bellissard1,Connes94} in the setting of $C^*$-algebras and K-theory on a variety of non-Hausdorff spaces, and also attempting to relate this theory to the quantum Hall effect. More specifically, as explained in \cite{Ypma1,Ypma2}.


\PMlinkexternal{Quantum Symmetry Bibliography}{http://planetphysics.us/encyclopedia/QuantumSymmetryBibliography.html}

\begin{thebibliography}{99}

\bibitem{Abragam-Bleaney70}
Abragam, A.; Bleaney, B. \emph{Electron paramagnetic resonance of transition ions.}; Clarendon Press: Oxford, 1970.

\bibitem{Aguiar-Andrusk2k4}
Aguiar, M.; Andruskiewitsch, N. Representations of matched pairs of groupoids and applications to weak Hopf algebras. {\it Contemp. Math.} {\bf 2005}, {\em 376}, 127--173.
%%$\href{http://arxiv.org/abs/math.QA/0402118}{math.QA/0402118}$.

\bibitem{Aguiar2k9a}
Aguiar, M.C.O.; Dobrosavljevic, V.; Abrahams, E.; Kotliar G. Critical behavior at Mott--Anderson transition: a TMT-DMFT perspective. \emph{Phys. Rev. Lett.} {\bf 2009}, {\em 102}, 156402, 4~pages.
%%$\href{http://arxiv.org/abs/0811.4612}{arXiv:0811.4612}$.

\bibitem{Aguiar2k9b}
Aguiar, M.C.O.; Dobrosavljevic, V.; Abrahams, E.; Kotliar G. Scaling behavior of an Anderson impurity close to the Mott-Anderson transition. \textit{Phys. Rev.~B} {\bf 2006}, {\em 73}, 115117, 7~pages.
%%\href{http://arxiv.org/abs/cond-mat/0509706}{cond-mat/0509706}.

\bibitem{Alfsen-Schultz2k3}
Alfsen, E.M.; Schultz, F.W. {\em Geometry of state spaces of operator algebras.}; 
Birkh$\"{a}$user: Boston-- Basel-- Berlin, 2003.

\bibitem{Altintas-Arika2k8}
Altintash, A.A.; Arika, M. The inhomogeneous invariance quantum supergroup of supersymmetry algebra. \textit{Phys. Lett. A} {\bf 2008}, {\em 372}, 5955--5958.

\bibitem{Anderson58}
Anderson, P.W. Absence of diffusion in certain random lattices. \textit{Phys. Rev.} {\bf 1958}, {\em 109}, 1492--1505.

\bibitem{Anderson77}
Anderson, P.W. Topology of Glasses and Mictomagnets. {\em Lecture presented at The Cavendish Laboratory}: Cambridge, UK, 1977.

\bibitem{Baez-Huerta2010}
Baez, J. and Huerta, J. {\em An Invitation to Higher Gauge Theory}. {\bf 2010}, {\em Preprint, March 23, 2010}: Riverside, CA;pp. 60. http://www.math.ucr.edu/home/baez/invitation.pdf
%%\href{http://arxiv.org/abs/hep-th/1003.4485}{arXiv:1003.4485v1}

\bibitem{Baez-Schr2k8}
Baez, J. and Schreiber, U. {\em Higher Gauge Theory II: 2-Connections} (JHEP Preprint), 2008; pp.75. 
http://math.ucr.edu/home/baez/2conn.pdf.

\bibitem{Baianu2010}
Baianu, I.C.; Editor. {\em Quantum Algebra and Symmetry: Quantum Algebraic Topology, Quantum Field Theories and Higher Dimensional Algebra}; PediaPress GmbH: Mainz, Second Edition, Vol. \\{\em 1}: {\em Quantum Algebra, Group Representations and Symmetry}; Vol.{\em 2}: {\em Quantum Algebraic Topology: \\ QFTs, SUSY, HDA};Vol.{\em 3}: {\em Quantum Logics and Quantum Biographies}, December 20, 2010;pp. 1,068.

\bibitem{Baianu71}
Baianu, I.C. Categories, functors and automata theory: a novel approach to quantum automata through algebraic-topological quantum computation. In {\em Proceedings of the 4th Intl. Congress of Logic, Methodology and Philosophy of Science, Bucharest, August~-- September, 1971}; University of Bucharest: Bucharest, 1971;pp. 256--257.

\bibitem{Baianu74}
Baianu, I.C. \emph{Structural studies of erythrocyte and bacterial cytoplasmic membranes by X-ray diffraction and electron microscopy.} PhD Thesis; Queen Elizabeth College: University of London, 1974.

\bibitem{Baianu78}
Baianu, I.C. X-ray scattering by partially disordered membrane systems.
\emph{Acta Cryst. A}, 1978, {\bf 34}: 751--753.

\bibitem{Baianu-etal78a}
Baianu, I.C.; Boden, N.; Levine, Y.K.;Lightowlers, D. Dipolar coupling between groups of three spin-1/2 undergoing hindered reorientation in solids. {\it Solid State Comm.} 1978, {\bf 27}, 474--478.

\bibitem{Baianu80}
Baianu, I.C. Structural order and partial disorder in biological systems. {\em Bull. Math. Biology} {\bf 1980}, {\em 42}, 464--468.

\bibitem{Baianu-etal78b}
Baianu, I.C.; Boden N.; Levine Y.K.; Ross S.M. Quasi-quadrupolar NMR spin-Echoes in solids containing dipolar coupled methyl groups. {\em J. Phys. C: Solid State Phys.}
{\bf 1978},{\em 11}, L37--L41.

\bibitem{Baianu-etal81}
Baianu, I.C.; Boden, N.;Lightowlers, D. NMR spin-echo responses of dipolar-coupled spin-1/2 triads in solids. {\em J. Magnetic Resonance} {\bf 1981},{\em 43}, 101--111.

\bibitem{Baianu-etal78c}
Baianu, I.C.; Boden, N.; Mortimer, M.; Lightowlers, D. A new approach to the structure of concentrated aqueous electrolyte solutions by pulsed N.M.R. {\em Chem. Phys. Lett.},{\bf 1978}, {\em 54}, 169--175.

\bibitem{Baianu-etal2k7}
Baianu, I.C.; Glazebrook J.F.; Brown, R. A non-Abelian, Categorical Ontology of Spacetimes and Quantum Gravity. {\em Axiomathes} {\bf 2007}, {\em 17}, 353--408.

\bibitem{BGB1}
Baianu, I. C.; Glazebrook, J. F.; Brown, R. Algebraic Topology Foundations of Supersymmetry and Symmetry Breaking in Quantum Field Theory and Quantum Gravity: A Review.
{\em Sigma} {\bf 2009}, {\em 5}, 70 pages.

\bibitem{Baianu-etal79a}
Baianu, I.C.; Rubinson, K.A.; Patterson, J. The observation of structural relaxation in a FeNiPB glass by $X$-ray scattering and ferromagnetic resonance. {\em Phys. Status
Solidi A} {\bf 1979}, {\em 53}, K133--K135.

\bibitem{Baianu-etal79b}
Baianu, I.C.; Rubinson, K.A.; Patterson, J. Ferromagnetic resonance and spin--wave excitations in metallic glasses. \emph{J. Phys. Chem. Solids} {\bf 1979}, {\em 40}, 940--951.

\bibitem{BSS2k2}
Bais, F. A.; Schroers, B. J.; and Slingerland, J. K. Broken quantum symmetry and confinement phases in planar physics. \emph{Phys. Rev. Lett.} \textbf{2002}, {\em 89}, No. 18 (1--4), 181--201.

%%\href{http://arxiv.org/abs/hep-th/0205117}{hep-th/0205117}.

\bibitem{Ball2k5}
Ball, R.C. Fermions without fermion fields. {\em Phys. Rev. Lett.} \textbf{95} (2005), 176407, 4~pages.

%% \href{http://arxiv.org/abs/cond-mat/0409485}{cond-mat/0409485}.

\bibitem{Banica96}
Banica, T. Theorie des representations du groupe quantique compact libre $O (n)$. {\em C. R. Acad. Sci. Paris.}, {\bf 1996}, {\em 322}, Serie I, 241--244.

\bibitem{Banica2k}
Banica, T. Compact Kac algebras and commuting squares. {\em J. Funct. Anal.} {\bf 2000} ,{\em 176}, no. 1, 80--99.

\bibitem{Barrett2k3}
Barrett, J.W. Geometrical measurements in three-dimensional quantum gravity. In {\em Proceedings of the Tenth Oporto Meeting on Geometry, Topology and Physics (2001)}, \textit{Internat. J. Modern Phys. A} {\bf 2003}, {\em 18}, October, suppl., 97--113.

%%\href{http://arxiv.org/abs/gr-qc/0203018}{gr-qc/0203018}.

\bibitem{Barrett-Mackaay2k6}
Barrett, J.W.; Mackaay, M. Categorical representation of categorical groups. {\em Theory Appl. Categ.} {\bf 2006},{\em 16}, 529--557.
 %%\href{http://arxiv.org/abs/math.CT/0407463}{math.CT/0407463}.

\bibitem{HJB-DC91}
Baues, H.J. ; Conduch\'{e}, D. On the tensor algebra of a nonabelian group and applications. {\em $K$-Theory} {\bf 1991/92}, {\em 5} ), 531--554.

\bibitem{Bellissard1}
Bellissard, J. K-theory of C*-algebras in solid state physics. {\em Statistical Mechanics and Field Theory: Mathematical Aspects}
;Dorlas, T. C. et al., Editors; Springer Verlag. {\em Lect. Notes in Physics}, {\bf 1986}, {\em 257}, 99--156.

\bibitem{Bichon2k8}
Bichon, J. Algebraic quantum permutation groups. {\em Asian-Eur. J. Math.} {\bf 2008}, {\em 1}, no. 1, 1--13. arXiv:0710.1521

\bibitem{Bichonetal2k6}
Bichon, J.; De Rijdt, A.; Vaes, S. Ergodic coactions with large multiplicity and monoidal equivalence of quantum groups. {\em Comm. Math. Phys.} {\bf 2006}, {\em 262}, 703--728.

\bibitem{Blaom2k7}
Blaom, A.D. Lie algebroids and Cartan's method of equivalence. 
%%\href{http://arxiv.org/abs/math.DG/0509071},{\em math.DG/0509071}.

\bibitem{Blok-Wen92}
Blok, B.; Wen X.-G. Many-body systems with non-Abelian statistics. {\em Nuclear Phys. B} {\bf 1992}, {\em 374}, 615--619.

\bibitem{HH88}
Hindeleh, A.M. and Hosemann, R. Paracrystals representing the physical state of matter. Solid State Phys. 1988, 21, 4155--4170.

\bibitem{Hosemann-Bagchi62}
Hosemann, R.; and Bagchi, R.N. {\em Direct analysis of diffraction by matter.}; North-Holland Publs.: Amsterdam~-- New York, 1962.

\bibitem{Hosemann-etal81}
Hosemann, R.; Vogel W.; Weick, D.; Balta-Calleja, F.J. Novel aspects of the real paracrystal. \emph{Acta Cryst.~A} {\bf 1981}, {\bf 376}, 85--91.

\bibitem{ThVetal69}
Ionescu, Th.V.; Parvan, R.; and Baianu, I. Les oscillations ioniques dans les cathodes creuses dans un champ magnetique. \emph{C.R.Acad.Sci.Paris}. {\bf 1969}, {\bf 270}, 1321--1324; ({\em paper communicated by Nobel laureate Louis Ne$\'{e}$l}).

\bibitem{Butterfield-Isham2k4}
Isham, C.J.; Butterfield, J. Some possible roles for topos theory in quantum theory and quantum gravity. {\em Found. Phys.} {\bf 2000},{\em 30}, 1707--1735. 
%%\href{http://arxiv.org/abs/gr-qc/9910005}{gr-qc/9910005}.

\bibitem{Janelidze91}
Janelidze, G. Precategories and Galois theory. In {\em Category theory}; Como, 1990. {\em Springer Lecture Notes in Math.}, Vol.~\emph{1488}; Springer: Berlin, 1991;pp. 157--173.

\bibitem{Janelidze90}
Janelidze, G. Pure Galois theory in categories. \emph{J. Algebra}(1990), {\bf 132}, 270--286.

\bibitem{Jimbo85}
Jimbo, M. A $q$-difference analogue of $U_g$ and the Yang--Baxter equations. \emph{Lett. Math. Phys.} (1985), {\bf 10},63--69.

\bibitem{Jones83}
Jones, V. F. R. Index for Subfactors. {\em Invetiones Math.} {\bf 1989} {\bf 72},1--25. Reprinted in: \emph{New Developments in the Theory of Knots.}; World Scientific Publishing: Singapore,1989.

\bibitem{Andre-Street93}
Joyal, A.; Street R. Braided tensor categories. \emph{Adv. Math.} (1995), \emph{102}, 20--78.

\bibitem{Kac77}
Kac, V. Lie superalgebras. \emph{Adv. Math.} {\bf 1977}, {\bf 26}, 8--96.

\bibitem{Kamps-Porter99}
Kamps, K. H.; Porter, T. A homotopy 2--groupoid from a fibration. \emph{Homotopy, Homology and Applications.} (1999), {\bf 1}, 79--93.

\bibitem{Kapitsa78}
Kapitsa, P. L. Plasma and the controlled thermonuclear reaction. \emph{The Nobel Prize lecture in Physics 1978}, in \emph{Nobel Lectures, Physics 1971--1980}; Editor S.~Lundqvist; World Scientific Publishing Co.: Singapore, 1992.

\bibitem{Kauffman89}
Kauffman, L. Spin Networks and the Jones Polynomial. \emph{Twistor Newsletter, No.29}; Mathematics Institute: Oxford, November 8th, 1989.

\bibitem{Kauffman91}
Kauffman, L. $SL(2)_q$--Spin Networks. \emph{Twistor Newsletter, No.32 }; Mathematics Institute: Oxford, March 12th, 1989.

\bibitem{Khoroshkin-Tolstoy91}
Khoroshkin, S.M.; Tolstoy, V.N. Universal $R$-matrix for quantum supergroups. In {\em Group Theoretical Methods in Physics}; Moscow, 1990; {\em Lecture Notes in Phys.}, Vol.~382; Springer: Berlin, 1991;pp. 229--232.

\bibitem{Kirilov-R89}
Kirilov, A.N.; and Reshetikhin, N. Yu. Representations of the Algebra $U_q(sl(2))$, $q$-Orthogonal Polynomials and Invariants of Links. Reprinted in: {\em New Developments in the Theory of Knots.}; Kohno, Editor; World Scientific Publishing, 1989.

\bibitem{KlimykSch97}
Klimyk, A. U.; and Schm\"udgen, K. {\em Quantum Groups and Their Representations.}; Springer--Verlag: Berlin, 1997.

\bibitem{Korepin}
Korepin, V. E. Completely integrable models in quasicrystals. {\em Commun. Math. Phys.} {\bf 1987}, {\em 110}, 157--171.

\bibitem{Kustermans-Vaes2k}
Kustermans, J.; Vaes, S. The operator algebra approach to quantum groups. {\em Proc. Natl. Acad. Sci. USA} {\bf 2000}, {\em 97}, 547--552.

\bibitem{Lambe-Redford97}
Lambe, L.A.; Radford, D.E. Introduction to the quantum Yang--Baxter equation and quantum groups: an algebraic approach. \emph{Mathematics and its Applications}, Vol.~{\em 423}; Kluwer Academic Publishers: Dordrecht, 1997.

\bibitem{Lance95}
Lance, E.C. Hilbert $C^*$-modules. A toolkit for operator algebraists. {\em London Mathematical Society Lecture Note Series}, Vol.~{\em 210}; Cambridge University Press: Cambridge, 1995.

\bibitem{Landsman98}
Landsman, N.P. Mathematical topics between classical and quantum mechanics. {\em Springer Monographs in Mathematics}; Springer-Verlag: New York, 1998.

\bibitem{Landsman2k}
Landsman, N.P. Compact quantum groupoids, in Quantum Theory and Symmetries (Goslar, July 18--22, 1999), Editors H.-D.~Doebner et al., World Sci. Publ., River Edge, NJ, 2000, 421--431,
%%\href{http://arxiv.org/abs/math-ph/9912006}{math-ph/9912006}.

\bibitem{Landsman-Ramazan2k1}
Landsman, N.P.; Ramazan, B. Quantization of Poisson algebras associated to Lie algebroids, in {\em Proc. Conf. on Groupoids in Physics, Analysis and Geometry: Boulder CO, 1999}; Editors: J.~Kaminker et al., \emph{Contemp. Math.}{\bf 2001},{\em 282}, 159--192.
%%\href{http://arxiv.org/abs/math-ph/0001005}{math-ph/0001005}.

\bibitem{Lawrence95}
Lawrence, R.L. Algebra and Triangle Relations. In: {\em Topological and Geometric Methods in Field Theory}; Editors: J. Michelsson and O.Pekonen; World Scientific Publishing,1992;pp.429-447. {\em J.Pure Appl. Alg.} {\bf 1995}, {\em 100}, 245-251.

\bibitem{Lee-etal2k4}
Lee, P.A.; Nagaosa, N.; Wen, X.-G. Doping a Mott insulator: physics of high temperature superconductivity.{\bf 2004}.
%%\href{http://arxiv.org/abs/cond-mat/0410445}{cond-mat/0410445}.

\bibitem{Levin-Olshanetsky08}
Levin, A.; Olshanetsky, M. Hamiltonian Algebroids and deformations of complex structures on Riemann curves. {\bf 2008}, hep--th/0301078v1.

\bibitem{Levin-Wen2k3}
Levin, M.; Wen, X.-G. Fermions, strings, and gauge fields in lattice spin models. {\em Phys. Rev. B} {\bf 2003}, {\em 67}, 245316, 4~pages. %%\href{http://arxiv.org/abs/cond-mat/0302460}{cond-mat/0302460}.

\bibitem{Levin-Wen2k6a}
Levin, M.; Wen, X.-G. Detecting topological order in a ground state wave function. {\em Phys. Rev. Lett.} {\bf 2006}, {\em 96}, 110405, 4~pages. %%\href{http://arxiv.org/abs/cond-mat/0510613}{cond-mat/0510613}.

\bibitem{Levin-Wen2k5}
Levin M.; Wen, X.-G. Colloquium: photons and electrons as emergent phenomena. {\em Rev. Modern Phys.} {\bf 2005}, {\em 77}, 871--879. %%\href{http://arxiv.org/abs/cond-mat/0407140}{cond-mat/0407140}.

\bibitem{Levin-Wen2k6b}
Levin, M.; Wen, X.-G. Quantum ether: photons and electrons from a rotor model. {\em Phys. Rev. B} {\bf 2006}, {\em 73}, 035122, 10~pages. %%\href{http://arxiv.org/abs/hep-th/0507118}{hep-th/0507118}.

\bibitem{Loday82}
Loday, J.L. Spaces with finitely many homotopy groups. {\em J. Pure Appl. Algebra} {\bf 1982},{\em 24}, 179--201.

\bibitem{Lu96}
Lu, J.-H. Hopf algebroids and quantum groupoids. {\em Internat. J. Math.} {\bf 1996}, {\em 7}, 47--70.
%%\href{http://arxiv.org/abs/q-alg/9505024}{q-alg/9505024}.

\bibitem{Mack-Schomerus92}
Mack, G.; Schomerus, V. Quasi Hopf quantum symmetry in quantum theory. {\em Nuclear Phys. B} {\bf 1992}, {\em 370}, 185--230.

\bibitem{Mackaay-Picken}
Mackaay, M.; Picken, P. State-sum models, gerbes and holonomy. In {\em Proceedings of the XII Fall Workshop on Geometry and Physics}, {\em Publ. R. Soc. Mat. Esp.}, Vol.~{\em 7}: Madrid, 2004;pp. 217--221.

\bibitem{Mackaay99}
Mackaay, M. Spherical 2-categories and 4-manifold invariants. \emph{Adv. Math.} {\bf 1999},{\em 143}, 288--348.
%%\href{http://arxiv.org/abs/math.QA/9805030}{math.QA/9805030}.

\bibitem{Mackenzie2k5}
Mackenzie, K. C. H. General theory of Lie groupoids and Lie algebroids. {\em London Mathematical Society Lecture Note Series}, Vol.~{\em 213}; Cambridge University Press: Cambridge, 2005.

\bibitem{Mackey92}
Mackey, G.W. The scope and history of commutative and noncommutative harmonic analysis. {\em History of Mathematics}, Vol.{\em 5}; American Mathematical Society: Providence, RI; London Mathematical Society: London, 1992.

\bibitem{MacLane65}
Mac Lane, S. Categorical algebra. \emph{Bull. Amer. Math. Soc.} (1965), 40--106.

\bibitem{MacLane-Moerdijk92}
Mac Lane, S.; Moerdijk, I. \emph{Sheaves in geometry and logic. A first introduction to topos theory.}; Springer Verlag: New York, 1994.

\bibitem{Majid95}
Majid,S. \emph{Foundations of Quantum Group Theory.};Cambridge Univ. Press: Cambridge UK, 1995.

\bibitem{Maltsiniotis92}
Maltsiniotis, G. Groupo\"{\i}des quantiques. {\em C. R. Acad. Sci. Paris S\'er. I Math.} {\bf 1927}, {\em 314}, 249--252.

\bibitem{Manoj}
Manojlovi\'{c}, N.; Sambleten, H. Schlesinger transformations and quantum R-matrices. \emph{Commun. Math. Phys.} {\bf 2002}, {\em 230}, 517--537.

\bibitem{Miranda2k3}
Miranda, E. Introduction to bosonization. {\em Brazil. J. Phys.} {\bf 2003}, {\em 33}, 1--33. (http://www.ifi.unicamp.br/~emiranda/papers/bos2.pdf).

\bibitem{Mitchell65}
Mitchell, B. {\em The theory of categories.} {\em Pure and Applied Mathematics}, Vol.{\em 17}; Academic Press: New York and London, 1965.

\bibitem{Mitchell72}
Mitchell, B. Rings with several objects. {\em Adv. Math.} {\bf 1972}, {\em 8}, 1--161.

\bibitem{Mitchell85}
Mitchell, B. Separable algebroids. {\em Mem. Amer. Math. Soc.}, {\bf 1985}, {\em 57}, no.333.

\bibitem{Moerdijk-Svensson93}
Moerdijk, I.; Svensson, J.A. Algebraic classification of equivariant 2-types. {\em J. Pure Appl. Algebra} {\bf 1993}, {\em 89}, 187--216.

\bibitem{Mosa86}
Mosa, G.H. \emph{Higher dimensional algebroids and Crossed complexes.} PhD Thesis; University of Wales: Bangor, 1986.

\bibitem{Mott77}
Mott, N.F. \emph{Electrons in glass}, Nobel Lecture, 11 pages, December 8, 1977.

\bibitem{Mott78}
Mott, N.F. Electrons in solids with partial disorder. \emph{Lecture presented at The Cavendish Laboratory}: Cambridge, UK, 1978.

\bibitem{Mott--Davis78}
Mott, N.F.; Davis, E.A. {Electronic processes in non-crystalline materials}; Oxford University Press: Oxford, 1971; 2nd ed., 1978.

\bibitem{Mott--etal75}
Mott, N.F.; et al. The Anderson transition. \emph{Proc. Royal Soc. London Ser. A} (1975), {\bf 345}, 169--178.

\bibitem{Moultaka-etal2k5}
Moultaka, G.; Rausch, M. de Traubenberg; T\u{a}nas\u{a}, A. Cubic supersymmetry and abelian gauge invariance.
{\em Internat. J. Modern Phys. A} {\bf 2005}, {\em 20}, 5779--5806. %%\href{http://arxiv.org/abs/hep-th/0411198}{hep-th/0411198}.

\bibitem{Mrcun2k2}
Mr\v{c}un, J. On spectral representation of coalgebras and Hopf algebroids. %%\href{http://arxiv.org/abs/math.QA/0208199}{math.QA/0208199}.

%%Mr\v{c}un, J. Sheaf coalgebras and duality. {Topology Appl.} {\bf 2007}, {\em 154}, 2795--2812.
%%Mr\v{c}un J., On duality between etale groupoids and Hopf algebroids. {\em J. Pure Appl. Algebra} {\bf 2007}, {\em 210}, 267--282.

\bibitem{Mrcun07}
Mr\v{c}un, J. On the duality between \'{e}tale groupoids and Hopf algebroids. \emph{J. Pure Appl. Algebra} (2007), {\bf 210}, 267--282.

\bibitem{MRW8717}
Muhly, P.; Renault, J.; Williams, D. Equivalence and isomorphism for groupoid $C^*$-algebras. {\em J. Operator Theory}, {\bf 1987}, {\em 6}, 3--22.

\bibitem{Meusberger}
Meusberger, C. Quantum double and $\kappa$-Poincar\'{e} symmetries in $(2+1)$-gravity and Chern-Simons theory. {\em Can. J. Phys.} \textbf{2009}, {\em 87}, 245--250.

\bibitem{Nayak-etal2k7}
Nayak, C.; Simon, S.H.; Stern, A.; Freedman, M.; Das Sarma, S. Non-Abelian anyons and topological quantum computation. {\it Rev. Modern Phys.} {\bf 2008}, {\em 80},1083, 77~pages.
%%\href{http://arxiv.org/abs/0707.1889}{arXiv:0707.1889}.

\bibitem{Nikshych-Vainerman2k}
Nikshych, D.A.; Vainerman, L. A characterization of depth 2 subfactors of ${\rm II}_1$ factors. {\em J. Funct. Anal.} {\bf 2000}, {\em 171}, 278--307. %%\href{http://arxiv.org/abs/math.QA/9810028}{math.QA/9810028}.

\bibitem{Nishimura96}
Nishimura, H. Logical quantization of topos theory. {\em Internat. J. Theoret. Phys.} {\bf 1996}, {\em 35}, 2555--2596.

\bibitem{Neuchl97}
Neuchl, M. {\em Representation theory of Hopf categories}. PhD Thesis: University of M$\"$unich, 1997.

\bibitem{Noether1918}
Noether, E. Invariante Variationsprobleme. {\em Nachr. D. K\"onig. Gesellsch. D. Wiss. Zu G$\"o$ttingen, Math-phys. Klasse} {\bf 1918}, 235--257.

\bibitem{Norrie90}
Norrie, K. Actions and automorphisms of crossed modules. {\em Bull. Soc. Math. France} {\bf 1990}, {\em 118}~(2), 129--146.

\bibitem{Ocneanu88}
Ocneanu, A. Quantized groups, string algebras and Galois theory for algebras. In {\em Operator algebras and applications.}, {\em 2} \textit{London Math. Soc. Lecture Note Ser., 136}; Cambridge Univ. Press: Cambridge,1988;pp. 119-172.

\bibitem{Ocneanu2k1}
Ocneanu, A. Operator algebras, topology and subgroups of quantum symmetry~-- construction of subgroups of quantum groups. In {\em Taniguchi Conference on Mathematics Nara '98}, {\em Adv. Stud. Pure Math.},{\bf 2001}, {\em 31}, {\em Math. Soc. Japan}: Tokyo;pp. 235--263.

\bibitem{Ostrik2k8}
Ostrik, V. Module categories over representations of ${\rm SL}_q(2)$ in the non-simple case. {\em Geom. Funct. Anal.} {\bf 2008}, {\em 17}, 2005--2017. %%\href{http://arxiv.org/abs/math.QA/0509530}{math.QA/0509530}.

\bibitem{Patera}
Patera, J. Orbit functions of compact semisimple, however, as special functions. Iin {\em Proceedings of Fifth International Conference ``Symmetry in Nonlinear Mathematical Physics''} (June 23--29, 2003: Kyiev); Editors: A.G.~Nikitin; V.M.~Boyko; R.O.~Popovych; and I.A.~Yehorchenko. {\em Proceedings of the Institute of Mathematics, Kyiev} {\bf 2004}, {\em 50}, Part~1, 1152--1160.

\bibitem{Paterson99}
Paterson, A.L.T. Groupoids, inverse semigroups, and their operator algebras. {\em Progress in Mathematics}, {\em 170}; Birkh\"auser Boston, Inc.: Boston, MA, 1999.

\bibitem{Paterson2k3a}
Paterson, A.L.T. The Fourier--Stieltjes and Fourier algebras for locally compact groupoids.
In {\em Trends in Banach Spaces and Operator Theory, (Memphis, TN, 2001}, \\emph{Contemp. Math.} {\bf 2003}, {\em 321}, 223--237.
%% \href{http://arxiv.org/abs/math.OA/0310138}{math.OA/0310138}.

\bibitem{Paterson2k3b}
Paterson, A.L.T. Graph inverse semigroups, groupoids and their $C^*$-algebras. {\em J. Operator Theory} {\bf 2002}, {\em 48}, 645--662.
 %%\href{http://arxiv.org/abs/math.OA/0304355}{math.OA/0304355}.

\bibitem{Penrose1}
Penrose, R. The role of aestetics in pure and applied mathematical research. \emph{Bull. Inst. Math. Appl.} {\bf 1974}, \emph{10}, 266--271.

\bibitem{Penrose71}
Penrose, R. {\em Applications of Negative Dimensional Tensors}; In Welsh, Editor. {\em Combinatorial Mathematics and its Applications}; Academic Press: London, 1971.

\bibitem{Perelman}
Perelman, G. The entropy formula for the Ricci flow and its geometric applications.
%% \href{http://arxiv.org/abs/math.DG/0211159}{math.DG/0211159}.

\bibitem{Plymen-Robinson94}
Plymen, R.J.; Robinson, P.L. Spinors in Hilbert space. {\em Cambridge Tracts in Mathematics}, Vol.{\bf 114}; Cambridge University Press: Cambrige, 1994.

\bibitem{Podles95}
Podles, P. Symmetries of quantum spaces. Subgroups and quotient spaces of quantum $SU(2)$ and $SO(3)$ groups. 
{\em Commun. Math. Phys.} {\bf 1995}, {\em 170}, 1-20.

\bibitem{Poi96}
Poincar$\'e$, H. {\em \OEuvres. Tome {\em VI}}; Les Grands Classiques Gauthier-Villars. $\'E$ditions; Jacques Gabay:  Sceaux ,1996. {\em G$\'{e}$om$\'{e}$trie. Analysis situs (topologie)};
reprint of the 1953 edition.

\bibitem{Popescu68}
Popescu, N. {\em The theory of Abelian categories with applications to rings and modules.} {\em London Mathematical Society Monographs}, no.3; Academic Press: London-- New York, 1973.

\bibitem{Porter98}
Porter, T. Topological quantum field theories from homotopy $n$--types. (English summary)
{\em J. London Math. Soc. (2) } {\bf 1998}, {\em 58 }, no. 3, 723--732.

\bibitem{Porter-T2k8}
Porter, T.; and Turaev, V. Formal Homotopy Quantum Field Theories {I}: Formal maps and crossed {$C^*$-}algebras.
{\em Journal Homotopy and Related Structures},{\bf 2008},{\em 3}~(1), 113--159.

\bibitem{Prigogine80}
Prigogine, I. {\em From being to becoming time and complexity in the physical sciences.}; W. H. Freeman and Co.: San Francisco, 1980.

\bibitem{Pronk95}
Pronk, D. {\em Groupoid representations for sheaves on orbifolds.} PhD Thesis: University of Utrecht, 1995.

\bibitem{Radin}
Radin, C. Symmetries of Quasicrystals. {\em Journal of Statistical Physics} {\bf 1999}, {\bf 95}, Nos 5/6, 827--833.

\bibitem{Ramsay82}
Ramsay, A. Topologies on measured groupoids. {\em J. Funct. Anal.}{\bf 1982}, {\em 47}, 314--343.

\bibitem{Ramsay-Walter97}
Ramsay, A.; Walter, M.E. Fourier--Stieltjes algebras of locally compact groupoids. {\em J. Funct. Anal.},{\bf 1997},{\em 148}, 314--367. %%\href{http://arxiv.org/abs/math.OA/9602219}{math.OA/9602219}.

\bibitem{Ran-Wen2k6}
Ran, Y.; Wen, X.-G. Detecting topological order through a continuous quantum phase transition. {\em Phys. Rev Lett.}, {\bf 2006}, {\em 96}, 026802, 4~pages.
%% \href{http://arxiv.org/abs/cond-mat/0509155}{cond-mat/0509155}.

\bibitem{Raptis-Zapatrin2k}
Raptis I.; Zapatrin, R.R. Quantization of discretized spacetimes and the correspondence principle. {\em Internat. J. Theoret. Phys.} {\bf 2000}, {\em 39}, 1--13.
%%\href{http://arxiv.org/abs/gr-qc/9904079}{gr-qc/9904079}.

\bibitem{Reshetikhin-T91}
Reshetikhin, N.;Tureaev, V. Invariants of $3$-Manifolds via Link Polynomials. {\em Inventiones Math.} {\bf 1991}, {\em 103}, 547-597.

\bibitem{Rehren97}
Rehren, H.-K. Weak $C^*$-Hopf symmetry. In {\em Proceedings of Quantum Group Symposium at Group 21} ; Goslar, 1996; Heron Press: Sofia, 1997;pp. 62--69. 
%%\href{http://arxiv.org/abs/q-alg/9611007}{q-alg/9611007}.

\bibitem{Regge61}
Regge, T. General relativity without coordinates. {\em Nuovo Cimento (10)} {\bf 1961}, {\em 19}, 558--571.

\bibitem{Renault80}
Renault, J. A groupoid approach to $C^*$-algebras. {\em Lecture Notes in Mathematics}, Vol.{\em 793}; Springer: Berlin,1980.

\bibitem{Renault87}
Renault, J. Repr$\'e$sentations des produits crois$\'{e}$s d'alg$\`{e}$bres de groupo$\"{\i}$des. {\em J. Operator Theory} {\bf 1987}, {\em 18}, 67--97.

\bibitem{Renault97}
Renault, J. The Fourier algebra of a measured groupoid and its multiplier. {\em J. Funct. Anal.} {\bf 1997}, {\em 145}, 455--490.

\bibitem{Ribeiro-Wen2k5}
Ribeiro, T.C.; Wen, X.-G. New mean field theory of the $tt't''J$ model applied to the high-$T_c$ superconductors. {\em Phys. Rev. Lett.} {\bf 2005}, {\em 95}, 057001, 4~pages.
%%\href{http://arxiv.org/abs/cond-mat/0410750}{cond-mat/0410750}.

\bibitem{Rieffel2k2}
Rieffel, M.A. Group $C^*$-algebras as compact quantum metric spaces. \emph{Doc. Math.} {\bf 2002}, {\em 7}, 605--651.
%%\href{http://arxiv.org/abs/math.OA/0205195}{math.OA/0205195}.

\bibitem{Rieffel74}
Rieffel, M.A. Induced representations of $C^*$-algebras. {\em Adv. Math.}{\bf 1974}, {\em 13}, 176--257.

\bibitem{Roberts2k4}
Roberts, J. E. More lectures on algebraic quantum field theory. In {\em Noncommutative Geometry};pp.263--342; Editors A. Connes et al., \textit{Springer lect. Notes in Math.} \emph{1831}; Springer: Berlin, 2004.

\bibitem{Roberts95}
Roberts, J.E. Skein theory and Turaev--Viro invariants. {\em Topology} {\bf 1995}, {\em 34},771--787.

\bibitem{Roberts97}
Roberts, J.E. Refined state-sum invariants of 3-and 4-manifolds. In {\em Geometric Topology}: Athens, GA, 1993, {\em AMS/IP Stud. Adv. Math.}, {\em 2.1}, {\em Amer. Math. Soc.}; Providence, RI, 1997;pp. 217--234.

\bibitem{Rovelli98}
Rovelli, C. Loop quantum gravity. {\em Living Rev. Relativ.} {\bf 1998}, {\em 1}, 68~pages.
%%\href{http://arxiv.org/abs/gr-qc/9710008}{gr-qc/9710008}.

\bibitem{Sato-Wakui2k1}
Sato, N.; Wakui, M. $(2 + 1)$-dimensional topological quantum field theory with a Verlinde basis and Turaev--Viro--Ocneanu invariants of 3-manifolds. In {\em Invariants of Knots and 3-Manifolds}; Kyoto, 2001. {\em Geom. Topol. Monogr.}, {\em 4}; {\em Geom. Topol. Publ.}: Coventry, 2002;pp. 281--294. 
%%\href{http://arxiv.org/abs/math.QA/0210368}{math.QA/0210368}.

\bibitem{Schreiber2k10}
Schreiber, U. Invited lecture in {\em Workshop and School on Higher Gauge Theory, TQFT and Quantum Gravity, 2011}; http://sites.google.com/site/hgtqgr/ ; {\em Conference on Higher Gauge Theory, Quantum Gravity, and Topological Field Theory, December 18, 2010};\\
http://sbseminar.wordpress.com/2010/12/18/conference-on-higher-gauge-theory-quantum-gravity-TFT

\bibitem{Schwartz45}
Schwartz, L. G\'en\'eralisation de la notion de fonction, de d\'erivation, de transformation de Fourier et applications math\'ematiques et physiques. {\em Ann. Univ. Grenoble. Sect. Sci. Math. Phys. (N.S.)} {\bf 1945}, {\em 21}, 57--74.

\bibitem{Schwartz52}
Schwartz, L. {\em Th\'eorie des distributions.} {\em Publications de l'Institut de Math\'ematique de l'Universit\'e de Strasbourg}, no.IX--X; Hermann: Paris, 1966.

\bibitem{Schechtman}
Schechtman, D.; Blech, L.; Gratias, D.; Cahn, J.W. Metallic phase with long-range orientational order and no translational symmetry. \emph{\em Phys. Rev. Lett.} {\bf 1984}, \emph{53}, 1951--1953.

\bibitem{Seda76}
Seda, A.K. Haar measures for groupoids. {\em Proc. Roy. Irish Acad. Sect. A} {\bf 1976}, {\em 76}, no.~5, 25--36.

\bibitem{Seda82}
Seda, A.K. Banach bundles of continuous functions and an integral representation theorem. {\em Trans. Amer. Math. Soc.} {\bf 1982}, {\em 270}, 327--332.

\bibitem{Seda86}
Seda, A.K. On the Continuity of Haar measures on topological groupoids. \emph{Proc. Amer. Math. Soc.}{\bf 1986}, {\em 96}, 115--120.

\bibitem{Segal47a}
Segal, I.E. Irreducible representations of operator algebras. \emph{Bull. Amer. Math. Soc.} {\bf 1947}, {\em 53}, 73--88.

\bibitem{Segal47b}
Segal, I.E. Postulates for general quantum mechanics. \emph{Ann. of Math. (2)} {\bf 1947}, {\em 4}, 930--948.

\bibitem{Seifert1}
Seifert, H. Konstruction drei dimensionaler geschlossener Raume. {\em Berichte S{$\"{a}$}chs. {A}kad. {L}eipzig, Math.-Phys. Kl.} {\bf 1931}, {\em 83}, 26--66.

\bibitem{Serre65}
Serre, J. P. {\em Lie Algebras and Lie Groups: 1964 Lectures given at Harvard University}. Lecture Notes in Mathematics, {\em 1500}; Springer: Berlin, 1965.

\bibitem{Sheu1}
Sheu, A. J. L. Compact quantum groups and groupoid $C^*$-algebras. \emph{J. Funct. Anal.} {\bf 1997}, {\em 144, no. 2}, 371--393.

\bibitem{Sheu2}
Sheu, A. J. L. The structure of quantum spheres. {\em J. Funct. Anal.} {\bf 2001}, {\em 129, no. 11}, 3307--3311.

\bibitem{Sklyanin83}
Sklyanin, E.K. Some algebraic structures connected with the Yang--Baxter equation. {\em Funktsional. Anal. i Prilozhen.} {\bf 1997}, {\em 6}; {\bf 16} (1982), no.4, 27--34.

\bibitem{Sklyanin84}
Sklyanin, E.K. Some algebraic structures connected with the Yang--Baxter equation. Representations of quantum algebras. {\em Funct. Anal. Appl.} {\bf 1984}, {\em 17}, 273--284.

\bibitem{Sklyanin83en}
Sklyanin, E. K. Some Algebraic Structures Connected with the Yang--Baxter equation. {\em Funct. Anal. Appl.} {\bf 1983}, {\em 16}, 263--270.

\bibitem{Street1}
Street, R. The quantum double and related constructions. {\em J. Pure Appl. Algebra} {\bf 1998}, {\em 132, no. 2}, 195--206.

\bibitem{Stern-Halperin2k6}
Stern, A.; Halperin, B.I. Proposed experiments to probe the non-Abelian $\nu=5/2$ quantum Hall state. {\em Phys. Rev. Lett.} {\bf 2006}, {\em 96}, 016802, 4~pages.
%%\href{http://arxiv.org/abs/cond-mat/0508447}{cond-mat/0508447}.

\bibitem{Stradling78}
Stradling, R.A. Quantum transport. {\em Phys. Bulletin} {\bf 1978}, {\em 29}, 559--562.

\bibitem{Szlachanyi2k4}
Szlach$\'{a}$nyi, K. The double algebraic view of finite quantum groupoids. {\em J. Algebra} {\bf 2004}, {\em280}, 249--294.

\bibitem{Sweedler96}
Sweedler, M.E. Hopf algebras. {\em Mathematics Lecture Note Series}; W.A. Benjamin, Inc.: New York, 1969.

\bibitem{Tn2k6}
T\u{a}$nas\u{a}$, A. Extension of the Poincar\'e symmetry and its field theoretical interpretation. {\em SIGMA} {\bf 2006}, {\em 2}, 056, 23~pages. 
%%\href{http://arxiv.org/abs/hep-th/0510268}{hep-th/0510268}.

\bibitem{Taylor88}
Taylor, J. Quotients of groupoids by the action of a group. {\em Math. Proc. Cambridge Philos. Soc.} (1988), {\bf 103}, 239--249.

\bibitem{Tonks93}
Tonks, A.P. {\em Theory and applications of crossed complexes}. PhD Thesis; University of Wales: Bangor, 1993.

\bibitem{Tsui-Allen2k7}
Tsui, D.C.; Allen, S.J. Jr. Mott--Anderson localization in the two-dimensional band tail of {\em Si} inversion layers. {\em Phys. Rev. Lett.} {\bf 1974}, {\em 32}, 1200--1203.

\bibitem{Turaev--Viro92}
Turaev, V.G.; Viro, O.Ya. State sum invariants of 3-manifolds and quantum {\em 6j}-symbols. \emph{Topology} (1992), {\bf 31}, 865--902.

\bibitem{van Kampen33}
van Kampen, E.H. On the connection between the fundamental groups of some related spaces. \emph{Amer. J. Math.} (1933), {\bf 55}, 261--267.

\bibitem{Varilly97}
V\'{a}rilly, J.C. An introduction to noncommutative geometry. {\em EMS Series of Lectures in Mathematics}; European Mathematical Society (EMS): Z\"{u}rich, 2006.

\bibitem{Weinberg96}
Weinberg, S. {\em The quantum theory of fields., Vol. I. Foundations and Vol. II. Modern applications.}; Cambridge University Press: Cambridge, 1996-1997.

\bibitem{Weinberg2005}
Weinberg, S. {\em The quantum theory of fields. Vol. III. Supersymmetry}; Cambridge University Press: Cambridge, 2005.

\bibitem{Weinstein96}
Weinstein, A. Groupoids: unifying internal and external symmetry. A tour through some examples.{\em Notices Amer. Math. Soc.} {\bf 1996}, {\em 43}, 744--752.
%%\href{http://arxiv.org/abs/math.RT/9602220}{math.RT/9602220}.

\bibitem{Wen91}
Wen, X.-G. Non-Abelian statistics in the fractional quantum Hall states. {\em Phys. Rev. Lett.} {\bf 1991}, {\em 66}, 802--805.

\bibitem{Wen99}
Wen, X.-G. Projective construction of non-Abelian quantum Hall liquids. {\em Phys. Rev. B} {\bf 1999}, {\em 60}, 8827, 4~pages. %%\href{http://arxiv.org/abs/cond-mat/9811111}{cond-mat/9811111}.

\bibitem{Wen2k3}
Wen, X.-G. Quantum order from string-net condensations and origin of light and massless fermions.\\ {\em Phys. Rev. D} {\bf 2003}, {\em 68}, 024501, 25~pages. %%\href{http://arxiv.org/abs/hep-th/0302201}{hep-th/0302201}.

\bibitem{Wen2k4}
Wen, X.-G.{\em Quantum field theory of many--body systems -- from the origin of sound to an origin of\\ light and electrons}; Oxford University Press: Oxford, 2004.

\bibitem{Wess-Bagger83}
Wess, J.; Bagger, J. Supersymmetry and supergravity. {\em Princeton Series in Physics}; Princeton University Press: Princeton, N.J., 1983.

\bibitem{WJ1}
Westman, J.J. Harmonic analysis on groupoids. {\em Pacific J. Math.} {\bf 1968}, {\em 27}, 621--632.

\bibitem{WJ2}
Westman, J.J. {\em Groupoid theory in algebra, topology and analysis}; University of California at Irvine, 1971.

\bibitem{Wickramasekara-Bohm2k2}
Wickramasekara, S.; Bohm, A. Symmetry representations in the rigged Hilbert space formulation of quantum mechanics. {\em J. Phys. A: Math. Gen.} {\bf 2002}, {\em 35}, 807--829.

%%\href{http://arxiv.org/abs/math-ph/0302018}{math-ph/0302018}.

\bibitem{Wightman56}
Wightman, A.S. Quantum field theory in terms of vacuum expectation values. {\em Phys. Rev.} {\bf 1956}, {\em 101}, 860--866.

\bibitem{Wightman76}
Wightman, A.S., Hilbert's sixth problem: mathematical treatment of the axioms of physics. In {\em Mathematical Developments Arising from Hilbert Problems : Proc. Sympos. Pure Math.}, Northern Illinois Univ., De Kalb, Ill., 1974. {\em Amer. Math. Soc.}: Providence, R.I., 1976;pp. 147--240.

\bibitem{Wightman-Garding64}
Wightman, A.S.; G\"arding, L. Fields as operator-valued distributions in relativistic quantum theory. {\em Ark. Fys.} {\bf 1964}, {\em 28}, 129--184.

\bibitem{WEP31}
Wigner, E. P. \emph {Gruppentheorie}; Friedrich Vieweg und Sohn: Braunschweig, Germany, 1931;pp. 251-254. {\em Group Theory}; Academic Press Inc.: New York, 1959;pp. 233-236.

\bibitem{WEP39}
Wigner, E. P. On unitary representations of the inhomogeneous Lorentz group. {\em Annals of Mathematics} {\bf 1939}, {\em 40} (1), 149--204. doi:10.2307/1968551.

\bibitem{Witten89}
Witten, E. Quantum field theory and the Jones polynomial. {\em Comm. Math. Phys.} {\bf 1989}, {\em 121}, 351--355.

\bibitem{Witten98}
Witten, E. Anti de Sitter space and holography. {\em Adv. Theor. Math. Phys.} {\bf 1998},{\em 2}, 253--291.
%%\mbox{\href{http://arxiv.org/abs/hep-th/9802150}{hep-th/9802150}}.

\bibitem{Wood97}
Wood, E.E. Reconstruction Theorem for Groupoids and Principal Fiber Bundles. {\em Intl. J. Theor. Physics} {\em 1997} {\em 36} (5), 1253-1267. DOI: 10.1007/BF02435815.

\bibitem{Woronowicz1}
Woronowicz, S.L. Twisted {\em SU(2)} group. An example of a non-commutative differential calculus.{\em Publ. Res. Inst. Math. Sci.}(1987), {\bf 23},117--181.

\bibitem{Woronowicz98}
Woronowicz, S. L. Compact quantum groups. In \emph{Quantum Symmetries.} Les Houches Summer School--1995, Session LXIV; Editors: A. Connes; K. Gawedzki; J. Zinn-Justin; Elsevier Science: Amsterdam,1998;pp. 845--884.

\bibitem{Xu97}
Xu, P. Quantum groupoids and deformation quantization. {\em C. R. Acad. Sci. Paris S$\'{e}$r. I Math.} (1998), {\bf 326}, 289--294. 
%%\href{http://arxiv.org/abs/q-alg/9708020}{q-alg/9708020}.

\bibitem{Yang-Mills54}
Yang, C.N.; Mills, R.L. Conservation of isotopic spin and isotopic gauge invariance. {\em Phys. Rev.} {\bf 1954}, {\em 96}, 191--195.

\bibitem{Yang62}
Yang, C.N. Concept of Off-Diagonal Long-Range Order and the Quantum Phases of Liquid He and of Superconductors.{\em Rev. Mod. Phys.} {\bf 1962}, {\em 34}, 694--704.

\bibitem{Yetter93}
Yetter, D.N. TQFTs from homotopy 2-types. \textit{J. Knot Theory Ramifications} {\bf 1993}, {\em 2}, 113--123.

\bibitem{Ypma1}
Ypma, F. K-theoretic gap labelling for quasicrystals, {\em Contemp. Math.}{\bf 2007}, {\em 434}, 247--255; \emph{Amer. Math. Soc.}: Providence, RI,.

\bibitem{Ypma2}
Ypma, F. {\em Quasicrystals, $C^*$-algebras and K-theory}. Msc. Thesis. {\bf 2004}: University of Amsterdam.

\bibitem{Zhang91}
Zhang, R.B. Invariants of the quantum supergroup $U_{q(gl(m/1))}$. {\em J. Phys. A: Math. Gen.} {\bf 1991}, {\em 24}, L1327--L1332.

\bibitem{Zhang-Gould99}
Zhang, Y.-Z.; Gould, M.D. Quasi-Hopf superalgebras and elliptic quantum supergroups, \emph{J. Math. Phys.} {\bf 1999}, {\em 40}, 5264--5282,
%%\href{http://arxiv.org/abs/math.QA/9809156}{math.QA/9809156}.
\end{thebibliography}


%%%%%
%%%%%
\end{document}
