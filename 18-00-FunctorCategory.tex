\documentclass[12pt]{article}
\usepackage{pmmeta}
\pmcanonicalname{FunctorCategory}
\pmcreated{2013-03-22 18:25:40}
\pmmodified{2013-03-22 18:25:40}
\pmowner{CWoo}{3771}
\pmmodifier{CWoo}{3771}
\pmtitle{functor category}
\pmrecord{8}{41082}
\pmprivacy{1}
\pmauthor{CWoo}{3771}
\pmtype{Definition}
\pmcomment{trigger rebuild}
\pmclassification{msc}{18-00}
\pmclassification{msc}{18A25}
\pmclassification{msc}{18A05}
\pmrelated{SmallCategory}
\pmrelated{ConcreteCategory}
\pmrelated{Supercategories3}
\pmdefines{category of functors}

\usepackage{amssymb,amscd}
\usepackage{amsmath}
\usepackage{amsfonts}
\usepackage{mathrsfs}

% used for TeXing text within eps files
%\usepackage{psfrag}
% need this for including graphics (\includegraphics)
%\usepackage{graphicx}
% for neatly defining theorems and propositions
\usepackage{amsthm}
% making logically defined graphics
%%\usepackage{xypic}
\usepackage{pst-plot}

% define commands here
\newcommand*{\abs}[1]{\left\lvert #1\right\rvert}
\newtheorem{prop}{Proposition}
\newtheorem{thm}{Theorem}
\newtheorem{ex}{Example}
\newcommand{\real}{\mathbb{R}}
\newcommand{\pdiff}[2]{\frac{\partial #1}{\partial #2}}
\newcommand{\mpdiff}[3]{\frac{\partial^#1 #2}{\partial #3^#1}}
\begin{document}
\PMlinkescapeword{functor category}
\PMlinkescapeword{category of functors}

Let $\mathcal{C},\mathcal{D}$ be categories.  Consider the class $O$ of all covariant functors $T:\mathcal{C}\to \mathcal{D}$, and the class $M$ of all natural transformations $\tau:S\dot{\to} T$ for every pair $S,T:\mathcal{C}\to \mathcal{D}$ of functors.  Write $\mathcal{D}^\mathcal{C}$ for the pair $(O,M)$.

For each pair of functors $S,T:\mathcal{C}\to \mathcal{D}$, write $\hom(S,T)$ the class of all natural transformations from $S$ to $T$.  If $\tau$ is in both $\hom(S,T)$ and $\hom(U,V)$, then $S=U$ and $T=V$.

Using the composition of natural transformations, we have a mapping $$\bullet:\hom(R,S)\times \hom(S,T)\to \hom(R,T),$$ for every triple $R,S,T:\mathcal{C}\to \mathcal{D}$.  Since composition of natural transformations is associative, the associativity of $\bullet$ applies.

In addition, for each $S:\mathcal{C}\to \mathcal{D}$, we have the identity natural transformation $1_S\in \hom(S,S)$.  For every $\tau\in \hom(S,T)$ and every $\eta\in \hom(T,S)$, we have $\tau \bullet 1_S=\tau$ and $1_S\bullet \eta=\eta$. 

From the discussion above, we are ready to call $\mathcal{D}^\mathcal{C}$ a category.  However, unless $\hom(S,T)$ is a set for every pair of functors in $O$, $\mathcal{D}^\mathcal{C}$ is not a category.  When $\mathcal{D}^\mathcal{C}$ is a category, we call it the \emph{category of functors} from $\mathcal{C}$ to $\mathcal{D}$, or simply a \emph{functor category}.

That $\mathcal{D}^\mathcal{C}$ is a functor category depends on various restrictions being placed on the ``sizes'' of $\mathcal{C}$ and $\mathcal{D}$:

\begin{prop} If $\mathcal{C}$ is \PMlinkname{$\mathcal{U}$-small}{SmallCategory}, then $\mathcal{D}^\mathcal{C}$ is a category. \end{prop}
\begin{proof}  Suppose $\mathcal{C}$ is $\mathcal{U}$-small.  Consider the class $\hom(S,T)$.  Each $\tau \in \hom(S,T)$ is determined by the collection of morphisms $S(A)\to T(A)$ for each object $A$ in $\mathcal{C}$.  This means that, for each $A$ in $\mathcal{C}$, $\hom(S(A),T(A))$ contains the image of every $\tau\in \hom(S,T)$ under $A$.  So the class of all these natural transformations is a subclass of the product 
\begin{equation}
\prod_{A\in \operatorname{Ob}(\mathcal{C})} \hom(S(A),T(A))
\end{equation}
Since $\operatorname{Ob}(\mathcal{C})$, as well as each $\hom(S(A),T(A))$ is a set, so is the product (1).  Hence $\hom(S,T)$, being a subclass of (1), is a set, or that $\mathcal{D}^{\mathcal{C}}$ is a category.
\end{proof}

\begin{prop} If in addition $\mathcal{D}$ is a \PMlinkname{$\mathcal{U}$-category}{SmallCategory}, then so is $\mathcal{D}^\mathcal{C}$. \end{prop}
\begin{proof}  $\mathcal{D}$ being a $\mathcal{U}$-category means that $\hom(S(A),T(A))$ is $\mathcal{U}$-small, for every object $A$ in $\mathcal{C}$.  Since $\operatorname{Ob}(\mathcal{C})$ is also $\mathcal{U}$-small (assumption in Proposition 1), the product (1) above is $\mathcal{U}$-small.  Consequently, $\hom(S,T)$, being a subclass of (1), is $\mathcal{U}$-small.  This shows that $\mathcal{D}^{\mathcal{C}}$ is a $\mathcal{U}$-category.
\end{proof}

\begin{prop} If $\mathcal{D}$ is furthermore $\mathcal{U}$-small, so is $\mathcal{D}^\mathcal{C}$. \end{prop}
\begin{proof}  We want to show that the class $\mathcal{M}$ of all functors from $\mathcal{C}$ to $\mathcal{D}$ is $\mathcal{U}$-small.  A functor $S:\mathcal{C}\to \mathcal{D}$ can be broken up into two components: a function $S_1: \operatorname{Ob}(\mathcal{C})\to \operatorname{Ob}(\mathcal{D})$, and a function $S_2:\operatorname{Mor}(\mathcal{C})\to \operatorname{Mor}(\mathcal{D})$, so that $S_2(A\to B)=S_1(A)\to S_1(B)$.  

Define a binary relation $\sim$ on $\mathcal{M}$ so that $S\sim T$ iff they have the same first component: $S_1=T_1$.  It is easy to see that $\sim$ is an equivalence relation on $\mathcal{M}$.  Let $[S]$ be the equivalence class containing the functor $S$.  For every morphism $A\to B$, its image under the second component of every functor in $[S]$ lies in $\hom(S_1(A),S_1(B))$.  So the size of $[S]$ can not exceed the size of $$\prod_{A,B\in \operatorname{Ob}(\mathcal{C})} \hom(S_1(A),S_1(B))$$  Since $\operatorname{Ob}(\mathcal{C})$ is $\mathcal{U}$-small (assumption in Prop 1), so is $\operatorname{Ob}(\mathcal{C})\times \operatorname{Ob}(\mathcal{C})$.  Furthermore, since each $\hom(S_1(A),S_1(B))$ is $\mathcal{U}$-small (assumption in Prop 2), $[S]$ is $\mathcal{U}$-small as well.  

Next, let us estimate the size of the class $\mathcal{M}/\sim$ of equivalence classes in $\mathcal{M}$.  First, note that for every functor $S:\mathcal{C}\to \mathcal{D}$, its first component is a function from the \emph{set} $\operatorname{Ob}(\mathcal{C})$ to the \emph{set} $\operatorname{Ob}(\mathcal{D})$ by assumption.  As $[S]\ne [T]$ iff $S_1\ne T_1$, the size can not exceed $$|\operatorname{Ob}(\mathcal{D})^{\operatorname{Ob}(\mathcal{C})}|$$ the cardinality of the set of all functions from $\operatorname{Ob}(\mathcal{C})$ to $\operatorname{Ob}(\mathcal{D})$.  By assumption, $\operatorname{Ob}(\mathcal{D})$ is $\mathcal{U}$-small, so is $\operatorname{Ob}(\mathcal{D})^{\operatorname{Ob}(\mathcal{C})}$.  As a result, $\mathcal{M}/\sim$ is $\mathcal{U}$-small.  Together with the fact that $[S]$ is $\mathcal{U}$-small for each functor $S$, we have that $\mathcal{M}$ itself must be $\mathcal{U}$-small, which completes the proof.
\end{proof}

%Finally, suppose $\mathcal{D}$ is $\mathcal{U}$-small.

%If $\mathcal{C}$ is a $\mathcal{U}$-category, then we can also define the \emph{functor category} $\mathcal{D}^ \mathcal{C}$; the objects of $\mathcal{D}^\mathcal{C}$ are the functors $T:\mathcal{C}\to\mathcal{D}$, and the morphisms are the natural transformations $\tau:S\dot{\to} T$.  The composition of two composable functions which are natural transformations is again a natural transformation, and so $\mathcal{D}^\mathcal{C}$ is a category.
%%%%%
%%%%%
\end{document}
