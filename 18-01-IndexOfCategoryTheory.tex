\documentclass[12pt]{article}
\usepackage{pmmeta}
\pmcanonicalname{IndexOfCategoryTheory}
\pmcreated{2013-03-22 16:40:00}
\pmmodified{2013-03-22 16:40:00}
\pmowner{rspuzio}{6075}
\pmmodifier{rspuzio}{6075}
\pmtitle{index of category theory}
\pmrecord{14}{38873}
\pmprivacy{1}
\pmauthor{rspuzio}{6075}
\pmtype{Topic}
\pmcomment{trigger rebuild}
\pmclassification{msc}{18-01}
\pmrelated{DualityInMathematics}
\pmrelated{IndexOfCategories}
\pmrelated{CategoryTheory}
\pmrelated{CategoricalOntologyABibliographyOfCategoryTheory}
\pmrelated{TopicEntryOnTheAlgebraicFoundationsOfMathematics}
\pmrelated{2CCategory}
\pmrelated{GrothendieckCategory}

% this is the default PlanetMath preamble.  as your knowledge
% of TeX increases, you will probably want to edit this, but
% it should be fine as is for beginners.

% almost certainly you want these
\usepackage{amssymb}
\usepackage{amsmath}
\usepackage{amsfonts}
\usepackage{multicol}

% used for TeXing text within eps files
%\usepackage{psfrag}
% need this for including graphics (\includegraphics)
%\usepackage{graphicx}
% for neatly defining theorems and propositions
%\usepackage{amsthm}
% making logically defined graphics
%%%\usepackage{xypic}

% there are many more packages, add them here as you need them

% define commands here

\begin{document}
\begin{multicols}{2}

\section{Foundations}

\subsection{Basic Definitions}
\begin{enumerate}
\item category theory
\item precategory
\item category
\item alternative definition of category
\item subcategory
\item automorphism
\item commutative diagram
\item concrete category
\item dual category
\item duality principle
\item endomorphism
\item epi
\item monic
\item extremal monomorphism
\item source
\item sink
\item initial source
\item final sink
\item isomorphism-closed subcategory
\item locally finite category
\item preimage of category
\item product of categories
\item types of morphisms
\item wellpowered category
\item zero object
\item \PMlinkid{$\mathcal{U}$-small}{5658}
\item equalizer
\item subobject
\item quotient object
\item categorical direct product
\item categorical direct sum
\item categorical pullback
\item direct limit
\item limiting cone
\item complete category
\item groupoid (category theoretic)
\end{enumerate}

\subsection{Maps of Categories}
\begin{enumerate}
\item functor
\item autofunctor
\item category isomorphism
\item diagonal functor 
\item endofunctor
\item forgetful functor 
\item identity functor
\item isomorphism
\item multifunctor
\item natural transformation
\item essentially surjective
\item faithful functor
\item full functor
\item natural equivalence
\item adjoint functor
\item equivalence of categories
\item \PMlinkname{isomorphic categories}{CategoryIsomorphism}
\item universal property
\item representable functor
\item Equivalent definition of a Representable Functor 
\item simplicial object
\end{enumerate}

\subsection{Fundamental Theorems}
\begin{enumerate}
\item properties of monomorphisms and epimorphisms
\item properties of regular and extremal monomorphisms
\item monomorphisms are pullback stable
\item proof that an equalizer is a monomorphism
\item Yoneda lemma
\item categorical direct product is an inverse limit
\item kernel is an inverse limit
\end{enumerate}

\subsection{Examples of Categories}
\begin{enumerate}
\item discrete category
\item category example (arrow category)
\item category associated to a partial order
\item category of matrices
\item Category of pseudomorphisms
\item Category of intermorphisms
\item examples of initial objects and terminal objects and zero objects
\item category of sets
\item monomorphisms of category of sets
\item monoid as a category
\item comma category
\item category of pointed topological spaces
\item simplicial category
\end{enumerate}

\subsection{Micellaneous}
\begin{enumerate}
\item algebra formed from a category
\item monad
\item comonad 
\item monoidal category
\item group object
\item nerve

\end{enumerate}

\section{Additive Categories and Homology}
\begin{enumerate}
\item preadditive category
\item additive category
\item Abelian category
\item supplemental axioms for an Abelian category
\item exact sequence
\item exact functor
\item Grothendieck spectral sequence
\item enough projectives 
\item enough injectives
\item projective object
\item injective object 
\item derived functor
\item derived category
\item Algebraic K-theory
\item examples of algebraic K-theory groups
\item Grothendieck group
\item delta functor
\item horseshoe lemma
\item syzygy
\item Ext
\item Tor
\item projective dimension
\item 5-lemma
\item proof of 5-lemma
\item 9-lemma
\item snake lemma
\item \PMlinkname{proof of snake lemma}{ProofOfSnakeLemma}
\item chain homotopy
\item chain homotopy equivalence
\item chain map
\item homology of a chain complex
\item Leray spectral sequence
\item spectral sequence
\end{enumerate}

\section{Sheaves, Topoi, and the like}
\begin{enumerate}
\item presheaf
\item sheaf 
\item sheafification
\item presheaf of a topological basis
\item stalk 
\item \'Etal\'e space
\item resolution of a sheaf
\item site
\item small site on a scheme
\item topos
\item cosmos
\item subobject classifier
\item well-pointed topos
\item power object
\item natural numbers object 
\item Cartesian closed category 
\item exponential object
\end{enumerate}

\end{multicols}
%%%%%
%%%%%
\end{document}
