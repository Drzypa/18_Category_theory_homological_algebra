\documentclass[12pt]{article}
\usepackage{pmmeta}
\pmcanonicalname{PartialOrderingOnSubobjectsOfAnObject}
\pmcreated{2013-03-22 18:20:40}
\pmmodified{2013-03-22 18:20:40}
\pmowner{CWoo}{3771}
\pmmodifier{CWoo}{3771}
\pmtitle{partial ordering on subobjects of an object}
\pmrecord{11}{40978}
\pmprivacy{1}
\pmauthor{CWoo}{3771}
\pmtype{Definition}
\pmcomment{trigger rebuild}
\pmclassification{msc}{18A20}
\pmdefines{union}
\pmdefines{intersection}
\pmdefines{union of subobjects}
\pmdefines{intersection of subobjects}

\usepackage{amssymb,amscd}
\usepackage{amsmath}
\usepackage{amsfonts}
\usepackage{mathrsfs}

% used for TeXing text within eps files
%\usepackage{psfrag}
% need this for including graphics (\includegraphics)
%\usepackage{graphicx}
% for neatly defining theorems and propositions
\usepackage{amsthm}
% making logically defined graphics
%%\usepackage{xypic}
\usepackage{pst-plot}

% define commands here
\newcommand*{\abs}[1]{\left\lvert #1\right\rvert}
\newtheorem{prop}{Proposition}
\newtheorem{thm}{Theorem}
\newtheorem{ex}{Example}
\newcommand{\real}{\mathbb{R}}
\newcommand{\pdiff}[2]{\frac{\partial #1}{\partial #2}}
\newcommand{\mpdiff}[3]{\frac{\partial^#1 #2}{\partial #3^#1}}

\newcommand{\Sub}{{\mathrm{Sub}}}
\begin{document}
Let $\mathcal{C}$ be a category and $A$ an object $\mathcal{C}$.  Recall that a subobject of $A$ is just an equivalent class of equivalent monomorphisms into $A$.  A subobject is denoted by $[f:B\to A]$, where $f$ is monomorphic, or simply $[B]$, whenever there is no confusion.  Furthermore, we write $B\subseteq A$ to mean that $B$ is a subobject of $A$, and write $B\cong C$ whenever $[B]=[C]$.  The class of all subobjects of $A$ is denoted by $\Sub(A)$.  We wish to put a partial order on $\Sub(A)$.

Given any two monomorphisms $f:B \to A$ and $g:C\to A$, define $g\le f$ iff there is a morphism $h:C\to B$ such that
$$
\xymatrix{
C\ar[dr]^g\ar[dd]_h\\
&A\\
B\ar[ur]_f
}
$$
Now, if $f':B'\to A$ is equivalent to $f$ and $g':C\to A$ is equivalent to $g$, then we have morphisms $x:C\to C'$ and $x':C'\to C$ and $y:B\to B'$ and $y':B'\to B$ such that the diagram below consisting of only the solid lines is commutative:
$$
\xymatrix{
C\ar[dr]_g\ar[dd]_h\ar[rr]_x & & C'\ar[dl]^{g'}\ar[ll]_{x'} \ar@{.>}[dd]^{h'} \\
&A &\\
B\ar[ur]^f\ar[rr]_y & & B'\ar[ul]_{f'}\ar[ll]_{y'}
}
$$
We define $h'=y\circ h\circ x'$ (the dotted line above).  Then the diagram above including the dotted line is commutative: as $f'\circ h'=f'\circ (y\circ h\circ x')= f\circ (h\circ x')=g \circ x' = g'$.  Hence $g'\le f'$.  As a result, $\le$ induces a binary relation on $\Sub(A)$: $$C\le B$$ iff there are representing monomorphisms $f:B\to A$ and $g:C\to A$ such that $g\le f$.  Let us use the same notation $\le$ for the induced relation.  Then $\le$ on $\Sub(A)$ is a partial ordering on $\Sub(A)$:

\begin{description}
\item[reflexivity:] $B\le B$, because the composition of the identity morphism $1_B$ with any representing monomorphism $f:B\to A$ is $f$ itself.
\item[anti-symmetry:] if $B\le C$ and $C\le B$, then there are representing monomorphisms $f:B\to A$ and $g:C\to A$ such that $f\le g$ and $g\le f$.  So there are morphisms $h:B\to C$ and $h':C\to B$ such that
$$
\xymatrix{
C\ar@<0.5ex>[dr]^g \ar@<0.5ex>[dd]^{h'}\\
&A\\
B\ar@<-0.5ex>[ur]_f \ar@<0.5ex>[uu]^h
}
$$
is commutative.  But this just means that $h$ and $h'$ are inverses of one other, or $B\cong C$, so that they are the same subobject of $A$.

\item[transitivity:] if $D\le C$ and $C\le B$, then there are representing monomorphisms $f:B\to A$, $g:C\to A$ and $h:D\to A$ such that $h\le g$ and $g\le f$.  This means there are morphisms $r:D\to C$ and $s:C\to B$ such that $g\circ r = h$ and $f\circ s=g$.  Since $f\circ (s\circ r)=g\circ r=h$, $h\le f$, and as a result, $D\le B$.
\end{description}

Thus, $\le$ turns $\Sub(A)$ into a partially ordered class, with top element $A$.  With this, we may form the notions of unions and intersections of subobjects.  Formally, let $\lbrace B_i \mid i\in I\rbrace$ be a collection of subobjects of $A$, indexed by a set $I$.  
\begin{itemize}
\item
The \emph{union} $C$ of $B_i$ is the supremum of the $B_i$'s with respect to $\le$, provided that it exists.  In notation, we write $$C=\bigvee_{i\in I} B_i.$$
\item
The \emph{intersection} $D$ of $B_i$ is the infimum of the $B_i$'s with respect to $\le$, provided that it exists.  In notation, we write $$D=\bigwedge_{i\in I} B_i.$$
\end{itemize}

For example, let $f:B \to A$ and $g:C\to A$ be subobjects of $A$.  Then the pullback of $f$ and $g$, if it exists, is the intersection of $B$ and $C$.

\textbf{Remark}.  It can be shown that in any Abelian category, $\Sub(A)$ under $\le$ is a lattice for any object $A$.
%%%%%
%%%%%
\end{document}
