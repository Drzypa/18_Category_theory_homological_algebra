\documentclass[12pt]{article}
\usepackage{pmmeta}
\pmcanonicalname{LongExactSequenceInCohomology}
\pmcreated{2013-03-22 19:03:49}
\pmmodified{2013-03-22 19:03:49}
\pmowner{rm50}{10146}
\pmmodifier{rm50}{10146}
\pmtitle{long exact sequence in cohomology}
\pmrecord{5}{41947}
\pmprivacy{1}
\pmauthor{rm50}{10146}
\pmtype{Theorem}
\pmcomment{trigger rebuild}
\pmclassification{msc}{18G35}
\pmdefines{morphism of cochain complexes}
\pmdefines{map of cochain complexes}
\pmdefines{exact sequence of cochain complexes}

\usepackage{amssymb}
\usepackage{amsmath}
\usepackage{amsfonts}

% used for TeXing text within eps files
%\usepackage{psfrag}
% need this for including graphics (\includegraphics)
%\usepackage{graphicx}
% for neatly defining theorems and propositions
\usepackage{amsthm}
% making logically defined graphics
%%\usepackage{xypic}

% there are many more packages, add them here as you need them

% define commands here
\newtheorem{thm}{Theorem}
\newtheorem{defn}{Definition}
\DeclareMathOperator{\im}{im}
\newcommand{\BQ}{\mathbb{Q}}
\newcommand{\BR}{\mathbb{R}}
\newcommand{\BZ}{\mathbb{Z}}
\begin{document}
\PMlinkescapeword{square}
\begin{defn} If $\mathcal{A}$ and $\mathcal{B}$ are cochain complexes, a morphism $f:\mathcal{A}\to\mathcal{B}$ is a collection of maps $f_n : A^n \to B^n$ such that for every $n$, the following diagram commutes:
\[\xymatrix{
  \cdots \ar[r] & A^n \ar[r]\ar[d]^{f_n} & A^{n+1} \ar[r]\ar[d]^{f_{n+1}} & \cdots \\
  \cdots \ar[r] & B^n \ar[r] & B^{n+1} \ar[r] & \cdots
 }
\]
\end{defn}

\begin{thm} A morphism of cochain complexes $f:\mathcal{A} \to \mathcal{B}$ induces group homomorphisms $H^n(\mathcal{A}) \to H^n(\mathcal{B})$ for every $n$.
\end{thm}
\begin{proof} This follows trivially from the fact that images and kernels of the cochain maps of $\mathcal{A}$ are mapped under $f$ to the images and kernels of the cochain maps of $\mathcal{B}$.
\end{proof}

\begin{defn} If $\alpha : \mathcal{A}\to\mathcal{B}$ and $\beta : \mathcal{B}\to\mathcal{C}$ are morphisms of cochain complexes, we say that
\[
  0 \to \mathcal{A} \xrightarrow{\alpha} \mathcal{B} \xrightarrow{\beta} \mathcal{C} \to 0
\]
is a short exact sequence of cochain complexes if for every $n$,
\[
  0 \to A^n \xrightarrow{\alpha_n} B^n \xrightarrow{\beta_n} C^n \to 0
\]
is short exact.
\end{defn}

\begin{thm} Let
\[
  0 \to \mathcal{A} \to \mathcal{B} \to \mathcal{C} \to 0
\]
be a short exact sequence of cochain complexes with $A^n = B^n = C^n = 0$ for $n<0$. Then there is a long exact sequence of cohomology groups
\[
  0 \to H^0(\mathcal{A}) \to H^0(\mathcal{B}) \to H^0(\mathcal{C}) \to H^1(\mathcal{A}) 
      \to H^1(\mathcal{B}) \to H^1(\mathcal{C}) \to H^2(\mathcal{A}) \to \cdots
\]

\end{thm}

The proof that for each $n$, $H^n(\mathcal{A}) \to H^n(\mathcal{B}) \to H^n(\mathcal{C})$ is exact is straightforward from the above. The interesting part of the proof is in defining the \emph{connecting homomorphism}
\[
  \delta_n : H^n(\mathcal{C}) \to H^{n+1}(\mathcal{A})
\]
To define $\delta_n$, consider the following portion of the diagram:
\[\xymatrix{
                & 0 \ar[d]         & 0 \ar[d]           & 0 \ar[d]\\
  \cdots \ar[r] & A^n \ar[r]^{d_n}\ar[d]^{\alpha_n} & A^{n+1} \ar[r]^{d_{n+1}}\ar[d]^{\alpha_{n+1}} & A^{n+2} \ar[r]\ar[d]^{\alpha_{n+2}} & \cdots \\
  \cdots \ar[r] & B^n \ar[r]^{d_n}\ar[d]^{\beta_n} & B^{n+1} \ar[r]^{d_{n+1}}\ar[d]^{\beta_{n+1}} & B^{n+2} \ar[r]\ar[d]^{\beta_{n+2}} & \cdots \\
  \cdots \ar[r] & C^n \ar[r]^{d_n}\ar[d] & C^{n+1} \ar[r]^{d_{n+1}}\ar[d] & C^{n+2} \ar[r]\ar[d] & \cdots \\
                & 0                & 0                          & 0
 }
\]
Given an element $\bar{c}$ of $H^n(\mathcal{C})$, choose a representative $c\in \ker d_n\subset C^n$. Since the vertical map from $B^n$ is surjective, choose $b\in B^n$ with $\beta_n(b)=c$. Then $d_n(b)\in\ker\beta_{n+1}$ since the square commutes and $c\in\ker d_n$; thus $d_n(b)\in\im \alpha_{n+1}$. Let $a\in A^{n+1}$ be the unique preimage of $d_n(b)$. Note that
\[
  \alpha_{n+2}d_{n+1}(a)= d_{n+1}\alpha_{n+1}(a) = d_{n+1}d_n(b)=0
\]
Since $\alpha_{n+2}$ is injective, $a\in\ker d_{n+1}$. Then define $\delta_n(\bar{c})$ to be the equivalence class of $a$ in $H^{n+1}(\mathcal{A})$. It is a straightforward diagram chase to verify that the result is independent of the choice of representative of $\bar{c}$ and of preimage of $c$.
%%%%%
%%%%%
\end{document}
