\documentclass[12pt]{article}
\usepackage{pmmeta}
\pmcanonicalname{AlternativeDefinitionOfCategory1}
\pmcreated{2013-03-22 18:53:17}
\pmmodified{2013-03-22 18:53:17}
\pmowner{rspuzio}{6075}
\pmmodifier{rspuzio}{6075}
\pmtitle{alternative definition of category}
\pmrecord{15}{41736}
\pmprivacy{1}
\pmauthor{rspuzio}{6075}
\pmtype{Definition}
\pmcomment{trigger rebuild}
\pmclassification{msc}{18A05}

% this is the default PlanetMath preamble.  as your knowledge
% of TeX increases, you will probably want to edit this, but
% it should be fine as is for beginners.

% almost certainly you want these
\usepackage{amssymb}
\usepackage{amsmath}
\usepackage{amsfonts}

% used for TeXing text within eps files
%\usepackage{psfrag}
% need this for including graphics (\includegraphics)
%\usepackage{graphicx}
% for neatly defining theorems and propositions
%\usepackage{amsthm}
% making logically defined graphics
%%%\usepackage{xypic}

% there are many more packages, add them here as you need them

% define commands here

\begin{document}
The notion of category may be defined in a form which only
involves morphisms and does not mention objects.  This
definition shows that categories are a generalization of
semigroups in which the closure axiom has been weakened;
rather than requiring that the product of two arbitrary
elements of the system be defined as an element of the 
system, we only require the product to be defined in
certain cases.

We define a \emph{category} to be a set\footnote{For 
simplicity, we will only consider small categories here, 
avoiding logical complications related to proper classes.} 
$M$ (whose elements we shall term \emph{morphisms}) and a
function $\circ$ (which we shall term \emph{composition})
from a subset $D$ of $M \times M$ to $M$ which satisfies the 
following properties:
\begin{itemize}
 \item{\bf 1.} If $a,b,c,d$ are elements of $M$ such that
  $(a,c) \in D$ and $(a,d) \in D$
  and $(b,c) \in D$, then $(b,d) \in D$.
 \item{\bf 2} If $a,b,c$ are elements of $M$ such that 
  $(a,b) \in D$ and $(b,c) \in D$,
  then $(a \circ b, c) \in D$ and
  $(a, b \circ c) \in D$ and
  $(a \circ b) \circ c = a \circ (b \circ c)$
 \item{\bf 3a} For every $a \in M$, there exists an element
  $e \in M$ such that
   \begin{enumerate}
    \item $(e,e) \in D$ and $e \circ e = e$
    \item $(a,e) \in D$ and $a \circ e = a$
    \item For all $x \in M$ such that $(x,e) \in D$, 
      we have $x \circ e = x$.
   \end{enumerate}
 \item{\bf 3a} For every $a \in M$, there exists an element
  $e \in M$ such that
   \begin{enumerate}
    \item $(e,e) \in D$ and $e \circ e = e$
    \item $(e,a) \in D$ and $e \circ a = a$
    \item For all $x \in M$ such that $(x,e) \in D$, 
      we have $e \circ a = x$.
   \end{enumerate}
\end{itemize}
This definition may also be stated in terms of predicate calculus.
Defining the three place predicate $P$ by $P(a,b,c)$ if and only if
$(a,b) \in D$ and $a \circ b = c$, our axioms look 
as follows:
\begin{itemize}
 \item{\bf 0.} $(\forall a,b,c,d) ~ P(a,b,c) \land P(a,b,d) \Rightarrow
  c = d$.
 \item{\bf 1.} $(\forall a,b,c,d) ~ ((\exists e)~P(a,c,e)) \land 
  ((\exists e)~P(a,d,e)) \land
  ((\exists e)~P(b,c,e))z \Rightarrow ((\exists e)~P(b,d,e))$
 \item{\bf 2.} $(\forall a,b,c,d,e) ~ P(a,b,d) \land P(b,c,e)
  \Rightarrow (\exists f) ~ P(d,c,f) \land P(a,e,f)$
 \item{\bf 3a.} $(\forall a) (\exists b) ~ P(b,b,b) \land P(b,a,a)
  \land ((\forall c,d) ~ P(b,c,d) \Rightarrow c = d)$
 \item{\bf 3b.} $(\forall a) (\exists b) ~ P(b,b,b) \land P(a,b,a)
  \land ((\forall c,d) ~ P(c,b,d) \Rightarrow c = d)$
\end{itemize}

That a category defined in the usual way satisfies these properties 
is easily enough established.  Given two morphisms $f$ and $g$,
the composition $f \circ g$ is only defined if $f \in {\rm Hom} (B,C)$
and $g \in {\rm Hom} (A,B)$ for suitable objects $A,B,C$, i.e if the
final object of $f$ equals the initial object of $g$.  The three
hypotheses of axiom 1 state that the initial object of $a$ equals the
final objects of $c$ and $d$ and that the initial object of $b$ also
equals the final object of $c$; hence the initial object of $b$ equals
the final object of $d$ so we may compose $b$ with $d$.
Axiom 2 states associativity of composition whilst axioms 3a and 3b
follow from existence of identity elements.

To show that the new definition implies the old one is not so easy
because we must first recover the objects of the category somehow.
The observation which makes this possible is that to each object $A$
we may associate two sets: the set ${\bf L}$ of morphisms which have
$A$ as initial object, ${\bf L} = \cup_{B \in {\rm Ob}} {\rm Hom} (A,B)$,
and the set ${\bf R}$ of morphisms which have $A$ as final object, 
${\bf R} = \cup_{B \in {\rm Ob}} {\rm Hom} (B,A)$.  Moreover, this
pair of sets $({\bf L},{\bf R})$ determines $A$ uniquely.  In order
for this observation to be useful for our purposes, we must somehow
characterize these pairs of sets without reference to objects, which
may be done by the further observation that, if we have two sets ${\bf L}$ 
and ${\bf R}$ of morphisms such that $x \in {\bf L}$ if and only if 
$x \circ y$ is defined for all $y \in {\bf R}$ and $x \in {\bf R}$ if 
and only if $y \circ x$ is defined for all $y \in {\bf L}$, then there 
exists an object $A$ which gives rise to ${\bf L}$ and ${\bf R}$ as above.  
This fact may be demonstrated easily enough from the usual definition of
category.  We will now reverse the procedure, using our axioms to show
that such pairs behave as objects should, justifying defining objects
as such pairs.

Returning to our new definition, let us now define $\ell \colon M \to {\cal P} (M)$, 
$r \colon m \to {\cal P} (M)$, ${\cal L} \subseteq {\cal P} (M)$, and
${\cal R} \subseteq {\cal P} (M)$ as follows:
\begin{align*}
 \ell (a) &= \{ b \in M \mid (b,a) \in D \} \\ 
 r (a) &= \{ b \in M \mid (a,b) \in D \} \\ 
 {\cal L} &= \{ \ell (a) \mid a \in M \} \\
 {\cal R} &= \{ r (a) \mid a \in M \}
\end{align*}

We now show that, if $U,V \in {\cal L}$ then either $U \cap V = \emptyset$
or $U = V$.  Suppose that $U,V \in {\cal L}$ and $U \cap V \neq \emptyset$.
Then there exists a morphism $a$ such that $a \in U$ and $a \in V$.  By the
definition of ${\cal L}$, there exist morphisms $b$ and $c$ such that 
$U = \ell (b)$ and $V = \ell (c)$.  By definition of $\ell$, we have $(a,b)
\in D$ and $(a,c) \in D$.  If $d \in U$, then $(d,b) \in D$ so, by axiom 1,
$(d,c) \in D$, i.e. $d \in \ell (c) = V$.  Likewise, switching the roles of
$U$ and $V$ we conclude that, if $d \in V$, then $d \in U$.  Hence $U = V$.

Making an argument similar to that of last paragraph, but with $r$ instead
of $\ell$ and ${\cal R}$ instead of ${\cal L}$, we also conclude that, if 
$U,V \in {\cal R}$ then either 
$U \cap V = \emptyset$ or $U = V$.  Because of axiom 3a, we know that, for
every $a \in M$, there exists $b \in M$ such that $a \in \ell (b)$ and, by 
axiom 3b, there exists $c \in M$ such that $a \in r (c)$.  Hence, the sets
${\cal L}$ and ${\cal R}$ are each partitions of $M$.

Next, we show that, if $S \in {\cal L}$ and $a,b \in S$, then $r(a) = r(b)$.
By definition, there exists a morphism $c$ such that $S = \ell (c)$, so
$(a,c) \in D$ and $(b,c) \in D$.  Now suppose that $d \in r(a)$.  This means
that $(a,d) \in D$.  By axiom 1, we conclude that $(b,d) \in D$, so $d \in
r(b)$.  Likewise, switching the roles of $a$ and $b$ in the foregoing argument,
we conclude that, if $d \in r(b)$, then $d \in r(a)$.  Thus, $r(a) = r(b)$. 

By a similar argument to that of the last paragraph, we may also show that,
if $S \in {\cal R}$ and $a,b \in S$, then $\ell(a) = \ell(b)$.  Taken together,
these results tell us that there is a one-to-one correspondence between 
of ${\cal L}$ and ${\cal R}$ --- to each $S \in {\cal L}$, there exists exactly
one $T \in {\cal R}$ such that $S \times T \in D$ and vice-versa.  In light
of this fact, we shall define and object of our category to be a pair $(P,Q)$
of subsets of $M$ such that $x \in P$ if and only if $(x,y) \in D$ for all
$y \in Q$ and $y \in Q$ if and only if $(x,y) \in D$ for all $x \in Q$.  Given
two objects $A = (P,Q)$ and $B = (R,S)$, we define ${\rm Hom} (A,B) = P \cap S$.
We now will verify that, with these definitions, our axioms reproduce the
defining properties of the standard definition of category.

Suppose that $A = (P,Q)$ and $B = (R,S)$ and $C = (U,V)$ are objects according 
to the above definition and that $f \in {\rm Hom} (A,B)$ and $g \in {\rm Hom} 
(B,C)$.  Then $f \in S$ and $g \in R$.  By the way we defined our pairs, $(g,f) 
\in D$, so $g \circ f$ is defined. Let $h$ be any element of $Q$.  Since 
$f \in P$, it follows that $(f,h) \in D$.  Since $(g,f) \in D$ as well, it follows
from axiom 2 that $(g \cdot f, h) \in D$, so $g \circ f\in P$.  Let $k$ be any
element of $U$.  Since $g \in V$, it follows that $(k,g) \in D$.  Since $(g,f) 
\in D$ as well, it follows from axiom 2 that $(k,g \circ f) \in D$, so $g \circ f
\in V$.  Hence, $g \circ f \in P \cap V = {\rm Hom} (A,C)$.  Thus, $\circ$ is 
defined as a function from ${\rm Hom} (A,B) \times {\rm Hom} (B,C) \to {\rm Hom} (A,C)$.

Next, suppose that $A = (P,Q)$ and $B = (R,S)$ are distinct objects.   By the
properties described earlier, $P \cap R = \emptyset$ and $Q \cap S = \emptyset$.
Let $E$ and $F$ be two objects.  Since ${\rm Hom} (A,E) \subset P$ and
${\rm Hom} (B,F) \subset R$, it follows that ${\rm Hom} (A,E) \cap {\rm Hom} (B,F)
= \emptyset$.  Likewise, since ${\rm Hom} (E,A) \subset Q$ and ${\rm Hom} (F,B) 
\subset S$, it follows that ${\rm Hom} (E.A) \cap {\rm Hom} (F,B) = \emptyset$.
Hence, it follows that, given four objects $A,B,E,F$, we have
${\rm Hom} (A,E) \cap {\rm Hom} (B,F) = \emptyset$ unless $A = B$ and $E = F$.

[more to come]


%%%%%
%%%%%
\end{document}
