\documentclass[12pt]{article}
\usepackage{pmmeta}
\pmcanonicalname{QuantumGravityTheories}
\pmcreated{2013-03-22 18:13:50}
\pmmodified{2013-03-22 18:13:50}
\pmowner{bci1}{20947}
\pmmodifier{bci1}{20947}
\pmtitle{quantum gravity theories}
\pmrecord{24}{40817}
\pmprivacy{1}
\pmauthor{bci1}{20947}
\pmtype{Topic}
\pmcomment{trigger rebuild}
\pmclassification{msc}{18D25}
\pmclassification{msc}{18-00}
\pmclassification{msc}{55U99}
\pmclassification{msc}{81-00}
\pmclassification{msc}{81P05}
\pmclassification{msc}{81Q05}
\pmsynonym{relativistic quantum theories and QFT including gravitational fields}{QuantumGravityTheories}
%\pmkeywords{quantum theories of gravitation}
%\pmkeywords{relativistic quantum theories and QFT including gravitational fields}
\pmrelated{FoundationsOfQuantumFieldTheories}
\pmrelated{LieSuperalgebra}
\pmrelated{HamiltonianAlgebroids}
\pmrelated{NoncommutativeGeometry}
\pmrelated{GroupoidCConvolutionAlgebra}
\pmrelated{MathematicalProgrammesForDevelopingQuantumGravityTheories}
\pmrelated{SpacetimeQuantizationProblemsInQuantumGravityTheories}
\pmrelated{SuperfieldsSu}

% this is the default PlanetMath preamble.  as your knowledge
% of TeX increases, you will probably want to edit this, but
% it should be fine as is for beginners.

% almost certainly you want these
\usepackage{amssymb}
\usepackage{amsmath}
\usepackage{amsfonts}

% used for TeXing text within eps files
%\usepackage{psfrag}
% need this for including graphics (\includegraphics)
%\usepackage{graphicx}
% for neatly defining theorems and propositions
%\usepackage{amsthm}
% making logically defined graphics
%%%\usepackage{xypic}

% there are many more packages, add them here as you need them

% define commands here

\begin{document}
\subsection{Quantum gravity theories}

The goal of several \emph{quantum gravity theories} is to define gravitational interactions in terms of relativistic quantum fields; this poses axiomatic, conceptual, logical and mathematical-fundamental problems and theoretical 
challenges. In spite of their universal span, gravitational interactions are the 
weakest known. Repeated experimental attempts failed so far to reliably detect the quanta of gravitational fields-
the {\em gravitons}- which are considered as `particles' expected to be associated
with `gravitational waves'. Solving the theoretical problem of defining and mathematically treating gravitational interactions in quantum terms is thus the aim of Quantum Gravity theories. Recently, there are several quite different
mathematical/theoretical physics approaches that involve either Hamiltonian algebroids or graded 'Lie' algebras/ superalgebras involving extensions of previous relativistic QFT approaches. Two such 
approaches to Quantum Gravity (and respectively, dark matter) problems are 
\PMlinkname{Noncommutative Geometry}{NoncommutativeGeometry} which was initially proposed by A. Connes, and (respectively) Quantum Geometry (mostly by theoreticians). 


\begin{thebibliography}{9}
\bibitem{AC94}
A. Connes. 1994. \emph{Noncommutative Geometry}, Academic Press: New York.
 
\bibitem{AK2k5}
Abhay Ashtekar and Jerzy Lewandowski.2005. Quantum Geometry and Its Applications,\\
\PMlinkexternal{Quantum Geometry with Applications}{http://cgpg.gravity.psu.edu/people/Ashtekar/articles/qgfinal.pdf} 

\bibitem{VJC97}
Várilly, J. C.: 1997, \emph{An introduction to noncommutative geometry}, arXiv: phys/9709045
\end{thebibliography} 
%%%%%
%%%%%
\end{document}
