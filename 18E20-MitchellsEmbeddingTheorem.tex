\documentclass[12pt]{article}
\usepackage{pmmeta}
\pmcanonicalname{MitchellsEmbeddingTheorem}
\pmcreated{2013-03-22 18:29:46}
\pmmodified{2013-03-22 18:29:46}
\pmowner{CWoo}{3771}
\pmmodifier{CWoo}{3771}
\pmtitle{Mitchell's embedding theorem}
\pmrecord{7}{41180}
\pmprivacy{1}
\pmauthor{CWoo}{3771}
\pmtype{Theorem}
\pmcomment{trigger rebuild}
\pmclassification{msc}{18E20}
\pmclassification{msc}{18E10}
\pmdefines{Freyd-Mitchell embedding theorem}

\usepackage{amssymb,amscd}
\usepackage{amsmath}
\usepackage{amsfonts}
\usepackage{mathrsfs}

% used for TeXing text within eps files
%\usepackage{psfrag}
% need this for including graphics (\includegraphics)
%\usepackage{graphicx}
% for neatly defining theorems and propositions
\usepackage{amsthm}
% making logically defined graphics
%%\usepackage{xypic}
\usepackage{pst-plot}

% define commands here
\newcommand*{\abs}[1]{\left\lvert #1\right\rvert}
\newtheorem{prop}{Proposition}
\newtheorem{thm}{Theorem}
\newtheorem{ex}{Example}
\newcommand{\real}{\mathbb{R}}
\newcommand{\pdiff}[2]{\frac{\partial #1}{\partial #2}}
\newcommand{\mpdiff}[3]{\frac{\partial^#1 #2}{\partial #3^#1}}
\begin{document}
\begin{thm}  Every small abelian category admits an \PMlinkname{exact}{ExactFunctor} and \PMlinkname{full}{FullFunctor} \PMlinkname{embedding}{FaithfulFunctor} into the category \textbf{$\mbox{Mod}_R$} of (left) modules over some ring $R$.
\end{thm}

As a consequence, this theorem says that certain facts about small abelian categories can be proved in the more concrete setting of $\mbox{Mod}_R$ (indeed a concrete category).  For example, in order to prove that a sequence is exact in an abelian category, it is enough to prove it in the context of $\mbox{Mod}_R$, by realizing the fact that objects in $\mbox{Mod}_R$ are sets (with structures) and utilizing the elements therein.  In particular, the diagram chasing technique popular in homological algebra may be formulated in small abelian categories as a result of this theorem.

\begin{thebibliography}{9}
\bibitem{fb} F. Borceux \emph{Categories and Structures, Handbook of Categorical Algebra II}, Cambridge University Press, Cambridge (1994)
\bibitem{pf} P. Freyd \emph{Abelian Categories}, Harper and Row, (1964) [\PMlinkexternal{online version}{http://www.emis.de/journals/TAC/reprints/articles/3/tr3.pdf}]
\bibitem{bm} B. Mitchell \emph{The Full Embedding Theorem}, American Journal of Math, 86, (1964) pp. 619-637
\end{thebibliography}
%%%%%
%%%%%
\end{document}
