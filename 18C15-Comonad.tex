\documentclass[12pt]{article}
\usepackage{pmmeta}
\pmcanonicalname{Comonad}
\pmcreated{2013-03-22 16:27:52}
\pmmodified{2013-03-22 16:27:52}
\pmowner{mps}{409}
\pmmodifier{mps}{409}
\pmtitle{comonad}
\pmrecord{10}{38623}
\pmprivacy{1}
\pmauthor{mps}{409}
\pmtype{Definition}
\pmcomment{trigger rebuild}
\pmclassification{msc}{18C15}

\endmetadata

% this is the default PlanetMath preamble.  as your knowledge
% of TeX increases, you will probably want to edit this, but
% it should be fine as is for beginners.

% almost certainly you want these
\usepackage{amssymb}
\usepackage{amsmath}
\usepackage{amsfonts}

% used for TeXing text within eps files
%\usepackage{psfrag}
% need this for including graphics (\includegraphics)
%\usepackage{graphicx}
% for neatly defining theorems and propositions
%\usepackage{amsthm}
% making logically defined graphics
%%\usepackage{xypic}

% there are many more packages, add them here as you need them

% define commands here
\DeclareMathOperator{\id}{id}
\newcommand{\opposite}[1]{{#1}^{\mathrm{op}}}
\begin{document}
Let $\mathcal{C}$ be a category.  A \emph{comonad} over $\mathcal{C}$ consists of an endofunctor $T\colon\mathcal{C}\to\mathcal{C}$ along with natural transformations $\Delta\colon T\dot{\to}T\circ T$ and $\varepsilon\colon T\dot{\to}\id_{\mathcal{C}}$.  The natural transformation $\Delta$ is coassociative, in the sense that the diagram
\[\xymatrix{
& T\ar[dl]_{\Delta}\ar[dr]^{\Delta} & \\
T\circ T\ar[dr]_{T\Delta} & & T\circ T\ar[dl]^{\Delta T} \\
& T\circ T\circ T &
}\]
is commutative.  Moreover, $\Delta$ is counitary, in the sense that the diagram
\[\xymatrix{
 & T\ar[dl]\ar[dr]\ar[dd]^{\Delta} & \\
\id_{\mathcal{C}}T & & T\id_{\mathcal{C}} \\
& G\circ G\ar[lu]^{\varepsilon T}\ar[ru]^{T\varepsilon} &
}\]
also commutes.  Observe that the defining diagrams for a comonad are precisely dual to those for a monad.  Thus one could more briefly define a comonad over a category $\mathcal{C}$ as a monad over the opposite category $\opposite{\mathcal{C}}$.

Just as one can define algebras over a monad it is possible to define coalgebras over a comonad. % add more to this later

\begin{thebibliography}{99}
\bibitem{JoRo}
S.~A.~Joni and G.-C.~Rota, {\it Coalgebras and bialgebras in combinatorics}, Stud.~Appl.~Math., 61 (1979), pp. 93--139.

\bibitem{Ma}
S.~Mac~Lane. {\it Categories for the Working Mathematician}, 2nd ed. Springer-Verlag, 1997

\bibitem{MaMo}
S.~Mac~Lane and I.~Moerdijk. {\it Sheaves and Geometry in Logic: A First Introduction to Topos Theory}, Springer-Verlag, 1992.
\end{thebibliography}

%%%%%
%%%%%
\end{document}
