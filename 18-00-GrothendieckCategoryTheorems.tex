\documentclass[12pt]{article}
\usepackage{pmmeta}
\pmcanonicalname{GrothendieckCategoryTheorems}
\pmcreated{2013-03-22 19:22:13}
\pmmodified{2013-03-22 19:22:13}
\pmowner{bci1}{20947}
\pmmodifier{bci1}{20947}
\pmtitle{Grothendieck category theorems}
\pmrecord{4}{42324}
\pmprivacy{1}
\pmauthor{bci1}{20947}
\pmtype{Theorem}
\pmcomment{trigger rebuild}
\pmclassification{msc}{18-00}
%\pmkeywords{Gabriel-Popescu theorem extensions}

\endmetadata

% this is the default PlanetMath preamble. as your knowledge
% of TeX increases, you will probably want to edit this, but
\usepackage{amsmath, amssymb, amsfonts, amsthm, amscd, latexsym}
%%\usepackage{xypic}
\usepackage[mathscr]{eucal}
% define commands here
\theoremstyle{plain}
\newtheorem{lemma}{Lemma}[section]
\newtheorem{proposition}{Proposition}[section]
\newtheorem{theorem}{Theorem}[section]
\newtheorem{corollary}{Corollary}[section]
\theoremstyle{definition}
\newtheorem{definition}{Definition}[section]
\newtheorem{example}{Example}[section]
%\theoremstyle{remark}
\newtheorem{remark}{Remark}[section]
\newtheorem*{notation}{Notation}
\newtheorem*{claim}{Claim}
\renewcommand{\thefootnote}{\ensuremath{\fnsymbol{footnote%%@
}}}
\numberwithin{equation}{section}
\newcommand{\Ad}{{\rm Ad}}
\newcommand{\Aut}{{\rm Aut}}
\newcommand{\Cl}{{\rm Cl}}
\newcommand{\Co}{{\rm Co}}
\newcommand{\DES}{{\rm DES}}
\newcommand{\Diff}{{\rm Diff}}
\newcommand{\Dom}{{\rm Dom}}
\newcommand{\Hol}{{\rm Hol}}
\newcommand{\Mon}{{\rm Mon}}
\newcommand{\Hom}{{\rm Hom}}
\newcommand{\Ker}{{\rm Ker}}
\newcommand{\Ind}{{\rm Ind}}
\newcommand{\IM}{{\rm Im}}
\newcommand{\Is}{{\rm Is}}
\newcommand{\ID}{{\rm id}}
\newcommand{\GL}{{\rm GL}}
\newcommand{\Iso}{{\rm Iso}}
\newcommand{\Sem}{{\rm Sem}}
\newcommand{\St}{{\rm St}}
\newcommand{\Sym}{{\rm Sym}}
\newcommand{\SU}{{\rm SU}}
\newcommand{\Tor}{{\rm Tor}}
\newcommand{\U}{{\rm U}}
\newcommand{\A}{\mathcal A}
\newcommand{\Ce}{\mathcal C}
\newcommand{\D}{\mathcal D}
\newcommand{\E}{\mathcal E}
\newcommand{\F}{\mathcal F}
\newcommand{\G}{\mathcal G}
\newcommand{\Q}{\mathcal Q}
\newcommand{\R}{\mathcal R}
\newcommand{\cS}{\mathcal S}
\newcommand{\cU}{\mathcal U}
\newcommand{\W}{\mathcal W}
\newcommand{\bA}{\mathbb{A}}
\newcommand{\bB}{\mathbb{B}}
\newcommand{\bC}{\mathbb{C}}
\newcommand{\bD}{\mathbb{D}}
\newcommand{\bE}{\mathbb{E}}
\newcommand{\bF}{\mathbb{F}}
\newcommand{\bG}{\mathbb{G}}
\newcommand{\bK}{\mathbb{K}}
\newcommand{\bM}{\mathbb{M}}
\newcommand{\bN}{\mathbb{N}}
\newcommand{\bO}{\mathbb{O}}
\newcommand{\bP}{\mathbb{P}}
\newcommand{\bR}{\mathbb{R}}
\newcommand{\bV}{\mathbb{V}}
\newcommand{\bZ}{\mathbb{Z}}
\newcommand{\bfE}{\mathbf{E}}
\newcommand{\bfX}{\mathbf{X}}
\newcommand{\bfY}{\mathbf{Y}}
\newcommand{\bfZ}{\mathbf{Z}}
\renewcommand{\O}{\Omega}
\renewcommand{\o}{\omega}
\newcommand{\vp}{\varphi}
\newcommand{\vep}{\varepsilon}
\newcommand{\diag}{{\rm diag}}
\newcommand{\grp}{{\mathbb G}}
\newcommand{\dgrp}{{\mathbb D}}
\newcommand{\desp}{{\mathbb D^{\rm{es}}}}
\newcommand{\Geod}{{\rm Geod}}
\newcommand{\geod}{{\rm geod}}
\newcommand{\hgr}{{\mathbb H}}
\newcommand{\mgr}{{\mathbb M}}
\newcommand{\ob}{{\rm Ob}}
\newcommand{\obg}{{\rm Ob(\mathbb G)}}
\newcommand{\obgp}{{\rm Ob(\mathbb G')}}
\newcommand{\obh}{{\rm Ob(\mathbb H)}}
\newcommand{\Osmooth}{{\Omega^{\infty}(X,*)}}
\newcommand{\ghomotop}{{\rho_2^{\square}}}
\newcommand{\gcalp}{{\mathbb G(\mathcal P)}}
\newcommand{\rf}{{R_{\mathcal F}}}
\newcommand{\glob}{{\rm glob}}
\newcommand{\loc}{{\rm loc}}
\newcommand{\TOP}{{\rm TOP}}
\newcommand{\wti}{\widetilde}
\newcommand{\what}{\widehat}
\renewcommand{\a}{\alpha}
\newcommand{\be}{\beta}
\newcommand{\ga}{\gamma}
\newcommand{\Ga}{\Gamma}
\newcommand{\de}{\delta}
\newcommand{\del}{\partial}
\newcommand{\ka}{\kappa}
\newcommand{\si}{\sigma}
\newcommand{\ta}{\tau}
\newcommand{\lra}{{\longrightarrow}}
\newcommand{\ra}{{\rightarrow}}
\newcommand{\rat}{{\rightarrowtail}}
\newcommand{\oset}[1]{\overset {#1}{\ra}}
\newcommand{\osetl}[1]{\overset {#1}{\lra}}
\newcommand{\hr}{{\hookrightarrow}}

\begin{document}
\section{Two New Theorems for Grothendieck categories}

\subsection{Introduction}

The theory has its origin in the work of Grothendieck~\cite{Gro} who
introduced the following notation of properties of abelian
categories:

\begin{itemize}
\item[Ab3.] An abelian category with coproducts or equivalently, a
            cocomplete abelian category.
\item[Ab5.] Ab3-category, in which for any directed family
            $\{A_i\}_{i\in I}$ of subobjects of an arbitrary object
            $X$ and for any subobject $B$ of $X$ the following relation holds:
            $$(\sum_{i\in I}A_i)\cap B=\sum_{i\in I}(A_i\cap B)$$
\end{itemize}

Ab5-categories possessing a family of generators are called {\em
Grothendieck categories}. They constitute a natural extension of the
class of module categories, with which they share a great number of
important properties.

The {\bf Popescu-Gabriel Theorem} is generalized as follows.

{\bf Theorem}[Popescu and Gabriel]
{\em Let ${\bf G}$ be a Grothendieck category with a family of generators
$\left\{U_i\right\}_{i \in I}$ and $T=(-,?): {\bf G} \to mcAb$ be the representation functor that takes each $X \in {\bf G}$ to $(-,X)$, where
$mcAb =\{h_{U_i}=(-,U_i)\}_{i\in I}$. 
Then:

(1.) $T$ is full and faithful.

(2.) $T$ induces an equivalence between ${\bf G}$ and the quotient category
$mcAb  {\bf S}$, where ${\bf S}$ denotes the largest localizing subcategory
in $mcAb$ for which all modules $TX=(-,X)$ are ${\bf S}$-closed.}
%\end{theorem}

This extension of the {\bf Popescu-Gabriel Theorem} is due to {\em Grigory Garkusha} from the Saint-Petersburg State University, Higher Algebra and Number Theory Department, School of Mathematics and Mechanics, Bibliotechnaya Sq. 2, 198904 (Russia).

The advantage of this Theorem is that we can freely choose a family of
generators {\em U} of ${\bf G}$. To be precise, if {\em M} is an arbitrary
family of objects of ${\bf G}$, then the family: 
$\left\{U_i\right\}_{i \in I}= U \cup M$ is also a family of generators.

We say that an object $C$ of ${\bf G}$ is {\em U-finitely generated}
(or respectively{\em U-finitely presented}) if there is an epimorphism
$\eta:\psi_{i=1}^n U_i \to C$ (if there is an exact sequence
$\psi_{i=1}^nU_i\to\psi_{j=1}^mU_i\to C$) where $U_i \in U$. The full
subcategory of $U$-finitely generated ($U$-finitely presented)
objects of ${\bf G}$  is denoted by ${\bf fg}_U {\bf G}$ (${\bf fp}_U {\bf G}$).
When every $U_i \in U$ is finitely generated (finitely presented), that is
the functor $(U_i,-)$ preserves direct unions (limits), we write
${\bf fg}_{\bf G} = {\bf fg} \in {\bf G}$ (${\bf fp}_{U({\bf G})} ={\bf fp} \in {\bf G}$). Then every Grothendieck category is locally $U$-finitely generated
(locally $U$-finitely presented) which means that every object $C$ of ${\bf G}$
is a direct union (limit) $$C= \sum_{i\in I}C_i$$ , 

\bigbreak
($C={\bf lp}_{i\in I}C_i$)

 
of $U$-finitely generated ($U$-finitely presented) objects $C_i$.

Recall also that a localizing subcategory ${\bf S}$ of ${\bf G}$ is of prefinite
(finite) type provided that the inclusion functor ${\bf J}: {\bf S} \to {\bf G}$
commutes with direct unions (limits). So the following proposition holds.


%\begin
{\bf Theorem}[Breitsprecher]
Let ${\bf G}$ be a Grothendieck category with a family of generators
$U=\{U_i\}_{i\in I}$. Then the representation functor

$T=(-,?): {\bf G} \to {\bf fp}_U{\bf G}^({\rm op},{\rm Ab})$

defines an equivalence between ${\bf G}$ and $({\bf fp}_U {\bf G})^{{\rm op}},{\rm Ab})/{\bf S}$,
where ${\bf S}$ is some localizing subcategory of 
${{\bf fp}_U {\bf G}}^{({\rm op},{\rm Ab})}$.

Moreover, ${\bf S}$
is of finite type if and only if ${\bf fp}_U {\bf G}= {\bf fp}{\bf G}$. 
In this case, ${\bf G}$ is equivalent to the category

${\bf Lex}(({\bf fp}_U {\bf G})^{({\rm op},{\rm Ab})})$ of contravariant
left exact functors from ${\bf fp}_U {\bf G}$ to ${\rm Ab}$.
%\end{thmm}



\begin{thebibliography}{99}
\bibitem[{Aus1}]{Au} {\sc M.~Auslander}, `Coherent functors', in {\em Proc. Conf. on
                  Categorical Algebra\/} (La Jolla, 1965), Springer, 1966, 189-231.
\bibitem[{Aus2}]{Au1} {\sc M.~Auslander}, `A functorial approach to
                  representation theory', {\em Lect. Notes Math.} 944 (1982), 105-179.
\bibitem[Aus3]{Au2} {\sc M.~Auslander}, `Isolated singularities
                  and almost split sequences', {\em Lect. Notes Math.} 1178 (1986), 194-242.
\bibitem[Br]{Br} {\sc S.~Breitsprecher}, `Lokal endlich pr\"asentierbare
                  Grothendieck-Kategorien', {\em Mitt. Math. Sem. Giessen\/} 85 (1970), 1-25.
\bibitem[BD]{BD} {\sc I.~Bucur, A.~Deleanu}, {\em Introduction to the theory
                  of categories and functors,} Wiley, London, 1968.
\bibitem[Bur]{Bur} {\sc K.~Burke}, `Some Model-Theoretic Properties of Functor
                  Categories for Modules', {\em Doctoral Thesis, University of
                  Manchester,} 1994.
\bibitem[Fa]{Fa} {\sc C.~Faith}, {\em Algebra: rings, modules and categories,} vol. 1,
                  Mir, Moscow, 1977 (in Russian).
\bibitem[Fr]{Fr} {\sc P.~Freyd}, {\em Abelian categories,} Harper and Row,
                  New-York, 1964.
\bibitem[Gbl]{Ga} {\sc P.~Gabriel}, `Des cat\'egories ab\'eliennes',
                  {\em Bull. Soc. Math. France\/} 90 (1962), 323-448.
\bibitem[GG1]{GG} 
{\sc G. A. Garkusha, A. I. Generalov}, ``Grothendieck categories as quotient categories of $R-mod,Ab$'' 
{\em Fund. Prikl. Mat.,} to appear (in Russian).
\bibitem[GG2]
{GG2} {\sc G.~A.~Garkusha, A.~I.~Generalov}, `Duality for categories of finitely presented modules', {\em Algebra i Analiz} to appear (in Russian).
\bibitem[Grk]{Gro} {\sc A.~Grothendieck}, `Sur quelques points d'alg\`ebre homologique', {\em Tohoku Math. J.} 9 (1957), 119-221.
\bibitem[GJ]
{GJ}{\sc L.~Gruson, C.U.~Jensen}, `Dimensions cohomologiques            reli\'ees aux foncteurs ${lo}^{(i)}$', {\em Lect. Notes Math.} 867 (1981), 234-294.
\bibitem[Hrz]
{He} {\sc I.~Herzog}, `The Ziegler spectrum of a locally coherent Grothendieck category', {\em Proc. London Math. Soc.} 74 (1997), 503-558.
\bibitem[JL]{JL}  
{\sc C.U.~Jensen, M.~Lenzing}, {\em Model theoretic algebra,}
Logic and its Applications 2, Gordon and Breach, New York, 1989.

\bibitem[Kap]{Ka} 
{\sc I.~Kaplansky}, {\em Infinite abelian groups,} Ann Arbor, University of Michigan Press, 1969.

\bibitem[Kr1]{Kr1} {\sc H.~Krause}, `The spectrum of a locally coherent category', {\em J. Pure Appl. Algebra\/} 114 (1997), 259-271.

\bibitem[Kr2]
{Kr2} {\sc H.~Krause}, `The spectrum of a module category',
{\em Habilitationsschrift, Universit\"at Bielefeld,} 1998.

\bibitem[Kr3]{Kr3} 
{\sc H.~Krause}, `Functors on locally finitely presented additive categories', {\em Colloq. Math.} 75 (1998), 105-132.
\bibitem[Laz]{La} {\sc D.~Lazard}, `Autour de la platitude', {\em Bull. Soc.
                  Math. France\/} 97 (1969), 81-128.
\bibitem[PG]{PG} {\sc N.~Popescu, P.~Gabriel}, `Caract\'erisation des cat\'egories
                  ab\'eliennes avec g\'en\'erateurs et limites inductives
                  exactes', {\em C.~R.~Acad. Sc. Paris\/} 258 (1964), 4188-4190.
\bibitem[{Pop}]{Pop} {\sc N.~Popescu}, {\em Abelian categories with applications
                  to rings and modules,} Academic Press, London and New-York, 1973.
\bibitem[{Pr1}]{Prr} {\sc M.~Prest}, `Elementary torsion theories and locally
                  finitely presented Abelian categories', {\em J. Pure
                  Appl. Algebra\/} 18 (1980), 205-212.
\bibitem[{Pr2}]{Pr} {\sc M.~Prest}, `Epimorphisms of rings, interpretations of
                  modules and strictly wild algebras', {\em Comm. Algebra\/}
                  24 (1996), 517-531.
\bibitem[{Pr3}]{Pr2} {\sc M.~Prest}, `The Zariski spectrum of the category of
                  finitely presented modules', {\em preprint, University of
                  Manchester,} 1998.
\bibitem[PRZ]{PRZ} {\sc M.~Prest, Ph.~Rothmaler, M.~Ziegler}, `Absolutely pure
                  and flat modules and ``indiscrete'' rings', {\em J.~Algebra\/}
                  174 (1995), 349-372.
\bibitem[{Rs}]{Ro} {\sc J.-E.~Roos}, `Locally noetherian categories',
                  {\em Lect. Notes Math.} 92 (1969), 197-277.
\bibitem[{Stm}]{St} {\sc B.~Stenstr\"om}, {\em Rings of quotients,} Springer-Verlag,
                  New York and Heidelberg, 1975.
\bibitem[{Zgr}]{Zi} {\sc M.~Ziegler}, `Model theory of modules',
                  {\em Ann. Pure Appl. Logic\/} 26 (1984), 149-213.
\end{thebibliography}


{\bf more to come...}

%%%%%
%%%%%
\end{document}
