\documentclass[12pt]{article}
\usepackage{pmmeta}
\pmcanonicalname{CategoryOfMatrices}
\pmcreated{2013-03-22 16:31:54}
\pmmodified{2013-03-22 16:31:54}
\pmowner{rspuzio}{6075}
\pmmodifier{rspuzio}{6075}
\pmtitle{category of matrices}
\pmrecord{9}{38712}
\pmprivacy{1}
\pmauthor{rspuzio}{6075}
\pmtype{Example}
\pmcomment{trigger rebuild}
\pmclassification{msc}{18A05}

% this is the default PlanetMath preamble.  as your knowledge
% of TeX increases, you will probably want to edit this, but
% it should be fine as is for beginners.

% almost certainly you want these
\usepackage{amssymb}
\usepackage{amsmath}
\usepackage{amsfonts}

% used for TeXing text within eps files
%\usepackage{psfrag}
% need this for including graphics (\includegraphics)
%\usepackage{graphicx}
% for neatly defining theorems and propositions
%\usepackage{amsthm}
% making logically defined graphics
%%%\usepackage{xypic}

% there are many more packages, add them here as you need them

% define commands here

\begin{document}
The set of all matrices (rectangular as well as square) 
over a given field (such as real or complex numbers; more
generally, one could consider matrices over a unital ring
such as integers or a suitable ring of polynomials) 
forms a category.  As objects of this category, we take the 
positive integers.  For any two integers $m$ and $n$, we 
take $\operatorname{Hom}(m,n)$ to be the set of $m \times n$
matrices.  The composition of morphisms is taken to be
matrix multiplication.  It is easy to see that the defining
properties of categories are satisfied:

\begin{enumerate}
\item The intersection of the set of $a \times b$ matrices and
the set of $c \times d$ matrices is empty unless $a = c$ and 
$b = d$.
\item Matrix multiplication is associative.
\item For every $n$, there is a special element of 
$\operatorname{Hom}(n,n)$, the $n \times n$ identity matrix,
which satisfies the properties of an identity morphism.
\end{enumerate}

This example illustrates an important \PMlinkescapetext{point} about 
the notion of category.  As with groups and semigroups, categories are
an algebraic structure defined by a single associative operation.
Where they differ is that the closure property no longer holds
--- given two morphisms in a category, it is not necessarily 
the case that they can be composed.  The reason for introducing
objects is to keep track of when it is possible to compose
the morphisms.
%%%%%
%%%%%
\end{document}
