\documentclass[12pt]{article}
\usepackage{pmmeta}
\pmcanonicalname{TypesOfMorphisms}
\pmcreated{2013-03-22 16:03:39}
\pmmodified{2013-03-22 16:03:39}
\pmowner{kompik}{10588}
\pmmodifier{kompik}{10588}
\pmtitle{types of morphisms}
\pmrecord{16}{38114}
\pmprivacy{1}
\pmauthor{kompik}{10588}
\pmtype{Definition}
\pmcomment{trigger rebuild}
\pmclassification{msc}{18A05}
\pmrelated{TypesOfHomomorphisms}
\pmrelated{SectionsAndRetractions}
\pmdefines{monomorphism}
\pmdefines{epimorphism}
\pmdefines{bimorphism}
\pmdefines{retraction}
\pmdefines{section}
\pmdefines{coretraction}
\pmdefines{isomorphism}
\pmdefines{inverse morphism}
\pmdefines{split monomorphism}
\pmdefines{split epimorphism}
\pmdefines{epimorphic extension}
\pmdefines{epimorphic monomorphism}

\endmetadata

% this is the default PlanetMath preamble. as your knowledge
% of TeX increases, you will probably want to edit this, but
% it should be fine as is for beginners.

% almost certainly you want these
\usepackage{amssymb}
\usepackage{amsmath}
\usepackage{amsfonts}
\usepackage{amsthm}

% used for TeXing text within eps files
%\usepackage{psfrag}
% need this for including graphics (\includegraphics)
%\usepackage{graphicx}
% for neatly defining theorems and propositions
%
% making logically defined graphics
%%\usepackage{xypic}

% there are many more packages, add them here as you need them

% define commands here

\newcommand{\sR}[0]{\mathbb{R}}
\newcommand{\sC}[0]{\mathbb{C}}
\newcommand{\sN}[0]{\mathbb{N}}
\newcommand{\sZ}[0]{\mathbb{Z}}

\newcommand{\R}[0]{\mathbb{R}}
\newcommand{\C}[0]{\mathbb{C}}
\newcommand{\N}[0]{\mathbb{N}}
\newcommand{\Z}[0]{\mathbb{Z}}


%\usepackage{bbm}
%\newcommand{\N}{\mathbbmss{N}}
%\newcommand{\Z}{\mathbbmss{Z}}
%\newcommand{\C}{\mathbbmss{C}}
%\newcommand{\R}{\mathbbmss{R}}
%\newcommand{\Q}{\mathbbmss{Q}}



\newcommand*{\norm}[1]{\lVert #1 \rVert}
\newcommand*{\abs}[1]{| #1 |}

\newcommand{\Map}[3]{#1:#2\to#3}
\newcommand{\Emb}[3]{#1:#2\hookrightarrow#3}
\newcommand{\Mor}[3]{#2\overset{#1}\to#3}

\newcommand{\Cat}[1]{\mathcal{#1}}
\newcommand{\Kat}[1]{\mathbf{#1}}
\newcommand{\Func}[3]{\Map{#1}{\Cat{#2}}{\Cat{#3}}}
\newcommand{\Funk}[3]{\Map{#1}{\Kat{#2}}{\Kat{#3}}}

\newcommand{\intrv}[2]{\langle #1,#2 \rangle}

\newcommand{\vp}{\varphi}
\newcommand{\ve}{\varepsilon}

\newcommand{\Invimg}[2]{\inv{#1}(#2)}
\newcommand{\Img}[2]{#1[#2]}
\newcommand{\ol}[1]{\overline{#1}}
\newcommand{\ul}[1]{\underline{#1}}
\newcommand{\inv}[1]{#1^{-1}}
\newcommand{\limti}[1]{\lim\limits_{#1\to\infty}}

\newcommand{\Ra}{\Rightarrow}

%fonts
\newcommand{\mc}{\mathcal}

%shortcuts
\newcommand{\Ob}{\mathrm{Ob}}
\newcommand{\Hom}{\mathrm{hom}}
\newcommand{\homs}[2]{\mathrm{hom(}{#1},{#2}\mathrm )}
\newcommand{\Eq}{\mathrm{Eq}}
\newcommand{\Coeq}{\mathrm{Coeq}}

%theorems
\newtheorem{THM}{Theorem}
\newtheorem{DEF}{Definition}
\newtheorem{PROP}{Proposition}
\newtheorem{LM}{Lemma}
\newtheorem{COR}{Corollary}
\newtheorem{EXA}{Example}

%categories
\newcommand{\Top}{\Kat{Top}}
\newcommand{\Haus}{\Kat{Haus}}
\newcommand{\Set}{\Kat{Set}}

%diagrams
\newcommand{\UnimorCD}[6]{
\xymatrix{ {#1} \ar[r]^{#2} \ar[rd]_{#4}& {#3} \ar@{-->}[d]^{#5} \\
& {#6} } }

\newcommand{\RovnostrCD}[6]{
\xymatrix@C=10pt@R=17pt{
& {#1} \ar[ld]_{#2} \ar[rd]^{#3} \\
{#4} \ar[rr]_{#5} && {#6} } }

\newcommand{\RovnostrCDii}[6]{
\xymatrix@C=10pt@R=17pt{
{#1} \ar[rr]^{#2} \ar[rd]_{#4}&& {#3} \ar[ld]^{#5} \\
& {#6} } }

\newcommand{\RovnostrCDiiop}[6]{
\xymatrix@C=10pt@R=17pt{
{#1}  && {#3} \ar[ll]_{#2}  \\
& {#6} \ar[lu]^{#4} \ar[ru]_{#5} } }

\newcommand{\StvorecCD}[8]{
\xymatrix{
{#1} \ar[r]^{#2} \ar[d]_{#4} & {#3} \ar[d]^{#5} \\
{#6} \ar[r]_{#7} & {#8}
}
}

\newcommand{\TriangCD}[6]{
\xymatrix{ {#1} \ar[r]^{#2} \ar[rd]_{#4}&
{#3} \ar[d]^{#5} \\
& {#6} } }
\begin{document}
\begin{DEF}
A morphism $\Map fAB$ is a \emph{monomorphism}, if for any two morphisms $\Map{g,h}CA$ the
equality $f\circ g=f\circ h$ implies $h=g$.

\PMlinkname{Dual}{DualityPrinciple} notion: Morphism $\Map fAB$ is an \emph{epimorphism}, if
for any two morphisms $\Map{g,h}BC$ the equality $g\circ f=h\circ f$ implies $h=g$.

A morphism $f$ is a \emph{bimorphism}, if it is monomorphism and epimorphism at the same
time. Also the names \emph{epimorphic extension} and \emph{epimorphic monomorphism} are used.
\end{DEF}

\begin{DEF}
A morphism $\Map fAB$ is called \emph{retraction} if there exists a morphism $\Map gBA$ such
that $f\circ g=id_B$.

Retractions are sometimes called \emph{split epimorphisms}.

Dual notion: a morphism $\Map fAB$ is a \emph{section} (or \emph{coretraction} or \emph{split monomorphism}) if there exists a morphism $\Map gBA$ such that $g\circ f=id_A$.

A morphism $\Map fAB$ is an \emph{\PMlinkname{isomorphism}{Isomorphism2}} if it is a
retraction and section at the same time.
\end{DEF}

Bimorphism and isomorphism are examples of self-dual properties.
The condition that $f$ is isomorphism is equivalent to the existence of a morphism $g$ with
$f\circ g=id_B$ and $g\circ f=id_A$ (for the proof see properties of monomorphisms and
epimorphisms).

\begin{DEF} If $f$ is an isomorphism then the morphism $\Map gBA$ such that $f\circ g=id_B$ and $g\circ f=id_A$ 
is called \emph{inverse} morphism of $f$ and denoted by $\inv f$.
\end{DEF}

\begin{DEF}
If there exists an isomorphism $\Map fAB$ we say that the objects $A$ and $B$ are
\emph{isomorphic}, denoted by $A\cong B$.
\end{DEF}

%%%%%
%%%%%
\end{document}
