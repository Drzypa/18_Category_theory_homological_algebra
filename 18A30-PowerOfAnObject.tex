\documentclass[12pt]{article}
\usepackage{pmmeta}
\pmcanonicalname{PowerOfAnObject}
\pmcreated{2013-03-22 18:31:32}
\pmmodified{2013-03-22 18:31:32}
\pmowner{CWoo}{3771}
\pmmodifier{CWoo}{3771}
\pmtitle{power of an object}
\pmrecord{5}{41223}
\pmprivacy{1}
\pmauthor{CWoo}{3771}
\pmtype{Definition}
\pmcomment{trigger rebuild}
\pmclassification{msc}{18A30}
\pmdefines{power}
\pmdefines{copower}

\usepackage{amssymb,amscd}
\usepackage{amsmath}
\usepackage{amsfonts}
\usepackage{mathrsfs}

% used for TeXing text within eps files
%\usepackage{psfrag}
% need this for including graphics (\includegraphics)
%\usepackage{graphicx}
% for neatly defining theorems and propositions
\usepackage{amsthm}
% making logically defined graphics
\usepackage[all]{xypic}
\usepackage{pst-plot}

% define commands here
\newcommand*{\abs}[1]{\left\lvert #1\right\rvert}
\newtheorem{prop}{Proposition}
\newtheorem{thm}{Theorem}
\newtheorem{ex}{Example}
\newcommand{\real}{\mathbb{R}}
\newcommand{\pdiff}[2]{\frac{\partial #1}{\partial #2}}
\newcommand{\mpdiff}[3]{\frac{\partial^#1 #2}{\partial #3^#1}}
\begin{document}
Let $\mathcal{C}$ be a category and $A$ an object in $\mathcal{C}$.  Suppose $n$ is a non-negative integer.  The \emph{$n$-th power} of $A$ is defined as the direct product of $A$ with itself $n$ times.  In other words, the \emph{$n$-th power} of $A$ is an object $P$ in $\mathcal{C}$, together with $n$ parallel morphisms $\pi_1, \ldots, \pi_n \in \hom(P,A)$, such that if there are $n$ parallel morphisms $p_1, \ldots, p_n \in \hom(B,A)$, then there is a unique morphism $h:B\to P$ such that $\pi_i\circ h=p_i$, where $i=1,\ldots, n$.  The commutative diagram below illustrates the situation:

$$\xymatrix@C=0.5cm{& B \ar@{.>}[d]^{h} \ar@/_4ex/[dddl]_{p_1} \ar@/^4ex/[dddr]^{p_n} & \\ & P \ar[ddl]_{\pi_1}="1" \ar[ddr]^{\pi_n}="2" & \\ & & \\ A & \cdots & A \ar@{}"1";"2"|-{\cdots}}$$

The $n$-th power of $A$ is denoted by $A^n$.

Below are some properties of the power of an object in a category:
\begin{itemize}
\item Each of the projection morphisms $\pi_i$ is a split epimorphism.
\item $A^1 \cong A$.
\item $A^0$ is a terminal object in $\mathcal{C}$.
\item $A^{m+n}\cong A^n \times A^m$, if the product exists.
\end{itemize}

For example, in the category of sets, the $n$-th power of a set $A$ is the set of $n$-tuples where each entry is an element of $A$.

\textbf{Remark}.  The \emph{copower} of an object is defined dually.  All of the properties above can be dualized.  For example, the $0$-th copower of an object is an initial object.  The $n$-th copower of an object $A$ in \textbf{Set} is the disjoint union of $n$-copies of $A$.
%%%%%
%%%%%
\end{document}
