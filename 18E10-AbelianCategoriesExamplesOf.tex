\documentclass[12pt]{article}
\usepackage{pmmeta}
\pmcanonicalname{AbelianCategoriesExamplesOf}
\pmcreated{2013-03-22 16:42:52}
\pmmodified{2013-03-22 16:42:52}
\pmowner{mps}{409}
\pmmodifier{mps}{409}
\pmtitle{abelian categories, examples of}
\pmrecord{28}{38931}
\pmprivacy{1}
\pmauthor{mps}{409}
\pmtype{Example}
\pmcomment{trigger rebuild}
\pmclassification{msc}{18E10}
%\pmkeywords{Abelian groups}
%\pmkeywords{Grothendieck categories}
%\pmkeywords{commutative groupoids}
%\pmkeywords{reversible automata}
%\pmkeywords{quantum automata}
%\pmkeywords{rings}
%\pmkeywords{modules}
\pmrelated{GrothendieckCategory}
\pmrelated{NonAbelianStructures}
\pmrelated{QuantumAutomataAndQuantumComputation2}
\pmrelated{AbelianCategory}
\pmrelated{AxiomsForAnAbelianCategory}
\pmrelated{GeneralizedVanKampenTheoremsHigherDimensional}
\pmrelated{AxiomaticTheoryOfSupercategories}
\pmrelated{AlgebraicCategoryOfLMnLogicAlgebras}
\pmrelated{Categoric}

\endmetadata

% this is the default PlanetMath preamble.  as your knowledge
% of TeX increases, you will probably want to edit this, but
% it should be fine as is for beginners.

% almost certainly you want these
\usepackage{amssymb}
\usepackage{amsmath}
\usepackage{amsfonts}

% used for TeXing text within eps files
%\usepackage{psfrag}
% need this for including graphics (\includegraphics)
%\usepackage{graphicx}
% for neatly defining theorems and propositions
\usepackage{amsthm}
% making logically defined graphics
%%\usepackage{xypic}
\xyoption{all}

% there are many more packages, add them here as you need them

% define commands here
\newcommand{\Ab}{\mathbf{Ab}}
\DeclareMathOperator{\Hom}{Hom}
\DeclareMathOperator{\coker}{coker}
\newcommand{\Mod}[1]{{}_{#1}\mathbf{Mod}}
\newcommand{\rMod}[1]{\mathbf{Mod}_{#1}}

\theoremstyle{example}
\newtheorem{example}{Example}
\begin{document}
\PMlinkescapeword{properties}
\PMlinkescapeword{structure}
\PMlinkescapeword{sum}
\PMlinkescapeword{satisfies}
\PMlinkescapeword{similar}
\PMlinkescapeword{terminal}

The axiomatization of abelian categories was intended to capture some of the useful properties of categories in homological algebra.  This entry gives some examples of abelian categories.

\begin{example}
The category $\Ab$ of abelian groups is an abelian category. [\PMlinkname{Proof}{ProofThatAbelianGroupsFormAnAbelianCategory}]
\end{example}

Since abelian groups are a special case of $R$-modules, one might ask whether categories of $R$-modules are also abelian categories.  This is the case.

\begin{example}
For any ring $R$, the category $\Mod{R}$ of left $R$-modules is an abelian category.
\end{example}

\begin{example}
For any ring $R$, the category $\mathcal{C}(R)$ of \PMlinkname{complexes}{ChainComplex} of left $R$-modules is an abelian category.
\end{example}

\begin{example}
For any topological space $X$, the category of sheaves of abelian groups over $X$ is an abelian category.
\end{example}

\begin{example}
Every Grothendieck category that satisfies the $\mathcal{A}b6$ axiom is Abelian.
\end{example}

\begin{example}
For any topological groupoid $\mathcal{G}$, the 2--category of sheaves of commutative groupoids over the groupoid space $X_G$ is an abelian 2-category.
\end{example}

\begin{example} 
The 2--category of commutative groupoids $\mathcal{G_C}$ is an Abelian 2--category. 
\end{example}

\begin{example}
For any reversible sequential machine, or automaton, $R_S$, (with all state transitions reversible), the category $\mathcal{A}(R_S)$ of such reversible automata is an Abelian category.
\end{example}

\textbf{Counter-example}
For general quantum automata $Q_A$s, the category $\mathcal{Q_A}$ of such quantum automata is in general, non-Abelian
(or nonabelian).

%%%%%
%%%%%
\end{document}
