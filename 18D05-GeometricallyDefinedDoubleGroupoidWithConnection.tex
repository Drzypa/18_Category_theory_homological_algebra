\documentclass[12pt]{article}
\usepackage{pmmeta}
\pmcanonicalname{GeometricallyDefinedDoubleGroupoidWithConnection}
\pmcreated{2013-03-22 18:14:45}
\pmmodified{2013-03-22 18:14:45}
\pmowner{bci1}{20947}
\pmmodifier{bci1}{20947}
\pmtitle{geometrically defined double groupoid with connection}
\pmrecord{33}{40839}
\pmprivacy{1}
\pmauthor{bci1}{20947}
\pmtype{Topic}
\pmcomment{trigger rebuild}
\pmclassification{msc}{18D05}
\pmclassification{msc}{55N33}
\pmclassification{msc}{55N20}
\pmclassification{msc}{55U40}
\pmsynonym{PWL map}{GeometricallyDefinedDoubleGroupoidWithConnection}
%\pmkeywords{geometrically and algebraically thin squares}
%\pmkeywords{geometrically defined fouble groupoid with connection}
%\pmkeywords{higher dimensional generalized VanKampen theorems}
\pmrelated{HomotopyGroupoidsAndCrossComplexesAsNonCommutativeStructuresInHigherDimensionalAlgebraHDA}
\pmrelated{ThinDoubleTracks}
\pmdefines{PWL map of simplicial complexes}
\pmdefines{piecewise linear map of simplicial complexes}

\endmetadata

% this is the default PlanetMath preamble.  as your knowledge
% of TeX increases, you will probably want to edit this, but
% it should be fine as is for beginners.

% almost certainly you want these
\usepackage{amssymb}
\usepackage{amsmath}
\usepackage{amsfonts}

% used for TeXing text within eps files
%\usepackage{psfrag}
% need this for including graphics (\includegraphics)
%\usepackage{graphicx}
% for neatly defining theorems and propositions
%\usepackage{amsthm}
% making logically defined graphics
%%%\usepackage{xypic}

% there are many more packages, add them here as you need them

% define commands here
\usepackage{amsmath, amssymb, amsfonts, amsthm, amscd, latexsym}
%%\usepackage{xypic}
\usepackage[mathscr]{eucal}

\setlength{\textwidth}{6.5in}
%\setlength{\textwidth}{16cm}
\setlength{\textheight}{9.0in}
%\setlength{\textheight}{24cm}

\hoffset=-.75in %%ps format
%\hoffset=-1.0in %%hp format
\voffset=-.4in

\theoremstyle{plain}
\newtheorem{lemma}{Lemma}[section]
\newtheorem{proposition}{Proposition}[section]
\newtheorem{theorem}{Theorem}[section]
\newtheorem{corollary}{Corollary}[section]

\theoremstyle{definition}
\newtheorem{definition}{Definition}[section]
\newtheorem{example}{Example}[section]
%\theoremstyle{remark}
\newtheorem{remark}{Remark}[section]
\newtheorem*{notation}{Notation}
\newtheorem*{claim}{Claim}

\renewcommand{\thefootnote}{\ensuremath{\fnsymbol{footnote%%@
}}}
\numberwithin{equation}{section}

\newcommand{\Ad}{{\rm Ad}}
\newcommand{\Aut}{{\rm Aut}}
\newcommand{\Cl}{{\rm Cl}}
\newcommand{\Co}{{\rm Co}}
\newcommand{\DES}{{\rm DES}}
\newcommand{\Diff}{{\rm Diff}}
\newcommand{\Dom}{{\rm Dom}}
\newcommand{\Hol}{{\rm Hol}}
\newcommand{\Mon}{{\rm Mon}}
\newcommand{\Hom}{{\rm Hom}}
\newcommand{\Ker}{{\rm Ker}}
\newcommand{\Ind}{{\rm Ind}}
\newcommand{\IM}{{\rm Im}}
\newcommand{\Is}{{\rm Is}}
\newcommand{\ID}{{\rm id}}
\newcommand{\GL}{{\rm GL}}
\newcommand{\Iso}{{\rm Iso}}
\newcommand{\Sem}{{\rm Sem}}
\newcommand{\St}{{\rm St}}
\newcommand{\Sym}{{\rm Sym}}
\newcommand{\SU}{{\rm SU}}
\newcommand{\Tor}{{\rm Tor}}
\newcommand{\U}{{\rm U}}

\newcommand{\A}{\mathcal A}
\newcommand{\Ce}{\mathcal C}
\newcommand{\D}{\mathcal D}
\newcommand{\E}{\mathcal E}
\newcommand{\F}{\mathcal F}
\newcommand{\G}{\mathcal G}
\newcommand{\Q}{\mathcal Q}
\newcommand{\R}{\mathcal R}
\newcommand{\cS}{\mathcal S}
\newcommand{\cU}{\mathcal U}
\newcommand{\W}{\mathcal W}

\newcommand{\bA}{\mathbb{A}}
\newcommand{\bB}{\mathbb{B}}
\newcommand{\bC}{\mathbb{C}}
\newcommand{\bD}{\mathbb{D}}
\newcommand{\bE}{\mathbb{E}}
\newcommand{\bF}{\mathbb{F}}
\newcommand{\bG}{\mathbb{G}}
\newcommand{\bK}{\mathbb{K}}
\newcommand{\bM}{\mathbb{M}}
\newcommand{\bN}{\mathbb{N}}
\newcommand{\bO}{\mathbb{O}}
\newcommand{\bP}{\mathbb{P}}
\newcommand{\bR}{\mathbb{R}}
\newcommand{\bV}{\mathbb{V}}
\newcommand{\bZ}{\mathbb{Z}}

\newcommand{\bfE}{\mathbf{E}}
\newcommand{\bfX}{\mathbf{X}}
\newcommand{\bfY}{\mathbf{Y}}
\newcommand{\bfZ}{\mathbf{Z}}

\renewcommand{\O}{\Omega}
\renewcommand{\o}{\omega}
\newcommand{\vp}{\varphi}
\newcommand{\vep}{\varepsilon}

\newcommand{\diag}{{\rm diag}}
\newcommand{\grp}{{\mathbb G}}
\newcommand{\dgrp}{{\mathbb D}}
\newcommand{\desp}{{\mathbb D^{\rm{es}}}}
\newcommand{\Geod}{{\rm Geod}}
\newcommand{\geod}{{\rm geod}}
\newcommand{\hgr}{{\mathbb H}}
\newcommand{\mgr}{{\mathbb M}}
\newcommand{\ob}{{\rm Ob}}
\newcommand{\obg}{{\rm Ob(\mathbb G)}}
\newcommand{\obgp}{{\rm Ob(\mathbb G')}}
\newcommand{\obh}{{\rm Ob(\mathbb H)}}
\newcommand{\Osmooth}{{\Omega^{\infty}(X,*)}}
\newcommand{\ghomotop}{{\rho_2^{\square}}}
\newcommand{\gcalp}{{\mathbb G(\mathcal P)}}

\newcommand{\rf}{{R_{\mathcal F}}}
\newcommand{\glob}{{\rm glob}}
\newcommand{\loc}{{\rm loc}}
\newcommand{\TOP}{{\rm TOP}}

\newcommand{\wti}{\widetilde}
\newcommand{\what}{\widehat}

\renewcommand{\a}{\alpha}
\newcommand{\be}{\beta}
\newcommand{\ga}{\gamma}
\newcommand{\Ga}{\Gamma}
\newcommand{\de}{\delta}
\newcommand{\del}{\partial}
\newcommand{\ka}{\kappa}
\newcommand{\si}{\sigma}
\newcommand{\ta}{\tau}
\newcommand{\med}{\medbreak}
\newcommand{\medn}{\medbreak \noindent}
\newcommand{\bign}{\bigbreak \noindent}
\newcommand{\lra}{{\longrightarrow}}
\newcommand{\ra}{{\rightarrow}}
\newcommand{\rat}{{\rightarrowtail}}
\newcommand{\oset}[1]{\overset {#1}{\ra}}
\newcommand{\osetl}[1]{\overset {#1}{\lra}}
\newcommand{\hr}{{\hookrightarrow}}
\begin{document}
\subsection{Introduction}  
 In the setting of a geometrically defined double groupoid with connection, as in \cite{BH2}, (resp. \cite{BHKP}), there is an appropriate notion of \emph{geometrically thin} square.  It was proven in \cite{BH2},
(Theorem 5.2 (resp. \cite{BHKP}, Proposition 4)), that in the cases there specified 
\emph{geometrically and algebraically thin squares coincide}.

\subsection{Geometrically defined double groupoid with connection} 

\subsubsection{Basic definitions}

\begin{definition}
A map $ \Phi : |K| \longrightarrow |L| $ where $ K $ and $ L $ are
(finite) simplicial complexes is \emph{PWL} ({\it piecewise linear}) if
there exist subdivisions of $ K $ and $ L $ relative to which $ \Phi$ is simplicial. 
\end{definition}

\subsubsection{Remarks} 

 We briefly recall here the related concepts involved:
 \begin{definition}
A \emph{square} $ u:I^{2} \longrightarrow X $ in a topological space $ X $ is \emph{thin} if there 
is a factorisation of $ u $, $$ u : I^{2} \stackrel{\Phi_{u}}{\longrightarrow}
J_{u} \stackrel{p_{u}}{\longrightarrow} X, $$  where $J_{u}$ is a
\emph{tree} and $ \Phi_{u} $ is piecewise linear (PWL, as defined next) on the
boundary $ \partial{I}^{2} $ of $ I^{2} $. 
\end{definition}

\begin{definition}
A {\it tree}, is defined here as the underlying space $ |K| $ of a
finite $ 1 $-connected $ 1 $-dimensional simplicial complex $ K $ boundary 
$ \partial{I}^{2} $ of $ I^{2} $. 
\end{definition}

\begin{thebibliography}{9}
\bibitem{BR2k6}
Ronald Brown: Topology and Groupoids, BookSurge LLC (2006).

\bibitem{BH2}
Brown, R., and Hardy, J.P.L.:1976, Topological groupoids I:
universal constructions, \emph{Math. Nachr.}, 71: 273-286.

\bibitem{BHKP}
Brown, R., Hardie, K., Kamps, H. and T. Porter: 2002, The homotopy
double groupoid of a Hausdorff space., 
\emph{Theory and pplications of Categories} \textbf{10}, 71-93.

\bibitem{BRHS}
Ronald Brown R, P.J. Higgins, and R. Sivera.: {\em Non-Abelian algebraic topology},({\em in preparation}),(2008).
\PMlinkexternal{(available here as PDF)}{http://www.bangor.ac.uk/~mas010/nonab-t/partI010604.pdf}
, \PMlinkexternal{see also other available, relevant papers at this website}{http://www.bangor.ac.uk/~mas010/publicfull.htm}.

\bibitem{BL87a}
R. Brown and J.-L. Loday: Homotopical excision, and Hurewicz theorems, for $n$-cubes of spaces, 
\emph{Proc. London Math. Soc.}, 54:(3), 176-192,(1987).

\bibitem{BL87b}
R. Brown and J.-L. Loday: Van Kampen Theorems for diagrams of spaces, \emph{Topology}, 26: 311-337 (1987).

\bibitem{BM86}
R. Brown and G. H. Mosa: Double algebroids and crossed modules of algebroids, University of Wales-Bangor, Maths 
({\em Preprint}), 1986.

\bibitem{BS76}
R. Brown and C.B. Spencer: Double groupoids and crossed modules, {\em Cahiers Top. G\'eom. Diff.}, 17 (1976), 343-362.

\end{thebibliography}

%%%%%
%%%%%
\end{document}
