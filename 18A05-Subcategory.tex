\documentclass[12pt]{article}
\usepackage{pmmeta}
\pmcanonicalname{Subcategory}
\pmcreated{2013-03-22 14:21:58}
\pmmodified{2013-03-22 14:21:58}
\pmowner{CWoo}{3771}
\pmmodifier{CWoo}{3771}
\pmtitle{subcategory}
\pmrecord{13}{35850}
\pmprivacy{1}
\pmauthor{CWoo}{3771}
\pmtype{Definition}
\pmcomment{trigger rebuild}
\pmclassification{msc}{18A05}
\pmrelated{Category}
\pmrelated{Functor}
\pmrelated{FaithfulFunctor}
\pmdefines{full subcategory}
\pmdefines{inclusion functor}

\endmetadata

% this is the default PlanetMath preamble.  as your knowledge
% of TeX increases, you will probably want to edit this, but
% it should be fine as is for beginners.

% almost certainly you want these
\usepackage{amssymb,amscd}
\usepackage{amsmath}
\usepackage{amsfonts}

% used for TeXing text within eps files
%\usepackage{psfrag}
% need this for including graphics (\includegraphics)
%\usepackage{graphicx}
% for neatly defining theorems and propositions
%\usepackage{amsthm}
% making logically defined graphics
%%%\usepackage{xypic}

% there are many more packages, add them here as you need them

% define commands here
\begin{document}
\PMlinkescapeword{embedding}

Let $\mathcal{C}$ be a (small) category.  If $\mathcal{S}$ is a collection of both a subset, call it $\operatorname{Ob}(\mathcal{S})$, of objects of $\mathcal{C}$ and a subset, call it $\operatorname{Mor}(\mathcal{S})$, of morphisms of $\mathcal{C}$ such that
\begin{enumerate}
\item
For each $S\in\operatorname{Ob}(\mathcal{S})$, the identity morphism of $S$, $id_S\in\operatorname{Mor}(\mathcal{S});$
\item
For each $f\in\operatorname{Mor}(\mathcal{S})$, $\operatorname{domain}(f)$ and 
$\operatorname{codomain}(f)\in\operatorname{Ob}(\mathcal{S});$ 
\item
For every pair $f,g\in\operatorname{Mor}(\mathcal{S})$ such that $f\circ g$ exists, then $f\circ g\in\operatorname{Mor}(\mathcal{S}).$ 
\end{enumerate}
Then $\mathcal{S}$ is readily seen to be a category.  It is called a \emph{subcategory} of the category $\mathcal{C}.$

Given a category $\mathcal{C}$ and a subcategory $\mathcal{S}$ of $\mathcal{C}$, a map $$\operatorname{Incl}:\mathcal{S}\hookrightarrow \mathcal{C}$$ that sends each object of $\mathcal{S}$ to itself (in $\mathcal{C}$), and each morphism of $\mathcal{S}$ to itself (in $\mathcal{C}$), is a functor.  $\operatorname{Incl}$ is called the \emph{inclusion functor}, or an \emph{embedding}.  This inclusion functor is a faithful functor.  If it is also \PMlinkname{full}{fullfunctor}, then we call the corresponding subcategory  $\mathcal{S}$ a \emph{full subcategory} of $\mathcal{C}$.  In other words, if $\mathcal{S}$ is a full subcategory of $\mathcal{C}$, then $$\operatorname{hom_{\mathcal{C}}}(S_1,S_2)=\operatorname{hom_{\mathcal{S}}}(S_1,S_2)$$ for pair of $S_1,S_2\in \operatorname{Ob}(\mathcal{S})$.

\textbf{Remarks}
\begin{enumerate}
\item
Let $T:\mathcal{C}\to\mathcal{D}$ be a full and faithful functor.  Then $T(\mathcal{C})$ is a full subcategory of $\mathcal{D}$.
\item
Again, let $T:\mathcal{C}\to\mathcal{D}$ be a full and faithful functor.  If $\mathcal{S}$ is a full subcategory of $\mathcal{D}$, then $T^{-1}(\mathcal{S})$ defined by: 
\begin{itemize}
\item
$\operatorname{Ob}(T^{-1}(\mathcal{S})):=\lbrace C\in\operatorname{Ob}(\mathcal{C})\mid T(C)\in\operatorname{Ob}(\mathcal{S})\rbrace$
\item
$\operatorname{Mor}(T^{-1}(\mathcal{S})):=\lbrace f\in\operatorname{Mor}(\mathcal{C})\mid T(f)\in\operatorname{Mor}(\mathcal{S})\rbrace$
\end{itemize}
is a subcategory of $\mathcal{C}$.
\end{enumerate}

\textbf{Examples of Subcategories}
\begin{enumerate}
\item
In \textbf{Set}, the category of finite sets is a full subcategory, and so is the category of $k$-element sets, where $k$ is any (possibly infinite) cardinality.  If $k$ is finite, then every morphism in the subcategory is invertible.
\item
In \textbf{Top}, we have the full subcategories whose objects are Euclidean spaces, compact spaces, or Hausdorff spaces.
\item
In \textbf{Grp}, there is the full subcategory whose objects are abelian groups with additive homomorphisms.
\item
\textbf{Grp} is in fact a subcategory of the category of topological groups, since every group may be viewed as a topological group with the discrete topology.
\item
In \PMlinkescapetext{\textbf{Ring}}, there are the subcategories of commutative rings, matrix rings, or fields.  Note that \textbf{Field} is not a full subcategory of \textbf{Ring}, since the ring homomorphism that maps every element to $0$ is not a field homomorphism.
\end{enumerate}

\begin{thebibliography}{9}
\bibitem{Ma}S. Mac Lane, \emph{Categories for the Working Mathematician} (2nd edition), Springer-Verlag, 1997.
\end{thebibliography}
%%%%%
%%%%%
\end{document}
