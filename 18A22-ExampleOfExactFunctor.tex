\documentclass[12pt]{article}
\usepackage{pmmeta}
\pmcanonicalname{ExampleOfExactFunctor}
\pmcreated{2013-03-22 18:08:29}
\pmmodified{2013-03-22 18:08:29}
\pmowner{CWoo}{3771}
\pmmodifier{CWoo}{3771}
\pmtitle{example of exact functor}
\pmrecord{5}{40696}
\pmprivacy{1}
\pmauthor{CWoo}{3771}
\pmtype{Example}
\pmcomment{trigger rebuild}
\pmclassification{msc}{18A22}

\endmetadata

\usepackage{amssymb,amscd}
\usepackage{amsmath}
\usepackage{amsfonts}
\usepackage{mathrsfs}

% used for TeXing text within eps files
%\usepackage{psfrag}
% need this for including graphics (\includegraphics)
%\usepackage{graphicx}
% for neatly defining theorems and propositions
\usepackage{amsthm}
% making logically defined graphics
%%\usepackage{xypic}
\usepackage{pst-plot}

% define commands here
\newcommand*{\abs}[1]{\left\lvert #1\right\rvert}
\newtheorem{prop}{Proposition}
\newtheorem{thm}{Theorem}
\newtheorem{lem}{Lemma}
\newtheorem{ex}{Example}
\newcommand{\real}{\mathbb{R}}
\newcommand{\pdiff}[2]{\frac{\partial #1}{\partial #2}}
\newcommand{\mpdiff}[3]{\frac{\partial^#1 #2}{\partial #3^#1}}
\newcommand{\im}{\operatorname{im}}
\begin{document}
In this entry, we show the following proposition:

\begin{prop} $\hom(F,-)$ is an exact functor for any free module $F$ over a ring $R$. \end{prop}

Before proving this, let us prove something related to a free module:
\begin{lem} If $f:F\to N$ is a module homomorphism where $F$ is free, then for any module homomorphism $\alpha: M\to N$  where $\im(f)\subseteq \im(\alpha)$, there is a module homomorphism $g:F\to M$ such that $\alpha \circ g = f$.  \end{lem}

\begin{proof}
The case when $F=0$ is trivial.  Suppose now $F\ne 0$ is free, $F$ has a basis, say $X$.  For each $x\in X$, let $M_x:=\alpha^{-1}(f(x))$.  Each $M_x$ is non-empty.  By the axiom of choice (one of its equivalents if necessary), there is a function $g$ from $X$ to $\bigcup \lbrace M_x\mid x\in F\rbrace \subseteq M$ such that $g(x)\in M_x$.  Since $X$ is a basis for $F$, this function can be extended to a module homomorphism from $F$ to $M$.  By abuse of notation, let us use $g$ for the extension.  Since $\alpha (g(x))=f(x)$ for all $x\in X$, this implies that $\alpha\circ g=f$ on all of $F$.
\end{proof}

\begin{proof}[Proof of Proposition 1]
Let 
$$
\xymatrix{
0 \ar[r] & A \ar[r]^{\alpha} & B \ar[r]^{\beta} & C \ar[r] & 0
}
$$
be a short exact sequence of $R$-modules.  We want to show that 
$$
\xymatrix{
0 \ar[r] & \hom(F,A) \ar[r]^{\alpha^*} & \hom(F,B) \ar[r]^{\beta^*} & \hom(F,C) \ar[r] & 0
}
$$
is short exact.  This amounts to establishing the following three equations:
\begin{itemize}
\item
$\ker(\alpha^*)=0$:  

If $\alpha^*(f)=0$, then $\alpha\circ f=0$, which implies that $f(x)=0$ for all $x\in F$.  This means that $f=0$.

\item
$\im(\beta^*)=\hom(F,C)$: 

By Lemma 1 above, for every $f:F\to C$, there is a $g:F\to B$ such that $\beta \circ g = f$, or $\beta^*(g)=f$.

\item
$\im(\alpha^*)=\ker(\beta^*)$: 

If $f\in \im(\alpha^*)$, then there is $g: F\to A$ such that $\alpha \circ g = f$.  Therefore, $\beta^*(f)=\beta\circ f= \beta \circ (\alpha \circ g)=(\beta\circ \alpha)\circ g=0\circ g=0$, showing that $\im(\alpha^*)\subseteq \ker(\beta^*)$.  

On the other hand, pick $f\in \ker(\beta^*)$, so $f(F)\subseteq \ker(\beta)=\im(\alpha)$.  Therefore, by Lemma 1 above, there is a $g:F\to A$ such that $\alpha\circ g=f$.  This means that $f\in \im(\alpha^*)$.
\end{itemize}
Thus, $\hom(F,-)$ is exact.
\end{proof}

\textbf{Remark}.  The converse of this is not true.  But we do have the following fact: $\hom(P,-)$ is an exact functor iff $P$ is a projective module (over some ring $R$).
%%%%%
%%%%%
\end{document}
