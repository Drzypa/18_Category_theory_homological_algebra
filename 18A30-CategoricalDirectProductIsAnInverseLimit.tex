\documentclass[12pt]{article}
\usepackage{pmmeta}
\pmcanonicalname{CategoricalDirectProductIsAnInverseLimit}
\pmcreated{2013-03-22 14:11:26}
\pmmodified{2013-03-22 14:11:26}
\pmowner{archibal}{4430}
\pmmodifier{archibal}{4430}
\pmtitle{categorical direct product is an inverse limit}
\pmrecord{4}{35620}
\pmprivacy{1}
\pmauthor{archibal}{4430}
\pmtype{Theorem}
\pmcomment{trigger rebuild}
\pmclassification{msc}{18A30}
\pmrelated{CategoricalDirectSum}
\pmrelated{CategoricalDirectProduct}

\endmetadata

% this is the default PlanetMath preamble.  as your knowledge
% of TeX increases, you will probably want to edit this, but
% it should be fine as is for beginners.

% almost certainly you want these
\usepackage{amssymb}
\usepackage{amsmath}
\usepackage{amsfonts}

% used for TeXing text within eps files
%\usepackage{psfrag}
% need this for including graphics (\includegraphics)
%\usepackage{graphicx}
% for neatly defining theorems and propositions
\usepackage{amsthm}
% making logically defined graphics
%%%\usepackage{xypic}

% there are many more packages, add them here as you need them

\usepackage{mathrsfs}

% define commands here

\newtheorem{theorem}{Theorem}
\newtheorem{defn}{Definition}
\newtheorem{prop}{Proposition}
\newtheorem{lemma}{Lemma}
\newtheorem{cor}{Corollary}


\newcommand{\Univ}{\mathscr{U}}
\DeclareMathOperator{\liminv}{\varprojlim}
\DeclareMathOperator{\limdir}{\varinjlim}
\DeclareMathOperator{\Funct}{Funct}
\DeclareMathOperator{\Hom}{Hom}
\begin{document}
\begin{theorem}
The categorical direct product can be realized as an example of an inverse limit.
\end{theorem}
\begin{proof}
Suppose we have a direct product of $\{C_i\}_{i\in I}$ for some ($\Univ$) set $I$. 
Consider $I$ as a category whose arrows are only the identity arrows.  Then we can 
define a functor $G$ by $G(i)=C_i$.  It is then clear that the universal property 
of an inverse limit is equivalent to the universal property defining a categorical 
direct product.
\end{proof}

Reversing the arrows, it is also clear that the categorical direct sum is an example of a direct limit.

These results are of interest when one is looking to prove exactness of sums and products in a category: often it is easier to address exactness of direct and inverse limits, and the result then applies to many other constructions as well.
%%%%%
%%%%%
\end{document}
