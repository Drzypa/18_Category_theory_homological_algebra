\documentclass[12pt]{article}
\usepackage{pmmeta}
\pmcanonicalname{CommutativeDiagram}
\pmcreated{2013-03-22 13:24:55}
\pmmodified{2013-03-22 13:24:55}
\pmowner{Dr_Absentius}{537}
\pmmodifier{Dr_Absentius}{537}
\pmtitle{commutative diagram}
\pmrecord{13}{33962}
\pmprivacy{1}
\pmauthor{Dr_Absentius}{537}
\pmtype{Definition}
\pmcomment{trigger rebuild}
\pmclassification{msc}{18A10}
\pmclassification{msc}{18A05}
\pmrelated{WeakHopfCAlgebra}
\pmrelated{CategoricalDiagramsAsFunctors}
\pmrelated{CategoricalSequence}
\pmdefines{diagram}

\endmetadata

%\documentclass{amsart}
\usepackage{amsmath}
\usepackage[all,poly,knot,dvips]{xy}
%\usepackage{pstricks,pst-poly,pst-node,pstcol}
\usepackage{amsthm,latexsym}


% THEOREM Environments --------------------------------------------------

\newtheorem{thm}{Theorem}
 \newtheorem*{mainthm}{Main~Theorem}
 \newtheorem{cor}[thm]{Corollary}
 \newtheorem{lem}[thm]{Lemma}
 \newtheorem{prop}[thm]{Proposition}
 \newtheorem{claim}[thm]{Claim}
 \theoremstyle{definition}
 \newtheorem{defn}[thm]{Definition}
 \theoremstyle{remark}
 \newtheorem{rem}[thm]{Remark}
 \numberwithin{equation}{subsection}


%---------------------  Greek letters, etc ------------------------- 

\newcommand{\CA}{\mathcal{A}}
\newcommand{\CC}{\mathcal{C}}
\newcommand{\CM}{\mathcal{M}}
\newcommand{\CP}{\mathcal{P}}
\newcommand{\CS}{\mathcal{S}}
\newcommand{\BC}{\mathbb{C}}
\newcommand{\BN}{\mathbb{N}}
\newcommand{\BR}{\mathbb{R}}
\newcommand{\BZ}{\mathbb{Z}}
\newcommand{\FF}{\mathfrak{F}}
\newcommand{\FL}{\mathfrak{L}}
\newcommand{\FM}{\mathfrak{M}}
\newcommand{\Ga}{\alpha}
\newcommand{\Gb}{\beta}
\newcommand{\Gg}{\gamma}
\newcommand{\GG}{\Gamma}
\newcommand{\Gd}{\delta}
\newcommand{\GD}{\Delta}
\newcommand{\Ge}{\varepsilon}
\newcommand{\Gz}{\zeta}
\newcommand{\Gh}{\eta}
\newcommand{\Gq}{\theta}
\newcommand{\GQ}{\Theta}
\newcommand{\Gi}{\iota}
\newcommand{\Gk}{\kappa}
\newcommand{\Gl}{\lambda}
\newcommand{\GL}{\Lamda}
\newcommand{\Gm}{\mu}
\newcommand{\Gn}{\nu}
\newcommand{\Gx}{\xi}
\newcommand{\GX}{\Xi}
\newcommand{\Gp}{\pi}
\newcommand{\GP}{\Pi}
\newcommand{\Gr}{\rho}
\newcommand{\Gs}{\sigma}
\newcommand{\GS}{\Sigma}
\newcommand{\Gt}{\tau}
\newcommand{\Gu}{\upsilon}
\newcommand{\GU}{\Upsilon}
\newcommand{\Gf}{\varphi}
\newcommand{\GF}{\Phi}
\newcommand{\Gc}{\chi}
\newcommand{\Gy}{\psi}
\newcommand{\GY}{\Psi}
\newcommand{\Gw}{\omega}
\newcommand{\GW}{\Omega}
\newcommand{\Gee}{\epsilon}
\newcommand{\Gpp}{\varpi}
\newcommand{\Grr}{\varrho}
\newcommand{\Gff}{\phi}
\newcommand{\Gss}{\varsigma}

\def\co{\colon\thinspace}
\begin{document}
\begin{defn}
Let $\mathcal{C}$ be a category. A \emph{diagram} in $\CC$ is a
directed graph $\GG$ with vertex set $V$ and  edge set $E$, (``loops''
and ``parallel edges'' are allowed) together with two maps 
$o\co V\to\mathrm{Obj}(\CC)$, $m\co E\to \mathrm{Morph}(\CC)$ such that
if $e\in E$ has source $s(e)\in V$ and target $t(e)\in V$ then 
$m(e) \in \text{Hom}_{\CC}\left(o\left(s(e)\right),o\left(t(e)\right)\right)$.
\end{defn}

Usually diagrams are denoted by drawing the corresponding graph
and labeling its vertices (respectively edges) with their images under $o$
(respectively $m$), for example if $f\co A\to B$ is a morphism
$$\xymatrix@1{ {A}\ar[r]^f&{B} }$$
is a diagram. Often (as in the previous example) the vertices themselves are
not drawn since their position can be deduced by the position of their
labels.

\begin{defn}
  Let $D=(\GG,o,m)$ be a diagram in the category $\CC$ and $\Gg=(e_1,\ldots,e_n)$
be a path in $\GG$. Then the \emph{composition along} $\Gg$ is the following
morphism of $\CC$
$$\circ(\Gg):=m(e_n)\circ\cdots\circ m(e_1)\,.$$  
 We say that $D$ is
  \emph{commutative} or that it \emph{commutes} if for any two objects in
  the image of $o$, say $A=o(v_1)$ and $B=o(v_2)$, and any two paths $\Gg_1$
 and $\Gg_2$ that connect $v_1$ to $v_2$ we have
$$\circ(\Gg_1)=\circ(\Gg_2)\,.$$ 
\end{defn}

For example the commutativity of the triangle
$$\xymatrix{
{A}\ar[rr]^{f}\ar[dr]_{h}&&{B}\ar[dl]^{g}\\
&{C}&
}
$$ 
translates to $h=g\circ f$, while the commutativity of the square
$$\xymatrix{
{A}\ar[r]^{f}\ar[d]_{k}&{B}\ar[d]^{g}\\
{C}\ar[r]_{h}&{D}
}
$$
translates to $g\circ f=h\circ k$.
%%%%%
%%%%%
\end{document}
