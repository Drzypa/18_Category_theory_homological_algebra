\documentclass[12pt]{article}
\usepackage{pmmeta}
\pmcanonicalname{Precategory}
\pmcreated{2013-03-22 16:16:39}
\pmmodified{2013-03-22 16:16:39}
\pmowner{CWoo}{3771}
\pmmodifier{CWoo}{3771}
\pmtitle{precategory}
\pmrecord{6}{38389}
\pmprivacy{1}
\pmauthor{CWoo}{3771}
\pmtype{Definition}
\pmcomment{trigger rebuild}
\pmclassification{msc}{18A10}
\pmsynonym{diagram scheme}{Precategory}
\pmrelated{Category}
\pmdefines{free category}

\usepackage{amssymb,amscd}
\usepackage{amsmath}
\usepackage{amsfonts}

% used for TeXing text within eps files
%\usepackage{psfrag}
% need this for including graphics (\includegraphics)
%\usepackage{graphicx}
% for neatly defining theorems and propositions
%\usepackage{amsthm}
% making logically defined graphics
%%\usepackage{xypic}
\usepackage{pst-plot}
\usepackage{psfrag}

% define commands here

\begin{document}
A \emph{precategory} $\mathcal{B}$ consists of the following 

\begin{enumerate}
\item a class of objects, called objects of $\mathcal{B}$, written $Ob(\mathcal{B})$
\item a set of elements, called arrows or morphisms, for \emph{each} ordered pair $(A,B)$ of objects in $\mathcal{B}$, usually written $\hom(A,B)$.  For any arrow $f\in\hom(A,B)$, $A$ is called the \emph{domain} of $f$ and $B$ is the \emph{codomain} of $f$.  It is required that $\hom(A,B)\cap\hom(C,D)=\varnothing$ if $(A,B)\neq(C,D)$.
\end{enumerate}

If $Ob(\mathcal{B})$ is a set, then we say that $\mathcal{B}$ is \emph{small}.  A small precategory is just a directed pseudograph (a digraph allowing multiple edges between pairs of vertices), indeed, for the collection of all arrows in $\mathcal{B}$ is a set, written $Mor(\mathcal{B})$.  In addition, there are two functions 
$$\operatorname{dom},\operatorname{codom}:Mor(\mathcal{B})\to Obj(\mathcal{B})$$
such that $\operatorname{dom}(f)$ is the domain of $f$ and $\operatorname{codom}(f)$ is the codomain of $f$.  Note that both $\operatorname{dom}$ and $\operatorname{codom}$ are well-defined functions because if $f\in \hom(A,B)\cap \hom(C,D)$, then $A=B$ and $C=D$, so that both $\operatorname{dom}$ and $\operatorname{codom}$ map $f$ to unique objects $A$ and $B$ respectively.

With the realization that a precategory is essentially a directed graph, we may use the language of graph theory to define concepts such as paths and loops in a precategory.  This will allow us to enlarge any precategory to a category.  We will carry out the construction below.

\subsubsection*{Paths Defined}
Let $\mathcal{B}$ be a precategory.  A \emph{path} $p$ (in $\mathcal{B}$) is a finite sequence of arrows $f_1,\ldots,f_n$ such that the codomain of $f_i$ is the domain of $f_{i+1}$.  Note that the definition here does not parallel the one given for a graph (as in graph theory), since we allow vertices (domains and codomains), as well as edges (arrows or morphisms) to coincide.  The \emph{length} of a path $p=(p_1,\ldots,p_n)$ is defined to be the non-negative integer $n$.

Given a path $p = (f_1,\ldots, f_n)$, we may set the domain of $p$, written $\operatorname{dom}(p)$, to be $\operatorname{dom}(f_1)$, and codomain of $p$, written $\operatorname{codom}(p)$, to be $\operatorname{codom}(f_n)$.  A \emph{loop} is a path $p$ where $\operatorname{dom}(p)=\operatorname{codom}(p)$.

Next, for each ordered pair of objects $(A,B)$ in a precategory $\mathcal{B}$, the collection of paths with with domain $A$ and codomain $B$ is a set, and we denote it by $\operatorname{Hom}(A,B)$.

\subsubsection*{Composition of Paths Defined}
Now, let $f\in \operatorname{Hom}(A,B)$ and $g\in \operatorname{Hom}(B,C)$.  So $f=(f_1,\ldots,f_n)$ and $g=(g_1,\ldots,g_m)$.  Since $\operatorname{codom}(f_n)=B=\operatorname{dom}(g_1)$, we can ``concatenate'' the two paths and form a new path $$(f_1,\ldots,f_n,g_1,\ldots,g_m),$$ and we write $g\circ f$ for this new path.  It is clear that $g\circ f\in \operatorname{Hom}(A,C)$.  It is also easy to see that $\circ$ is a function from $\operatorname{Hom}(A,B)\times \operatorname{Hom}(B,C)$ to $\operatorname{Hom}(A,C)$, if we set $\circ(f,g):= g\circ f$.  As the ``concatenation'' operation is evidently associative, $(h\circ g)\circ f=h\circ (g\circ f)$.

\subsubsection*{Empty Paths Defined}
Finally, for each object $A$ in $Ob(\mathcal{B})$, we can artificially associate an \emph{empty path} $1_A$ with $A$, with the following properties
\begin{itemize}
\item $1_A$ is a path with length $0$
\item $\operatorname{dom}(1_A)=\operatorname{codom}(1_A):=A$; in other words, $1_A\in \operatorname{Hom}(A,A)$
\item for any $f\in\operatorname{Hom}(A,B)$ and $g\in\operatorname{Hom}(C,A)$, $f\circ 1_A:=f$ and $1_A\circ g:=g$.
\end{itemize}
The class of all paths, including every empty path for each object, in $\mathcal{B}$ is written $Path(\mathcal{B})$.

\subsubsection*{Precategory Enlarged to a Category}
So if we start out with a precategory $\mathcal{B}$, we end up with a category $\mathcal{\overline{B}}$ such that 
\begin{enumerate}
\item $Ob(\mathcal{\overline{B}})=Ob(\mathcal{B})$
\item $Mor(\mathcal{\overline{B}})=Path(\mathcal{B})$, such that
\begin{itemize}
\item domain and codomain of each morphism are defined to be the domain and codomain of the underlying path
\item for each ordered pair $(A,B)$ of objects in $\mathcal{\overline{B}}$, the collection of morphisms with domain $A$ and codomain $B$ is a set, and is denoted by $\operatorname{Hom}(A,B)$
\item for every triple of objects $A,B,C$, a function $\circ$ is defined to be the ``concatenation'' of a path from $A$ to $B$ and a path from $B$ to $C$
\item the identity morphism $1_A$ each object $A$ is just the empty path associated with $A$.
\end{itemize}
\end{enumerate}
We may embed $\mathcal{B}$ in $\mathcal{\overline{B}}$ so that $\mathcal{B}$ is just a diagram of $\mathcal{\overline{B}}$.  Because of this, $\mathcal{B}$ is also known as a \emph{diagram scheme}.  $\mathcal{\overline{B}}$, also written $F(\mathcal{B})$, is known as the \emph{free category} freely generated by $\mathcal{B}$.
%%%%%
%%%%%
\end{document}
