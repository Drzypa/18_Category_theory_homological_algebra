\documentclass[12pt]{article}
\usepackage{pmmeta}
\pmcanonicalname{Automorphism}
\pmcreated{2013-03-22 13:48:29}
\pmmodified{2013-03-22 13:48:29}
\pmowner{mathcam}{2727}
\pmmodifier{mathcam}{2727}
\pmtitle{automorphism}
\pmrecord{10}{34529}
\pmprivacy{1}
\pmauthor{mathcam}{2727}
\pmtype{Definition}
\pmcomment{trigger rebuild}
\pmclassification{msc}{18A05}
%\pmkeywords{morphism}
%\pmkeywords{isomorphism}
%\pmkeywords{endomorphism}
%\pmkeywords{epimorphism}
%\pmkeywords{monomorphism}
\pmrelated{OppositeNumber}

% this is the default PlanetMath preamble.  as your knowledge
% of TeX increases, you will probably want to edit this, but
% it should be fine as is for beginners.

% almost certainly you want these
\usepackage{amssymb}
\usepackage{amsmath}
\usepackage{amsfonts}

% used for TeXing text within eps files
%\usepackage{psfrag}
% need this for including graphics (\includegraphics)
%\usepackage{graphicx}
% for neatly defining theorems and propositions
%\usepackage{amsthm}
% making logically defined graphics
%%%\usepackage{xypic}

% there are many more packages, add them here as you need them

% define commands here
\begin{document}
Roughly, an automorphism is a map from a mathematical object onto itself such that: 1. There exists an ``inverse'' map such that the composition of the two is the identity map of the object, and 2. any relevant structure related to the object in question is preserved.

In category theory, an automorphism of an object $A$ in a category $\mathcal{C}$ is a morphism $\psi \in Mor(A, A)$ such that there exists another morphism $\phi \in Mor(A, A)$ and $\psi \circ \phi = \phi \circ \psi = id_{A}$. 

For example in the category of groups an automorphism is just a bijective (inverse exists and composition gives the identity) group homomorphism (group structure is preserved). Concretely, the map: $x \mapsto -x$ is an automorphism of the additive group of real numbers. In the category of topological spaces an automorphism would be a bijective, continuous map such that its inverse map is also continuous (not guaranteed as in the group case). Concretely, the map $\psi: S^1 \to S^1$ where $\psi(\alpha) = \alpha + \theta$ for some fixed angle $\theta$ is an automorphism of the topological space that is the circle.
%%%%%
%%%%%
\end{document}
