\documentclass[12pt]{article}
\usepackage{pmmeta}
\pmcanonicalname{Supercategory}
\pmcreated{2013-03-22 18:12:59}
\pmmodified{2013-03-22 18:12:59}
\pmowner{bci1}{20947}
\pmmodifier{bci1}{20947}
\pmtitle{supercategory}
\pmrecord{113}{40800}
\pmprivacy{1}
\pmauthor{bci1}{20947}
\pmtype{Topic}
\pmcomment{trigger rebuild}
\pmclassification{msc}{18A30}
\pmclassification{msc}{03C99}
\pmclassification{msc}{18A25}
\pmsynonym{organismic supercategories}{Supercategory}
\pmsynonym{higher dimensional algebra(HDA)}{Supercategory}
\pmsynonym{super-categories}{Supercategory}
%\pmkeywords{organismic supercategories}
%\pmkeywords{axiomatic theory of supercategories}
%\pmkeywords{higher dimensional algebra (HDA)}
%\pmkeywords{meta-categories}
%\pmkeywords{non-Abelian superstructures with heterofunctors}
%\pmkeywords{n-categories}
%\pmkeywords{super-categories}
\pmrelated{ETAS}
\pmrelated{SupercategoriesOfComplexSystems}
\pmrelated{2Category2}
\pmrelated{AxiomatizableClass}
\pmrelated{HigherDimensionalAlgebraHDA}
\pmrelated{VariableTopology8}
\pmrelated{VariableTopology3}
\pmrelated{2Category}
\pmrelated{2Category2}
\pmrelated{Supercategory}
\pmrelated{NaturalTransformationsOfOrganismicStructures}
\pmrelated{FunctorCategory2}
\pmdefines{order of a supercategory}
\pmdefines{structural dimension}
\pmdefines{meta-category}
\pmdefines{super-category}
\pmdefines{multiple composition laws}
\pmdefines{non-Abelian superstructure with heterofunctors}
\pmdefines{vector field S1-supercategory}

% this is the default PlanetMath preamble.
% almost certainly you want these
\usepackage{amssymb}
\usepackage{amsmath}
\usepackage{amsfonts}
% used for TeXing text within eps files
%\usepackage{psfrag}
% need this for including graphics (\includegraphics)
%\usepackage{graphicx}
% for neatly defining theorems and propositions
%\usepackage{amsthm}
% making logically defined graphics
%%%\usepackage{xypic}
% define commands here
\usepackage{amsmath, amssymb, amsfonts, amsthm, amscd,  enumerate}
\usepackage{xypic, xspace}
\usepackage[mathscr]{eucal}
\usepackage[dvips]{graphicx}
\usepackage[curve]{xy}
\setlength{\textwidth}{6.5in}
\setlength{\textheight}{9.0in}
\voffset=-.4in
\theoremstyle{plain}
\newtheorem{lemma}{Lemma}[section]
\newtheorem{proposition}{Proposition}[section]
\newtheorem{theorem}{Theorem}[section]
\newtheorem{corollary}{Corollary}[section]
\theoremstyle{definition}
\newtheorem{definition}{Definition}[section]
\newtheorem{example}{Example}[section]
\newtheorem{remark}{Remark}[section]
\newtheorem*{notation}{Notation}
\newtheorem*{claim}{Claim}
\renewcommand{\thefootnote}{\ensuremath{\fnsymbol{footnote}}}
\numberwithin{equation}{section}
\newcommand{\Ad}{{\rm Ad}}
\newcommand{\Aut}{{\rm Aut}}
\newcommand{\Cl}{{\rm Cl}}
\newcommand{\Co}{{\rm Co}}
\newcommand{\DES}{{\rm DES}}
\newcommand{\Diff}{{\rm Diff}}
\newcommand{\Dom}{{\rm Dom}}
\newcommand{\Hol}{{\rm Hol}}
\newcommand{\Mon}{{\rm Mon}}
\newcommand{\Hom}{{\rm Hom}}
\newcommand{\Ker}{{\rm Ker}}
\newcommand{\Ind}{{\rm Ind}}
\newcommand{\IM}{{\rm Im}}
\newcommand{\Is}{{\rm Is}}
\newcommand{\ID}{{\rm id}}
\newcommand{\grpL}{{\rm GL}}
\newcommand{\Iso}{{\rm Iso}}
\newcommand{\rO}{{\rm O}}
\newcommand{\Sem}{{\rm Sem}}
\newcommand{\SL}{{\rm Sl}}
\newcommand{\St}{{\rm St}}
\newcommand{\Sym}{{\rm Sym}}
\newcommand{\Symb}{{\rm Symb}}
\newcommand{\SU}{{\rm SU}}
\newcommand{\Tor}{{\rm Tor}}
\newcommand{\U}{{\rm U}}
\newcommand{\A}{\mathcal A}
\newcommand{\Ce}{\mathcal C}
\newcommand{\E}{\mathcal E}
\newcommand{\F}{\mathcal F}
%\newcommand{\grp}{\mathcal G}
\renewcommand{\H}{\mathcal H}
\renewcommand{\cL}{\mathcal L}
\newcommand{\Q}{\mathcal Q}
\newcommand{\R}{\mathcal R}
\newcommand{\cS}{\mathcal S}
\newcommand{\cU}{\mathcal U}
\newcommand{\W}{\mathcal W}
\newcommand{\bA}{\mathbb{A}}
\newcommand{\bB}{\mathbb{B}}
\newcommand{\bC}{\mathbb{C}}
\newcommand{\bD}{\mathbb{D}}
\newcommand{\bE}{\mathbb{E}}
\newcommand{\bF}{\mathbb{F}}
\newcommand{\bG}{\mathbb{G}}
\newcommand{\bK}{\mathbb{K}}
\newcommand{\bM}{\mathbb{M}}
\newcommand{\bN}{\mathbb{N}}
\newcommand{\bO}{\mathbb{O}}
\newcommand{\bP}{\mathbb{P}}
\newcommand{\bR}{\mathbb{R}}
\newcommand{\bV}{\mathbb{V}}
\newcommand{\bZ}{\mathbb{Z}}
\newcommand{\bfE}{\mathbf{E}}
\newcommand{\bfX}{\mathbf{X}}
\newcommand{\bfY}{\mathbf{Y}}
\newcommand{\bfZ}{\mathbf{Z}}
\renewcommand{\O}{\Omega}
\renewcommand{\o}{\omega}
\newcommand{\vp}{\varphi}
\newcommand{\vep}{\varepsilon}
\newcommand{\diag}{{\rm diag}}
\newcommand{\grp}{\mathcal G}
\newcommand{\dgrp}{{\mathsf{D}}}
\newcommand{\desp}{{\mathsf{D}^{\rm{es}}}}
\newcommand{\hgr}{{\mathsf{H}}}
\newcommand{\mgr}{{\mathsf{M}}}
\newcommand{\ob}{{\rm Ob}}
\newcommand{\obg}{{\rm Ob(\mathsf{G)}}}
\newcommand{\obgp}{{\rm Ob(\mathsf{G}')}}
\newcommand{\obh}{{\rm Ob(\mathsf{H})}}
\newcommand{\Osmooth}{{\Omega^{\infty}(X,*)}}
\newcommand{\grphomotop}{{\rho_2^{\square}}}
\newcommand{\grpcalp}{{\mathsf{G}(\mathcal P)}}
\newcommand{\rf}{{R_{\mathcal F}}}
\newcommand{\grplob}{{\rm glob}}
\newcommand{\loc}{{\rm loc}}
\newcommand{\TOP}{{\rm TOP}}
\newcommand{\wti}{\widetilde}
\renewcommand{\a}{\alpha}
\newcommand{\be}{\beta}
\newcommand{\de}{\delta}
\newcommand{\del}{\partial}
\newcommand{\ka}{\kappa}
\newcommand{\si}{\sigma}
\newcommand{\ta}{\tau}
\newcommand{\med}{\medbreak}
\newcommand{\medn}{\medbreak \noindent}
\newcommand{\bign}{\bigbreak \noindent}
\newcommand{\lra}{{\longrightarrow}}
\newcommand{\ra}{{\rightarrow}}
\newcommand{\rat}{{\rightarrowtail}}
\newcommand{\ovset}[1]{\overset {#1}{\ra}}
\newcommand{\ovsetl}[1]{\overset {#1}{\lra}}
\newcommand{\hr}{{\hookrightarrow}}
\begin{document}
\begin{definition} 
A \emph{supercategory} is defined axiomatically in terms of ETAS (\cite{ICB3}) as a non-Abelian structure consisting of classes of objects $\mathcal{O}$ and/or classes of categorical diagrams $\mathcal{D}$ with multiple structures $S_0, S_1, ..., S_n$ (with $(n+1)$ being an integer called its \emph{structural dimension} or \emph{order})-- such as, algebraic, topological, geometric, and analytical-- that also have defined distinct \emph{ETAS composition laws} $\Gamma_1 := \circ, \Gamma_2 := *, \Gamma_3:= **, ...,\Gamma_n$, defined for each type of structure, subject to the ETAS composition law axioms for $\Gamma$'s. Such composition laws form a finite collection denoted as 
$\mathcal{L}_{Sn}$.
\end{definition}

\subsection{Examples}

\begin{example}

Let us consider first a simple example of an $\mathcal{S}_1$ supercategory of 1st order defined with a single composition law of functors as in the standard definition of a \PMlinkname{functor category}{FunctorCategory2}. Thus, a less general definition of supercategory was recently introduced in mathematical (`categorified') physics, on the web's n-Category caf\'e s web site under \PMlinkexternal{``Supercategories''}{http://golem.ph.utexas.edu/category/2007/07/supercategories.html}. This is a rather `simple' example of supercategory, albeit in a much more restricted sense as it still involves only one composition law for
the functors, and also its components are all small categories; this functor category is sometimes called a 
`\emph{super-category}' in the published literature. Thus, the following specific example of an $\mathcal{S}_1$ supercategory begins with a standard definiton of such a super-category, or `super category' (functor category) from category theory; it becomes interesting as it is being tailored to supersymmetry and extensions of `Lie' superalgebras, or superalgebroids, which are sometimes called graded `Lie' algebras that are thought to be relevant to quantum gravity (\cite{BGB2} and references cited therein). The following is an almost exact quote from the above n-Category cafe's website posted by Dr. Urs Schreiber: 
\end{example}

\begin{definition}
A \PMlinkname{super-category}{2Category2} is a \emph{diagram} of the form:
$\diamond  \diamond Id_C \diamond \textbf{C} \diamond \diamond s$ in \textbf{Cat}--the category of categories and (homo-) functors between categories-- such that: $\diamond  \diamond \textsl{Id} \diamond \diamond Id_C \diamond \textbf{C} \diamond \textbf{C}\diamond \diamond s \diamond \diamond s = \diamond  \diamond Id_C \diamond Id_C  \diamond  \diamond \textsl{Id}$, \\
(where the `diamond' symbol should be replaced by the symbol `square', as in the original Dr.Urs Schreiber's postings.) 
\end{definition} 
This specific instance is that of a supercategory which has only \textbf{one object}-- the above quoted superdiagram of diamonds, an arbitrary abstract category \textbf{C} (subject to all ETAC axioms), and the standard category identity (homo-) functor; it can be further specialized to the previously introduced concepts of \textit{supergroupoids} (also definable as crossed complexes of groupoids), and \textit{supergroups} (also definable as crossed modules of groups), which seem to be of great interest to mathematicians involved in `Categorified' Mathematical Physics or Physical Mathematics. The geometric picture that one is tempted to associate with this `linear super-category' is that of
a `string of pearls'.  This definition of a linear `supercategory' was then followed up with an even more interesting example. ``What, in this sense, is a \textit{braided monoidal supercategory ?}''.  Urs, suggested the following answer: like an ordinary braided monoidal category is a 3-category which in lowest degrees looks like the trivial 2-group, a 
\PMlinkexternal{braided monoidal supercategory}{http://www.math.uni-hamburg.de/home/schreiber/scat.pdf} is a 3-category which in lowest degree looks like the strict 2-group that comes from 
the crossed module $G(2)=(\diamond 2 \diamond \textsl{Id} \diamond 2)$. Urs called this generalization of stabilization of n-categories, $G(2)$-\textit{stabilization}. So the claim would be that braided monoidal supercategories come from 
$G(2)$-stabilized 3-categories, with $G(2)$ the above strict 2-group. In view of the interpretation of this Urs type
of $\mathcal{S}_1$ supercategory as a category equipped with an ``odd vector field'', it may be more appropriately called a `\emph{vector field $\mathcal{S}_1$-supercategory}'.



\begin{remark}
One can readily extend the above definition of a \emph{one-object super-category} to a two-object, and... 
\textbf{n-object}, proper supercategory using more that one category, more than one $Id_X$, and several types of diamond, square, and so on, to define an $n$--supercategory; thus, one ends up with several, connected `strings of pearls into a tangled strand'--the intuitive picture of a proper n-supercategory, $\S_n$. 
\end{remark}
\begin{example}
A bipartite graph, and k-partite graphs in general, are examples of supercategories where
distinct composition laws are represented by different colors. 
\end{example}

\begin{example}
The 2-category of double groupoids is another example of an $\mathcal{S}_1$ supercategory.
\end{example}

The following subsection lists examples of supercategories, including types of supercategories that
have applications in categorical logics and general system dynamics (i.e. beyond ODE's).

\subsection{A partial list of examples of supercategories}
\begin{enumerate}
\item functor categories
\item \PMlinkexternal{R-supercatgeories}{http://planetphysics.org/?op=getobj&from=objects&name=RSupercategory}
\item hypergraphs
\item bipartite graphs and colored graphs
\item k-partite graphs
\item \PMlinkname{2-categories}{2Category}, super-categories and n-categories; categories of categories (that are often considered for the foundation of mathematics) 
\item higher dimensional algebras (HDA): double groupoids,double algebroids, crossed complexes of groupoids, crossed complexes of algebroids, superalgebroids, double categories, multiple categories, $\omega$ and cubic structures, 
\item \PMlinkname{organismic supercategories}{SupercategoriesOfComplexSystems};
\item algebraic theories of organismic sets;
\item standard `super-categories' redefined as supercategories \cite{Urs2k7P};
\item super-categories of metabolic-repair, or $(M,R)$-systems;
\item super-categories of \L{}ukasiewicz algebraic (3-valued) logics;
\item supercategories of quantum automata;
\item $LM_n$-generalized toposes
\item toposes (or topoi) of genetic networks and human interactomes;
\item supercategories of Post algebraic logics;
\item supercategories of MV-algebraic logics.
\end{enumerate}

{\em Several specific examples of organismic supercategories} were developed in a series of mathematical biology publications listed in the bibliography. 

\begin{remark}
 Supercategories, and especially, {\em organismic supercategories} provide an unified conceptual framework for abstract relational biology that utilizes flexible, algebraic and topological structures which transform naturally under heteromorphisms or heterofunctors, and natural transformations. 


 One of the major advantages of the ETAS axiomatic approach, which was inspired by the work of 
Lawvere (\cite{LW1}, \cite{LW2}), is that ETAS avoids all the antimonies/paradoxes previously reported for sets,
sets of sets, involving for example the elementhood relation or infinite sets, the axiom of choice, and so on (Russell and Whitehead, 1925, and Russell, 1937; \cite{BBGG1}). ETAS also provides an axiomatic approach to recent higher dimensional algebra (\cite{BHS2}, \cite{BGB2}) applications to complex systems biology 
(\cite{Bgg2}, \cite{BBGG1}, and references cited therein). 
\end{remark}

\subsection{A simple example of ETAS axioms (\cite{ICB3}):} 

\begin{itemize}
\item (a). The eight ETAC axioms introduced by W. F. Lawvere for same type morphisms, functors, natural transformations
and other arrows in higher dimensions. 
\item (b). A family of composition laws $\left\{\Gamma_i\right\}$, with $i= 1, 2, ..., n$, and
each $\Gamma_i$ being subject to ETAC axioms 5 to 7; 
\item (c). The composition law conditions in the ETAC axioms 8a to 8c will apply only to (small) subclasses of homo-morphisms and homo-functors (but will not apply to hetero-morphisms and hetero-functors); 
\item (d). Hetero-morphism and hetero-functor axioms specifying how two composition laws interact, involving also objects with different structure or type. 
\end{itemize}

\subsection{Generic example of an ETAS with ten axioms:}  


0. For any letters $x, y, u, A, B$, and \emph{unary function} symbols $\Delta_0$ and $\Delta_1$,
and \emph{composition law} $\Gamma$, the following are defined as \emph{formulas}: $\Delta_0 (x) = A$,
$\Delta_1 (x) = B$, $\Gamma (x,y;u)$, and $ x = y$; These formulas are to be, respectively, interpreted as
``$A$ is the domain of $x$", ``$B$ is the codomain, or range, of $x$", ``$u$ is the composition $x$ followed by $y$",
and ``$x$ equals $y$". 

1. If $\Phi$ and $\Psi$ are formulas, then ``$[\Phi]$ and $[\Psi]$'' , ``$[\Phi]$ or$[\Psi]$'', ``$[\Phi] \Rightarrow [\Psi]$'', and ``$[not \Phi]$''  are also formulas.

2. If $\Phi$ is a formula and $x$ is a letter, then ``$ \forall x[\Phi]$'', 
``$ \exists x[\Phi]$'' are also formulas.

3. A string of symbols is a formula in ETAS iff it follows from the above axioms 0 to 2.

A \emph{sentence} is then defined as any formula in which every occurrence of each letter $x$ is within the scope of a quantifier, such as $\forall x$  or $\exists x $.  The \emph{theorems} of ETAS are defined as all those sentences which can be derived through logical inference from the following ETAS axioms:

4. $\Delta_i(\Delta_j(x))=\Delta_j(x)$ for  $i,j = 0, 1$. 

5. $\Gamma(x,y;u)$ and $\Gamma(x,y;u')\Rightarrow u = u'$.

5. $ \exists u [\Gamma(x,y;u)] \Rightarrow \Delta_1(x) =  \Delta_0(y)$;

7. $\Gamma(x,y;u) \Rightarrow \Delta_0(u) =  \Delta_0(x)$ and $\Delta_1(u) =  \Delta_1(y)$.

8. Identity axiom:
$ \Gamma(\Delta_0 (x), x;x)$ and  $ \Gamma(x, \Delta_1 (x);x)$  yield always the same result.

7. Associativity axiom: $\Gamma(x,y;u)$ and $\Gamma(y,z;w)$ and $\Gamma(x,w;f)$ and $\Gamma(u,z;g)\Rightarrow f = g $.
With these axioms in mind, one can see that commutative diagrams can be now regarded as certain 
\textit{abbreviated} formulas corresponding to systems of equations such as:  
$\Delta_0(f) = \Delta_0(h) = A$, $\Delta_1(f) = \Delta_0(g) = B$, $\Delta_1(g) = \Delta_1(h) = C$ 
and $\Gamma(f,g;h)$, instead of $g\circ f = h$ for the arrows f, g, and h, drawn respectively between the 
`objects' A, B and C, thus forming a `triangular commutative super-diagram` in the usual sense of category theory. Compared with the ETAS formulas such diagrams have the advantage of a geometric--intuitive image of their equivalent underlying equations. The common property of A of being an object is written in shorthand as the abbreviated formula Obj(A) standing for the following three equations:

8. $A = \Delta_0(A) = \Delta_1(A)$,

9. $ \exists x[A = \Delta_0 (x)] \exists y[A = \Delta_1 (y)]$,

and 

10. $\forall x \forall u [\Gamma (x,A; u)\Rightarrow x = u]$ and 
$ \forall y  \forall v [\Gamma (A,y; v)] \Rightarrow y = v$ .  

Intuitively, with this terminology and axioms a \textit{supercategory} is meant to be any structure which is a direct interpretation of these ten ETAS axioms. A \textit{heterofunctor} is then  understood to be a \textit{triple} consisting of two such supercategories (or classes of objects, diagrams, etc.) and of a set, or proper class,  of rules $\S_H$ (`the hetero-functor') which assigns to each arrow or morphism $x$ of the first supercategory,
a unique morphism, written as `$\S_H (x)$' of the second category, in such a way that the usual conditions on objects (but not on arrows) are fulfilled (see for example \cite {ICBM})--  the functor is well behaved, it carries object identities to image object identities, and commutative super-diagrams (or classes) to image commmutative super-diagrams of the corresponding image objects and image (hetero)morphisms betwen objects with different structures, as well as homo-morphisms (between objects with the {\em same type of structure}, such as groupoid homomorphisms, or topological space homeomorphisms, automata homomorphisms, etc).  At the next level, one then defines \emph{natural transformations} or \emph{functorial morphisms} between functors as metalevel abbreviated formulas and equations pertaining to commutative diagrams of the distinct images of two functors acting on both objects and morphisms. As the name indicates natural transformations are also well--behaved in terms of the ETAC equations satisfied. \\
{\em Remark:} The example shown above is the ETAS formulation closest to ETAC--
axioms for categories and super-categories, or categories of categories in volved in the
foundations of most Mathematics, (cf. Lawvere and others); see also \cite{LW2}
and the entry on axiomatic theory of supercategories. \\

\subsubsection{Additional explanation of the examples of supercategories listed above: } 


In the usual sense, super-categories are defined as categories of categories,
and the process is repeated in higher dimensions in n-categories. There is however
a `more geometric', or `gluing' construction of double categories, double groupoids (\cite{BS}, and 
double algebroids \cite{BM}) that involves additional conditions or axioms leading to non-Abelian higher dimensional structures and Higher Dimensional Algebra (HDA;\cite{BS},\cite{BM} and \cite{BGB2}). Similarly, the construction of \textit{supercategories} (\cite{ICB3}), unlike that of $n$-categories, allows for `gluing' together distinct structures
and/or their corresponding diagrams (such as algebraic and topological ones), through hetero-morphisms or hetero-functors, which are arrows (not necessarily subject to all of the ETAC axioms) linking distinct structures into the superstructure called a \textit{`supercategory}'; the latter is a generalized type of double groupoid, double category, 2-category,..., n-category or super--category/meta--category, that always involves only diagrams of homo-morphisms at each level, but also includes hetero-morphisms or hetero-functors, and so on, between different types of structures. 
Proper supercategories could also be called \textit{`n-ary'} super-categories, or \textit{multi-categories}, in the sense of extending double groupoid, double algebroid and double category structures to higher dimensions. 
Note however that even in a general, abstract supercategory at the level of diagrams of homo-morphisms, homo-functors, natural transformations, or any n-categories with only one type of arrows at each level, all the ETAC axioms still hold. On the other hand, at the levels of supercategorical diagrams, or superdiagrams, involving several types of morphisms, or hetero-morphisms and hetero-functors, several of the rules for connecting diagrams are weakened, and the result is a superstructure which does not have all naturality conditions satisfied by all arrows, and one has additional composition laws, such as $\Gamma', \Gamma'', ...$, and so on,  satisfying new ETAS axioms that are not allowed in ETAC. Thus, additional ETAS axioms are needed to also specify how such distinct composition laws are combined within the same superstructure. Any interpretation of ETAS axioms (that may include also the ETAC axioms for the special cases of categories, n-categories, double categories, etc) then defines a \textit{supercategory}.

\begin{definition} 

 \emph{Organismic supercategory (\cite{ICBM},\cite{ICB3} and \cite{ICB1}.}
An example of a supercategory interpreting such ETAS axioms as those stated above
was previously defined for organismic structures with different levels of complexity (\cite{ICB3}); an \textit{organismic supercategory} was thus defined as a \textit{superstructure interpretation of ETAS} (including ETAC, as appropriate) in terms of triples $\textbf{K} = (\emph{C}, \Pi, \textit{N})$, where \emph{C} is an arbitrary category (interpretation of ETAC axioms, formulas, etc.), $\Pi$ is a category of complete self--reproducing entities, $\pi$, (\cite{LO68}) subject to the negation of the axiom of restriction (for elements of sets):
$ \exists S: (S \neq \oslash) ~ and ~ \forall u: [u \in S) \Rightarrow \exists v: (v \in u)~ and ~( v \in S)]$, (which is known to be independent from the ordinary logico-mathematical and biological reasoning), 
and $\textit{N}$ is a category of non-atomic expressions, defined as follows.  An  \textit{atomically self--reproducing entity} is a unit class relation $u$ such that  $\pi \pi \left\langle \pi \right\rangle$, which means 
``$\pi$ stands in the relation $\pi$ to $\pi$'', $\pi \pi \left\langle \pi , \pi \right\rangle$, etc. 
An expression that does not contain any such atomically self--reproducing entity is called a \textit{non-atomic expression}. 
\end{definition}

\textbf{Note:}
Supercategories are mapped ('morphed', transformed or linked) by hetero-functors (or 'heterofunctors'); {\em hetero-functors} are defined as a natural extension of the concept of functor between categories, in the sense that they link objects from different categorical diagrams or categories with different structures following rules and axioms specified by the elementary theory of abstract supercategories, ETAS. Heterofunctors can be imaged for example as the colored links or edges in a bipartite or k-partite graph.

\begin{thebibliography}{99}
\bibitem{AR2k5}
Assadollahi, R. and Brigitte Rockstroh. 2005. Neuromagnetic brain responses to words from semantic sub- and supercategories., {\em BMC Neurosci.},{6}: 57. 
\PMlinkexternal{PDF download}{http://www.pubmedcentral.nih.gov/articlerender.fcgi?artid=1236933}

\bibitem{Urs2k7P}
Urs Schreiber. 2007. Supercategories., 
\PMlinkexternal{(8 pp Preprint)}{http://www.math.uni-hamburg.de/home/schreiber/scat.pdf}.

\bibitem{}
E. Lowen-Colebunders and R. Lowen.: 1997. 
\PMlinkexternal{Supercategories of Top and the Inevitable Emergence of Topological Constructs.}{http://books.google.com/books?id=dV6WtepcZLkC&pg=PA969&lpg=PA969&dq=supercategories}.
 pp. 969-1027 in {\em Handbook of the History of General Topology},
Charles E. Aull, R. Lowen, Eds. Springer: Berlin, ISBN 079236970X, 9780792369707.

\bibitem{ICBM}
I.C.Baianu and M. Marinescu: 1968, Organismic Supercategories: Towards a Unitary Theory of Systems. \emph{Bulletin of Mathematical Biophysics} \textbf{30}, 148-159.

\bibitem{ICB3}
I.C. Baianu: 1970, Organismic Supercategories: II. On Multistable Systems. \emph{Bulletin of Mathematical Biophysics}, \textbf{32}: 539-561.

\bibitem{ICB1}
I.C. Baianu : 1971a, Organismic Supercategories and Qualitative Dynamics of Systems. \emph{Ibid.}, \textbf{33} (3), 339--354.

\bibitem{ICB71}
I.C. Baianu: 1971b, Categories, Functors and Quantum Algebraic Computations, in P. Suppes (ed.), \emph{Proceed. Fourth Intl. Congress Logic-Mathematics-Philosophy of Science}, September 1--4, 1971, University of Bucharest.

\bibitem{ICB04b}
I.C. Baianu: \L ukasiewicz-Topos Models of Neural Networks, Cell Genome and Interactome Nonlinear Dynamics). CERN Preprint EXT-2004-059. \textit{Health Physics and Radiation Effects} (June 29, 2004). 

\bibitem{BBGG1}
I.C. Baianu, Brown R., J. F. Glazebrook, and Georgescu G.: 2006, Complex Nonlinear Biodynamics in 
Categories, Higher Dimensional Algebra and \L ukasiewicz--Moisil Topos: Transformations of
Neuronal, Genetic and Neoplastic networks, \emph{Axiomathes} \textbf{16} Nos. 1--2, 65--122.

\bibitem{ICBs5}
I.C. Baianu and D. Scripcariu: 1973, On Adjoint Dynamical Systems. \emph{The Bulletin of Mathematical Biophysics}, \textbf{35}(4), 475--486.

\bibitem{ICB5}
I.C. Baianu: 1973, Some Algebraic Properties of \emph{\textbf{(M,R)}} -- Systems. \emph{Bulletin of Mathematical Biophysics} \textbf{35}, 213-217.

\bibitem{ICBm2}
I.C. Baianu and M. Marinescu: 1974, A Functorial Construction of \emph{\textbf{(M,R)}}-- Systems. \emph{Revue Roumaine de Mathematiques Pures et Appliquees} \textbf{19}: 388-391.

\bibitem{ICB6}
I.C. Baianu: 1977, A Logical Model of Genetic Activities in \L ukasiewicz Algebras: The Non-linear Theory. \emph{Bulletin of Mathematical Biophysics}, \textbf{39}: 249-258.

\bibitem{ICB7}
I.C. Baianu: 1980, Natural Transformations of Organismic Structures. \emph{Bulletin of Mathematical Biophysics}
\textbf{42}: 431-446.

\bibitem{ICB8}
I.C. Baianu: 1983, Natural Transformation Models in Molecular Biology., in \emph{Proceedings of the SIAM Natl. Meet}., Denver, CO.; Eprint at cogprints.org as No. 3675/0l (Naturaltransfmolbionu6.pdf).

\bibitem{ICB9}
I.C. Baianu: 1984, A Molecular-Set-Variable Model of Structural and Regulatory Activities in Metabolic and Genetic Networks., \emph{FASEB Proceedings} \textbf{43}, 917.

\bibitem{ICB2}
I.C. Baianu: 1987a, Computer Models and Automata Theory in Biology and Medicine.,  in M. Witten (ed.), 
\emph{Mathematical Models in Medicine}, vol. 7., Pergamon Press, New York, 1513--1577; \emph{CERN Preprint No. EXT-2004-072}. 

\bibitem{ICB9b}
I.C. Baianu: 1987b, Molecular Models of Genetic and Organismic Structures, in \emph{Proceed. Relational Biology Symp.} Argentina; \emph{CERN Preprint No.EXT-2004-067}. 

\bibitem{BGG2}
I.C. Baianu, Glazebrook, J. F. and G. Georgescu: 2004, Categories of Quantum Automata and 
N-Valued \L ukasiewicz Algebras in Relation to Dynamic Bionetworks, \textbf{(M,R)}--Systems and
Their Higher Dimensional Algebra, \emph{Abstract and Preprint of Report}. 

\bibitem{BHS2}
R. Brown R, P.J. Higgins, and R. Sivera.: \textit{``Non--Abelian Algebraic Topology''}. (\textit{vol.2 in preparation}
(2008)

\bibitem{BGB2}
R. Brown, J. F. Glazebrook and I. C. Baianu: A categorical and higher dimensional algebra framework for complex systems and spacetime structures, \emph{Axiomathes} \textbf{17}:409--493, (2007).

\bibitem{BM}
R. Brown and G. H. Mosa: Double algebroids and crossed modules of algebroids, University of Wales--Bangor, Maths Preprint, 1986.

\bibitem{BS}
R. Brown  and C.B. Spencer: Double groupoids and crossed modules,
\emph{Cahiers Top. G\'eom.Diff.} \textbf{17} (1976), 343--362.

\bibitem{LW1}
W.F. Lawvere: 1963. Functorial Semantics of Algebraic Theories. \emph{Proc. Natl. Acad. Sci. USA}, 50: 869--872

\bibitem{LW2}
W. F. Lawvere: 1966. The Category of Categories as a Foundation for Mathematics. , In {\em Proc. Conf. Categorical Algebra--La Jolla}, 1965, Eilenberg, S et al., eds. Springer --Verlag: Berlin, Heidelberg and New York, pp. 1--20.

\bibitem{LO68}
L. L$\ddot{o}$fgren: 1968. On Axiomatic Explanation of Complete Self--Reproduction. \emph{Bull. Math. Biophysics}, 
\textbf{30}: 317--348. 

\end{thebibliography}
%%%%%
%%%%%
\end{document}
