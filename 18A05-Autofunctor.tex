\documentclass[12pt]{article}
\usepackage{pmmeta}
\pmcanonicalname{Autofunctor}
\pmcreated{2013-03-22 13:07:33}
\pmmodified{2013-03-22 13:07:33}
\pmowner{mathcam}{2727}
\pmmodifier{mathcam}{2727}
\pmtitle{autofunctor}
\pmrecord{17}{33560}
\pmprivacy{1}
\pmauthor{mathcam}{2727}
\pmtype{Definition}
\pmcomment{trigger rebuild}
\pmclassification{msc}{18A05}

\endmetadata

% this is the default PlanetMath preamble.  as your knowledge
% of TeX increases, you will probably want to edit this, but
% it should be fine as is for beginners.

% almost certainly you want these
\usepackage{amssymb}
\usepackage{amsmath}
\usepackage{amsfonts}

% used for TeXing text within eps files
%\usepackage{psfrag}
% need this for including graphics (\includegraphics)
%\usepackage{graphicx}
% for neatly defining theorems and propositions
%\usepackage{amsthm}
% making logically defined graphics
%%%\usepackage{xypic}

% there are many more packages, add them here as you need them

% define commands here
\begin{document}
\PMlinkescapeword{induced}
Let $F\colon \mathcal{C}\to \mathcal{C}$ be an endofunctor on a category $\mathcal{C}$.  If $F$ is a bijection on both objects, $\mathrm{Ob}(\mathcal{C})$, and morphisms, $\mathrm{Mor}(\mathcal{C})$, then it is an autofunctor.

In short, an autofunctor is a full and faithful endofunctor $F\colon \mathcal{C}\to \mathcal{C}$ such that the mapping $b: \mathrm{Ob}(\mathcal{C})\to \mathrm{Ob}(\mathcal{C})$ which is induced by $F$ is a bijection.

An autofunctor $F\colon \mathcal{C}\to\mathcal{C}$ is naturally isomorphic to the identity functor $\mathrm{Id}_{\mathcal{C}}: \mathcal{C} \to \mathcal{C}$.
%%%%%
%%%%%
\end{document}
