\documentclass[12pt]{article}
\usepackage{pmmeta}
\pmcanonicalname{ExampleOfUniverse}
\pmcreated{2013-03-22 14:09:06}
\pmmodified{2013-03-22 14:09:06}
\pmowner{rspuzio}{6075}
\pmmodifier{rspuzio}{6075}
\pmtitle{example of universe}
\pmrecord{8}{35569}
\pmprivacy{1}
\pmauthor{rspuzio}{6075}
\pmtype{Example}
\pmcomment{trigger rebuild}
\pmclassification{msc}{18A15}
\pmclassification{msc}{03E30}
\pmrelated{CumulativeHierarchy}

% this is the default PlanetMath preamble.  as your knowledge
% of TeX increases, you will probably want to edit this, but
% it should be fine as is for beginners.

% almost certainly you want these
\usepackage{amssymb}
\usepackage{amsmath}
\usepackage{amsfonts}

% used for TeXing text within eps files
%\usepackage{psfrag}
% need this for including graphics (\includegraphics)
%\usepackage{graphicx}
% for neatly defining theorems and propositions
%\usepackage{amsthm}
% making logically defined graphics
%%%\usepackage{xypic}

% there are many more packages, add them here as you need them

% define commands here
\begin{document}
The simplest example of a universe consists of all sets gotten by starting with the empty set and repeatedly aggregating sets already known to lie in the universe.  That is to say, starting with $\emptyset$, we first form $\{\emptyset\}$.  Given these two elements of our universe, we can then form 
$\{\{\emptyset\}\}$ and $\{\emptyset,\{\emptyset\}\}$.  Given these four elements,
we then can form $\{\{\{\emptyset\}\}\}$, $\{\emptyset,\{\{\emptyset\}\}\}$, 
$\{\{\emptyset,\{\emptyset\}\}\}$ and several other sets.  We can repeat this process to obtain an infinite collection of finite sets. 

The set considered here also happens to be the set $V_\omega$ in the cumulative hierarchy.
%%%%%
%%%%%
\end{document}
