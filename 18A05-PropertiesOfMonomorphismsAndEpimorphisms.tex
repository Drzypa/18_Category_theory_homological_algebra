\documentclass[12pt]{article}
\usepackage{pmmeta}
\pmcanonicalname{PropertiesOfMonomorphismsAndEpimorphisms}
\pmcreated{2013-03-22 16:03:41}
\pmmodified{2013-03-22 16:03:41}
\pmowner{kompik}{10588}
\pmmodifier{kompik}{10588}
\pmtitle{properties of monomorphisms and epimorphisms}
\pmrecord{7}{38115}
\pmprivacy{1}
\pmauthor{kompik}{10588}
\pmtype{Theorem}
\pmcomment{trigger rebuild}
\pmclassification{msc}{18A05}

\endmetadata

% this is the default PlanetMath preamble. as your knowledge
% of TeX increases, you will probably want to edit this, but
% it should be fine as is for beginners.

% almost certainly you want these
\usepackage{amssymb}
\usepackage{amsmath}
\usepackage{amsfonts}
\usepackage{amsthm}

% used for TeXing text within eps files
%\usepackage{psfrag}
% need this for including graphics (\includegraphics)
%\usepackage{graphicx}
% for neatly defining theorems and propositions
%
% making logically defined graphics
%%\usepackage{xypic}

% there are many more packages, add them here as you need them

% define commands here

\newcommand{\sR}[0]{\mathbb{R}}
\newcommand{\sC}[0]{\mathbb{C}}
\newcommand{\sN}[0]{\mathbb{N}}
\newcommand{\sZ}[0]{\mathbb{Z}}

\newcommand{\R}[0]{\mathbb{R}}
\newcommand{\C}[0]{\mathbb{C}}
\newcommand{\N}[0]{\mathbb{N}}
\newcommand{\Z}[0]{\mathbb{Z}}


%\usepackage{bbm}
%\newcommand{\N}{\mathbbmss{N}}
%\newcommand{\Z}{\mathbbmss{Z}}
%\newcommand{\C}{\mathbbmss{C}}
%\newcommand{\R}{\mathbbmss{R}}
%\newcommand{\Q}{\mathbbmss{Q}}



\newcommand*{\norm}[1]{\lVert #1 \rVert}
\newcommand*{\abs}[1]{| #1 |}

\newcommand{\Map}[3]{#1:#2\to#3}
\newcommand{\Emb}[3]{#1:#2\hookrightarrow#3}
\newcommand{\Mor}[3]{#2\overset{#1}\to#3}

\newcommand{\Cat}[1]{\mathcal{#1}}
\newcommand{\Kat}[1]{\mathbf{#1}}
\newcommand{\Func}[3]{\Map{#1}{\Cat{#2}}{\Cat{#3}}}
\newcommand{\Funk}[3]{\Map{#1}{\Kat{#2}}{\Kat{#3}}}

\newcommand{\intrv}[2]{\langle #1,#2 \rangle}

\newcommand{\vp}{\varphi}
\newcommand{\ve}{\varepsilon}

\newcommand{\Invimg}[2]{\inv{#1}(#2)}
\newcommand{\Img}[2]{#1[#2]}
\newcommand{\ol}[1]{\overline{#1}}
\newcommand{\ul}[1]{\underline{#1}}
\newcommand{\inv}[1]{#1^{-1}}
\newcommand{\limti}[1]{\lim\limits_{#1\to\infty}}

\newcommand{\Ra}{\Rightarrow}

%fonts
\newcommand{\mc}{\mathcal}

%shortcuts
\newcommand{\Ob}{\mathrm{Ob}}
\newcommand{\Hom}{\mathrm{hom}}
\newcommand{\homs}[2]{\mathrm{hom(}{#1},{#2}\mathrm )}
\newcommand{\Eq}{\mathrm{Eq}}
\newcommand{\Coeq}{\mathrm{Coeq}}

%theorems
\newtheorem{THM}{Theorem}
\newtheorem{DEF}{Definition}
\newtheorem{PROP}{Proposition}
\newtheorem{LM}{Lemma}
\newtheorem{COR}{Corollary}
\newtheorem{EXA}{Example}

%categories
\newcommand{\Top}{\Kat{Top}}
\newcommand{\Haus}{\Kat{Haus}}
\newcommand{\Set}{\Kat{Set}}

%diagrams
\newcommand{\UnimorCD}[6]{
\xymatrix{ {#1} \ar[r]^{#2} \ar[rd]_{#4}& {#3} \ar@{-->}[d]^{#5} \\
& {#6} } }

\newcommand{\RovnostrCD}[6]{
\xymatrix@C=10pt@R=17pt{
& {#1} \ar[ld]_{#2} \ar[rd]^{#3} \\
{#4} \ar[rr]_{#5} && {#6} } }

\newcommand{\RovnostrCDii}[6]{
\xymatrix@C=10pt@R=17pt{
{#1} \ar[rr]^{#2} \ar[rd]_{#4}&& {#3} \ar[ld]^{#5} \\
& {#6} } }

\newcommand{\RovnostrCDiiop}[6]{
\xymatrix@C=10pt@R=17pt{
{#1}  && {#3} \ar[ll]_{#2}  \\
& {#6} \ar[lu]^{#4} \ar[ru]_{#5} } }

\newcommand{\StvorecCD}[8]{
\xymatrix{
{#1} \ar[r]^{#2} \ar[d]_{#4} & {#3} \ar[d]^{#5} \\
{#6} \ar[r]_{#7} & {#8}
}
}

\newcommand{\TriangCD}[6]{
\xymatrix{ {#1} \ar[r]^{#2} \ar[rd]_{#4}&
{#3} \ar[d]^{#5} \\
& {#6} } }
\begin{document}
This entry deals with basic properties of monomorphisms and related notions (as extremal and
regular monomorphisms, retractions etc) as well as the dual notions.

Monomorphisms (epimorphisms, bimorphisms) are closed under composition.

\begin{PROP}
If $\Map fAB$, $\Map gBC$ are monomorphisms (epimorphisms, bimorphisms) then $g\circ f$ is
a monomorphism (epimorphism, bimorphism).
\end{PROP}

\begin{proof}
a) $g\circ f\circ h=g\circ f\circ k$ $\Rightarrow$ $f\circ h=f\circ k$ $\Rightarrow$ $h=k$ \\
b) $h\circ g\circ f=k\circ g\circ f$ $\Rightarrow$ $h\circ g=k\circ g$ $\Rightarrow$ $h=k$ \\
c) An easy corollary of a) and b).
\end{proof}

\begin{PROP} %\label{GFJEMONOIMPFJEMONO}
Let $f\circ g$ be a monomorphism. Then $g$ is a monomorphism.

\PMlinkname{Dual}{DualityPrinciple} claim: If $g\circ f$ is an epimorphism, then $g$ is an
epimorphism.
\end{PROP}

\begin{proof}
$g\circ h=g\circ k$ $\Rightarrow$ $f\circ g\circ h=f\circ g\circ k$ $\Rightarrow$ $h=k$
\end{proof}

Retractions and sections are closed under composition.

\begin{PROP} %\label{SKLADANIERETR}
If $\Map fAB$, $\Map gBC$ are retractions (sections, isomorphisms), then $g\circ f$ is
a retraction (section, isomorphism).
\end{PROP}

\begin{proof}
Suppose we are given $\Map hBA$, $\Map kCB$ such that $f\circ h=id_B$, $g\circ k=id_C$. Then
$(g\circ f)\circ(h\circ k)=g\circ(f\circ h)\circ k=g\circ id_B\circ k=g\circ k=id_C$. Thus we
have shown the first part of the claim. The second part is dual to the first one and the
third one follows from the first two.
\end{proof}

\begin{PROP} %\label{GFJERETRIMPFJERETR}
Let $\Map fAB$, $\Map gBC$ be morphisms. If $g\circ f$ is a section then $f$ is a section. If
$g\circ f$ is a retraction then $g$ is a retraction.
\end{PROP}

\begin{proof}
If $g\circ f$ is a section then there exists a morphism $h$ such that $h\circ(g\circ
f)=(h\circ g)\circ f=id_A$, thus $f$ is a section as well.
\end{proof}

\begin{PROP} %\label{korejemonom}
Every section is a monomorphism. Every retraction is an epimorphism.
\end{PROP}

\begin{proof}
Let $\Map fAB$ be a section and $\Map gBA$ be the left inverse to $f$, i.e., $\Map gBA$,
$g\circ f=id_A$. If $f\circ h=f\circ k$ then $h=id_A\circ h=g\circ f\circ h=g\circ f\circ
k=id_A\circ k=k$. The duality principle yields the second part of the claim.
\end{proof}

Recall that a morphism is called an isomorphism if it is a section and a retraction at the same time.

\begin{LM}
If $\Map fAB$, $\Map{g,h}BA$ are morphisms such that $g\circ f=id_A$ and $f\circ h=id_B$ then
$g=h$.
\end{LM}

\begin{proof}
$h=id_A\circ h=(g\circ f)\circ h=g\circ(f\circ h)=g\circ id_B=g$
\end{proof}

\begin{PROP}
A morphism $\Map fAB$ is an isomorphism if and only if there exists a morphism $\Map gBA$
such that $g\circ f=id_A$, $f\circ g=id_B$. The morphism $g$ is determined uniquely.
\end{PROP}

\begin{proof}
The implication $\boxed{\Leftarrow}$ is obvious. The implication $\boxed{\Rightarrow}$
follows from the above lemma.
\end{proof}

The morphism $g$ from the above proposition is called the inverse of $f$ and denoted by $f^{-1}$.

As an easy corollary we get:

\begin{PROP} %\label{FJEIZOMIMPINVFJEIZOM}
If $f$ is an isomorphism then also $\inv f$ is an isomorphism.
\end{PROP}

%%%%%
%%%%%
\end{document}
