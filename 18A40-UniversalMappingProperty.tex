\documentclass[12pt]{article}
\usepackage{pmmeta}
\pmcanonicalname{UniversalMappingProperty}
\pmcreated{2013-03-22 15:38:10}
\pmmodified{2013-03-22 15:38:10}
\pmowner{MichaelMcCliment}{20205}
\pmmodifier{MichaelMcCliment}{20205}
\pmtitle{universal mapping property}
\pmrecord{15}{37566}
\pmprivacy{1}
\pmauthor{MichaelMcCliment}{20205}
\pmtype{Definition}
\pmcomment{trigger rebuild}
\pmclassification{msc}{18A40}
\pmsynonym{universal}{UniversalMappingProperty}
\pmsynonym{co-universal}{UniversalMappingProperty}
\pmrelated{AdjointFunctor}
\pmrelated{UnitOfAnAdjunction}
\pmrelated{LimitingCone}
\pmdefines{universal morphism}
\pmdefines{universality}
\pmdefines{universal arrow}
\pmdefines{universal property}

\endmetadata

\usepackage{latexsym}
\usepackage{amssymb}
\usepackage{amsmath}
\usepackage{amsfonts}
\usepackage{amsthm}
\usepackage{graphicx}

%%\usepackage{xypic}

%-----------------------------------------------------

%       Standard theoremlike environments.

%       Stolen directly from AMSLaTeX sample

%-----------------------------------------------------

%% \theoremstyle{plain} %% This is the default

\newtheorem{thm}{Theorem}

\newtheorem{coro}[thm]{Corollary}

\newtheorem{lem}[thm]{Lemma}

\newtheorem{lemma}[thm]{Lemma}

\newtheorem{prop}[thm]{Proposition}

\newtheorem{conjecture}[thm]{Conjecture}

\newtheorem{conj}[thm]{Conjecture}

\newtheorem{defn}[thm]{Definition}

\newtheorem{remark}[thm]{Remark}

\newtheorem{ex}[thm]{Example}



%\countstyle[equation]{thm}



%--------------------------------------------------

%       Item references.

%--------------------------------------------------


\newcommand{\exref}[1]{Example-\ref{#1}}

\newcommand{\thmref}[1]{Theorem-\ref{#1}}

\newcommand{\defref}[1]{Definition-\ref{#1}}

\newcommand{\eqnref}[1]{(\ref{#1})}

\newcommand{\secref}[1]{Section-\ref{#1}}

\newcommand{\lemref}[1]{Lemma-\ref{#1}}

\newcommand{\propref}[1]{Prop\-o\-si\-tion-\ref{#1}}

\newcommand{\corref}[1]{Cor\-ol\-lary-\ref{#1}}

\newcommand{\figref}[1]{Fig\-ure-\ref{#1}}

\newcommand{\conjref}[1]{Conjecture-\ref{#1}}


% Normal subgroup or equal.

\providecommand{\normaleq}{\unlhd}

% Normal subgroup.

\providecommand{\normal}{\lhd}

\providecommand{\rnormal}{\rhd}
% Divides, does not divide.

\providecommand{\divides}{\mid}

\providecommand{\ndivides}{\nmid}


\providecommand{\union}{\cup}

\providecommand{\bigunion}{\bigcup}

\providecommand{\intersect}{\cap}

\providecommand{\bigintersect}{\bigcap}








\newcommand{\Hom}{\operatorname{Hom}}
\newcommand{\KVect}{\operatorname{\mathit{K}-Vect}}
\newcommand{\KAlg}{\operatorname{\mathit{K}-Alg}}


\begin{document}
\PMlinkescapeword{areas}
\PMlinkescapeword{attribute}

\section{Introduction.}

A strong attribute of categories is a uniform description of many seemingly
unrelated concepts from diverse interest areas of mathematics.  For example, 
Cartesian products produce new sets from old just as direct product produce
new groups from old and Cartesian products of topologies are again topologies.
To make these notions uniform one can provide categorical definitions.  Because
categories assume as their constituents only abstract objects and morphisms,
all the properties of such constructions must be given as properties of morphisms
between objects.  Those constructions which can be characterized by means of the
existence of a \emph{unique} morphism are generally grouped under the heading
of \emph{universal mapping properties} \cite[p. 57]{Hungerford}.  A precise
definition will follow below.

As is common with category theory, there are generally not many theorems
that can be proved at a categorical level and set to apply to all universal 
mapping properties.  The exception is a generic proof of uniqueness of objects 
with a universal mapping property.  It should be emphasized that most universal 
mapping properties are a result of the underlying structure of the categories in 
question and not provided for abstract reasons.  (For concrete examples 
consider the construction of free groups or free Lie algebras.)

\begin{remark}
There are contingents that use the phrase \emph{universal property} especially in
the vernacular.  Mac Lane does so in passing but his use is not as a defined term but
rather as an intuitive concept to be refined to a strict definition later \cite[p. 2]{MacLane}

Because the property applies to the uniqueness of mappings the title \emph{universal 
mapping property} remains the choice of most mathematical references in print.
Authors interested in shorter titles will sometimes substitute Mac Lane's concept
of \emph{universal arrows}\cite[III.1]{MacLane} -- a rigorous definition which will 
be explained below and captures the spirit of the universal mapping property as
presently described.  The preferred term of morphism for arrow motivates the modern
title \emph{universal morphism}.
\end{remark}

Universal mapping properties arise in such constructions
as: direct products and direct sums, free groups, free algebras, product topology,
Stone-\v{C}ech compactification, tensor product, exterior algebra, inverse and direct limit, 
pullbacks and pushouts.

\section{Universal Morphisms and the Universal Mapping Property}

We follow \cite[III.1]{MacLane} for our treatment of universal arrows and 
universal mapping properties.

\begin{defn}
Let $U: \mathcal{D} \to \mathcal{C}$ be a functor from a category
$\mathcal{D}$ to a category $\mathcal{C}$, and let $X$ be an
object of $\mathcal{C}$. A \emph{universal morphism} from $X$ to $U$
consists of a pair $(A, \phi)$ where $A$ is an object of $\mathcal{D}$
and $\phi : X \to U(A)$ is a morphism in $\mathcal{C}$, such
that the following \emph{universal mapping property} is satisfied: 
\begin{quote}
  Whenever $Y$ is an object of $\mathcal{D}$ and $f \colon X \to U(Y)$ is a
  morphism in $\mathcal{C}$, then there exists a \textbf{unique} morphism 
  $g:A\to Y$ such that the following diagram commutes.
  \begin{figure}[h]
    \[\xymatrix{
                         & U(A)\ar[dd] &  A\ar@{-->}[dd]^g\\
         X\ar[ur]^{\phi}\ar[dr]_f &			   & 			\\
         				 & U(Y)	       & Y.           
    }\]
    \caption{A universal morphism from $X$ to $U$}
  \end{figure}
\end{quote}
\end{defn}

The existence of the morphism g intuitively expresses the fact
that $A$ is ``general enough'', while the uniqueness of the morphism
ensures that $A$ is ``not too general''.

One can also consider the categorical dual of the above definition by
reversing all the arrows. Let $F:\mathcal{C}\to \mathcal{D}$ be a functor, and let $X$
be an object of $\mathcal{D}$. A
co-universal morphism from $F$ to $X$ consists of a pair $(A, \phi)$
where $A$ is an object of $\mathcal{C}$ and $\phi:F(A) \to X$ is a
morphism in $\mathcal{D}$, such that the following universal mapping property is
satisfied:
\begin{quote}
  Whenever $Y$ is an object of $\mathcal{C}$ and $f \colon F(Y) \to X$ is a morphism
  in $\mathcal{D}$, then there exists a \textbf{unique} morphism $g \colon Y \to A$
  such that the following diagram commutes.
  \begin{figure}[h]
    \centering
    \[\xymatrix{
 		 Y\ar@{-->}[dd]^g & F(Y)\ar[dd]^{F(g)}\ar[rd]^f & \\
 		   &      &  X.\\
 		 A & F(A)\ar[ur]_{\phi} &     
    }\]
    \caption{A co-universal morphism from $F$ to $X$}
  \end{figure}
\end{quote}
To avoid ambiguity, some authors may call one of these constructions a
universal morphism and the other one a co-universal morphism.

\section{Properties}

\paragraph{Existence and uniqueness.}
Defining a quantity does not guarantee its existence. Given a functor
$U$ and an object $X$ as above, there may or may not exist a universal
morphism from $X$ to $U$ (or from $U$ to $X$). If, however, a universal
morphism $(A, \phi)$ does exists, then it is unique up to a unique
isomorphism. That is, if $(A', \phi')$ is another such pair
then there exists a unique isomorphism $g \colon A \to A'$ such
that $\phi' = U(g)\phi$. This is easily seen by substituting
$(A', \phi')$ for $(Y, f)$ in the definition of the universal mapping
property.

\paragraph{Equivalent formulations.}
The definition of a universal morphism can be rephrased in a variety
of ways. Let $U$ be a functor from $\mathcal{D}$ to $\mathcal{C}$, and let $X$ be
an object of $\mathcal{C}$. Then the following statements are equivalent
\begin{itemize}
\item $(A, \phi)$ is a universal morphism from $X$ to $U$;
\item $(A, \phi)$ is an initial object of the comma category $(X
  \downarrow U)$;
\item $(A, \phi)$ is
  a representable functor of $\Hom_{\mathcal{C}}(X,U(-))$.
\end{itemize}

The dual statements are also equivalent
\begin{itemize}
\item $(A, \phi)$ is a universal morphism from $F$ to $X$;
\item $(A, \phi)$ is a terminal object of the comma category $(F
  \downarrow X)$;
\item $(A, \phi)$ is a representable functor|representation of
  $\Hom_{\mathcal{C}}(F(-),X)$.
\end{itemize}


\paragraph{Relation to adjoint functors.}
Suppose that $(A_1, \phi_1)$ is a universal morphism from $X_1$ to $U$
and that $(A_2,\phi_2)$ is also a universal morphism from $X_2$ to $U$.
By the universal mapping property, given any morphism $h \colon X_1 \to X_2$
there exists a unique morphism $g \colon A_1 \to A_2$ such that the
following diagram commutes
\begin{figure}[h]
  \[\xymatrix{
     X_1\ar[r]^{\phi_1}\ar[d]_h & U(A_1)\ar[d]_{U(g)} & A_1\ar@{-->}[d]^g \\
     X_2\ar[r]_{\phi_2} & U(A_2) & A_2 \\
   }\]
\end{figure}
If \textbf{every} object $X_i$ of $\mathcal{C}$ admits a universal morphism to
$U$, then the assignment $X_i\mapsto A_i$ and $h \mapsto g$ defines a
functor $V\colon \mathcal{C} \to \mathcal{D}$. The maps $\phi_i$ then define a natural
transformation from $1_{\mathcal{C}}$ (the identity functor on $\mathcal{C}$) to $U
V$. The functors $(V, U)$ are then a pair of adjoint functors, with
$V$ left-adjoint to $U$.
Similar statements apply to the dual situation of morphisms from
$U$. If such morphisms exist for every $X\in \mathcal{C}$ one obtains a
functor $V \colon \mathcal{C} \to \mathcal{D}$ which is right-adjoint to $U$.

Indeed, all pairs of adjoint functors arise from universal
constructions in this manner. Let $F$ and $G$ be a pair of adjoint
functors with unit $\eta$ and co-unit $\epsilon$ (see the article on
adjoint functors for the definitions). Then we have a universal
morphism for each object in $\mathcal{C}$ and $\mathcal{D}$.
\begin{itemize}
\item For each object $X\in\mathcal{C}$, the pair $(F(X), \eta_X)$ is a
  universal morphism from $X$ to $G$. That is, for all $f \colon X \to
  G(Y)$, there exists a unique $g \colon F(X) \to Y$ for which the
  diagrams below commute.
\item Dually, for each object $Y \in \mathcal{D}$, the pair $(G(Y),\epsilon_Y)$
  is a universal morphism from $F$ to $Y$. That is, 
  for all $g \colon F(X) \to Y$ there exists a unique $f \colon X \to
  G(Y)$ for which the following diagrams commute.
  \begin{figure}[t]
    \[
     \xymatrix{
           & GF(X)\ar[dd]^{G(g)}     &  F(X)\ar[dd]^{F(f)}\ar[dr]^g  \\
        X\ar[ur]^{\eta_X}\ar[dr]_f  & &  & Y      \\
           & G(Y)  & FG(Y)\ar[ur]_{\epsilon_Y}
    }
    \]
    \caption{Universal mapping property of a pair of adjoint functors}
  \end{figure}
\end{itemize}

Universal constructions are more general than adjoint functor pairs.  A
universal construction is like an optimization problem; it gives rise
to an adjoint pair if and only if this problem has a solution for
every object of $\mathcal{C}$ (equivalently, for every object of $\mathcal{D}$).

\section{Examples}

\paragraph{Tensor algebras.}
Let $\KVect$ be the category of vector spaces over a field $K$, and
let $\KAlg$ be the category of $K$-algebras (assumed to be unital and
associative). Let $U$ be the forgetful functor which assigns to each
algebra its underlying vector space.  Given any vector space $V$ over
$K$ we can construct the tensor algebra $T(V)$ of $V$. The universal
mapping property of the tensor algebra expresses the fact that the pair
$(T(V), \iota)$, where $\iota \colon V \to T(V)$ is the natural
inclusion map, is a universal morphism from $V$ to $U$.  Since this
construction works for any vector space $V$, we conclude that $T$ is a
functor from $\KVect$ to $\KAlg$. This functor is left-adjoint to the
forgetful functor $U$.

\paragraph{Kernels.}
Suppose $\mathcal{C}$ is a category with zero morphisms $0_{AB}\in\Hom_{\mathcal{C}}(A,B)$ (such
as the category of groups) and let $f \colon X \to Y$ be a morphism in
$\mathcal{C}$. A kernel of $f$ is any morphism $k\colon K \to X$ such that
\begin{itemize}
\item the composition $f k$ is the zero morphism from $K$ to $Y$;
\item given any morphism $k'\colon K' \to X$ such that $f k'$ is the
  zero morphism, there is a unique morphism $u\colon K' \to K$ such
  that $k u = k'$.
\end{itemize}
To understand this in the framework of the general setting above, we
let $\mathcal{D}$ be the category of morphisms in $\mathcal{C}$. The objects of $\mathcal{D}$
are morphisms $f \colon X \to Y$ in $\mathcal{C}$, and a morphism from $f
\colon X \to Y$ to $g \colon S \to T$ is a commutative square whose
sides are a pair of morphisms $\alpha \colon X \to S$ and $\beta
\colon Y \to T$.  Diagramatically,
\[\xymatrix{
   X\ar[r]^f\ar[d]_{\alpha} & Y\ar[d]^{\beta} \\
   S\ar[r]^g  & T
}\]

Let $F \colon \mathcal{C} \to \mathcal{D}$ be the functor that maps an object
$K\in\mathcal{C}$ to the zero morphism $0_{\scriptscriptstyle KK} \colon K \to
K $, and that maps a morphism $r \colon K \to L$ to the trivial square
with sides $(r,0_{KL})$.  Now,  let $f$ be
an object $\mathcal{D}$ (which is the same thing as a morphism $f \colon X \to
Y$ in the category $\mathcal{C}$).  A kernel $k\colon K\to X$, if it exists,
is the same thing as an object $K\in \mathcal{C}$ and a morphism $(k,0)$ in
$\mathcal{D}$ that satisfies the co-universal property expressed by the
diagram below. 
I.e., a kernel is the same thing as a universal morphism
from $F$ to $f$.
\[\xymatrix{
  K'\ar@{-->}[dr]_r  & & K'\ar[dd]_{k'}\ar@{-->}[dr]^r\ar[rr]^0 & & K'\ar[dd]^0 \ar@{-->}[dr]^0\\
  & K   &    &  K\ar[ld]_k \ar[rr]_{0~~~}  & & K\ar[ld]^0 \\
     & & X\ar[rr]_f & & Y
}\]

\paragraph{Limits and colimits.}
Limits and colimits are important special
cases of universal constructions. Let $\mathcal{J}$ and $\mathcal{C}$ be categories
with $\mathcal{J}$ small ($\mathcal{J}$ is to be thought of as an index
category) and let $\mathcal{C}^\mathcal{J}$ be the corresponding functor
category of functors from $\mathcal{J}$ to $\mathcal{C}$. The \emph{diagonal functor}
$\Delta \colon \mathcal{C} \to 
\mathcal{C}^\mathcal{J}$ is the functor that maps each object $N\in\mathcal{C}$
to the constant functor $\Delta(N) \colon \mathcal{J} \to \mathcal{C}$
(i.e., $\Delta(N)(X) = N$ for each $X\in \mathcal{J}$.)
Given a functor $F \colon \mathcal{J} \to \mathcal{C}$ (thought of as an object in
$\mathcal{C}^\mathcal{J}$), the limit of $F$, if it exists, is
nothing but a universal morphism from $\Delta$ to $F$. Dually, the
colimit of $F$ is a co-universal morphism from $F$ to $\Delta$.

\section{Motivational remarks.}

For the most part, the objects in a category which are constructed in terms of universal mapping properties are not new to the theorist of these categories.  For example,
topologist long knew that a new topology results from a Cartesian product
of topologies.  However the general vocabulary allows for easier detection and understanding of functors between different categories.  For example, the
homology functors take the coproducts -- \PMlinkname{wedge products}{WedgeProductOfPointedTopologicalSpaces} in topology -- to
coproducts of modules -- direct sums in module categories.  Proofs of these
facts are in no way facilitated by category theory but the terminology is
at least uniform and thus easier to conceptualize and express.

Once one recognizes a certain construction as given by a universal mapping
property, one gains several benefits.
\begin{itemize}
\item Universal mapping properties define objects up to a unique isomorphism.
  One strategy to prove that two objects are isomorphic is therefore
  to show that they satisfy the same universal mapping property.
\item The concrete details of a given construction may be messy, but
  if the construction satisfies a universal mapping property, one can forget
  all those details; all there is to know about the construct is
  already contained in the universal mapping property. Proofs often become
  short and elegant if the universal mapping property is used rather than the
  concrete details.
\item If the universal construction can be carried out for every $X$
  in $\mathcal{C}$, then we know that we obtain a functor from $\mathcal{C}$ to $D$.
  For example, forming kernels is functorial; every 
  commutative square
  $(\alpha,\beta)$ from the morphism $f$ to the morphism $g$ induces a
  morphism from the kernel of $f$ to the kernel of $g$.
\item Furthermore, if such a functor can be formed, it is a right or
  left adjoint to $U$. But right adjoints commute with limits, and
  left adjoints commute with colimits! So, looking back at the
  previous example, we can immediately conclude that the kernel of a
  product is equal to the product of the kernels.
\end{itemize}

Universal mapping properties of various topological constructions were
presented by Pierre Samuel in 1948. They were later used extensively
by Bourbaki. The closely related concept of adjoint functors was
introduced independently by Daniel Kan in 1958.

\bibliographystyle{amsplain}
\begin{thebibliography}{10}
\bibitem{Hungerford}
Thomas W. Hungerford \emph{Algebra}, Springer-Verlag, New York, (1974).
\bibitem{Cohen}
Paul M. Cohen, \emph{Universal Algebra}, D.Reidel
  Publishing, Holland, (1981). ISBN 90-277-1213-1.
\bibitem{MacLane}
Saunders Mac Lane \emph{Categories for the Working Mathematician} 2nd ed. Graduate 
Texts in Mathematics 5. Springer, (1998). ISBN 0-387-98403-8.
\end{thebibliography}

\section{Acknowledgements and Notes}
This entries was adapted, for the most part, from the Wikipedia entry
entitled \PMlinkexternal{Universal
  Property}{http://en.wikipedia.org/wiki/Universal_property}.  In turn
  much of Wikipedia's entry appears in \cite[III.1]{MacLane} were the interested reader is directed for further details.
%%%%%
%%%%%
\end{document}
