\documentclass[12pt]{article}
\usepackage{pmmeta}
\pmcanonicalname{GrothendieckCategory}
\pmcreated{2013-03-22 18:12:08}
\pmmodified{2013-03-22 18:12:08}
\pmowner{bci1}{20947}
\pmmodifier{bci1}{20947}
\pmtitle{Grothendieck category}
\pmrecord{135}{40781}
\pmprivacy{1}
\pmauthor{bci1}{20947}
\pmtype{Definition}
\pmcomment{trigger rebuild}
\pmclassification{msc}{18A99}
\pmclassification{msc}{18-00}
\pmclassification{msc}{18E15}
\pmsynonym{Ab5 category}{GrothendieckCategory}
\pmsynonym{Abelian category with generators}{GrothendieckCategory}
%\pmkeywords{$Ab5$- category}
%\pmkeywords{definition of abelian categories}
%\pmkeywords{definition of Grothendieck categories}
%\pmkeywords{localization in Grothendieck categories}
\pmrelated{AbelianCategory}
\pmrelated{GrothendieckGroup}
\pmrelated{DirectLimit}
\pmrelated{DualityInMathematics}
\pmrelated{IndexOfCategoryTheory}
\pmrelated{GeneratorOfACategory}
\pmrelated{PreAdditiveFunctors}
\pmrelated{ExamplesOfAbelianCategory}
\pmrelated{CategoryTheory}
\pmrelated{CategoricalSequence}
\pmrelated{FundamentalGroupoidFunctor}
\pmrelated{C_2Category}
\pmrelated{C_3Category}
\pmrelated{Gab}
\pmdefines{$Ab3$- category}
\pmdefines{preabelian category}
\pmdefines{cogenerator}
\pmdefines{co-Grothendieck category}
\pmdefines{Abelian category}
\pmdefines{Ab3 category}
\pmdefines{Ab5-category}
\pmdefines{Ab5* category}
\pmdefines{generator}
\pmdefines{family of category generators}
\pmdefines{canonical injection}

\endmetadata

% this is the default PlanetMath preamble.  as your 
% almost certainly you want these
\usepackage{amssymb}
\usepackage{amsmath}
\usepackage{amsfonts}
% used for TeXing text within eps files
%\usepackage{psfrag}
% need this for including graphics (\includegraphics)
%\usepackage{graphicx}
% for neatly defining theorems and propositions
%\usepackage{amsthm}
% making logically defined graphics
%%%\usepackage{xypic}

% there are many more packages, add them here as you need them

% define commands here
\usepackage{amsmath, amssymb, amsfonts, amsthm, amscd, latexsym, enumerate}
\usepackage{xypic, xspace}
\usepackage[mathscr]{eucal}
\usepackage[dvips]{graphicx}
\usepackage[curve]{xy}

\setlength{\textwidth}{6.5in}
%\setlength{\textwidth}{16cm}
\setlength{\textheight}{9.0in}
%\setlength{\textheight}{24cm}

\hoffset=-.75in     %%ps format
%\hoffset=-1.0in     %%hp format
\voffset=-.4in


\theoremstyle{plain}
\newtheorem{lemma}{Lemma}[section]
\newtheorem{proposition}{Proposition}[section]
\newtheorem{theorem}{Theorem}[section]
\newtheorem{corollary}{Corollary}[section]

\theoremstyle{definition}
\newtheorem{definition}{Definition}[section]
\newtheorem{example}{Example}[section]
%\theoremstyle{remark}
\newtheorem{remark}{Remark}[section]
\newtheorem*{notation}{Notation}
\newtheorem*{claim}{Claim}

\renewcommand{\thefootnote}{\ensuremath{\fnsymbol{footnote}}}
\numberwithin{equation}{section}

\newcommand{\Ad}{{\rm Ad}}
\newcommand{\Aut}{{\rm Aut}}
\newcommand{\Cl}{{\rm Cl}}
\newcommand{\Co}{{\rm Co}}
\newcommand{\DES}{{\rm DES}}
\newcommand{\Diff}{{\rm Diff}}
\newcommand{\Dom}{{\rm Dom}}
\newcommand{\Hol}{{\rm Hol}}
\newcommand{\Mon}{{\rm Mon}}
\newcommand{\Hom}{{\rm Hom}}
\newcommand{\Ker}{{\rm Ker}}
\newcommand{\Ind}{{\rm Ind}}
\newcommand{\IM}{{\rm Im}}
\newcommand{\Is}{{\rm Is}}
\newcommand{\ID}{{\rm id}}
\newcommand{\grpL}{{\rm GL}}
\newcommand{\Iso}{{\rm Iso}}
\newcommand{\rO}{{\rm O}}
\newcommand{\Sem}{{\rm Sem}}
\newcommand{\SL}{{\rm Sl}}
\newcommand{\St}{{\rm St}}
\newcommand{\Sym}{{\rm Sym}}
\newcommand{\Symb}{{\rm Symb}}
\newcommand{\SU}{{\rm SU}}
\newcommand{\Tor}{{\rm Tor}}
\newcommand{\U}{{\rm U}}

\newcommand{\A}{\mathcal A}
\newcommand{\Ce}{\mathcal C}
\newcommand{\D}{\mathcal D}
\newcommand{\E}{\mathcal E}
\newcommand{\F}{\mathcal F}
%\newcommand{\grp}{\mathcal G}
\renewcommand{\H}{\mathcal H}
\renewcommand{\cL}{\mathcal L}
\newcommand{\Q}{\mathcal Q}
\newcommand{\R}{\mathcal R}
\newcommand{\cS}{\mathcal S}
\newcommand{\cU}{\mathcal U}
\newcommand{\W}{\mathcal W}

\newcommand{\bA}{\mathbb{A}}
\newcommand{\bB}{\mathbb{B}}
\newcommand{\bC}{\mathbb{C}}
\newcommand{\bD}{\mathbb{D}}
\newcommand{\bE}{\mathbb{E}}
\newcommand{\bF}{\mathbb{F}}
\newcommand{\bG}{\mathbb{G}}
\newcommand{\bK}{\mathbb{K}}
\newcommand{\bM}{\mathbb{M}}
\newcommand{\bN}{\mathbb{N}}
\newcommand{\bO}{\mathbb{O}}
\newcommand{\bP}{\mathbb{P}}
\newcommand{\bR}{\mathbb{R}}
\newcommand{\bV}{\mathbb{V}}
\newcommand{\bZ}{\mathbb{Z}}

\newcommand{\bfE}{\mathbf{E}}
\newcommand{\bfX}{\mathbf{X}}
\newcommand{\bfY}{\mathbf{Y}}
\newcommand{\bfZ}{\mathbf{Z}}

\renewcommand{\O}{\Omega}
\renewcommand{\o}{\omega}
\newcommand{\vp}{\varphi}
\newcommand{\vep}{\varepsilon}

\newcommand{\diag}{{\rm diag}}
\newcommand{\grp}{\mathcal G}
\newcommand{\dgrp}{{\mathsf{D}}}
\newcommand{\desp}{{\mathsf{D}^{\rm{es}}}}
\newcommand{\grpeod}{{\rm Geod}}
%\newcommand{\grpeod}{{\rm geod}}
\newcommand{\hgr}{{\mathsf{H}}}
\newcommand{\mgr}{{\mathsf{M}}}
\newcommand{\ob}{{\rm Ob}}
\newcommand{\obg}{{\rm Ob(\mathsf{G)}}}
\newcommand{\obgp}{{\rm Ob(\mathsf{G}')}}
\newcommand{\obh}{{\rm Ob(\mathsf{H})}}
\newcommand{\Osmooth}{{\Omega^{\infty}(X,*)}}
\newcommand{\grphomotop}{{\rho_2^{\square}}}
\newcommand{\grpcalp}{{\mathsf{G}(\mathcal P)}}

\newcommand{\rf}{{R_{\mathcal F}}}
\newcommand{\grplob}{{\rm glob}}
\newcommand{\loc}{{\rm loc}}
\newcommand{\TOP}{{\rm TOP}}

\newcommand{\wti}{\widetilde}
\newcommand{\what}{\widehat}

\renewcommand{\a}{\alpha}
\newcommand{\be}{\beta}
\newcommand{\grpa}{\grpamma}
%\newcommand{\grpa}{\grpamma}
\newcommand{\de}{\delta}
\newcommand{\del}{\partial}
\newcommand{\ka}{\kappa}
\newcommand{\si}{\sigma}
\newcommand{\ta}{\tau}

\newcommand{\med}{\medbreak}
\newcommand{\medn}{\medbreak \noindent}
\newcommand{\bign}{\bigbreak \noindent}

\newcommand{\lra}{{\longrightarrow}}
\newcommand{\ra}{{\rightarrow}}
\newcommand{\rat}{{\rightarrowtail}}
\newcommand{\ovset}[1]{\overset {#1}{\ra}}
\newcommand{\ovsetl}[1]{\overset {#1}{\lra}}
\newcommand{\hr}{{\hookrightarrow}}
\newcommand{\Univ}{\mathscr{U}}
\DeclareMathOperator{\liminv}{\varprojlim}
\DeclareMathOperator{\limdir}{\varinjlim}
\newcommand{\<}{{\langle}}

\def\baselinestretch{1.1}

\hyphenation{prod-ucts}

%\grpeometry{textwidth= 16 cm, textheight=21 cm}

\newcommand{\sqdiagram}[9]{$$ \diagram  #1  \rto^{#2} \dto_{#4}&
#3  \dto^{#5} \\ #6    \rto_{#7}  &  #8   \enddiagram
\eqno{\mbox{#9}}$$ }

\def\C{C^{\ast}}

\newcommand{\labto}[1]{\stackrel{#1}{\longrightarrow}}

%\newenvironment{proof}{\noindent {\bf Proof} }{ \hfill $\Box$
%{\mbox{}}
\newcommand{\midsqn}[1]{\ar@{}[dr]|{#1}}
\newcommand{\quadr}[4]
{\begin{pmatrix} & #1& \\[-1.1ex] #2 & & #3\\[-1.1ex]& #4&
 \end{pmatrix}}
\def\D{\mathsf{D}}

\begin{document}
\subsection{Introduction: generator, generator family and cogenerator definitions}

 Let $\mathcal{C}$ be a category. Moreover, let $\left\{U\right\}= \left\{U_i\right\}_{i \in I}$ be a family of objects of $\mathcal{C}$. The \emph{family} $\left\{U\right\}$ is said to be a \emph{family of generators} of the category $\mathcal{C}$ if for any object $A$ of $\mathcal{C}$ and any subobject $B$ of $A$, distinct from $A$, there is at least an index $i \in I$, and a morphism, $u : U_i \to A$, that cannot be factorized through the canonical injection $i : B \to A$. Then, an object $U$ of $\mathcal{C}$ is said to be a \emph{generator} of the category $\mathcal{C}$ provided that $U$ belongs to the family of generators $\left\{U_i\right\}_{i \in I}$ of $\mathcal{C}$ (\cite{NP65}).

 By duality, that is, by simply reversing all arrows in the above definition one obtains the notion of a 
{\em family of cogenerators} $\left\{U^*\right\}$ of the same category $\mathcal{C}$, and also the notion of {\em cogenerator} $U^*$ of $\mathcal{C}$, if all of the required, reverse arrows exist. Notably, in a groupoid-- regarded as a small category with all its morphisms invertible-- this is always possible, and thus a groupoid can always be cogenerated {\em via} duality. Moreover, any generator in the dual category $\mathcal{C}^{op}$ is a cogenerator of $\mathcal{C}$.  


\subsection{Ab-conditions: Ab3 and Ab5 conditions}
\begin{enumerate}

\item \emph{(Ab3)}. Let us recall that an \emph{Abelian} category $\mathcal{A}b$ is \emph{cocomplete} 
(or an $\mathcal{A}b3$-category) if it has arbitrary direct sums. 

\item \emph{(Ab5).}  A \emph{cocomplete Abelian category} $\mathcal{A}b$ is said to be an $\mathcal{A}b5$-category if for any directed family $\left\{A_i\right\}_{i \in I}$ of subobjects of $\mathcal{A}$, and for any subobject $B$ of 
$\mathcal{A}$, the following equation holds

$(\sum_{i \in I}A_i) \bigcap B = \sum_{i \in I} (A_i \bigcap B).$ 

\end{enumerate}

\subsubsection{Remarks}

\begin{itemize}
 
\item One notes that the condition \emph{Ab3} is \emph{equivalent to the existence of arbitrary direct limits}. 

\item Furthermore, \emph{Ab5} is equivalent to the following proposition: 
\emph{there exist inductive limits and the inductive limits over directed families of indices are exact}, 
that is, if $I$ is a directed set and  $0 \to A_i \to B_i \to C_i \to 0$ is an exact 
sequence for any $i \in I$, then 
$$0 \to \limdir{(A_i)} \to \limdir{(B_i)} \to  \limdir{(C_i)} \to 0$$ 
is also an exact sequence.

\item By duality, one readily obtains conditions \emph{Ab3*} and \emph{Ab5*} simply 
by reversing the arrows in the above conditions defining \emph{Ab3} and \emph{Ab5}.
\end{itemize}

\subsection{Grothendieck and co-Grothendieck categories}

\begin{definition} A {\em Grothendieck category} is an $\mathcal{\A}b5$ category
with a \PMlinkname{generator}{GeneratorOfACategory}. 
\end{definition}

As an example consider the category $\mathcal{\A}b$ of Abelian groups
such that if $\left\{X_i \right\}_{i \in I}$ is a family of abelian groups, then
a {\em direct product} $\Pi$ is defined by the Cartesian product $\Pi _i (X_i)$
with addition defined by the rule: $(x_i) + (y_i) = (x_i + y_i)$. 
One then defines a projection $\rho : \Pi _i (X_i) \rightarrow X_i$ given by 
$p_i ((x_i)) = x_i$. A {\em direct sum} is obtained by taking the appropriate subgroup
consisting of all elements $(x_i)$ such that $x_i = 0$ for all but a finite number of indices 
$i$. Then one also defines a {\em structural injection} , and it is straightforward
to prove that $\mathcal{\A}b$ is an $\mathcal{\A}b6$ and $\mathcal{\A}b4^*$
category. (\emph{viz}. p 61 in ref. \cite{NP65}). 

\begin{definition} A \emph{co-Grothendieck category} is an $\mathcal{A}b5^*$ category that has a set of cogenerators,
i.e., a category whose dual is a Grothendieck category.
\end{definition} 

\subsubsection{Remarks}

\begin{enumerate}

\item  Let $\mathcal{\A}$ be an Abelian category and $\mathcal{C}$ a small category. 
One defines then a functor $k_c: \mathcal{\A} \rightarrow [\mathcal{C},\mathcal{\A}]$ 
as follows: for any $X \in Ob \mathcal{\A}$, $k_{\mathcal{C}}(X) : \mathcal{C} \rightarrow \mathcal{\A}$ is the 
{\em constant functor} which is associated to $X$.  Then $\mathcal{\A}$ is an {\em $\mathcal{\A}b5$ category} (respectively, $\mathcal{\A}b5^*$), if and only if for any directed set $I$, as above, the functor $k_I$ has an exact left (or respectively, right) adjoint. 
\item  With $\mathcal{\A}b4$, $\mathcal{\A}b5$, $\mathcal{\A}b4^*$, and $\mathcal{\A}b6$ 
one can construct categories of (pre) additive functors.
\item  A \emph{preabelian category} is {\em an additive category with the additional ($\mathcal{\A}b1$) condition} that for any morphism $f$ in the category there exist also \emph{both} $ker f$ and $coker f$;
\item An \emph{Abelian category} can be then also defined as a \em{preabelian category} in which for any morphism $f:X \to Y$, the morphism $ \overline{f} : coim f \to im f$ is an isomorphism (the $\mathcal{\A}b2$ condition).

\end{enumerate}



\begin{thebibliography}{9}

\bibitem{AG4-sga}
Alexander Grothendieck et al. \emph{S\'eminaires en G\'eometrie Alg\`ebrique- 4}, Tome 1, Expos\'e 1 
(or the Appendix to Expos\'ee 1, by `N. Bourbaki' for more detail and a large number of results.),
AG4 is \PMlinkexternal{freely available}{http://modular.fas.harvard.edu/sga/sga/pdf/index.html} in French;
also available here is an extensive 
\PMlinkexternal{Abstract in English}{http://planetmath.org/?op=getobj&from=books&id=158}.


\bibitem{Alex84}
Alexander \PMlinkname{Grothendieck}{AlexanderGrothendieckABiographyOf}, 1984. ``Esquisse d'un Programme'', (1984 manuscript), {\em finally published in ``Geometric Galois Actions''}, L. Schneps, P. Lochak, eds., 
{\em London Math. Soc. Lecture Notes} {\bf 242}, Cambridge University Press, 1997, pp.5-48;
English transl., ibid., pp. 243-283. MR 99c:14034 .

\bibitem{Alex81}
Alexander Grothendieck, ``La longue marche in \'a travers la th\'eorie de Galois'' 
\emph{= ``The Long March Towards/Across the Theory of Galois''}, 1981 manuscript, University of Montpellier preprint series 1996, edited by J. Malgoire. 

\bibitem{NP65}
Nicolae Popescu. {\em Abelian Categories with Applications to Rings and Modules.},
Academic Press: New York and London, 1973 and 1976 edns., ({\em English translation by I. C. Baianu}.)

\bibitem{LS94}
Leila Schneps. 1994. 
\PMlinkexternal{The Grothendieck Theory of Dessins d'Enfants}{http://planetmath.org/?op=getobj&from=books&id=163}.
(London Mathematical Society Lecture Note Series), Cambridge University Press, 376 pp.

\bibitem{DHSL2k}
David Harbater and Leila Schneps. 2000.
\PMlinkexternal{Fundamental groups of moduli and the Grothendieck-Teichm\"uller group}{http://www.ams.org/tran/2000-352-07/S0002-9947-00-02347-3/home.html}, \emph{Trans. Amer. Math. Soc}. 352 (2000), 3117-3148. 
MSC: Primary 11R32, 14E20, 14H10; Secondary 20F29, 20F34, 32G15.

\end{thebibliography}

%%%%%
%%%%%
\end{document}
