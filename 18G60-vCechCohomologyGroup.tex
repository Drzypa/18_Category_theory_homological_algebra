\documentclass[12pt]{article}
\usepackage{pmmeta}
\pmcanonicalname{vCechCohomologyGroup}
\pmcreated{2013-03-22 14:43:14}
\pmmodified{2013-03-22 14:43:14}
\pmowner{Simone}{5904}
\pmmodifier{Simone}{5904}
\pmtitle{\v{C}ech cohomology group}
\pmrecord{5}{36346}
\pmprivacy{1}
\pmauthor{Simone}{5904}
\pmtype{Definition}
\pmcomment{trigger rebuild}
\pmclassification{msc}{18G60}
\pmsynonym{Cech cohomology group}{vCechCohomologyGroup}

\endmetadata

% this is the default PlanetMath preamble.  as your knowledge
% of TeX increases, you will probably want to edit this, but
% it should be fine as is for beginners.

% almost certainly you want these
\usepackage{amssymb}
\usepackage{amsmath}
\usepackage{amsfonts}

% used for TeXing text within eps files
%\usepackage{psfrag}
% need this for including graphics (\includegraphics)
%\usepackage{graphicx}
% for neatly defining theorems and propositions
%\usepackage{amsthm}
% making logically defined graphics
%%%\usepackage{xypic}

% there are many more packages, add them here as you need them

% define commands here
\begin{document}
Let $\mathcal F$ be a sheaf of abelian groups on a topological space $X$ and consider an \PMlinkescapetext{open covering} $\mathcal U=\{U_i\}_{i\in I}$ of $X$. For the sake of simplicity denote
$$
U_{i_0i_1\cdots i_q}=U_{i_0}\cap U_{i_1}\cap\dots\cap U_{i_q}.
$$
The group $\check C^q(\mathcal U,\mathcal F)$ of \v{C}ech $q$-cochains is the set of families
$$
c=(c_{i_0i_1\cdots i_q})\in\prod_{(i_0,\dots,i_q)\in I^{q+1}}\mathcal F(U_{i_0i_1\cdots i_q}).
$$
The group \PMlinkescapetext{structure} on $\check C^q(\mathcal U,\mathcal F)$ is the obvious one deduced from the addition law on sections of $\mathcal F$.

The \v{C}ech differential 
$$
\delta^q\colon\check C^q(\mathcal U,\mathcal F)\to\check C^{q+1}(\mathcal U,\mathcal F)
$$
is defined by the \PMlinkescapetext{formula}
$$
(\delta^q c)_{i_0\cdots i_{q+1}}=\sum_{0\le j\le q+1}(-1)^j c_{i_0\cdots\widehat{i_j}\cdots i_{q+1}}|_{U_{i_0\cdots i_{q+1}}},
$$
and we set $\check C^{q}(\mathcal U,\mathcal F)=0$, $\delta^q=0$ for $q<0$.
Easy computations show that $\delta^{q+1}\circ\delta^q=0$. We get therefore a cochain complex $(\check C^\bullet(\mathcal U,\mathcal F),\delta)$, called the complex of \v{C}ech cochains relative to the \PMlinkescapetext{covering} $\mathcal U$.

The $q$-th \emph{\v{C}ech cohomology group} of $\mathcal F$ relative to $\mathcal U$ is
$$
\check H^q(\mathcal U,\mathcal F)=H^q(\check C^\bullet(\mathcal U,\mathcal F)).
$$
%%%%%
%%%%%
\end{document}
