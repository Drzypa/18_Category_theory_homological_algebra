\documentclass[12pt]{article}
\usepackage{pmmeta}
\pmcanonicalname{LieAlgebraCohomology}
\pmcreated{2013-03-22 13:51:06}
\pmmodified{2013-03-22 13:51:06}
\pmowner{rmilson}{146}
\pmmodifier{rmilson}{146}
\pmtitle{Lie algebra cohomology}
\pmrecord{13}{34589}
\pmprivacy{1}
\pmauthor{rmilson}{146}
\pmtype{Definition}
\pmcomment{trigger rebuild}
\pmclassification{msc}{18G10}
\pmclassification{msc}{17B56}
\pmsynonym{cohomology}{LieAlgebraCohomology}
\pmrelated{LieAlgebra}

\usepackage{amsmath}
\usepackage{amsfonts}
\usepackage{amssymb}

\newtheorem{proposition}{Proposition}



\newcommand{\fg}{\mathfrak{g}}
\newcommand{\Hom}{\operatorname{Hom}}
\begin{document}
\section{Definition.}
Let $\fg$ be a finite-dimensional Lie algebra over a field $K$, and
let $M$ be a \PMlinkname{$\fg$-module}{RepresentationLieAlgebra}.  Our goal is to define a cochain complex
$C^{k}(\fg,M)$, whose cohomology is known as the Lie algebra cohomology
of $\fg$ with coefficients in $M$.  We define the space of $k$-cochains to be
\[C^k(\fg,M)=\Hom_K(\Lambda^k \fg,M),\quad k=0,1,\dots,\dim \fg,\]
the vector space of multilinear, alternating mappings from
$\overbrace{\fg\times \cdots\times \fg}^{k\text{ times}}$ to $M$.
\footnote{It should be understood that $C^0(\fg,M)\cong M$.}  
The coboundary operator 
$\delta: C^k(\fg,M)\to C^{k+1}(\fg,M)$ is defined to be
\begin{align*}
    (\delta \omega)(a_0,a_1,\dots, a_k) =& \sum_{0\leq i\leq k} (-1)^k 
    a_i \cdot\omega(\dots,\widehat{a_i},\dots) + \\ 
    &\ +
    \sum_{0\leq i<j\leq k}
    (-1)^{i+j} \omega([a_i,a_j],\dots, \widehat{a_i},\dots
    \widehat{a_j},\dots ),\quad \omega\in C^k(\fg,M),
\end{align*}
where $\widehat{a_i}$ indicates the omission of the argument $a_i$.
We leave it as an exercise for the reader to check that $\delta
\omega$, as defined above, is multi-linear, and alternating.  Another,
only slightly more challenging exercise, is to prove that $\delta^2
=0$.  The proof involves, in an essential way, the Jacobi identity,
and the commutator identity for the action of $\fg$ on $M$, namely
\[ a\cdot (b\cdot u) - b\cdot(a\cdot u) = [a,b] \cdot u,\quad a,b\in
\fg,\; u\in M.\]

If $\fg$ is the tangent space of a Lie group, and $M$ is the reals with $0$ $\fg$- action, then any alternating map on $\fg$ extends to a smooth alternating form on the Lie group, via (left)- translation by the group action.  The above formula for the coboundary gives the exterior derivative with respect to this extension.  

\section{Infinite-dimensional generalizations.}  

The above definition generalizes readily to infinite-dimensional Lie
algebras.  The notion of a linear mapping with an infinite-dimensional
domain is quite tricky, and so the key requirement of any such
generalization is some kind of restriction on the space of cochains.
Thus, de Rham cohomology of a manifold $X$ can be regarded as
$H^k(V(X), C^\infty(X))$, the cohomology of the Lie algebra of vector
fields with coefficients that are smooth functions, with the caveat
that the cochain spaces $C^k(V(X),C^\infty(X))$ are restricted to
$\Omega^k(X)$, the smooth differential forms\footnote{The
  evaluation of a cochain on a list of vector fields is given by
  contraction.}.  Another interesting infinite-dimensional
generalization is the Gelfand-Fuchs cohomology of $V(X)$.  Here we are
allowed cochains that are not simply linear combinations of the
components of a vector field, but that also include the derivatives of
these components.

\section{Applications.}
Owing to numerous and useful various applications, it's useful to list
the formulas for the first few coboundary operators:
\begin{align*}
  (\delta \alpha)(a) =& a\cdot \alpha;\\
  (\delta \beta)(a,b) =& a\cdot \beta(b) - b\cdot \beta(a)-
  \beta([a,b]);\\
  (\delta \gamma)(a,b,c) =& a\cdot \gamma(b,c) +b\cdot\gamma(c,a) +
  c\cdot\gamma(a,b) - \gamma([a,b],c) - \gamma([b,c],a) - \gamma([c,a],b),
\end{align*}
where $a,b,c\in \fg$, and where $\alpha,\beta,\gamma$ are $0, 1$, and
$2$-cochains, respectively.  In particular, for small $k$, the
cohomology groups $H^k(\fg,M)$ have certain interesting
interpretations. 

The first cohomology space, $H^1(\fg,K)$ is isomorphic as a vector
space to $\fg/[\fg,\fg]$, the abelianization of $\fg$.  More generally
$H^1(\fg,M)$ classifies, up to natural equivalence, Lie algebras
consisting of inhomogeneous operators
\[ a+\beta(a),\quad a\in \fg,\; \beta\in C^1(\fg,M). \] In more fancy
language, such operators are called derivations of $T(M)$, the tensor
algebra of $M$.  

The second cohomology space, $H^2(\fg,M)$, is naturally isomorphic to
the vector space of abelian extensions of $\fg$ by $M$.  Thinking of
$M$ as an abelian Lie algebra, such an extension is a Lie algebra
$\hat{\fg}$ that occurs in the short-exact sequence
\[0\to M\to\hat{\fg}\to\fg\to 0.\]

The third cohomology space has an interesting interpretation in terms
of deformations of something or other.  This is due to Murray
Gerstenhaber of U. Penn, but I've forgotten the details.

\section{Homological algebra.}
Generalizing a bit, Lie algebra cohomology is just the cohomology of a
particular kind of algebraic theory.  There are analogous cohomology
theories for groups, associative algebras, and commutative rings.  All
these theories can be unified by employing the notion of an injective
resolution.

Broadening the scope even further, we can employ category theory and
re-conceptualize Lie algebra cohomology as a functor from the category
of $\fg$-modules to the category of cochain complexes.  One begins with
the covariant, left-exact functor
\[ (-)^\fg: M\mapsto M^{\fg}=\{u\in M : \fg \cdot u=0 \}\] from
the category of $\fg$-modules to the category of $K$-vector spaces.
One then defines $H^k(\fg,-)$ to be the right-derived functors
$R^k((-)^{\fg})$.

\section{Historical notes.}
Lie algebra cohomology was first formalized in an influential 1948
paper by C. Chevalley and S. Eilenberg\cite{CE48}.  The aim was to
calculate the cohomology, in the topological sense, of a compact Lie
group by using the finite-dimensional data of the corresponding Lie
algebra.  In this they were inspired by an even earlier idea of Elie
Cartan, who was the first to announce that there was a connection
between the topology of a Lie group and the algebraic structure of the
underlying Lie algebra \cite{C28}.  What makes this story particularly
interesting is that Homological Algebra, as a subject, was launched by
the remarkable 1956 book\cite{CE56} by Cartan and Eilenberg called,
oddly enough ``Homological Algebra''.  However the Cartan involved
this time is not Elie, but Henri, the equally remarkable son of the
very remarkable Elie.  A survey of the history of homological algebra
by Charles Weibel is available at the K-theory archive\cite{CW99}.
\begin{thebibliography}{99}
\bibitem{C28} E. Cartan, Sur les nombres de Betti des espaces de
  groupes clos, \textit{C. R. Acad. Sci. (Paris)} \textbf{187} (1928),
  196--198.
\bibitem{CE48} C. Chevalley and S. Eilenberg, Cohomology theory of Lie
  groups and Lie algebras, \textit{Trans. Amer. Math. Soc.} \textbf{63}
  (1948), 85-124.
\bibitem{CE56}H. Cartan and S. Eilenberg, \textit{Homological
    Algebra}, Princeton U. Press, 1956.
\bibitem{CW99} C. Weibel, History of homological algebra, in
  \textit{History of topology}, 797--836, North-Holland, Amsterdam,
  1999. Available online at
  \PMlinkexternal{http://www.math.uiuc.edu/K-theory/0245/}{http://www.math.uiuc.edu/K-theory/0245/}
\end{thebibliography}
%%%%%
%%%%%
\end{document}
