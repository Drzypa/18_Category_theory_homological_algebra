\documentclass[12pt]{article}
\usepackage{pmmeta}
\pmcanonicalname{IdentityFunctor}
\pmcreated{2013-03-22 16:37:08}
\pmmodified{2013-03-22 16:37:08}
\pmowner{CWoo}{3771}
\pmmodifier{CWoo}{3771}
\pmtitle{identity functor}
\pmrecord{9}{38817}
\pmprivacy{1}
\pmauthor{CWoo}{3771}
\pmtype{Definition}
\pmcomment{trigger rebuild}
\pmclassification{msc}{18A05}
\pmclassification{msc}{18-00}

\usepackage{amssymb,amscd}
\usepackage{amsmath}
\usepackage{amsfonts}

% used for TeXing text within eps files
%\usepackage{psfrag}
% need this for including graphics (\includegraphics)
%\usepackage{graphicx}
% for neatly defining theorems and propositions
%\usepackage{amsthm}
% making logically defined graphics
%%\usepackage{xypic}
\usepackage{pst-plot}
\usepackage{psfrag}

% define commands here

\begin{document}
Let $\mathcal{C}$ be a category.  The \emph{identity functor} of $\mathcal{C}$ is the unique functor, written $I_{\mathcal{C}}$, such that for every object $A$ and every morphism $\alpha$ in $\mathcal{C}$, we have 
$$I_{\mathcal{C}}(A)=A\quad\mbox{ and }\quad I_{\mathcal{C}}(\alpha)=\alpha.$$  To verify that $I_{\mathcal{C}}$ is indeed a functor, we note that $I_{\mathcal{C}}(1_A)=1_A=1_{I_{\mathcal{C}}(A)}$, where $1_A$ is the identity morphism of $A$, and $I_{\mathcal{C}}(\alpha\circ\beta)=\alpha\circ \beta=I_{\mathcal{C}}(\alpha)\circ I_{\mathcal{C}}(\beta)$.

For any functor $F:\mathcal{C}\to \mathcal{D}$, we have $F\circ I_{\mathcal{C}}= I_{\mathcal{D}}\circ F=F$.

Since every category gives rise to its unique identity functor, we can think of \emph{the identity functor} $I$ as a (covariant) functor on \textbf{Cat}, the category of (small) categories.  It is given by taking any category $\mathcal{C}$ to itself and any functor $F:\mathcal{C}\to \mathcal{D}$ to itself.
%%%%%
%%%%%
\end{document}
