\documentclass[12pt]{article}
\usepackage{pmmeta}
\pmcanonicalname{PropertiesOfDirectProduct}
\pmcreated{2013-03-22 18:29:15}
\pmmodified{2013-03-22 18:29:15}
\pmowner{CWoo}{3771}
\pmmodifier{CWoo}{3771}
\pmtitle{properties of direct product}
\pmrecord{10}{41164}
\pmprivacy{1}
\pmauthor{CWoo}{3771}
\pmtype{Derivation}
\pmcomment{trigger rebuild}
\pmclassification{msc}{18A30}
\pmrelated{CategoricalDirectSum}

\usepackage{amssymb,amscd}
\usepackage{amsmath}
\usepackage{amsfonts}
\usepackage{mathrsfs}

% used for TeXing text within eps files
%\usepackage{psfrag}
% need this for including graphics (\includegraphics)
%\usepackage{graphicx}
% for neatly defining theorems and propositions
\usepackage{amsthm}
% making logically defined graphics
%%\usepackage{xypic}
\usepackage{pst-plot}

% define commands here
\newcommand*{\abs}[1]{\left\lvert #1\right\rvert}
\newtheorem{prop}{Proposition}
\newtheorem{thm}{Theorem}
\newtheorem{cor}{Corollary}
\newtheorem{ex}{Example}
\newcommand{\real}{\mathbb{R}}
\newcommand{\pdiff}[2]{\frac{\partial #1}{\partial #2}}
\newcommand{\mpdiff}[3]{\frac{\partial^#1 #2}{\partial #3^#1}}
\begin{document}
Let $\mathcal{C}$ be a category.  This entry lists some of the basic properties of categorical direct product:

\begin{prop} (uniqueness of products)  A product $(C,\lbrace \pi_i\rbrace_{i\in I})$ of objects $\lbrace C_i\rbrace_{i\in I}$, if it exists, is unique up to isomorphism. \end{prop}

Before proving this, let us observe first that if $f:C\to C$ is a morphism such that 
\begin{equation} \pi_i=\pi_i\circ f \end{equation}
then $f=1_C$ necessarily, since the universal property of product $C$, $f$ is the unique morphism such that (1) holds, but then $\pi_1=\pi_1\circ 1_C$ as well, and this forces $f=1_C$.

\begin{proof}  If $(D,\lbrace g_i\rbrace_{i\in I})$ is another product of $\lbrace C_i\rbrace_{i\in I}$, we get two unique morphisms $x:D\to C$ and $y:C\to D$ such that $\pi_i=g_i\circ y$ and $g_i=\pi_i\circ x$ for all $i\in I$.  So $\pi_i=(\pi_i\circ x)\circ y = \pi_i\circ (x\circ y)$.  From the previous paragraph, we see that $x\circ y=1_C$.  Similarly, $g_i=g_i\circ (y\circ x)$, so that $y\circ x=1_D$.  This shows that $C$ is isomorphic to $D$.
\end{proof}
This justifies writing $\prod_{i\in I} C_i$ (with $\pi_i$) as \emph{the} product of $\lbrace C_i\rbrace_{i\in I}$.  In case $I$ has cardinality $2$, we write $C_1\times C_2$ as the product of $C_1$ and $C_2$.  Also, when $I=\varnothing$, we set the product as any terminal object $T$ in $\mathcal{C}$.

\begin{prop}  If $I$ is the disjoint union of $J$ and $K$, then $$\prod_{i\in I} C_i \cong \prod_{j\in J} C_j \times \prod_{k\in K} C_k,$$ assuming all products exist.
\end{prop}
\begin{proof}  Let $$C=\prod_{i\in I} C_i, \quad D=\prod_{j\in J} C_j, \quad\mbox{and}\quad E=\prod_{k\in K} C_k.$$  We break down the proof into two cases:
\begin{itemize}
\item
Suppose one of $J$ and $K$ is the empty set, say, $J=\varnothing$.  Then $D$ is a terminal object, and $K=I$, so that $E=C$.  In other words, we want to show that $$C\cong D\times C.$$  First, notice that we have morphisms $1_C:C\to C$ and $e_C:C\to D$ (where $e_C$ is unique since $D$ is terminal).  If $A$ is any object with morphisms $f:A\to C$ and $e_A: A\to D$.  Any $g:A\to C$ with $f=1_C\circ g$ and $e_A = e_C\circ g$ must result in $f=g$.  This shows that $C$ may be viewed as the product of $D$ and $C$, or $C\cong D\times C$.
\item
Now, suppose neither $J$ nor $K$ is empty.  We have projection morphisms $f_i:C\to C_i$ for all $i\in I$, $g_j:D\to C_j$ for all $j\in J$, and $h_k:E\to C_k$ for all $k\in K$.  Write $F=D\times E$ with projections $p_1:F\to D$ and $p_2: F\to E$.  

For every $i\in I$, define morphisms $x_i:F\to C_i$ as follows: if $i\in J$, then $x_i=g_i\circ p_1$.  Otherwise, $x_i=h_i\circ p_2$.  Since $J$ and $K$ are disjoint, this $I$-indexed set of morphisms is well-defined.  By the universality of the product $C$, we get a unique morphism $x:F\to C$ such that $x_i=f_i\circ x$.  

Next, from the universal properties of the products $D$ and $E$, we have two unique morphisms $y:C\to D$ and $z:C\to E$ such that $f_j = g_j\circ y$ and $f_k=h_k\circ z$ for any $j\in J$ and $k\in K$.  From the morphisms $y:C\to D$ and $z:C\to E$ and the universality of the product $F$, we have another unique morphism $f:C\to F$ such that $y=p_1\circ f$ and $z=p_2\circ f$.

Then $p_1\circ (f \circ x) = (p_1\circ f)\circ x=y\circ x$.  Since $g_j\circ y \circ x=f_j\circ x=x_j=g_j\circ p_1$ for any $j\in J$, we have $y\circ x=p_1$, so that $p_1\circ (f\circ x)=p_1$.  Similarly, $p_2\circ (f\circ x)=p_2$.  This shows that $f\circ x=1_F$.  Also, $f_i\circ ( x\circ f) = (f_i\circ x)\circ f = x_i \circ f$.  Now, if $i\in J$, then $x_i\circ f = g_i\circ p_1 \circ f = g_i \circ y = f_i$, and if $i\in K$, then $x_i\circ f = h_i\circ p_2 \circ f = h_i \circ z = f_i$.  As a result, $f_i\circ (x\circ f)=f_i$ for all $i\in I$, which implies $x\circ f = 1_C$.  This shows that $C\cong F=D\times E$.
\end{itemize}
This completes the proof.
\end{proof}

\begin{cor} (commutativity of products)  $A\times B\cong B\times A$, if one (and hence the other) exists. \end{cor}

This shows that it does not matter whether we say $A\times B$ as the product of $A$ and $B$, or the product of $B$ and $A$.

\begin{cor} (associativity of products)  $A\times (B\times C)\cong A\times B\times C \cong (A\times B)\times C$, whenever the products are defined. \end{cor}

\textbf{Remarks}.  All of the properties can be dualized, so that coproducts are unique up to isomorphism, and commutativity and associativity laws hold as well.
%%%%%
%%%%%
\end{document}
