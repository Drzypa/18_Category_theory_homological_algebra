\documentclass[12pt]{article}
\usepackage{pmmeta}
\pmcanonicalname{SectionsAndRetractions}
\pmcreated{2013-03-22 18:42:26}
\pmmodified{2013-03-22 18:42:26}
\pmowner{porton}{9363}
\pmmodifier{porton}{9363}
\pmtitle{sections and retractions}
\pmrecord{4}{41471}
\pmprivacy{1}
\pmauthor{porton}{9363}
\pmtype{Definition}
\pmcomment{trigger rebuild}
\pmclassification{msc}{18A05}
%\pmkeywords{morphism}
%\pmkeywords{category theory}
\pmrelated{TypesOfMorphisms}
\pmdefines{retraction}
\pmdefines{section}
\pmdefines{retractable}
\pmdefines{sectionable}

% this is the default PlanetMath preamble.  as your knowledge
% of TeX increases, you will probably want to edit this, but
% it should be fine as is for beginners.

% almost certainly you want these
\usepackage{amssymb}
\usepackage{amsmath}
\usepackage{amsfonts}

% used for TeXing text within eps files
%\usepackage{psfrag}
% need this for including graphics (\includegraphics)
%\usepackage{graphicx}
% for neatly defining theorems and propositions
%\usepackage{amsthm}
% making logically defined graphics
%%%\usepackage{xypic}

% there are many more packages, add them here as you need them

% define commands here

\begin{document}
Let $f: A\rightarrow B$ is a morphism of a category.

\begin{itemize}
\item If there exists morphism $g: B\rightarrow A$ such that $g\circ f=1_A$ then $f$ is called \emph{retractable} and $g$ is called \emph{retraction} of $f$.
\item If there exists morphism $g: B\rightarrow A$ such that $f\circ g=1_B$ then $f$ is called \emph{sectionable} and $g$ is called \emph{section} of $f$.
\end{itemize}

\section{References}

\begin{itemize}
\item \PMlinkexternal{Yet Another Text on general nonsense}{http://www.madore.org/~david/math/cat2.ps.gz}
\end{itemize}
%%%%%
%%%%%
\end{document}
