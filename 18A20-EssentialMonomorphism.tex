\documentclass[12pt]{article}
\usepackage{pmmeta}
\pmcanonicalname{EssentialMonomorphism}
\pmcreated{2013-03-22 18:27:30}
\pmmodified{2013-03-22 18:27:30}
\pmowner{jocaps}{12118}
\pmmodifier{jocaps}{12118}
\pmtitle{essential monomorphism}
\pmrecord{9}{41122}
\pmprivacy{1}
\pmauthor{jocaps}{12118}
\pmtype{Definition}
\pmcomment{trigger rebuild}
\pmclassification{msc}{18A20}
\pmsynonym{essential extension}{EssentialMonomorphism}
\pmrelated{essentialextension}
\pmdefines{essential monomorphism}

% this is the default PlanetMath preamble.  as your knowledge
% of TeX increases, you will probably want to edit this, but
% it should be fine as is for beginners.

% almost certainly you want these
\usepackage{amssymb}
\usepackage{amsmath}
\usepackage{amsfonts}

% used for TeXing text within eps files
%\usepackage{psfrag}
% need this for including graphics (\includegraphics)
%\usepackage{graphicx}
% for neatly defining theorems and propositions
%\usepackage{amsthm}
% making logically defined graphics
%%%\usepackage{xypic}

% there are many more packages, add them here as you need them

% define commands here

\begin{document}
Let $\mathcal C$ be a category and $A,B \in \mathrm{Ob}(\mathcal C)$.  Then a monomorphism 
$f \in \mathcal C(A,B)$ is said to be an \emph{essential extension} (or \emph{essential monomorphism}) iff : 


If $g\in\mathcal C(B,C)$ be such that $g \circ f : A\rightarrow B\rightarrow C$ is a monomorphism then 
$g$ is a monomorphism.

Some examples:
\begin{itemize}

\item Consider the category of commutative rings. One way to redefine essential extension in this category is in the following way: 

If $A$ is a subring of another ring $B$ then $B$ is an essential extension of $A$ iff for any $b\in B\backslash\{0\}$ there exists a $c\in B$ such that $bc \in A\backslash\{0\}$.

\item In the category of modules see \PMlinkname{essential submodules}{EssentialSubmodule}.

\item In the category of groups see essential subgroups.

\end{itemize}
%%%%%
%%%%%
\end{document}
