\documentclass[12pt]{article}
\usepackage{pmmeta}
\pmcanonicalname{SplicingTogetherExactSequences}
\pmcreated{2013-03-22 19:03:35}
\pmmodified{2013-03-22 19:03:35}
\pmowner{rm50}{10146}
\pmmodifier{rm50}{10146}
\pmtitle{splicing together exact sequences}
\pmrecord{4}{41943}
\pmprivacy{1}
\pmauthor{rm50}{10146}
\pmtype{Theorem}
\pmcomment{trigger rebuild}
\pmclassification{msc}{18G35}
\pmsynonym{splicing lemma}{SplicingTogetherExactSequences}

\endmetadata

\usepackage{amssymb}
\usepackage{amsmath}
\usepackage{amsfonts}

% used for TeXing text within eps files
%\usepackage{psfrag}
% need this for including graphics (\includegraphics)
%\usepackage{graphicx}
% for neatly defining theorems and propositions
\usepackage{amsthm}
% making logically defined graphics
%%%\usepackage{xypic}

% there are many more packages, add them here as you need them

% define commands here
\newcommand{\coker}{\mathrm{coker}}
\newcommand{\im}{\mathrm{im}}
\DeclareMathOperator{\Hom}{Hom}

\theoremstyle{plain} %% This is the default
\newtheorem{thm}{Theorem}
\newtheorem{cor}[thm]{Corollary}
\newtheorem{lem}[thm]{Lemma}
\newtheorem{prop}[thm]{Proposition}
\begin{document}
\PMlinkescapeword{simple}
\PMlinkescapeword{useful}
This article proves a simple but very useful result about ``splicing'' together two exact sequences. Assume we are working in an abelian category such as groups, rings, or modules.

\begin{prop} Let
\[
  A \to B \xrightarrow{f} C
\]
and
\[
  D \xrightarrow{g} E \to F
\]
be exact, and assume that there is an isomorphism $\varphi : \coker f \to \ker g$. Define $\psi : C\to D: c\mapsto \varphi(\bar{c})$, where $\bar{c}$ is the image of $c$ in $\coker f$. Then the following is exact:
\[
  A \to B \xrightarrow{f} C \xrightarrow{\psi} D \xrightarrow{g} E \to F
\]
\end{prop}
\begin{proof} Exactness at $C$:
\[
  c\in \ker \psi \iff \psi(c) = \varphi(\bar{c})=0 \iff \bar{c} = 0 \iff c\in\im f.
\]
Exactness at $D$:
\[
  d\in \ker g \iff d = \varphi(\bar{c}) \text{ for some } c\in C \iff d = \psi(c)\text{ for some } c\in C.
\]

\end{proof}

%%%%%
%%%%%
\end{document}
