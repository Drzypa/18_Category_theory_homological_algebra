\documentclass[12pt]{article}
\usepackage{pmmeta}
\pmcanonicalname{KernelIsAnInverseLimit}
\pmcreated{2013-03-22 14:11:28}
\pmmodified{2013-03-22 14:11:28}
\pmowner{mathcam}{2727}
\pmmodifier{mathcam}{2727}
\pmtitle{kernel is an inverse limit}
\pmrecord{5}{35621}
\pmprivacy{1}
\pmauthor{mathcam}{2727}
\pmtype{Theorem}
\pmcomment{trigger rebuild}
\pmclassification{msc}{18A30}
\pmrelated{Kernel}
\pmrelated{AbelianCategory}
\pmrelated{KernelOfAGroupHomomorphism}
\pmrelated{KernelOfAMorphism}

% this is the default PlanetMath preamble.  as your knowledge
% of TeX increases, you will probably want to edit this, but
% it should be fine as is for beginners.

% almost certainly you want these
\usepackage{amssymb}
\usepackage{amsmath}
\usepackage{amsfonts}

% used for TeXing text within eps files
%\usepackage{psfrag}
% need this for including graphics (\includegraphics)
%\usepackage{graphicx}
% for neatly defining theorems and propositions
\usepackage{amsthm}
% making logically defined graphics
%%\usepackage{xypic}

% there are many more packages, add them here as you need them

\usepackage{mathrsfs}

% define commands here

\newtheorem{theorem}{Theorem}
\newtheorem{defn}{Definition}
\newtheorem{prop}{Proposition}
\newtheorem{lemma}{Lemma}
\newtheorem{cor}{Corollary}
\newtheorem*{warning}{Warning}

\newcommand{\Univ}{\mathscr{U}}
\DeclareMathOperator{\liminv}{\varprojlim}
\DeclareMathOperator{\limdir}{\varinjlim}
\DeclareMathOperator{\Funct}{Funct}
\DeclareMathOperator{\Hom}{Hom}
\begin{document}
\begin{theorem}
Let $C$ be an abelian category, and $f\colon A\to B$ be a morphism of $C$.  Then $\ker f$ is an inverse limit. 
\end{theorem}
\begin{proof}
Recalling the definition of a kernel in an abelian category, we can see that $i\colon X\to A$ is a kernel of $f$ if and only if the following diagrams commute:
\[
\xymatrix{
& X \ar[dl]_i \ar[dr]^{f\circ i}&\\
A \ar[rr]^f & & B 
}
\]
\[
\xymatrix{
& X \ar[dl]_i \ar[dr]^{0\circ i}&\\
A \ar[rr]^0 & & B 
}
\]
and $f\circ i = 0\circ i$. 

Let $I$ be the category
\[
\xymatrix{
\diamondsuit \ar@<0.5ex>[r]^\dagger \ar@<-0.5ex>[r]_\ddagger & \heartsuit
}
\]
and define a functor $G\colon I\to C$ by 
$G(\diamondsuit)=A$, 
$G(\heartsuit)=B$, 
$G(\dagger) = f$, and 
$G(\ddagger) = 0$.  

Suppose that $\liminv G$ exists and is equal to $Y$.  Then there are maps $\pi_\diamondsuit\colon Y\to A$ and $\pi_\heartsuit\colon Y\to A$ that make the following diagrams commute:
\[
\xymatrix{
& Y \ar[dl]_{\pi_\diamondsuit} \ar[dr]^{\pi_\heartsuit}&\\
A \ar[rr]^{G(\dagger)=f} & & B 
}
\]
\[
\xymatrix{
& Y \ar[dl]_{\pi_\diamondsuit} \ar[dr]^{\pi_\heartsuit}&\\
A \ar[rr]^{G(\ddagger)=0} & & B 
}
\]
This is exactly the universal condition for a kernel in an abelian category.
\end{proof}

By reversing arrows, we can see that a cokernel is a direct limit.

This result can be extremely useful in proving exactness results: one shows that finite inverse and direct limits exist and are exact in a particular category, and one immediately obtains the fact that sums, products, kernels and cokernels are all exact.
%%%%%
%%%%%
\end{document}
