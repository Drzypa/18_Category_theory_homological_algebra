\documentclass[12pt]{article}
\usepackage{pmmeta}
\pmcanonicalname{FunctorCategoryExamples}
\pmcreated{2013-03-22 18:12:40}
\pmmodified{2013-03-22 18:12:40}
\pmowner{bci1}{20947}
\pmmodifier{bci1}{20947}
\pmtitle{functor category examples}
\pmrecord{57}{40792}
\pmprivacy{1}
\pmauthor{bci1}{20947}
\pmtype{Feature}
\pmcomment{trigger rebuild}
\pmclassification{msc}{18D05}
\pmclassification{msc}{18-00}
\pmclassification{msc}{18A25}
\pmsynonym{categories of functors and natural transformations}{FunctorCategoryExamples}
%\pmkeywords{small category}
%\pmkeywords{categories of functors and natural transformations}
\pmrelated{2Category}
\pmrelated{HigherDimensionalAlgebraHDA}
\pmrelated{SupercategoriesOfComplexSystems}
\pmrelated{SupercategoriesOfComplexSystems}
\pmrelated{TopicEntryOnFoundationsOfMathematics}
\pmrelated{QuantumFundamentalGroupoids}
\pmrelated{ETAS}
\pmrelated{ETAC}
\pmrelated{2Category2}
\pmrelated{GroupoidHomomorphisms}
\pmrelated{CategoryTheory}
\pmrelated{IndexOfCategories}
\pmdefines{group-groupoid}
\pmdefines{Abelian functor category}
\pmdefines{group-groupoid functor category}
\pmdefines{supercategory of all functor categories}

\endmetadata

% this is the default PlanetMath preamble.
\usepackage{amssymb}
\usepackage{amsmath}
\usepackage{amsfonts}

% define commands here
\usepackage{amsmath, amssymb, amsfonts, amsthm, amscd, latexsym,enumerate}
%%\usepackage{xypic}
\usepackage[mathscr]{eucal}

\setlength{\textwidth}{6.5in}
%\setlength{\textwidth}{16cm}
\setlength{\textheight}{9.0in}
%\setlength{\textheight}{24cm}

\hoffset=-.75in     %%ps format
%\hoffset=-1.0in     %%hp format
\voffset=-.4in


\theoremstyle{plain}
\newtheorem{lemma}{Lemma}[section]
\newtheorem{proposition}{Proposition}[section]
\newtheorem{theorem}{Theorem}[section]
\newtheorem{corollary}{Corollary}[section]

\theoremstyle{definition}
\newtheorem{definition}{Definition}[section]
\newtheorem{example}{Example}[section]
%\theoremstyle{remark}
\newtheorem{remark}{Remark}[section]
\newtheorem*{notation}{Notation}
\newtheorem*{claim}{Claim}

\renewcommand{\thefootnote}{\ensuremath{\fnsymbol{footnote}}}
\numberwithin{equation}{section}

\newcommand{\Ad}{{\rm Ad}}
\newcommand{\Aut}{{\rm Aut}}
\newcommand{\Cl}{{\rm Cl}}
\newcommand{\Co}{{\rm Co}}
\newcommand{\DES}{{\rm DES}}
\newcommand{\Diff}{{\rm Diff}}
\newcommand{\Dom}{{\rm Dom}}
\newcommand{\Hol}{{\rm Hol}}
\newcommand{\Mon}{{\rm Mon}}
\newcommand{\Hom}{{\rm Hom}}
\newcommand{\Ker}{{\rm Ker}}
\newcommand{\Ind}{{\rm Ind}}
\newcommand{\IM}{{\rm Im}}
\newcommand{\Is}{{\rm Is}}
\newcommand{\ID}{{\rm id}}
\newcommand{\GL}{{\rm GL}}
\newcommand{\Iso}{{\rm Iso}}
\newcommand{\rO}{{\rm O}}
\newcommand{\Sem}{{\rm Sem}}
\newcommand{\St}{{\rm St}}
\newcommand{\Sym}{{\rm Sym}}
\newcommand{\SU}{{\rm SU}}
\newcommand{\Tor}{{\rm Tor}}
\newcommand{\U}{{\rm U}}

\newcommand{\A}{\mathcal A}
\newcommand{\Ce}{\mathcal C}
\newcommand{\D}{\mathcal D}
\newcommand{\E}{\mathcal E}
\newcommand{\F}{\mathcal F}
\newcommand{\G}{\mathcal G}
\renewcommand{\H}{\mathcal H}
\renewcommand{\cL}{\mathcal L}
\newcommand{\Q}{\mathcal Q}
\newcommand{\R}{\mathcal R}
\newcommand{\cS}{\mathcal S}
\newcommand{\cU}{\mathcal U}
\newcommand{\W}{\mathcal W}

\newcommand{\bA}{\mathbb{A}}
\newcommand{\bB}{\mathbb{B}}
\newcommand{\bC}{\mathbb{C}}
\newcommand{\bD}{\mathbb{D}}
\newcommand{\bE}{\mathbb{E}}
\newcommand{\bF}{\mathbb{F}}
\newcommand{\bG}{\mathbb{G}}
\newcommand{\bK}{\mathbb{K}}
\newcommand{\bM}{\mathbb{M}}
\newcommand{\bN}{\mathbb{N}}
\newcommand{\bO}{\mathbb{O}}
\newcommand{\bP}{\mathbb{P}}
\newcommand{\bR}{\mathbb{R}}
\newcommand{\bV}{\mathbb{V}}
\newcommand{\bZ}{\mathbb{Z}}

\newcommand{\bfE}{\mathbf{E}}
\newcommand{\bfX}{\mathbf{X}}
\newcommand{\bfY}{\mathbf{Y}}
\newcommand{\bfZ}{\mathbf{Z}}

\renewcommand{\O}{\Omega}
\renewcommand{\o}{\omega}
\newcommand{\vp}{\varphi}
\newcommand{\vep}{\varepsilon}

\newcommand{\diag}{{\rm diag}}
\newcommand{\grp}{{\mathsf{G}}}
\newcommand{\dgrp}{{\mathsf{D}}}
\newcommand{\desp}{{\mathsf{D}^{\rm{es}}}}
\newcommand{\Geod}{{\rm Geod}}
\newcommand{\geod}{{\rm geod}}
\newcommand{\hgr}{{\mathsf{H}}}
\newcommand{\mgr}{{\mathsf{M}}}
\newcommand{\ob}{{\rm Ob}}
\newcommand{\obg}{{\rm Ob(\mathsf{G)}}}
\newcommand{\obgp}{{\rm Ob(\mathsf{G}')}}
\newcommand{\obh}{{\rm Ob(\mathsf{H})}}
\newcommand{\Osmooth}{{\Omega^{\infty}(X,*)}}
\newcommand{\ghomotop}{{\rho_2^{\square}}}
\newcommand{\gcalp}{{\mathsf{G}(\mathcal P)}}

\newcommand{\rf}{{R_{\mathcal F}}}
\newcommand{\glob}{{\rm glob}}
\newcommand{\loc}{{\rm loc}}
\newcommand{\TOP}{{\rm TOP}}

\newcommand{\wti}{\widetilde}
\newcommand{\what}{\widehat}

\renewcommand{\a}{\alpha}
\newcommand{\be}{\beta}
\newcommand{\ga}{\gamma}
\newcommand{\Ga}{\Gamma}
\newcommand{\de}{\delta}
\newcommand{\del}{\partial}
\newcommand{\ka}{\kappa}
\newcommand{\si}{\sigma}
\newcommand{\ta}{\tau}

\newcommand{\med}{\medbreak}
\newcommand{\medn}{\medbreak \noindent}
\newcommand{\bign}{\bigbreak \noindent}

\newcommand{\lra}{{\longrightarrow}}
\newcommand{\ra}{{\rightarrow}}
\newcommand{\rat}{{\rightarrowtail}}
\newcommand{\ovset}[1]{\overset {#1}{\ra}}
\newcommand{\ovsetl}[1]{\overset {#1}{\lra}}
\newcommand{\hr}{{\hookrightarrow}}

%\usepackage{geometry, amsmath,amssymb,latexsym,enumerate}
%%%\usepackage{xypic}

\def\baselinestretch{1.1}


\def\C{C^{\ast}}

\newcommand{\labto}[1]{\stackrel{#1}{\longrightarrow}}

\begin{document}
\subsection{Introduction}

 Let us recall the essential data required to define functor categories. One requires two arbitrary categories that, in principle, could be large categories, $\mathcal{\A}$ and $\mathcal{C}$, and also the class 
$$\textbf{M} = [\mathcal{\A},\mathcal{C}]$$ 
(alternatively denoted as $\mathcal{C}^{\mathcal{\A}}$) of all covariant functors from $\mathcal{\A}$ to $\mathcal{C}$. For any two such functors $F, K \in [\mathcal{\A}, \mathcal{C}]$, $ F: \mathcal{\A} \rightarrow \mathcal{C}$ and $ K: \mathcal{\A} \rightarrow \mathcal{C}$, the class of all natural transformations from $F$ to $K$ is denoted by $[F, K]$, (or simply denoted by $K^F$). In the particular case when $[F,K]$ is a \textbf{set} one can still define for a small category $\mathcal{\A}$, the set $Hom(F,K)$. Thus, (cf. p. 62 in \cite{Mitchell65}), when $\mathcal{\A}$ is a {\em small} category the class  $[F, K]$ of natural transformations from $F$ to $K$ may be viewed as a subclass of the cartesian product $\prod_{A \in \mathcal{\A}}[F(A), K(A)]$, and because the latter is a {\em set} so is $[F, K]$ as well.  Therefore, with the categorical law of composition of natural transformations of functors, and for $\mathcal{\A}$ being small, $\textbf{M} = [\mathcal{\A},\mathcal{C}]$ satisfies the conditions for the definition of a category, and it is in fact a \PMlinkname{functor category}{FunctorCategory2}. 
 

\subsection{Examples}

\begin{enumerate}
\item Let us consider $\mathcal{A}b$ to be a small Abelian category and let $\mathbb{G}_{Ab}$ be the category of finite Abelian (or commutative) groups, as well as the set of all covariant functors from $\mathcal{A}b$ to 
$\mathbb{G}_{Ab}$. Then, one can show by following the steps defined in the definition of a 
\PMlinkname{functor category}{FunctorCategory2} that $[\mathcal{A}b,\mathbb{G}_{Ab}]$, or 
${\mathbb{G}_{Ab}}^{\mathcal{A}b}$ thus defined is an \emph{Abelian functor category}. 

\item Let $\mathbb{G}_{Ab}$ be a small category of finite Abelian (or commutative) groups and, also let $\grp_G$ be a small category of group-groupoids, that is, group objects in the category of groupoids. Then, one can show that the imbedding functors  
$\textbf{I}$: from $\mathbb{G}_{Ab}$ into $\grp_G$ form a \PMlinkname{functor category}{FunctorCategory2} 
${\grp_G}^{\mathbb{G}_{Ab}}$.

\item In the general case when $\mathcal{\A}$ is not small, the proper class 
$$\textbf{M} = [\mathcal{\A}, \mathcal{\A'}]$$ may be endowed with the structure of a {\em supercategory} defined as any formal interpretation of ETAS with the usual categorical composition law for natural transformations of functors; similarly, one can construct a meta-category called the \emph{supercategory of all functor categories}.
\end{enumerate}



\begin{thebibliography}{9}

\bibitem{Mitchell65}
Mitchell, B.: 1965, \emph{Theory of Categories}, Academic Press: London.

\bibitem{NP1975}
Ref.$288$ in the 
\PMlinkname{Bibliography of Category Theory and Algebraic Topology}{CategoricalOntologyABibliographyOfCategoryTheory}.
 
\end{thebibliography}

%%%%%
%%%%%
\end{document}
