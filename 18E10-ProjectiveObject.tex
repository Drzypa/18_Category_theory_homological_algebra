\documentclass[12pt]{article}
\usepackage{pmmeta}
\pmcanonicalname{ProjectiveObject}
\pmcreated{2013-03-22 14:47:09}
\pmmodified{2013-03-22 14:47:09}
\pmowner{CWoo}{3771}
\pmmodifier{CWoo}{3771}
\pmtitle{projective object}
\pmrecord{10}{36437}
\pmprivacy{1}
\pmauthor{CWoo}{3771}
\pmtype{Definition}
\pmcomment{trigger rebuild}
\pmclassification{msc}{18E10}
\pmrelated{EnoughProjectives}
\pmrelated{EnoughInjectives}
\pmdefines{injective object}

\endmetadata

% this is the default PlanetMath preamble.  as your knowledge
% of TeX increases, you will probably want to edit this, but
% it should be fine as is for beginners.

% almost certainly you want these
\usepackage{amssymb,amscd}
\usepackage{amsmath}
\usepackage{amsfonts}

% used for TeXing text within eps files
%\usepackage{psfrag}
% need this for including graphics (\includegraphics)
%\usepackage{graphicx}
% for neatly defining theorems and propositions
%\usepackage{amsthm}
% making logically defined graphics
%%\usepackage{xypic}

% there are many more packages, add them here as you need them

% define commands here
\begin{document}
Let $\mathcal{C}$ be a category.  An object $P$ in $\mathcal{C}$ is said to be \emph{projective} if, given a diagram on the left, with $g$ a strong epimorphism, there is a morphism $h:P\to A$ making the diagram on the right commutative:
$$\xymatrix@+=4pc{
&{P}\ar[d]^{f}\\
{A}\ar[r]_{g}&{B}
}
\hspace{4cm}
\xymatrix@+=4pc{
&{P}\ar[d]^{f} \ar@{.>}[dl]_h \\
{A}\ar[r]_{g}&{B}
}
$$
An \emph{injective object} can be defined dually: an object $Q$ in $\mathcal{C}$ is \emph{injective} if, given a diagram on the left, with $g$ a strong monomorphism, there is a morphism $h:B\to Q$ making the digram on the right commutative:
$$\xymatrix@+=4pc{
{A}\ar[r]^{g} \ar[d] &{B} \\
Q &
}
\hspace{4cm}
\xymatrix@+=4pc{
{A}\ar[r]^{g} \ar[d] &{B} \ar@{.>}[dl]^h \\
Q &
}
$$

When $\mathcal{C}$ is an abelian category, we have the following: an object $P$ in $\mathcal{C}$ is projective iff 
$$\operatorname{Hom}(P,-)\colon\mathcal{C}\to\mathbf{Ab}$$ 
is an exact functor, where $\mathbf{Ab}$ is the category of abelian groups.  Dually, an object $Q$ is injective iff the $\operatorname{Hom}(-,Q)$ functor from $\mathcal{C}$ to $\mathbf{Ab}$ is exact.

\textbf{Example.}  Let $R$ be a ring with 1.  Consider the category of left $R$-modules $\mathcal{M}_R$.  $\mathcal{M}_R$ is an abelian category.  The projective objects in $\mathcal{M}_R$ are precisely the \PMlinkname{projective left $R$-modules}{ProjectiveModule}.  So $R$ is itself a projective object in $\mathcal{M}_R$.  

Dually, the injective objects in $\mathcal{M}_R$ are exactly the \PMlinkname{injective left $R$-modules}{InjectiveModule}.

\begin{thebibliography}{9}
\bibitem{fb} F. Borceux \emph{Basic Category Theory, Handbook of Categorical Algebra I}, Cambridge University Press, Cambridge (1994)
\end{thebibliography}
%%%%%
%%%%%
\end{document}
