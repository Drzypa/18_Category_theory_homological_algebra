\documentclass[12pt]{article}
\usepackage{pmmeta}
\pmcanonicalname{ExamplesOfEpis}
\pmcreated{2013-03-22 18:22:18}
\pmmodified{2013-03-22 18:22:18}
\pmowner{CWoo}{3771}
\pmmodifier{CWoo}{3771}
\pmtitle{examples of epis}
\pmrecord{6}{41013}
\pmprivacy{1}
\pmauthor{CWoo}{3771}
\pmtype{Example}
\pmcomment{trigger rebuild}
\pmclassification{msc}{18A05}
\pmclassification{msc}{18A20}

\usepackage{amssymb,amscd}
\usepackage{amsmath}
\usepackage{amsfonts}
\usepackage{mathrsfs}

% used for TeXing text within eps files
%\usepackage{psfrag}
% need this for including graphics (\includegraphics)
%\usepackage{graphicx}
% for neatly defining theorems and propositions
\usepackage{amsthm}
% making logically defined graphics
%%\usepackage{xypic}
\usepackage{pst-plot}

% define commands here
\newcommand*{\abs}[1]{\left\lvert #1\right\rvert}
\newtheorem{prop}{Proposition}
\newtheorem{thm}{Theorem}
\newtheorem{ex}{Example}
\newcommand{\real}{\mathbb{R}}
\newcommand{\pdiff}[2]{\frac{\partial #1}{\partial #2}}
\newcommand{\mpdiff}[3]{\frac{\partial^#1 #2}{\partial #3^#1}}
\begin{document}
The entry lists some of the common examples of epis (epimorphisms).  The examples also demonstrate some of the techniques used in finding epis.

\begin{enumerate}
\item In \textbf{Set}, the category of sets, the epis are exactly the onto functions.  First, suppose $f:A\to B$ is onto and that $g,h:B\to C$ are functions such that $g\circ f=h\circ f$.  Then for any $b\in B$, there is $a\in A$ such that $f(a)=b$ since $f$ is onto.  This means that $g(b)=g(f(a))= h(f(a))=h(b)$, or $g=h$, showing that $f$ is epi.  Conversely, suppose $f:A\to B$ is epi.  Define functions $g,h:B\to \lbrace 0,1\rbrace$ as follows: $g(x)=0$ for all $x\in B$, and $h(x)=0$ if $x\in f(A)$ and $h(x)=1$ otherwise.  Then $g(f(a))=0=h(f(a))$.  This means that $g=h$, or $x\in f(A)$ for all $x\in B$.  In other words, $f$ is onto.
\item In \textbf{Ab}, the category of abelian groups, the epis are exactly the onto abelian group homomorphisms.  If $f$ is onto and $g\circ f =h\circ f$, then for any $b\in B$, there is $a\in A$ such that $f(a)=b$.  This means that $g(b)=g(f(a))=h(f(a))=h(b)$, or $g=h$, showing that $f$ is epi.  On the other hand, suppose $f:A\to B$ is epi.  Define $g,h:B\to B/f(A)$ as follows: $g(x)=f(A)$ and $h(x)=x+f(A)$ for all $x\in B$.  Then $g(f(a))=f(A)=f(a)+f(A)=h(f(a))$.  This implies that $g=h$, so that $x\in f(A)$ for all $x\in B$, or $f$ is onto.
\item In \textbf{Top}, the category of topological spaces, the epis are exactly the surjective continuous functions.  If $f$ is onto and $g\circ f=h\circ f$, then then for any $b\in B$, there is $a\in A$ such that $f(a)=b$.  This means that $g(b)=g(f(a))=h(f(a))=h(b)$, or $g=h$, showing that $f$ is epi.  On the other hand, suppose $f:X\to Y$ is epi.  Equip $\lbrace 0,1\rbrace$ with the trivial topology.  Define $g,h:Y\to \lbrace 0,1\rbrace$ as in example 1 above.  Then $g$ and $h$ are both continuous.  We also have $g(f(a))=0=h(f(a))$, so that $g=h$, or $x\in f(A)$ for all $x\in B$.  Therefore, $f$ is onto.
\end{enumerate}

Not all epimorphisms are surjections.  For example, in the category \textbf{CommRng} of commutative rings with 1, the natural injection $i:\mathbb{Z}\to \mathbb{Q}$ is clearly not a surjection, and yet it is epimorphic.  To see this, let $R$ be any commutative ring with characteristic $0$.  Suppose $g,h:\mathbb{Q}\to R$ are ring homomorphisms such that $g\circ i=h\circ i$, in other words, $g(n)=h(n)$ for all $n\in \mathbb{Z}$.  Set $f:=g-h$.  Then $f(n)=0$ for all $n\in \mathbb{Z}$.  Then $0=f(n)=mf(n/m)$, where $m$ is an arbitrary positive integer.  Since $\operatorname{char}(R)=0$, this shows that $f(n/m)=0$.  Since $n/m$ is an arbitrary rational number, $f=0$, or $g=h$.  Hence $i$ is an epi.

For another counterexample, it can be shown that in \textbf{HausTop}, the category of Hausdorff topological spaces and continuous functions, the epimorphisms are precisely the continuous functions with dense images.  As such, surjections are not a requirement.
%%%%%
%%%%%
\end{document}
