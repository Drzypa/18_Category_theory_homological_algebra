\documentclass[12pt]{article}
\usepackage{pmmeta}
\pmcanonicalname{NicolaePopescumathematician}
\pmcreated{2013-03-22 19:22:10}
\pmmodified{2013-03-22 19:22:10}
\pmowner{bci1}{20947}
\pmmodifier{bci1}{20947}
\pmtitle{Nicolae Popescu (mathematician)}
\pmrecord{5}{42323}
\pmprivacy{1}
\pmauthor{bci1}{20947}
\pmtype{Biography}
\pmcomment{trigger rebuild}
\pmclassification{msc}{18-00}
%\pmkeywords{category theory}
%\pmkeywords{number theory}
%\pmkeywords{sheaves}
%\pmkeywords{valuation theory}
%\pmkeywords{localization and colocalization}
%\pmkeywords{Gabriel-Popescu Theorem}

\endmetadata

% this is the default PlanetMath preamble. as your knowledge

\usepackage{amsmath, amssymb, amsfonts, amsthm, amscd, latexsym}
%%\usepackage{xypic}
\usepackage[mathscr]{eucal}
% define commands here
\theoremstyle{plain}
\newtheorem{lemma}{Lemma}[section]
\newtheorem{proposition}{Proposition}[section]
\newtheorem{theorem}{Theorem}[section]
\newtheorem{corollary}{Corollary}[section]
\theoremstyle{definition}
\newtheorem{definition}{Definition}[section]
\newtheorem{example}{Example}[section]
%\theoremstyle{remark}
\newtheorem{remark}{Remark}[section]
\newtheorem*{notation}{Notation}
\newtheorem*{claim}{Claim}
\renewcommand{\thefootnote}{\ensuremath{\fnsymbol{footnote%%@
}}}
\numberwithin{equation}{section}
\newcommand{\Ad}{{\rm Ad}}
\newcommand{\Aut}{{\rm Aut}}
\newcommand{\Cl}{{\rm Cl}}
\newcommand{\Co}{{\rm Co}}
\newcommand{\DES}{{\rm DES}}
\newcommand{\Diff}{{\rm Diff}}
\newcommand{\Dom}{{\rm Dom}}
\newcommand{\Hol}{{\rm Hol}}
\newcommand{\Mon}{{\rm Mon}}
\newcommand{\Hom}{{\rm Hom}}
\newcommand{\Ker}{{\rm Ker}}
\newcommand{\Ind}{{\rm Ind}}
\newcommand{\IM}{{\rm Im}}
\newcommand{\Is}{{\rm Is}}
\newcommand{\ID}{{\rm id}}
\newcommand{\GL}{{\rm GL}}
\newcommand{\Iso}{{\rm Iso}}
\newcommand{\Sem}{{\rm Sem}}
\newcommand{\St}{{\rm St}}
\newcommand{\Sym}{{\rm Sym}}
\newcommand{\SU}{{\rm SU}}
\newcommand{\Tor}{{\rm Tor}}
\newcommand{\U}{{\rm U}}
\newcommand{\A}{\mathcal A}
\newcommand{\Ce}{\mathcal C}
\newcommand{\D}{\mathcal D}
\newcommand{\E}{\mathcal E}
\newcommand{\F}{\mathcal F}
\newcommand{\G}{\mathcal G}
\newcommand{\Q}{\mathcal Q}
\newcommand{\R}{\mathcal R}
\newcommand{\cS}{\mathcal S}
\newcommand{\cU}{\mathcal U}
\newcommand{\W}{\mathcal W}
\newcommand{\bA}{\mathbb{A}}
\newcommand{\bB}{\mathbb{B}}
\newcommand{\bC}{\mathbb{C}}
\newcommand{\bD}{\mathbb{D}}
\newcommand{\bE}{\mathbb{E}}
\newcommand{\bF}{\mathbb{F}}
\newcommand{\bG}{\mathbb{G}}
\newcommand{\bK}{\mathbb{K}}
\newcommand{\bM}{\mathbb{M}}
\newcommand{\bN}{\mathbb{N}}
\newcommand{\bO}{\mathbb{O}}
\newcommand{\bP}{\mathbb{P}}
\newcommand{\bR}{\mathbb{R}}
\newcommand{\bV}{\mathbb{V}}
\newcommand{\bZ}{\mathbb{Z}}
\newcommand{\bfE}{\mathbf{E}}
\newcommand{\bfX}{\mathbf{X}}
\newcommand{\bfY}{\mathbf{Y}}
\newcommand{\bfZ}{\mathbf{Z}}
\renewcommand{\O}{\Omega}
\renewcommand{\o}{\omega}
\newcommand{\vp}{\varphi}
\newcommand{\vep}{\varepsilon}
\newcommand{\diag}{{\rm diag}}
\newcommand{\grp}{{\mathbb G}}
\newcommand{\dgrp}{{\mathbb D}}
\newcommand{\desp}{{\mathbb D^{\rm{es}}}}
\newcommand{\Geod}{{\rm Geod}}
\newcommand{\geod}{{\rm geod}}
\newcommand{\hgr}{{\mathbb H}}
\newcommand{\mgr}{{\mathbb M}}
\newcommand{\ob}{{\rm Ob}}
\newcommand{\obg}{{\rm Ob(\mathbb G)}}
\newcommand{\obgp}{{\rm Ob(\mathbb G')}}
\newcommand{\obh}{{\rm Ob(\mathbb H)}}
\newcommand{\Osmooth}{{\Omega^{\infty}(X,*)}}
\newcommand{\ghomotop}{{\rho_2^{\square}}}
\newcommand{\gcalp}{{\mathbb G(\mathcal P)}}
\newcommand{\rf}{{R_{\mathcal F}}}
\newcommand{\glob}{{\rm glob}}
\newcommand{\loc}{{\rm loc}}
\newcommand{\TOP}{{\rm TOP}}
\newcommand{\wti}{\widetilde}
\newcommand{\what}{\widehat}
\renewcommand{\a}{\alpha}
\newcommand{\be}{\beta}
\newcommand{\ga}{\gamma}
\newcommand{\Ga}{\Gamma}
\newcommand{\de}{\delta}
\newcommand{\del}{\partial}
\newcommand{\ka}{\kappa}
\newcommand{\si}{\sigma}
\newcommand{\ta}{\tau}
\newcommand{\lra}{{\longrightarrow}}
\newcommand{\ra}{{\rightarrow}}
\newcommand{\rat}{{\rightarrowtail}}
\newcommand{\oset}[1]{\overset {#1}{\ra}}
\newcommand{\osetl}[1]{\overset {#1}{\lra}}
\newcommand{\hr}{{\hookrightarrow}}

\begin{document}
\subsection{Acad. Nicolae Popescu, PhD,D.Phil.} (b. 22 September 1937 at Strehaia, Comanda --d. 29 July 2010 in Bucharest, Romania), was a Romanian mathematician. He became a Member of the Romanian Academy in 1991 and an Emeritus Professor in 2001.
 
His areas of expertise were: Category theory, Abelian categories with applications to Rings and Modules, 
\PMlinkexternal{http://planetmath.org/encyclopedia/AdjointFunctor.html}{Adjoint Functors} and limits of a functor/colimits.($http://planetmath.org/encyclopedia/AdjointFunctor.html$), Theory of Rings, Fields, Polynomials, and Valuation Theory.  He also had interests and published in the following areas: Algebraic Topology, Algebraic Geometry, Commutative Algebra, K-Theory,Class-Field theory, and Functional Analysis/Algebraic Function Theory. He published between 1962 and 2008 more than 102 papers in peer-reviewed, mathematics journals, several monographs on the theory of sheaves, and also six books on Abelian category theory and abstract algebra. In a Grothendieck-like, energetic style, he initiated and provided scientific leadership to several seminars on category theory ( http://planetphysics.org/encyclopedia/FunctorialAlgebraicGeometry.html ; see Seminars on Algebraic Geometry and Category theory), sheaves and abstract algebra which resulted in a continuous stream of high-quality mathematical publications in international, peer-reviewed mathematics journals by several members participating in his Seminar series. His book ``Abelian Categories with Applications to Rings and Modules''(Nicolae Popescu. Abelian Categories with Applications to Rings and Modules, Academic Press, L. M. S. Monograph No.3, London, 1973,$ISBN 0-12-561550$) continues to provide valuable information to mathematicians around the world. His latest contributions have also branched into valuation and number theory. He has published over 100 original peer-reviewed articles on mathematics, mostly in category theory, algebraic geometry, Galois and number theory.

Nicolae Popescu was elected a Member of the Romanian Academy in 1992, and he was best known for his contributions to Algebra, the theory of Abelian categories and Number Theory. Since 1964 he collaborated on the characterization of abelian categories with the well-known French mathematician Pierre Gabriel.

\subsection{Biography}
Acad. Nicolae Popescu was married, and there are a surviving wife Liliana and three children. He earned his M.S. degree in mathematics in 1964, and his Ph.D. degree in mathematics in 1967, both at the University of Bucharest. He was also awarded a D. Phil. degree (Doctor Docent) in 1972 by the University of Bucharest. 

In 2009, and prior to 1976, he carried out mathematics studies at the Institute of Mathematics of the Romanian Academy in the ($http://www.imar.ro/prez/prez_algebra.html$ in the Algebra research group), and also had international collaborations on fourcontinents. One found from conversations with former Academician Nicolae Popescu that he shared many moral, ethical and religious values with another renown mathematician French-German-Jewish, Alexander Grothendieck who visited the School of Mathematics in Bucharest in 1968 at the invitation of Acad. Miron Nicolescu and at the urging of Dr. Nicolae Popescu. Like Grothendieck, he had a long-standing interest in category theory, number theory, practicing Yoga, and supporting promising young mathematicians in his fields of interest. He also supported the early developments of category theory applications in mathematical biology, relational biology and mathematical biophysics.

\subsection{Academic positions}
Popescu was appointed as a Lecturer at the University of Bucharest in 1968 where he taught graduate students until 1972. Since 1964, he also held a  Research Professorship at the Institute of Mathematics ($http://www.imar.ro/~nipopesc/$) of the Romanian Academy, which institute was ruthlessly eliminated by former dictator and president of S.R. Romania, Nicolae Ceausescu in 1976 for reasons related to his daughter Zoe Ceausescu who was 'hired' by the Mathematics Institute in Bucharest two years before.

\subsection{Books published}

Elemente de teoria analitica a numerelor, Univ. Bucuresti, 1968.

Teoria categoriilor si teoria fascicolelor, Ed. Stiintifica, 1971.
 
Nicolae Popescu. Categorii Abeliene, Ed. Academiei, 1971. 

Nicolae Popescu. Abelian categories with Applications to Rings and Modules, Academic Press, L. M. S. Monograph No.3, London, 1973, $ISBN=0125615507$.
Popescu, Nicolae and Popescu, Liliana. Theory of categories. Martinus Nijhoff Publishers, The Hague; Sijthoff \& Noordhoff International Publishers, Alphen aan den Rijn, 1979. $x+337 pp.ETnbsp;ISBN 90-286-0168-6.$
Selected topics in valuation theory (to appear).

\subsection{Original articles and references}
1.     On the continuity of the trace (Proceedings of the Romanian Academy, Series A, Volume 5, Number 2 (2004),  117-122 (with V. Alexandru, E. L Popescu)
 
2.     A new characterization of spectral extension of p-adic valuation, Proc. Conference in Math. Lahore, 18-20 mars 2004 (with E. L Popescu) 

3.     Norms on R[X1, ...., Xr] which are multiplicative von R, Resultate der Mathematik, 51, 229-247 (2008)  (with G. Groza and A. Zaharescu)

4.     All non-archimedean norms on K[X1, ...., Xr] (to appear) (with G. Groza and A. Zaharescu)

*5.     On the structure of compact subsets of Cp, Acta Arithmetica, 123. 3, 253-266 (2006) (with Alla Ditta Raza Choudary, and A. Popescu)

*6.     A basis of Sigma over s , Rev. Roum. Math. Pures Appl. Tome LI, 51 (2006), 87-88. (with E. L. Popescu)

*7.     Sur les categories preabeliens, Rev. Roum. Math. Pures Appl., 10(1965), 621-633. (with C. Banica) 

*8.     Caractérisation des catgories abeliennes avec generateurs et limites inductives exactes, C. R. Acad. Sci. Paris 258(1964), 4188-4191. (with Pierre Gabriel)

*9.     La localisation pour des sites, Rev. Roum. Math. Pures et Appl. 10(1965), 1031-1044.
 
*10.    La structure des modules injectifs sur an anneau et ideal principal, Bull. Math. de la Soc. Sci. Math. Phys. de la RPR Tome 8(56), Nr. 1-2 (1964), 67-73. (with A. Radu) 

*11.    Morphismes et co-morphismes des topos abeliens, Bull. Math. de la Soc. Sci. Math. de la R.S.R. Tom 10(58) W 2 =4 (1996), 319-328. (with A. Radu) 

*12.    Elemente de teoria fascicolelor I, St. Cerc. Mat. (1966) 267-296. (Sheaf theory elements: I)

*13.    Elemente de teoria fascicolelor II, St. Cerc. Mat. (1966) 407-456. Sheaf theory elements: II

*14.    Elemente de teoria fascicolelor III, St. Cerc. Mat. (1966) 547-583. Sheaf theory elements: III

*15.    Elemente de teoria fascicolelor IV, St. Cerc. Mat. (1966) 647-669. Sheaf theory elements: IV

*16.    Elemente de teoria fascicolelor V, St. Cerc. Mat. (1966) 945-991. Sheaf theory elements: V

*17.    Elemente de teoria fascicolelor VI, St. Cerc. Mat. (1967) 205-240. Sheaf theory elements: VI

*18.    Sur la structure des objets de certaines catégories abéliennes, C.R. Acad. Sci. Paris 262(1966), 1295-1297. (with C. Nastasescu) 

*19.    Quelques observations sur les topes abéliens, Rev. Roum. Math. Pures Appl.12 (1967), 553-563. (with C. Nastasescu) 

*20.    Théorie générale de la décomposition, Rev. Roum Math. Pures et Appl. 12 (1967), 1365-1371. 

*21.    Les anneaux semi-artiniens, Bull. Soc. Math. France 96 (1968) 357-368. (with C. Nastasescu). 

*22.    Sur les epimorphismes plants d'anneaux, C.R. Acad. Sci. Paris 268 (1969) 376-379. (with T. Spircu). 

*23.    On the localization ring of a ring, ''J. of Algebra'' 15 (1970) 41-56. (with C. Nastasescu)

*24.    Quelques observations sur les morphismes plats des anneaux, J. Algebra 16(1970), 40-59. (with T. Spircu)

*25.    Le spectre gauche d'un anneau, J. Algebra 18(1971) 213-228. 

*26.    Les quasi-ordres (á gauche) des anneaux, J. Algebra 17(1971), 474-481. (with D. Spulber) 

*27.    Les anneaux semi-noethériens, C.R. Acad. Sci. Paris 272 (1971), 1439-1441. 

*28.    Sur les C. P. anneaux, C.R. Acad. Sci. Paris 272 (1971) 1493-1496. 

*29.    Théorie de la décomposition primaire dans les anneaux semi-noethériens, J. Algebre 2399172), 482-492. 

*30.    Some remarks about semi-artinian rings, Rev. Roumaine Math. Pures et Appl. 17, nr. 9(1973), 1413-1422. (with C. Vraciu) 

*31.    Exemple de inele semi-artiniene, St. Cerc. Math. 26, nr.8 (1974), 1153-1157. (with T. Spircu) 

*32.    Quelques considérations sur les anneaux semi-artiniens commutatifs, C. R. Acad. Sci. Paris 276(1973), 1545-1548. 

*33.    Permanence Theorems for semi-artinian rings, Rev. Roum. Math. Pures et Appl. 21, nr.2 (1976), 227-231. (with T. Spircu)
 
*34.    Sur la Structure des Anneaux Absolument plats commutatifs, J. Algebra 40 (1976), 364-383. (with C. Vraciu) 

*35.    Sur l'anneau des quotients d'un anneau noethérien (à droite) par rapport au système localisant associe à un idéal bilatéral premier, C.R. Acad. Sci. Paris. 

*36.    Some remarks about the regular ring associated to a commutative ring, Rev. Roumaine Math. Pures et Appl. 23(1978), 269-277. (with C. Vraciu) 

*37.    Sur la sous-catégorie localisant associée a un idéal bilatéral premier dans un anneau noethérien (à droite), Rev. Roum. Math. Pures et Appl. T. XXVi, nr. 7 (1981), 1033-1042. 

*38.    Sur un problème d'Arens et Kaplansky concernant la structure de quelques anneaux absolument plats commutatifs, Rev. Roum. Math. Pures Appl., t. 27, nr.8 (1982), 867-874. (with C. Vraciu) 

*39.    Sur une classe de polynômes irréductibles, C.R. Acad. Sci. Paris, t. 297 (1983), 9-11. 

*40.    Galois Theory of permitted extensions of commutative regular rings, Bull. Math. Soc. Sci. Math. R.S. Roumanie, t. 29 (77), nr.1 (1985), 121-135. (with C. Vraciu) 

*41.    On Dedekind domains in infinite algebraic extensions, Rend. Sem. Math. Univ. Padova, vol. 74 (1985), 39-44. (with C. Vraciu) 

*42.    On a problem of Nagata in valuation theory, Rev. Roum. Math. Pures et Appl. 31 (1986), 639-641. 

*43.    On subfields of k(X), Red. Sem. Mat. Univ. Padova, vol. 75 (1986), 257-273. (with V. Alexandru) 

*44.    On a class of intermediate Subfields, Studii si Cercetari Matematice, tom 39, Nr. 2 (1987), p.156-162. (with E.L. Popescu) 

*45.    Sur une clase de prolongements a $K(X)$ d'une valuation sur un corps K, Rev. Roum. Math. Pures Appl. 33 (1988), 393-400. (with V. Alexandru)
 
*46.    A theorem of characterization of residual transcendental extensions of a valuation, J. Math. Kyoto Univ. 28-4 (1988), 579-592. (with V. Alexandru and A. Zaharescu) 

*47.    Sur la définition des prolongements résiduels transcendent d'une valuation sur un corps K \'a $K(X)$, Bull. Math. Soc. Sci. Math. Roumanie, t. 33 (1989), 257-264. (with E. L. Popescu) 

*48.    Minimal pairs of a residual transcendental extension of a valuation, J. Math. Kyoto, Univ., Vol. 30, (1990), 207-225. (with V. Alexandru si A. Zaharescu) 

*49.    All valuations on K(X), J. Math. Kyoto Univ., Vol. 30 (1990), 281-296. (with V. Alexandru si A. Zaharescu) 

*50.    On the residual transcendental extensions of a valuation. Key polynomials and augmented valuation, Math. Tsukuba Univ. Vol. 15, No.1 (1991), 57-78. (with E.L. Popescu) 

*51.    On the extension of valuation on a field K to $K(X)$I Red. Sem. Mat. Univ. Padova, vol. 87 (1992), 151-168. (with C. Vraciu) 

*52.    On the structure of the irreducible polynomials over local fields, J. Number Theory, Vol. 52 No.1 (1993), 98-118. (with A. Zaharescu) 

*53.    The valuations on k(x,y) which are trivial on k, Proc. Conf. Algebra, Constanta, 1994.
 
*54.    Some elementary remarks about n-local fields, Rend. Sem. Math. Univ. Padova, Vol. 91 (1994). (with V. Alexandru) 

*55.    A characterization of Generalized Dedekind Domains, Bull. Math. de la Soc. Sci. Math de la Roumanie, tome 35 (83), Nr.1-2 (1991), 139-141. (with E. L. Popescu)
 
*56.    On a class of Prufer domains, Rev. Roumaine Math. Pures et Appl. 29 (1984), 777-786.

*57.    Sur une classe d'anneaux qui g\'en\'eralisent les anneaux de Dedekind, J. of Algebra, Vol.173, (1995), 44-66. (with M. Fontana) 

*58.    On the extension of a valuation on a Field K to K(X), II, Rend. Sem. Mat, Univ. Padova, Vol 96 (1996), 1-14. (with C. Vraciu)
 
*59.    On the roots of a class of lifting polynomials, Fachbereich Math. Univ. Hagen, Band 63 (1998), 586-600. (with A. Zaharescu)
 
*60.    Completion of a r. t. extension of a local field, I, Math. Z., Vol 221 (1996), 675-682. (with V. Alexandru and A. Popescu) 

*61.    Completion of a r. t. extension of a local field”, II, Rend. Sem. Mat. Univ. Padova, (1998). (with V. Alexandru and A. Popescu)
 
*62.    On the main invariant of an element over a local field, Portugalia Mathematica, Vol. 54, Fasc. 1 (1997) 73-83. (with A. Zaharescu) 

*63.    On the closed subfields of Cp, J. Number Theory, Vol. 68 (1997), 131-150. (with V. Alexandru and A. Zaharescu) 

*64.    Sur une classe d'anneaux de Prüfer avec groupe de classes de torsion, Comm. Alg., 23 (1975), 4521-4533. (with M. Fontana)
 
*65.    On a class of Domains Having Prüfer Integral closure, The FOR-Domains, Commutative ring Theory, Vol. 185, Lecture Notes in Pure and Appl. Math. Dekker 1996. (with M. Fontana) 

*66.    Invertible ideals and Picard group of generalized Dedekind domains, J. Pure and Appl. Alg., Vol 135, Nr. 3 (1999), 237-251. (with S. Gabelli) 

*67.    The Lüroth's Theorem for some complete fields, in ''Abelian Groups, Module Theory and Topology, Editors Dikranian-Salce, Marcel Dekker Inc., 1998, 55-58. (with V. Alexandru)
 
*68.    On minimal pairs and residually transcendental extensions of valuations (Mathematika, 49(2002), 93-106 ) (with S. Khanduja and K.W. Roggenkamp) 

*69.    Nagata Transform and Localizing Systems, Comm. in Algebra, 30(5), (2002), 2297-2308. (with Marco Fontana) 

*70.    Spectral extensions of p-adic valuation, Rev. Roum. Math. Pures et Apll, Vol. 46, Nr.6 (2001), 805-817. (with E. L. Popescu and C. Vraciu) 

*71.    Trace on Cp, J. Number Theory 88 (2001), Nr.1, 13-48. (with V. Alexandru and A. Zaharescu) 

*72.    Spectral norms on valued fields, Math. Z., Vol 238 (2001), 101-114. (with V. Pasol and A. Popescu) 

*73.    The generating degree of Cp, Canad. Math. Bull. Vol. 44, (2001), 3-11. (with V. Alexandru and A. Zaharescu)
 
74.     Metric invariants in $BdR+$ associated to differential operators, Rev. Roum. Math. Pures Appl. 33 (1988), 393-400. (with V. Alexandru and *A. Zaharescu) 

*75.    Good elements and metric invariants in BdR+, J. Math. Kyoto Univ, vol 43, Nr. 1 (2003), 125-137. (with V. Alexandru and A. Zaharescu) 

*76.    On afine subdomains (to appear) (with G. Groza) 

*77.    A representation theorem for a class of rigid analytic functions, (with V. Alexandru and A. Zaharescu) (J. Th. Nombres Bordeaux 15 (2003), 639-650. 

*78.    On the spectral norm of Algebraic Numbers (to appear Math. Nachtr.) (with A. Popescu and A. Zaharescu) 

*79.    Universal Property of the Kaplansky Ideal Transform and Affiness of Open Subsets, J. Pure and Appl. Alg., 173, (2002), 121-134. (with Marco Fontana)
 
*80.    Metric invariants over Henselian valued Fields, J. Of Algebra, 266 (1), (2003), 14-26. (with A. Popescu and A. Zaharescu) 

*81.    Chains of metric invariants over a local field, Acta Arithmetica, 103 (1), (2002), 27-40. (with A. Popescu, M. Vajaitu and A. Zaharescu) 

*82.    Transcendental divisors and their critical functions, Manuscripta Math., 110 (4), (2003), 527-541. (with A. Popescu and A. Zaharescu) 

*83.    The Galois Action on Plane Compacts, J. Of Algebra, 270, (2003), 238-248. (with A. Popescu and A. Zaharescu) 

*84.    Total valuation rings of $K(X, σ)$ containing K, Communications in Algebra, Volume 30, Number 11  (2002), 5535 – 5546 (with S. Kobayashi, H. Marubayashi and C. Vraciu)
 
*85.    Total valuation Rings of Ore extensions, Result Math., 43 (2003), 373-379. (with S. Kobayashi, H. Marubayashi and C. Vraciu and G. Xie)
 
*86.    Non-commutative valuation rings of the quotient artinian ring of a skew polynomial ring, Algebra and Representation Theory (2005), 8; 57-68 (with S. Kobayashi, H. Marubayashi and C. Vraciu and G. Xie)
 
*87.    The structure of localization systems of a class of Prüfer Domain (to appear) (with H. Marubayashi and E. L Popescu)
 
*88.    On the existence of trace for elements of Cp Algebra and Representation Theory (2006), 9; 47-66 (with M. Vajaitu and A. Zaharescu)
 
*89.    Trace Series on $Qk$, Result in Math., 43 (2003), Nr 3-4, 331-341 (with A. Popescu and A. Zaharescu)
 
*90.    A Galois Theory for the Banach Algebra of continuous symmetric functions on absolute Galois Group., Result. Math. 45, No. 3-4, 349-358 (2004) (with A. Popescu and A. Zaharescu) 

*91.    The p-adic measure on the orbit of an element of Cp, Rend. Sem.Mat. Univ. Padova, Vol.118, 197-216 (2007)  (with V. Alexandru, M. Vajaitu and A. Zaharescu) 

*92.    Analytic Normal Basis Theorem Cent. Eur. J. Math., 6(3), 351-356 (2008) (with V. Alexandru and A. Zaharescu)

*93.    On the automorphisms of the spectral completion of the algebraic numbers field,  Journal of Pure and Applied Algebra, 212 (2008), 1427–1431 (with E. L Popescu  and A. Popescu)

*94.    The behaviour of rigid analytic functions around orbits of elements of Cp (to appear) (with S. Achimescu, V. Alexandru, M. Vajaitu and A. Zaharescu)

*95.    A Galois Theory for the field extensions $K((X))$/ K (to appear) (with Asim Naseen and A. Popescu)


96.     On the structure of compact subsets of Cp, Acta Arithmetica, 123. 3 (2006), 253-266. (with A. D. R. Choudary, and A. Popescu) 

97.     On the existence of trace for elements of Cp, Algebras and Representation Theory (2006) 9: 47-66. (with M. Vajaitu and A. Zaharescu) 

98.     The p-adic measure on the orbit of an element of Cp, Rend. Sem. Mat. Univ. Padova, Vol.118 (2007), 197-216. (with V. Alexandru, M. Vajaitu and A. Zaharescu) 

99.     Analytic Normal Basis Theorem Cent. Eur. J. Math., 6 (3) (2008), 351-356. (with V. Alexandru and A. Zaharescu) 

100.    Norms on K[X1, . . . ,Xr], which are multiplicative on R, Result. Math., 51 (2008), 229-247.  (with G. Groza and A. Zaharescu) 

101.    On the automorphisms of the spectral completion of the algebraic numbers field,  Journal of Pure and Applied Algebra, 212 (2008), 1427–1431. (with E. L Popescu and A. Popescu) 

102.    All non–Archimedean norms on K[X1, . . . ,Xr],  Glasg. Math. J. 52, (2010), No.1, 1-18 (with G. Groza and A. Zaharescu) 

103.    On the Iwasawa algebra associated to a normal element of Cp, Proc. Indian Acad. Sci. Math. Sci. 120, (2010), No. 1, 45-55. (with V. Alexandru, M. Vajaitu and A. Zaharescu) 

104.    A Galois Theory for the field extensions K((X))/ K, Glasg. Math. J. 52,(2010), 447-451 (with Asim Naseen and A. Popescu)
 
105.    On the spectral norm of algebraic numbers (to appear in Math. Nachtr.) (with A. Popescu and A. Zaharescu)
 
106.    The behavior of rigid analytic functions around orbits of elements of Cp (to appear) (with S. Achimescu, V. Alexandru, M. Vajaitu and A. Zaharescu) 

107.    On localizing systems in a Prüfer Domain  (to appear in Communications in Algebra) (with H. Marubayashi and E.L. Popescu)
 
108.    The study of the spectral p-adic extension (to appear in Proc. Rom. Acad.) 

109.    Some compact subsets of Qp (to appear in Rev. Roum. Math. Pures et Apll.) 

110.    Representation results for equivariant rigid analytic functions (to appear) (V. Alexandru, N. Popescu, M. Vajaitu and A. Zaharescu) 

111.    On the zeros of rigid analytic functions (to appear) (V. Alexandru, N. Popescu, M. Vajaitu and A. Zaharescu).



%%%%%
%%%%%
\end{document}
