\documentclass[12pt]{article}
\usepackage{pmmeta}
\pmcanonicalname{HorseshoeLemma}
\pmcreated{2013-03-22 15:49:46}
\pmmodified{2013-03-22 15:49:46}
\pmowner{mps}{409}
\pmmodifier{mps}{409}
\pmtitle{horseshoe lemma}
\pmrecord{5}{37799}
\pmprivacy{1}
\pmauthor{mps}{409}
\pmtype{Theorem}
\pmcomment{trigger rebuild}
\pmclassification{msc}{18G10}
\pmsynonym{simultaneous resolution theorem}{HorseshoeLemma}

\endmetadata

% this is the default PlanetMath preamble.  as your knowledge
% of TeX increases, you will probably want to edit this, but
% it should be fine as is for beginners.

% almost certainly you want these
\usepackage{amssymb}
\usepackage{amsmath}
\usepackage{amsfonts}

% used for TeXing text within eps files
%\usepackage{psfrag}
% need this for including graphics (\includegraphics)
%\usepackage{graphicx}
% for neatly defining theorems and propositions
\usepackage{amsthm}
% making logically defined graphics
%%\usepackage{xypic}

% there are many more packages, add them here as you need them

% define commands here
%\newtheorem*{theorem*}{Theorem}
\begin{document}
\PMlinkescapeword{column}
\PMlinkescapeword{columns}
\PMlinkescapeword{row}
\PMlinkescapeword{rows}
\PMlinkescapeword{commutative}
\PMlinkescapeword{sequence}
\PMlinkescapeword{completes}
Let $\mathcal{A}$ be an abelian category with enough projectives.  
The {\em horseshoe lemma}, also called the 
{\em simultaneous resolution theorem}, says that if
\[\xymatrix{
             &             &             & 0\ar[d]         &   \\
\cdots\ar[r] & P'_1\ar[r]  & P'_0\ar[r]  & A'\ar[r]\ar[d]  & 0 \\
             &             &             & A\ar[d]         &   \\
\cdots\ar[r] & P''_1\ar[r] & P''_0\ar[r] & A''\ar[r]\ar[d] & 0 \\
             &             &             & 0               &
}\]
is a diagram in $\mathcal{A}$ such that the column is exact and the
rows are projective resolutions of $A'$ and $A''$ respectively, then
it can be filled to a commutative diagram
\[\xymatrix{
             & 0\ar[d]           & 0\ar[d]           & 0\ar[d]         &   \\
\cdots\ar[r] & P'_1\ar[r]\ar[d]  & P'_0\ar[r]\ar[d]  & A'\ar[r]\ar[d]  & 0 \\
\cdots\ar[r] & P_1\ar[r]\ar[d]   & P_0\ar[r]\ar[d]   & A\ar[r]\ar[d]  & 0 \\
\cdots\ar[r] & P''_1\ar[r]\ar[d] & P''_0\ar[r]\ar[d] & A''\ar[r]\ar[d] & 0 \\
             & 0                 & 0                 & 0               &
}\]
where all columns are exact, the middle row is a projective resolution 
of $A$, and $P_n=P'_n\oplus P''_n$ for all $n$.  If $\mathcal{A}$ is an
abelian category with enough injectives, the dual statement also holds.

\begin{proof}
We fill in the diagram a column at a time, proving exactness along the way.
First we construct a surjective map $\pi\colon P_0\to A$.  There is a map
from $P'_0$ to $A$, the composition $P'_0\to A'\to A$.  Since $P''_0$ is projective, there is a filler for the diagram
\[\xymatrix{
        & P''_0\ar[d]\ar@{.>}[ld] & \\
A\ar[r] & A''\ar[r]               & 0
}\]
Since $P_0$ is the coproduct of $P'_0$ and $P''_0$, there is a
unique filler $\pi$ for the diagram
\[\xymatrix{
P'_0\ar[r]\ar[rd] & P_0\ar@{.>}[d]^{\pi} & P''_0\ar[l]\ar[ld] \\
                  & A                    &
}\]
The diagram
\[\xymatrix{
0\ar[r] &
P'_0\ar[r]\ar[d] &
P_0\ar[r]\ar[d]^{\pi} &
P''_0\ar[r]\ar[d] &
0 \\
0\ar[r] & A'\ar[r] & A\ar[r] & A''\ar[r] & 0
}\]
is commutative and has exact rows, so we may apply the 
\PMlinkname{short five lemma}{5Lemma}
to conclude that $\pi$ is surjective.

Now assume for induction that we have constructed a partial 
resolution of $A$ in this manner, yielding a diagram
\[\xymatrix{
0\ar[d] & 0\ar[d] &            & 0\ar[d]         &   \\
P'_n\ar[r]^{d'_n}\ar[d] & P'_{n-1}\ar[r]\ar[d] & \cdots\ar[r] & A'\ar[r]\ar[d]  & 0 \\
P_n\ar[r]^{d_n}\ar[d] & P_{n-1}\ar[r]\ar[d] & \cdots\ar[r] & A\ar[r]\ar[d]   & 0 \\
P''_n\ar[r]^{d''_n}\ar[d] & P''_{n-1}\ar[r]\ar[d] & \cdots\ar[r] & A''\ar[r]\ar[d] & 0 \\
0 & 0             &                   & 0                 &                
}\]
with exact rows and columns.  (We are not assuming here that $n>0$; if $n=0$, then $P'_{n-1}$ denotes $A'$ and $P'_{n-2}$ denotes 0; similar substitutions apply).  By the snake lemma, there is a commutative diagram
\[\xymatrix{
0\ar[r] & \ker d'_n\ar[r]\ar[d] & \ker d_n\ar[r]\ar[d] & \ker d''_n\ar[d] & \\
0\ar[r] & P'_n\ar[r]\ar[d]^{d'_n} & P_n\ar[r]\ar[d]^{d_n} & P''_n\ar[r]\ar[d]^{d''_n} & 0 \\
0\ar[r] & P'_{n-1}\ar[r]\ar[d]^{d'_{n-1}} & P_{n-1}\ar[r]\ar[d]^{d_{n-1}} & P''_{n-1}\ar[r]\ar[d]^{d''_{n-1}} & 0 \\
0\ar[r] & P'_{n-2}\ar[r] & P_{n-2}\ar[r] & P''_{n-2}\ar[r] & 0 
}\]
with exact rows and columns.  In fact, the map $\ker d_n\to\ker d''_n$ is
surjective; this can be verified by a diagram chase.  We can construct a
surjective map $b_{n+1}\colon P_{n+1}\to\ker d_n$ in the same way we 
constructed $\pi$.  Specifically, there is a map $P'_{n+1}\to\ker d'_n$
obtained by composition, and there is a filler for the diagram
\[\xymatrix{
& P''_n\ar[d]\ar@{.>}[ld] & \\
\ker d_n\ar[r] & \ker d''_n\ar[r] & 0
}\]
These maps determine $b_{n+1}$ uniquely.  Define $d_{n+1}\colon P_{n+1}\to P_n$
as the composition $P_{n+1}\xrightarrow{b_{n+1}} \ker d_n\hookrightarrow P_n$;
since $b_{n+1}$ is surjective, the sequence $P_{n+1}\to P_n\to P_{n-1}$ is
exact.  Hence we have managed to construct a diagram
\[\xymatrix{
0\ar[d] & 0\ar[d] & 0\ar[d] &            & 0\ar[d]         &   \\
P'_{n+1}\ar[r]^{d'_{n+1}}\ar[d] & P'_n\ar[r]^{d'_n}\ar[d] & P'_{n-1}\ar[r]\ar[d] & \cdots\ar[r] & A'\ar[r]\ar[d]  & 0 \\
P_{n+1}\ar[r]^{d_{n+1}}\ar[d] & P_n\ar[r]^{d_n}\ar[d] & P_{n-1}\ar[r]\ar[d] & \cdots\ar[r] & A\ar[r]\ar[d]   & 0 \\
P''_{n+1}\ar[r]^{d''_{n+1}}\ar[d] & P''_n\ar[r]^{d''_n}\ar[d] & P''_{n-1}\ar[r]\ar[d] & \cdots\ar[r] & A''\ar[r]\ar[d] & 0 \\
0 & 0 & 0             &                   & 0                 &                
}\]
with exact rows and columns.  This completes the proof.
\end{proof}

\begin{thebibliography}{9}
\bibitem{cite:CE}
Cartan, H. and S. Eilenberg, \emph{Homological algebra}, Princeton University Press, 1956.
\bibitem{cite:O}
Osborne, M.\ S., \emph{Basic homological algebra}, Springer-Verlag, 2000.
\end{thebibliography}
%%%%%
%%%%%
\end{document}
