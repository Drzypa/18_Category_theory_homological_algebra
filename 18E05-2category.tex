\documentclass[12pt]{article}
\usepackage{pmmeta}
\pmcanonicalname{2category}
\pmcreated{2013-03-22 18:18:30}
\pmmodified{2013-03-22 18:18:30}
\pmowner{bci1}{20947}
\pmmodifier{bci1}{20947}
\pmtitle{2-category}
\pmrecord{68}{40930}
\pmprivacy{1}
\pmauthor{bci1}{20947}
\pmtype{Definition}
\pmcomment{trigger rebuild}
\pmclassification{msc}{18E05}
\pmclassification{msc}{18-00}
\pmclassification{msc}{18D05}
\pmsynonym{small $Cat$-category}{2category}
\pmsynonym{$\mathcal{C}_2$}{2category}
%\pmkeywords{2-category}
%\pmkeywords{small $Cat$-category}
%\pmkeywords{functor categories}
%\pmkeywords{higher order categories}
%\pmkeywords{higher dimensional algebra}
%\pmkeywords{2- and 3- cells and morphisms}
\pmrelated{HigherDimensionalAlgebraHDA}
\pmrelated{FunctorCategories}
\pmrelated{AxiomsOfMetacategoriesAndSupercategories}
\pmrelated{ETAC}
\pmrelated{ETAS}
\pmrelated{FunctorCategories}
\pmrelated{Supercategories3}
\pmrelated{2Category2}
\pmrelated{GroupoidCategory}
\pmrelated{FundamentalGroupoidFunctor}
\pmrelated{CategoryTheory}
\pmrelated{FunctorCategory2}
\pmrelated{2CCategory}
\pmrelated{IndexOfCatego}
\pmdefines{2-morphism}
\pmdefines{small $Cat$-category}
\pmdefines{0-cell}
\pmdefines{$2$-cell}
\pmdefines{$3$-cell}
\pmdefines{$2$-morphism}
\pmdefines{2-categorical composition}
\pmdefines{horizontal composition}
\pmdefines{vertical composition}
\pmdefines{2-morphism}
\pmdefines{0-cell}
\pmdefines{multifunctor}

\endmetadata

% this is the default PlanetMath preamble. as your knowledge
% of TeX increases, you will probably want to edit this, but
% it should be fine as is for beginners.

% almost certainly you want these
\usepackage{amssymb}
\usepackage{amsmath}
\usepackage{amsfonts}

% used for TeXing text within eps files
%\usepackage{psfrag}
% need this for including graphics (\includegraphics)
%\usepackage{graphicx}
% for neatly defining theorems and propositions
%\usepackage{amsthm}
% making logically defined graphics
%%%\usepackage{xypic}

% there are many more packages, add them here as you need them

% define commands here
\usepackage{amsmath, amssymb, amsfonts, amsthm, amscd, latexsym}
%%\usepackage{xypic}
\usepackage[mathscr]{eucal}

\setlength{\textwidth}{6.5in}
%\setlength{\textwidth}{16cm}
\setlength{\textheight}{9.0in}
%\setlength{\textheight}{24cm}

\hoffset=-.75in %%ps format
%\hoffset=-1.0in %%hp format
\voffset=-.4in

\theoremstyle{plain}
\newtheorem{lemma}{Lemma}[section]
\newtheorem{proposition}{Proposition}[section]
\newtheorem{theorem}{Theorem}[section]
\newtheorem{corollary}{Corollary}[section]

\theoremstyle{definition}
\newtheorem{definition}{Definition}[section]
\newtheorem{example}{Example}[section]
%\theoremstyle{remark}
\newtheorem{remark}{Remark}[section]
\newtheorem*{notation}{Notation}
\newtheorem*{claim}{Claim}

\renewcommand{\thefootnote}{\ensuremath{\fnsymbol{footnote%%@
}}}
\numberwithin{equation}{section}

\newcommand{\Ad}{{\rm Ad}}
\newcommand{\Aut}{{\rm Aut}}
\newcommand{\Cl}{{\rm Cl}}
\newcommand{\Co}{{\rm Co}}
\newcommand{\DES}{{\rm DES}}
\newcommand{\Diff}{{\rm Diff}}
\newcommand{\Dom}{{\rm Dom}}
\newcommand{\Hol}{{\rm Hol}}
\newcommand{\Mon}{{\rm Mon}}
\newcommand{\Hom}{{\rm Hom}}
\newcommand{\Ker}{{\rm Ker}}
\newcommand{\Ind}{{\rm Ind}}
\newcommand{\IM}{{\rm Im}}
\newcommand{\Is}{{\rm Is}}
\newcommand{\ID}{{\rm id}}
\newcommand{\GL}{{\rm GL}}
\newcommand{\Iso}{{\rm Iso}}
\newcommand{\Sem}{{\rm Sem}}
\newcommand{\St}{{\rm St}}
\newcommand{\Sym}{{\rm Sym}}
\newcommand{\SU}{{\rm SU}}
\newcommand{\Tor}{{\rm Tor}}
\newcommand{\U}{{\rm U}}

\newcommand{\A}{\mathcal A}
\newcommand{\Ce}{\mathcal C}
\newcommand{\D}{\mathcal D}
\newcommand{\E}{\mathcal E}
\newcommand{\F}{\mathcal F}
\newcommand{\G}{\mathcal G}
\newcommand{\Q}{\mathcal Q}
\newcommand{\R}{\mathcal R}
\newcommand{\cS}{\mathcal S}
\newcommand{\cU}{\mathcal U}
\newcommand{\W}{\mathcal W}

\newcommand{\bA}{\mathbb{A}}
\newcommand{\bB}{\mathbb{B}}
\newcommand{\bC}{\mathbb{C}}
\newcommand{\bD}{\mathbb{D}}
\newcommand{\bE}{\mathbb{E}}
\newcommand{\bF}{\mathbb{F}}
\newcommand{\bG}{\mathbb{G}}
\newcommand{\bK}{\mathbb{K}}
\newcommand{\bM}{\mathbb{M}}
\newcommand{\bN}{\mathbb{N}}
\newcommand{\bO}{\mathbb{O}}
\newcommand{\bP}{\mathbb{P}}
\newcommand{\bR}{\mathbb{R}}
\newcommand{\bV}{\mathbb{V}}
\newcommand{\bZ}{\mathbb{Z}}

\newcommand{\bfE}{\mathbf{E}}
\newcommand{\bfX}{\mathbf{X}}
\newcommand{\bfY}{\mathbf{Y}}
\newcommand{\bfZ}{\mathbf{Z}}

\renewcommand{\O}{\Omega}
\renewcommand{\o}{\omega}
\newcommand{\vp}{\varphi}
\newcommand{\vep}{\varepsilon}

\newcommand{\diag}{{\rm diag}}
\newcommand{\grp}{{\mathbb G}}
\newcommand{\dgrp}{{\mathbb D}}
\newcommand{\desp}{{\mathbb D^{\rm{es}}}}
\newcommand{\Geod}{{\rm Geod}}
\newcommand{\geod}{{\rm geod}}
\newcommand{\hgr}{{\mathbb H}}
\newcommand{\mgr}{{\mathbb M}}
\newcommand{\ob}{{\rm Ob}}
\newcommand{\obg}{{\rm Ob(\mathbb G)}}
\newcommand{\obgp}{{\rm Ob(\mathbb G')}}
\newcommand{\obh}{{\rm Ob(\mathbb H)}}
\newcommand{\Osmooth}{{\Omega^{\infty}(X,*)}}
\newcommand{\ghomotop}{{\rho_2^{\square}}}
\newcommand{\gcalp}{{\mathbb G(\mathcal P)}}

\newcommand{\rf}{{R_{\mathcal F}}}
\newcommand{\glob}{{\rm glob}}
\newcommand{\loc}{{\rm loc}}
\newcommand{\TOP}{{\rm TOP}}

\newcommand{\wti}{\widetilde}
\newcommand{\what}{\widehat}

\renewcommand{\a}{\alpha}
\newcommand{\be}{\beta}
\newcommand{\ga}{\gamma}
\newcommand{\Ga}{\Gamma}
\newcommand{\de}{\delta}
\newcommand{\del}{\partial}
\newcommand{\ka}{\kappa}
\newcommand{\si}{\sigma}
\newcommand{\ta}{\tau}
\newcommand{\med}{\medbreak}
\newcommand{\medn}{\medbreak \noindent}
\newcommand{\bign}{\bigbreak \noindent}
\newcommand{\lra}{{\longrightarrow}}
\newcommand{\ra}{{\rightarrow}}
\newcommand{\rat}{{\rightarrowtail}}
\newcommand{\oset}[1]{\overset {#1}{\ra}}
\newcommand{\osetl}[1]{\overset {#1}{\lra}}
\newcommand{\hr}{{\hookrightarrow}}

\begin{document}
\begin{definition} 

A small $2$-category, $\mathcal{C}_2$, is the first of {\em higher-order} n-categories 
constructed as follows.

\begin{enumerate}
\item Define $\mathcal{C}at$ as the category of small categories and functors
\item Define a class of objects $A, B,...$ in $\mathcal{C}at$ called `$0$- \emph{cells}' 
\item For all `$0$-cells' $A$, $B$, consider a set denoted as ``$\mathcal{C}_2 (A,B)$'' that is defined as 
\PMlinkname{$\hom_{\mathcal{C}_2}(A,B)$}{Multifunctor}, with the elements of the latter set being the functors between the $0$-cells $A$ and $B$; the latter is then organized as a small category whose \PMlinkname{$2$-`morphisms'}{FunctorCategories}, or `$1$-cells' are defined by the natural transformations $\eta: F \to G$ for any two morphisms of $\mathcal{C}at$, (with $F$ and $G$ being functors between the `$0$-cells' $A$ and $B$, that is, $F,G: A \to  B$); as the `$2$-cells' can be considered as `$2$-morphisms' between $1$-morphisms, they are also written as: $\eta : F \Rightarrow  G$, and are depicted as labelled faces in the plane determined by their domains and codomains 
\item The $2$-categorical composition of $2$-morphisms is denoted as ``$\bullet$'' and is called the \emph{vertical composition}
\item A \emph{horizontal composition}, ``$\circ$'', is also defined for all triples of $0$-cells, $A$, $B$ and
$C$ in $\mathcal{C}at$ as the functor $$\circ: \mathcal{C}_2(B,C) \times \mathcal{C}_2(A,B) = \mathcal{C}_2(A,C),$$
which is \emph{associative}
\item The identities under horizontal composition are the identities of the $2$-cells of $1_X$
for any $X$ in $\mathcal{C}at$
\item For any object $A$ in $\mathcal{C}at$ there is a functor from the one-object/one-arrow category
$\textbf{1}$ (terminal object) to $\mathcal{C}_2(A,A)$. 
\end{enumerate}
\end{definition}

\subsection{Examples of 2-categories}

\begin{enumerate}
\item The $2$-category $\mathcal{C}at$ of small categories, functors, and natural transformations;
\item The $2$-category $\mathcal{C}at(\mathcal{E})$ of \emph{internal categories in any category $\mathcal{E}$ with
finite limits}, together with the internal functors and the internal natural transformations between such internal
functors; 
\item When $\mathcal{E} = \mathcal{S}et$, this yields again the category $\mathcal{C}at$, but if 
$\mathcal{E} = \mathcal{C}at$, then one obtains the 2-category of small \emph{double categories}; 
\item When $\mathcal{E} = \textbf{Group}$, one obtains the \emph{$2$-category of crossed modules}.
\end{enumerate} 

\subsection{Remarks}
\begin{itemize}
\item  In a manner similar to the (alternative) definition of small categories, one can 
describe $2$-categories in terms of $2$-arrows. Thus, let us consider a set with two defined operations 
$\otimes$, $\circ$, and also with units such that each operation endows the set with the structure of a 
(strict) category. Moreover, one needs to assume that all $\otimes$-units are also $\circ$-units, and that an associativity relation holds for the two products: 
$$(S \otimes T ) \circ (S \otimes T) = (S \circ S) \otimes (T \circ T);$$
\item A $2$-category is an example of a supercategory with just two composition laws, and it
is therefore an $\S_1$-supercategory, because the $\S_0$ supercategory is defined as a standard `$1$'-category subject only to the ETAC axioms.

\end{itemize}

%%%%%
%%%%%
\end{document}
