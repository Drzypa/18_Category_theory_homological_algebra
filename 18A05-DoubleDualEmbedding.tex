\documentclass[12pt]{article}
\usepackage{pmmeta}
\pmcanonicalname{DoubleDualEmbedding}
\pmcreated{2015-01-13 20:35:35}
\pmmodified{2015-01-13 20:35:35}
\pmowner{rmilson}{146}
\pmmodifier{rmilson}{146}
\pmtitle{double dual embedding}
\pmrecord{23}{32916}
\pmprivacy{1}
\pmauthor{rmilson}{146}
\pmtype{Example}
\pmcomment{trigger rebuild}
\pmclassification{msc}{18A05}
\pmclassification{msc}{15A04}
\pmrelated{DualHomomorphism}
\pmrelated{DualSpace}

\usepackage{amsfonts}
\usepackage{amsmath}

\usepackage{amssymb}

\usepackage{xypic}






\begin{document}
\PMlinkescapeword{language}
\PMlinkescapeword{identity}
\PMlinkescapeword{associates}


\newcommand{\rI}{\mathrm{I}}
\newcommand{\rD}{\mathrm{D}}
\newcommand{\cV}{\mathcal{V}}
\newcommand{\Hom}{\operatorname{Hom}}

Let $V$ be a vector space over a field $K$.  Recall that $V^*$, the
dual space, is defined to be the vector space of all linear forms on
$V$.  There is a natural embedding of $V$ into $V^{**}$, the dual of
its dual space.  In the language of categories, this embedding is a
natural transformation between the identity functor and the double
dual functor, both endofunctors operating on $\mathcal{V}_K$, the category of
vector spaces over $K$.

Turning to the details, let 
\[\rI, \rD:\cV_K\rightarrow \cV_K\]
denote the identity and the dual functors, respectively.  Recall that for a
linear mapping $L:U\rightarrow V$ (a morphism in $\cV_K$), the dual
homomorphism $D[L]:V^*\rightarrow U^*$ is defined by
\[ D[L](\alpha): u \mapsto \alpha(Lu),\quad u\in U,\; \alpha\in V^*.\]

The double dual embedding is a natural transformation
\[\delta:\rI\rightarrow \rD^2,\]
that associates to every $V\in \cV_K$ a linear homomorphism
$\delta_V\in\operatorname{Hom}(V,V^{**})$ described by
\[\delta_V(v): \alpha\mapsto \alpha(v),\quad v\in V,\;\alpha\in V^*\]
To show that this transformation is natural, let $L:U\rightarrow
V$ be a linear mapping.  We must show that the following diagram
commutes:

$$
\xymatrix{%
U \ar[r]^{\delta_U} \ar[d]^L & U^{**} \ar[d]^{D^2[L]} \\
V \ar[r]^{\delta_V} & V^{**}
}
$$

Let $u\in U$ and $\alpha\in V^*$ be given.  Following the arrows down
and right we have that
$$(\delta_V\circ L)(u): \alpha\mapsto \alpha(Lu).$$
Following the arrows right, then down we have that
\begin{eqnarray*}
(\rD[\rD[L]]\circ \delta_U)(u):
\alpha &\mapsto& (\delta_U u)(\rD[L]\alpha) \\
&=& (\rD[L]\alpha)(u) \\
&=& \alpha(Lu),
\end{eqnarray*}
as desired.

Let us also note that for every non-zero $v\in V$, there exists an
$\alpha\in V^*$ such that $\alpha(v)\neq 0$. Hence $\delta_V(v)\neq
0$, and hence $\delta_V$ is an embedding, i.e. it is one-to-one.  If
$V$ is finite dimensional, then $V^*$ has the same dimension as $V$.
Consequently, for finite-dimensional $V$, the natural embedding
$\delta_V$ is, in fact, an isomorphism.
%%%%%
%%%%%
\end{document}
