\documentclass[12pt]{article}
\usepackage{pmmeta}
\pmcanonicalname{RobertRosen}
\pmcreated{2013-03-22 18:18:50}
\pmmodified{2013-03-22 18:18:50}
\pmowner{bci1}{20947}
\pmmodifier{bci1}{20947}
\pmtitle{Robert Rosen}
\pmrecord{12}{40938}
\pmprivacy{1}
\pmauthor{bci1}{20947}
\pmtype{Biography}
\pmcomment{trigger rebuild}
\pmclassification{msc}{18-00}
\pmclassification{msc}{01A60}
\pmclassification{msc}{01A70}
%\pmkeywords{MR-systems}
%\pmkeywords{$(M}
%\pmkeywords{R)$--systems}
%\pmkeywords{metabolic-replication systems}
%\pmkeywords{categories of sets}
%\pmkeywords{categories of complex systems}
%\pmkeywords{complex systems biology}
%\pmkeywords{abstract relational biology}
%\pmkeywords{mathematical biology and mathematical biophysics}
\pmrelated{AbstractRelationalBiology}
\pmrelated{MathematicalBiology}
\pmrelated{NicolasRashevsky}

% this is the default PlanetMath preamble.  as your knowledge
% of TeX increases, you will probably want to edit this, but
% it should be fine as is for beginners.

% almost certainly you want these
\usepackage{amssymb}
\usepackage{amsmath}
\usepackage{amsfonts}

% used for TeXing text within eps files
%\usepackage{psfrag}
% need this for including graphics (\includegraphics)
%\usepackage{graphicx}
% for neatly defining theorems and propositions
%\usepackage{amsthm}
% making logically defined graphics
%%%\usepackage{xypic}

% there are many more packages, add them here as you need them

% define commands here
\usepackage{amsmath, amssymb, amsfonts, amsthm, amscd, latexsym}
%%\usepackage{xypic}
\usepackage[mathscr]{eucal}

\setlength{\textwidth}{6.5in}
%\setlength{\textwidth}{16cm}
\setlength{\textheight}{9.0in}
%\setlength{\textheight}{24cm}

\hoffset=-.75in     %%ps format
%\hoffset=-1.0in     %%hp format
\voffset=-.4in

\theoremstyle{plain}
\newtheorem{lemma}{Lemma}[section]
\newtheorem{proposition}{Proposition}[section]
\newtheorem{theorem}{Theorem}[section]
\newtheorem{corollary}{Corollary}[section]

\theoremstyle{definition}
\newtheorem{definition}{Definition}[section]
\newtheorem{example}{Example}[section]
%\theoremstyle{remark}
\newtheorem{remark}{Remark}[section]
\newtheorem*{notation}{Notation}
\newtheorem*{claim}{Claim}

\renewcommand{\thefootnote}{\ensuremath{\fnsymbol{footnote%%@
}}}
\numberwithin{equation}{section}

\newcommand{\Ad}{{\rm Ad}}
\newcommand{\Aut}{{\rm Aut}}
\newcommand{\Cl}{{\rm Cl}}
\newcommand{\Co}{{\rm Co}}
\newcommand{\DES}{{\rm DES}}
\newcommand{\Diff}{{\rm Diff}}
\newcommand{\Dom}{{\rm Dom}}
\newcommand{\Hol}{{\rm Hol}}
\newcommand{\Mon}{{\rm Mon}}
\newcommand{\Hom}{{\rm Hom}}
\newcommand{\Ker}{{\rm Ker}}
\newcommand{\Ind}{{\rm Ind}}
\newcommand{\IM}{{\rm Im}}
\newcommand{\Is}{{\rm Is}}
\newcommand{\ID}{{\rm id}}
\newcommand{\GL}{{\rm GL}}
\newcommand{\Iso}{{\rm Iso}}
\newcommand{\Sem}{{\rm Sem}}
\newcommand{\St}{{\rm St}}
\newcommand{\Sym}{{\rm Sym}}
\newcommand{\SU}{{\rm SU}}
\newcommand{\Tor}{{\rm Tor}}
\newcommand{\U}{{\rm U}}

\newcommand{\A}{\mathcal A}
\newcommand{\Ce}{\mathcal C}
\newcommand{\D}{\mathcal D}
\newcommand{\E}{\mathcal E}
\newcommand{\F}{\mathcal F}
\newcommand{\G}{\mathcal G}
\newcommand{\Q}{\mathcal Q}
\newcommand{\R}{\mathcal R}
\newcommand{\cS}{\mathcal S}
\newcommand{\cU}{\mathcal U}
\newcommand{\W}{\mathcal W}

\newcommand{\bA}{\mathbb{A}}
\newcommand{\bB}{\mathbb{B}}
\newcommand{\bC}{\mathbb{C}}
\newcommand{\bD}{\mathbb{D}}
\newcommand{\bE}{\mathbb{E}}
\newcommand{\bF}{\mathbb{F}}
\newcommand{\bG}{\mathbb{G}}
\newcommand{\bK}{\mathbb{K}}
\newcommand{\bM}{\mathbb{M}}
\newcommand{\bN}{\mathbb{N}}
\newcommand{\bO}{\mathbb{O}}
\newcommand{\bP}{\mathbb{P}}
\newcommand{\bR}{\mathbb{R}}
\newcommand{\bV}{\mathbb{V}}
\newcommand{\bZ}{\mathbb{Z}}

\newcommand{\bfE}{\mathbf{E}}
\newcommand{\bfX}{\mathbf{X}}
\newcommand{\bfY}{\mathbf{Y}}
\newcommand{\bfZ}{\mathbf{Z}}

\renewcommand{\O}{\Omega}
\renewcommand{\o}{\omega}
\newcommand{\vp}{\varphi}
\newcommand{\vep}{\varepsilon}

\newcommand{\diag}{{\rm diag}}
\newcommand{\grp}{{\mathbb G}}
\newcommand{\dgrp}{{\mathbb D}}
\newcommand{\desp}{{\mathbb D^{\rm{es}}}}
\newcommand{\Geod}{{\rm Geod}}
\newcommand{\geod}{{\rm geod}}
\newcommand{\hgr}{{\mathbb H}}
\newcommand{\mgr}{{\mathbb M}}
\newcommand{\ob}{{\rm Ob}}
\newcommand{\obg}{{\rm Ob(\mathbb G)}}
\newcommand{\obgp}{{\rm Ob(\mathbb G')}}
\newcommand{\obh}{{\rm Ob(\mathbb H)}}
\newcommand{\Osmooth}{{\Omega^{\infty}(X,*)}}
\newcommand{\ghomotop}{{\rho_2^{\square}}}
\newcommand{\gcalp}{{\mathbb G(\mathcal P)}}

\newcommand{\rf}{{R_{\mathcal F}}}
\newcommand{\glob}{{\rm glob}}
\newcommand{\loc}{{\rm loc}}
\newcommand{\TOP}{{\rm TOP}}

\newcommand{\wti}{\widetilde}
\newcommand{\what}{\widehat}

\renewcommand{\a}{\alpha}
\newcommand{\be}{\beta}
\newcommand{\ga}{\gamma}
\newcommand{\Ga}{\Gamma}
\newcommand{\de}{\delta}
\newcommand{\del}{\partial}
\newcommand{\ka}{\kappa}
\newcommand{\si}{\sigma}
\newcommand{\ta}{\tau}
\newcommand{\med}{\medbreak}
\newcommand{\medn}{\medbreak \noindent}
\newcommand{\bign}{\bigbreak \noindent}
\newcommand{\lra}{{\longrightarrow}}
\newcommand{\ra}{{\rightarrow}}
\newcommand{\rat}{{\rightarrowtail}}
\newcommand{\oset}[1]{\overset {#1}{\ra}}
\newcommand{\osetl}[1]{\overset {#1}{\lra}}
\newcommand{\hr}{{\hookrightarrow}}
\begin{document}
\emph{Robert Rosen} (1934-1997) American mathematician, theoretical biologist and Professor of Biophysics, worked first between 1964 and 1973 at the Center for Theoretical Biology at Buffalo, New York, and after 1974 at Dalhousie University. He completed his PhD studies in Mathematical Biophysics with Professor Nicolas Rashevsky, the Founder of Mathematical Biophysics and Mathematical Biology at the University of Chicago in the Committee for Mathematical Biology. During this very productive period he developed a categorical representation of Metabolic-Repair systems or $(M,R)$--systems for organisms, their relational structure and biodynamics. His last books entitled \emph{``Life Itself"} and \emph{``Essays on Life Itself} address the fundamental question ``What is Life?", biological complexity and logical-mathematical foundations of theoretical biology. He served as President of the Society for General Systems Research, (now \emph{ISSS}), during 1980-82.  Rosen's \emph{abstract relational biology} approach focuses on a definition of living organisms, and all complex systems, in terms of their internal ``organization" as open systems that cannot be reduced to their interacting components because of the multiple relations between metabolic and repair components that govern the organism's complex biodynamics. He deliberately chose the `simplest' graphs and categories for his representations of Metabolic-Replication Systems in small categories of sets endowed only with the discrete topology of sets, envisaging this choice as the most general and less restrictive. It turns out however that the categories of $(M,R)$-systems are \emph{Cartesian closed}, and may be considered in a very strict mathematical sense as subcategories of the category of sequential machines or automata, in a somewhat ironical vindication of the French philosopher Descartes' supposition that `all animals are only ellaborate machines' or mechanisms. The latter, mechanistic view prevails even today in most of general biology, but not in sociology or psychology.\\

 On the other hand, Rosen's metabolic-replication systems-- when extended to generalized, variable $(M,R)$-systems-- that were endowed with unique algebraic and variable topological structures, are different from any type of classical automaton, with the possible exception of \emph{quantum automata}. Rosen's relational approach to Biology is an extension and amplification of Nicolas Rashevsky's treatment of $n$-relations in, and among, organismic sets that he developed over two decades as a representation of both biological and social `organisms'.  

 Robert Rosen's contributions to abstract relational biology as well as theoretical biology are however much more numerous and substantial than the concise, outline presentation of his work provided in this brief biography. 
The reader may find many more interesting details in the following bibliography related to Robert Rosen's contributions
to Mathematical Biophysics, Mathematical and Theoretical Biology. 

\begin{thebibliography}{99}

\bibitem{RRosen1}
Rosen, R.: 1958a, A Relational Theory of Biological Systems \emph{Bulletin of Mathematical Biophysics} 
\textbf{20}: 245-260.

\bibitem{RRosen2}
Rosen, R.: 1958b, The Representation of Biological Systems from the Standpoint of the 
Theory of Categories., \emph{ Bulletin of Mathematical Biophysics} \textbf{20}: 317-341.

\bibitem{RRosen60}
Rosen, R. 1960. A quantum-theoretic approach to genetic problems. \emph{Bulletin of Mathematical Biophysics} 
22: 227-255.

\bibitem{RR70}
Rosen,R. 1970, \emph{Dynamical Systems Theory in Biology}. New York: Wiley Interscience. 

\bibitem{RR70}
Rosen, R. 1970, \emph{Optimality Principles in Biology}, New York and London: Academic Press. 

\bibitem{RR70}
Rosen, R. 1978, \emph{Fundamentals of Measurement and Representation of Natural Systems}, Elsevier Science Ltd, 

\bibitem{RR70}
Rosen, R. 1985, \emph{Anticipatory Systems: Philosophical, Mathematical and Methodological Foundations}. Pergamon Press:
New York. 

\bibitem{RRosen87}
Rosen, R.: 1987, On Complex Systems, \emph{European Journal of Operational Research} 
\textbf{30}, 129-134.

\bibitem{RR91}
Rosen, R. 1991, \emph{Life Itself: A Comprehensive Inquiry into the Nature, Origin, and Fabrication of Life}, Columbia University Press 

\bibitem{EACV2}
Ehresmann, A. C. and J.-P. Vanbremersch: 2006, The Memory Evolutive Systems as a Model of Rosen's Organisms, 
in \emph{Complex Systems Biology}, I.C. Baianu, Editor, \emph{Axiomathes} \textbf{16} (1--2), pp. 13-50.

\bibitem{Elsasser}
Elsasser, M.W.: 1981, A Form of Logic Suited for Biology., In: Robert, Rosen, ed., \emph{Progress in Theoretical Biology},  Volume 6, Academic Press, New York and London, pp 23-62.

\bibitem{Rashevsky1-yr1965}
Rashevsky, N.: 1965, The Representation of Organisms in Terms of Predicates, \emph{Bulletin of Mathematical Biophysics} \textbf{27}: 477-491.

\bibitem{Rashevsky2-1969}
Rashevsky, N.: 1969, Outline of a Unified Approach to Physics, Biology and Sociology., \emph{Bulletin of Mathematical Biophysics} \textbf{31}: 159--198.

\bibitem{ICB2k6}
Baianu, I. C.: 2006, Robert Rosen's Work and Complex Systems Biology, \emph{Axiomathes} \textbf{16}(1--2):25--34.

\bibitem{ICB70}
Baianu, I.C.: 1970, Organismic Supercategories: II. On Multistable Systems. \emph{Bulletin of Mathematical Biophysics}, \textbf{32}: 539-561.

\bibitem{ICB73}
Baianu, I.C.: 1973, Some Algebraic Properties of \emph{\textbf{(M,R)}} -- Systems. \emph{Bulletin of Mathematical Biophysics} \textbf{35}, 213-217.

\bibitem{ICBM74}
Baianu, I.C. and M. Marinescu: 1974, On A Functorial Construction of \emph{\textbf{(M,R)}}-- Systems. \emph{Revue Roumaine de Mathematiques Pures et Appliqu\'ees} \textbf{19}: 388-391.

\bibitem{ICB2}
Baianu, I.C.: 1980, Natural Transformations of Organismic Structures.,
\emph{Bulletin of Mathematical Biology},\textbf{42}: 431-446.

\bibitem{ICB87}
Baianu, I. C.: 1986--1987, Computer Models and Automata Theory in Biology and Medicine.,  in M. Witten (ed.), \emph{Mathematical Models in Medicine}, vol. 7., Ch.11 Pergamon Press, New York, 1513 -1577; URLs: \emph{CERN Preprint No. EXT-2004-072: } http://doe.cern.ch//archive/electronic/other/ext/ext-2004-072.pdf ;
http://en.scientificcommons.org/1857371 .

\bibitem{BBGG2k6}
Baianu I. C., Brown R., Georgescu G. and J. F. Glazebrook: 2006b, Complex Nonlinear Biodynamics in Categories, Higher Dimensional Algebra and \L ukasiewicz--Moisil Topos: Transformations of Neuronal, Genetic and Neoplastic Networks., \emph{Axiomathes}, \textbf{16} Nos. 1--2: 65--122.

\end{thebibliography}



 


%%%%%
%%%%%
\end{document}
