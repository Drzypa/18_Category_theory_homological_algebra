\documentclass[12pt]{article}
\usepackage{pmmeta}
\pmcanonicalname{EnoughProjectives}
\pmcreated{2013-03-22 14:50:16}
\pmmodified{2013-03-22 14:50:16}
\pmowner{CWoo}{3771}
\pmmodifier{CWoo}{3771}
\pmtitle{enough projectives}
\pmrecord{10}{36506}
\pmprivacy{1}
\pmauthor{CWoo}{3771}
\pmtype{Definition}
\pmcomment{trigger rebuild}
\pmclassification{msc}{18G05}
\pmclassification{msc}{18E10}
\pmrelated{EnoughInjectives}
\pmrelated{ProjectiveObject}

% this is the default PlanetMath preamble.  as your knowledge
% of TeX increases, you will probably want to edit this, but
% it should be fine as is for beginners.

% almost certainly you want these
\usepackage{amssymb,amscd}
\usepackage{amsmath}
\usepackage{amsfonts}

% used for TeXing text within eps files
%\usepackage{psfrag}
% need this for including graphics (\includegraphics)
%\usepackage{graphicx}
% for neatly defining theorems and propositions
%\usepackage{amsthm}
% making logically defined graphics
%%\usepackage{xypic}

% there are many more packages, add them here as you need them

% define commands here
\begin{document}
Let $\mathcal{A}$ be an abelian category.  $\mathcal{A}$ is said to have \emph{enough projectives} if, for every object $A$ of $\mathcal{A}$, there is a projective object $P$ of $\mathcal{A}$ and an exact sequence 
$$\xymatrix{P \ar[r]^p & A \ar[r] & 0}.$$  In other words, the map $p\colon P \to A$ is epi, or an epimorphism.

\textbf{Example.}  Let $R$ be a ring.  The category of left (right) $R$-modules is an abelian category having enough projectives.  This is true since, for every left (right) $R$-module $M$, we can take $F$ to be the free (and hence projective) $R$-module generated by a generating set $X$ for $M$ (we can in fact take $X$ to be $M$).  Then the canonical projection $\pi\colon F\to M$ is the required surjection.

More generally, a category $\mathcal{C}$ is said to have \emph{enough projectives} if every object is a strong quotient object of a projective object.


\begin{thebibliography}{9}
\bibitem{fb} F. Borceux \emph{Basic Category Theory, Handbook of Categorical Algebra I}, Cambridge University Press, Cambridge (1994)
\end{thebibliography}
%%%%%
%%%%%
\end{document}
