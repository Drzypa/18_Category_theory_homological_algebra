\documentclass[12pt]{article}
\usepackage{pmmeta}
\pmcanonicalname{ReferencesListForMMPostnikov}
\pmcreated{2013-03-22 18:21:50}
\pmmodified{2013-03-22 18:21:50}
\pmowner{bci1}{20947}
\pmmodifier{bci1}{20947}
\pmtitle{references list for MMPostnikov}
\pmrecord{13}{41003}
\pmprivacy{1}
\pmauthor{bci1}{20947}
\pmtype{Bibliography}
\pmcomment{trigger rebuild}
\pmclassification{msc}{18-00}
\pmclassification{msc}{55-00}
%\pmkeywords{M.M. Postnikov List of Publications translated in English}
\pmrelated{BibliographyForPhysicalMathematicsOfOperatorAlgebrasAndAQFTLToZ2}
\pmrelated{BernoulliPolynomial}
\pmrelated{BernoulliPolynomialsAndNumbers}

% this is the default PlanetMath preamble.  as your knowledge
% of TeX increases, you will probably want to edit this, but
% it should be fine as is for beginners.

% almost certainly you want these
\usepackage{amssymb}
\usepackage{amsmath}
\usepackage{amsfonts}

% used for TeXing text within eps files
%\usepackage{psfrag}
% need this for including graphics (\includegraphics)
%\usepackage{graphicx}
% for neatly defining theorems and propositions
%\usepackage{amsthm}
% making logically defined graphics
%%%\usepackage{xypic}

% there are many more packages, add them here as you need them

% define commands here
% this is the default PlanetMath preamble. as your knowledge
% of TeX increases, you will probably want to edit this, but
% it should be fine as is for beginners.

% almost certainly you want these
\usepackage{amssymb}
\usepackage{amsmath}
\usepackage{amsfonts}

% used for TeXing text within eps files
%\usepackage{psfrag}
% need this for including graphics (\includegraphics)
%\usepackage{graphicx}
% for neatly defining theorems and propositions
\usepackage{amsthm}
\usepackage[T2A]{fontenc}
\usepackage[russian, english]{babel}

% making logically defined graphics
%%%\usepackage{xypic}

% there are many more packages, add them here as you need them

% define commands here

\theoremstyle{definition}
\newtheorem*{thmplain}{Theorem}
\begin{document}
\subsection{M.M. Postnikov's List of Publications translated in English}

\begin{thebibliography}{9}


\bibitem{MMPList}
M.M. Postnikov. 
\PMlinkexternal{List of Mathematical papers in both Russian and English.}{http://www.math.sciences.univ-nantes.fr/~pajitnov/PMC/PostnikovsWorks.pdf}


\bibitem{Postnikov82}
M. M. Postnikov. 1982. \emph{Introduction to algebraic number theory}. Science Publs (``Nauka''),
Moscow. 

\bibitem{MMP} 
\CYRM. \CYRM. \CYRP\cyro\cyrs\cyrt\cyrn\cyri\cyrk\cyro\cyrv:
{\em \CYRV\cyrv\cyre\cyrd\cyre\cyrn\cyri\cyre\, \cyrv\, \cyrt\cyre\cyro\cyrr\cyri\cyryu\, \cyra\cyrl\cyrg\cyre\cyrb\cyrr\cyra\cyri\cyrch\cyre\cyrs\cyrk\cyri\cyrh \,
\cyrch\cyri\cyrs\cyre\cyrl}. \,\CYRI\cyrz\cyrd\cyra\cyrt\cyre\cyrl\cyrsftsn\cyrs\cyrt\cyrv\cyro \,
``\CYRN\cyra\cyru\cyrk\cyra''. \CYRM\cyro\cyrs\cyrk\cyrv\cyra \,(1982).

\end{thebibliography}


%%%%%
%%%%%
\end{document}
