\documentclass[12pt]{article}
\usepackage{pmmeta}
\pmcanonicalname{NormalCategory}
\pmcreated{2013-03-22 18:20:31}
\pmmodified{2013-03-22 18:20:31}
\pmowner{CWoo}{3771}
\pmmodifier{CWoo}{3771}
\pmtitle{normal category}
\pmrecord{8}{40975}
\pmprivacy{1}
\pmauthor{CWoo}{3771}
\pmtype{Definition}
\pmcomment{trigger rebuild}
\pmclassification{msc}{18E10}
\pmsynonym{normal monic}{NormalCategory}
\pmsynonym{conormal epi}{NormalCategory}
\pmdefines{normal}
\pmdefines{normal monomorphism}
\pmdefines{normal subobject}
\pmdefines{conormal}
\pmdefines{conormal epimorphism}
\pmdefines{conormal category}
\pmdefines{conormal quotient object}

\endmetadata

\usepackage{amssymb,amscd}
\usepackage{amsmath}
\usepackage{amsfonts}
\usepackage{mathrsfs}

% used for TeXing text within eps files
%\usepackage{psfrag}
% need this for including graphics (\includegraphics)
%\usepackage{graphicx}
% for neatly defining theorems and propositions
\usepackage{amsthm}
% making logically defined graphics
%%\usepackage{xypic}
\usepackage{pst-plot}

% define commands here
\newcommand*{\abs}[1]{\left\lvert #1\right\rvert}
\newtheorem{prop}{Proposition}
\newtheorem{thm}{Theorem}
\newtheorem{ex}{Example}
\newcommand{\real}{\mathbb{R}}
\newcommand{\pdiff}[2]{\frac{\partial #1}{\partial #2}}
\newcommand{\mpdiff}[3]{\frac{\partial^#1 #2}{\partial #3^#1}}
\begin{document}
A monomorphism is a category is said to be \emph{normal} if it is a kernel (of a morphism).  A subobject of an object is \emph{normal} if any (and hence all) of its representing monomorphisms is normal.

For example, in \textbf{Grp}, the category of groups, the inclusion of a subgroup $H\subseteq G$ into $G$ is normal iff $H$ is a normal subgroup of $G$.

A category is said to be \emph{normal} if every monic is a kernel.  Equivalently, a normal category is a category in which every subobject of every object is normal.

Dually, an epimorphism is \emph{conormal} if it is a cokernel (of a morphism).  A quotient object of an object is \emph{conormal} if any (and hence all) of its representing epimorphisms is conormal.  A category is said to be \emph{conormal} if every epimorphism is conormal.

The category $\textbf{AbGrp}$ of abelian groups, and more generally, any abelian category, is normal and conormal.

\begin{thebibliography}{9}
\bibitem{cf} C. Faith \emph{Algebra: Rings, Modules, and Categories I}, Springer-Verlag, New York (1973)
\end{thebibliography}
%%%%%
%%%%%
\end{document}
