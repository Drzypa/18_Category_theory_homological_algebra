\documentclass[12pt]{article}
\usepackage{pmmeta}
\pmcanonicalname{DualityPrinciple}
\pmcreated{2013-03-22 12:28:42}
\pmmodified{2013-03-22 12:28:42}
\pmowner{mathcam}{2727}
\pmmodifier{mathcam}{2727}
\pmtitle{duality principle}
\pmrecord{8}{32688}
\pmprivacy{1}
\pmauthor{mathcam}{2727}
\pmtype{Definition}
\pmcomment{trigger rebuild}
\pmclassification{msc}{18A05}
\pmdefines{self-dual statement}

\endmetadata

% this is the default PlanetMath preamble.  as your knowledge
% of TeX increases, you will probably want to edit this, but
% it should be fine as is for beginners.

% almost certainly you want these
\usepackage{amssymb}
\usepackage{amsmath}
\usepackage{amsfonts}

% used for TeXing text within eps files
%\usepackage{psfrag}
% need this for including graphics (\includegraphics)
%\usepackage{graphicx}
% for neatly defining theorems and propositions
%\usepackage{amsthm}
% making logically defined graphics
%%%\usepackage{xypic} 

% there are many more packages, add them here as you need them

% define commands here
\begin{document}
Let $\Sigma$ be any statement of the elementary theory of an abstract category. We form the dual of $\Sigma$ as follows:
\begin{enumerate}
\item Replace each occurrence of ``domain'' in $\Sigma$ with ``codomain'' and conversely.
\item Replace each occurrence of $g \circ f =h$ with $f \circ g = h$
\end{enumerate} 
Informally, these conditions state that the dual of a statement is formed by reversing arrows and compositions. For example, consider the following statements about a category $\mathcal{C}$:
\begin{itemize}
\item $f:A \to B$
\item $f$ is monic, i.e. for all morphisms $g,h$ for which composition makes sense, $f \circ g = f \circ h$ implies $g=h$.
\end{itemize}
The respective dual statements are
\begin{itemize}
\item $f:B \to A$
\item $f$ is epi, i.e. for all morphisms $g,h$ for which composition makes sense, $g \circ f = h \circ f$ implies $g=h$.
\end{itemize}
The \emph{duality principle} asserts that if a statement is a theorem, then the dual statment is also a theorem. We take "theorem" here to mean provable from the axioms of the elementary theory of an abstract category. In practice, for a valid statement about a particular category $\mathcal{C}$, the dual statement is valid in the dual category $\mathcal{C}^{*}$ ($\mathcal{C}^{op}$).

If the property $\Sigma$ is the same as its dual, then it is called \emph{self-dual}.
%%%%%
%%%%%
\end{document}
