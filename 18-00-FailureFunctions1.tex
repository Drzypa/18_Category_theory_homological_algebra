\documentclass[12pt]{article}
\usepackage{pmmeta}
\pmcanonicalname{FailureFunctions}
\pmcreated{2013-03-22 19:33:32}
\pmmodified{2013-03-22 19:33:32}
\pmowner{bci1}{20947}
\pmmodifier{bci1}{20947}
\pmtitle{failure functions}
\pmrecord{2}{42544}
\pmprivacy{1}
\pmauthor{bci1}{20947}
\pmcomment{trigger rebuild}
\pmclassification{msc}{18-00}

NULL
\begin{document}
\section{FAILURE FUNCTIONS}
I have written about failure functions on several maths sites, in my paper "A theorem a la Ramanujan " and also on my blogsite. I now propose to present a comprehensive piece.
Abstract definition: Let $phi(x)$ be a function of x. Then
$x = psi(x_0)$ is a failure function if the values of x generated by $psi(x_0)$ , when substituted in phi(x), generate only failures in accordance with our definition of a failure. Here $x_0$ is a specific value of x.
Examples: (i) Let the mother function be a polynomial 
in x (coeffficients belong to Z ), say phi(x).
Let our definition of a failure be a 
composite number. Then $x = psi (x_0) = 
x_0 + k(phi(x_0))$ is a failure function
since the values of x generated by $phi (x_0)$, when substituted in $phi(x)$ ,
generate only failures.
(ii) Let the mother function be an expon
-ential function, say $phi(x) = a^x + c$.
Then $x= psi(x_0) = x_0 + k.Eulerphi(phi(
x_0))$ is a failure function since the val
ues of x generated by $psi(x_0)$, when 
substitsuted in the mother function,
generate only failures. Note: Here too 
our definition of a failure is a compos
-ite number and  k belongs to N.
(iii) Let our definition of a failure be a 
non-Carmichael number. Let the mother function be $2^n + 49$. 
Then, $n = 5 + 
6k$ is a failure function.

\end{document}
