\documentclass[12pt]{article}
\usepackage{pmmeta}
\pmcanonicalname{MathematicalFoundationsOfQuantumFieldTheories}
\pmcreated{2013-03-22 18:29:48}
\pmmodified{2013-03-22 18:29:48}
\pmowner{bci1}{20947}
\pmmodifier{bci1}{20947}
\pmtitle{mathematical foundations of quantum field theories}
\pmrecord{23}{41181}
\pmprivacy{1}
\pmauthor{bci1}{20947}
\pmtype{Topic}
\pmcomment{trigger rebuild}
\pmclassification{msc}{18A15}
\pmclassification{msc}{55U99}
\pmclassification{msc}{55U40}
\pmclassification{msc}{08A99}
\pmsynonym{quantum field theories}{MathematicalFoundationsOfQuantumFieldTheories}
\pmsynonym{QFT}{MathematicalFoundationsOfQuantumFieldTheories}
\pmsynonym{QED}{MathematicalFoundationsOfQuantumFieldTheories}
%\pmkeywords{quantum field theories}
%\pmkeywords{QFT}
%\pmkeywords{QED}
%\pmkeywords{QCD}
\pmrelated{QEDInTheoreticalAndMathematicalPhysics}
\pmrelated{QED}
\pmrelated{QuantumGravityTheories}
\pmrelated{StateOnTheTetrahedron}
\pmrelated{QuantumAutomataAndQuantumComputation2}
\pmrelated{OverviewOfTheContentOfPlanetMath}

% this is the default PlanetMath preamble.  as your knowledge
% of TeX increases, you will probably want to edit this, but
% it should be fine as is for beginners.

% almost certainly you want these
\usepackage{amssymb}
\usepackage{amsmath}
\usepackage{amsfonts}

% used for TeXing text within eps files
%\usepackage{psfrag}
% need this for including graphics (\includegraphics)
%\usepackage{graphicx}
% for neatly defining theorems and propositions
%\usepackage{amsthm}
% making logically defined graphics
%%%\usepackage{xypic}

% there are many more packages, add them here as you need them

% define commands here
\usepackage{amsmath, amssymb, amsfonts, amsthm, amscd, latexsym}
%%\usepackage{xypic}
\usepackage[mathscr]{eucal}

\setlength{\textwidth}{6.5in}
%\setlength{\textwidth}{16cm}
\setlength{\textheight}{9.0in}
%\setlength{\textheight}{24cm}

\hoffset=-.75in     %%ps format
%\hoffset=-1.0in     %%hp format
\voffset=-.4in

\theoremstyle{plain}
\newtheorem{lemma}{Lemma}[section]
\newtheorem{proposition}{Proposition}[section]
\newtheorem{theorem}{Theorem}[section]
\newtheorem{corollary}{Corollary}[section]

\theoremstyle{definition}
\newtheorem{definition}{Definition}[section]
\newtheorem{example}{Example}[section]
%\theoremstyle{remark}
\newtheorem{remark}{Remark}[section]
\newtheorem*{notation}{Notation}
\newtheorem*{claim}{Claim}

\renewcommand{\thefootnote}{\ensuremath{\fnsymbol{footnote%%@
}}}
\numberwithin{equation}{section}

\newcommand{\Ad}{{\rm Ad}}
\newcommand{\Aut}{{\rm Aut}}
\newcommand{\Cl}{{\rm Cl}}
\newcommand{\Co}{{\rm Co}}
\newcommand{\DES}{{\rm DES}}
\newcommand{\Diff}{{\rm Diff}}
\newcommand{\Dom}{{\rm Dom}}
\newcommand{\Hol}{{\rm Hol}}
\newcommand{\Mon}{{\rm Mon}}
\newcommand{\Hom}{{\rm Hom}}
\newcommand{\Ker}{{\rm Ker}}
\newcommand{\Ind}{{\rm Ind}}
\newcommand{\IM}{{\rm Im}}
\newcommand{\Is}{{\rm Is}}
\newcommand{\ID}{{\rm id}}
\newcommand{\GL}{{\rm GL}}
\newcommand{\Iso}{{\rm Iso}}
\newcommand{\Sem}{{\rm Sem}}
\newcommand{\St}{{\rm St}}
\newcommand{\Sym}{{\rm Sym}}
\newcommand{\SU}{{\rm SU}}
\newcommand{\Tor}{{\rm Tor}}
\newcommand{\U}{{\rm U}}

\newcommand{\A}{\mathcal A}
\newcommand{\Ce}{\mathcal C}
\newcommand{\D}{\mathcal D}
\newcommand{\E}{\mathcal E}
\newcommand{\F}{\mathcal F}
\newcommand{\G}{\mathcal G}
\newcommand{\Q}{\mathcal Q}
\newcommand{\R}{\mathcal R}
\newcommand{\cS}{\mathcal S}
\newcommand{\cU}{\mathcal U}
\newcommand{\W}{\mathcal W}

\newcommand{\bA}{\mathbb{A}}
\newcommand{\bB}{\mathbb{B}}
\newcommand{\bC}{\mathbb{C}}
\newcommand{\bD}{\mathbb{D}}
\newcommand{\bE}{\mathbb{E}}
\newcommand{\bF}{\mathbb{F}}
\newcommand{\bG}{\mathbb{G}}
\newcommand{\bK}{\mathbb{K}}
\newcommand{\bM}{\mathbb{M}}
\newcommand{\bN}{\mathbb{N}}
\newcommand{\bO}{\mathbb{O}}
\newcommand{\bP}{\mathbb{P}}
\newcommand{\bR}{\mathbb{R}}
\newcommand{\bV}{\mathbb{V}}
\newcommand{\bZ}{\mathbb{Z}}

\newcommand{\bfE}{\mathbf{E}}
\newcommand{\bfX}{\mathbf{X}}
\newcommand{\bfY}{\mathbf{Y}}
\newcommand{\bfZ}{\mathbf{Z}}

\renewcommand{\O}{\Omega}
\renewcommand{\o}{\omega}
\newcommand{\vp}{\varphi}
\newcommand{\vep}{\varepsilon}

\newcommand{\diag}{{\rm diag}}
\newcommand{\grp}{{\mathbb G}}
\newcommand{\dgrp}{{\mathbb D}}
\newcommand{\desp}{{\mathbb D^{\rm{es}}}}
\newcommand{\Geod}{{\rm Geod}}
\newcommand{\geod}{{\rm geod}}
\newcommand{\hgr}{{\mathbb H}}
\newcommand{\mgr}{{\mathbb M}}
\newcommand{\ob}{{\rm Ob}}
\newcommand{\obg}{{\rm Ob(\mathbb G)}}
\newcommand{\obgp}{{\rm Ob(\mathbb G')}}
\newcommand{\obh}{{\rm Ob(\mathbb H)}}
\newcommand{\Osmooth}{{\Omega^{\infty}(X,*)}}
\newcommand{\ghomotop}{{\rho_2^{\square}}}
\newcommand{\gcalp}{{\mathbb G(\mathcal P)}}

\newcommand{\rf}{{R_{\mathcal F}}}
\newcommand{\glob}{{\rm glob}}
\newcommand{\loc}{{\rm loc}}
\newcommand{\TOP}{{\rm TOP}}

\newcommand{\wti}{\widetilde}
\newcommand{\what}{\widehat}

\renewcommand{\a}{\alpha}
\newcommand{\be}{\beta}
\newcommand{\ga}{\gamma}
\newcommand{\Ga}{\Gamma}
\newcommand{\de}{\delta}
\newcommand{\del}{\partial}
\newcommand{\ka}{\kappa}
\newcommand{\si}{\sigma}
\newcommand{\ta}{\tau}
\newcommand{\med}{\medbreak}
\newcommand{\medn}{\medbreak \noindent}
\newcommand{\bign}{\bigbreak \noindent}
\newcommand{\lra}{{\longrightarrow}}
\newcommand{\ra}{{\rightarrow}}
\newcommand{\rat}{{\rightarrowtail}}
\newcommand{\oset}[1]{\overset {#1}{\ra}}
\newcommand{\osetl}[1]{\overset {#1}{\lra}}
\newcommand{\hr}{{\hookrightarrow}}
\begin{document}
\subsection{Mathematical Foundations of Quantum Field Theories (QFT)}

\subsubsection{QED, QCD, Electroweak and Other Quantum Field Theories}

\begin{enumerate}
\item \textit{Quantum chromodynamics or QCD:} the advanced, standard mathematical and quantum physics treatment of strong force or nuclear interactions such as those among quarks and gluons, (or partons and mesons), that have an intrinsic threefold, or eightfold quantum symmetry described by the `quantum' group {\em SU(3)} (which was first reported in 1964 by the US Nobel Laureate Murray Gell-Mann and others); 
\item {\em Quantum electrodynamics QED}: that involves {\em U(1)} symmetry, is the advanced, standard mathematical and quantum physics treatment of electromagnetic interactions through several approaches, the more advanced including the path-integral approach by Feynman, Dirac's operator and QED equations, thus including either special or general
relativity formulations of electromagnetic phenomena;
\item Young--Mills theories;
\item Electroweak interactions: {\em SU(2)} Symmetry;
\item Algebraic quantum field theories (AQFT);
\item Homotopy quantum field theories (HQFT) and topological QFT's (TQFT); 
\item Quantum gravity (QG) and related theories.
\end{enumerate}

\subsubsection{Extended Quantum Symmetries}

 This obviates the need for `more fundamental' , or extended quantum symmetries, such as those afforded by either several larger groups such as $SU(3)  \times SU(2) \times U(1)$ (and their representations) in SUSY, or by spontaneously broken, multiple (`or localized') symmetries of a less restrictive kind present in `quantum groupoids' as for example in weak Hopf algebra representations. More generally, such extended quantum symmetries can be realized as locally compact groupoid, {\em $G_{lc}$} {\em unitary} representations, and even more `powerful' structures to the higher dimensional (quantum) symmetries of quantum double groupoids, quantum double algebroids, quantum categories/quantum supercategories in HDA, and/or quantum supersymmetry superalgebras (or graded `Lie' algebras, see- for example- the QFT ref. \cite{Weinberg2003} discussing superalgebras in quantum gravity). 
 
 Thus, certain finite irreducible representations correspond to `elementary' (quantum) particles and spin symmetry
representations have corresponding quantum obsevable operators, such as the Casimir operators. A well-known case is that of Pauli matrices that are representations of the special unitary group $SU(2)$. Supersymmetry, supergroups and
superoperators further expand SUSY to quantum gravity and quantum statistical mechanics.

\begin{thebibliography}{9}
\bibitem{Weinberg2003}
S. Weinberg. 2003. Quantum Field Theories, vol. 1-3, Cambridge University Press: Cambridge, UK.
\end{thebibliography}
%%%%%
%%%%%
\end{document}
