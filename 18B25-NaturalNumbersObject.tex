\documentclass[12pt]{article}
\usepackage{pmmeta}
\pmcanonicalname{NaturalNumbersObject}
\pmcreated{2013-03-22 16:36:34}
\pmmodified{2013-03-22 16:36:34}
\pmowner{mps}{409}
\pmmodifier{mps}{409}
\pmtitle{natural numbers object}
\pmrecord{4}{38806}
\pmprivacy{1}
\pmauthor{mps}{409}
\pmtype{Definition}
\pmcomment{trigger rebuild}
\pmclassification{msc}{18B25}
\pmdefines{strong natural numbers object}

% this is the default PlanetMath preamble.  as your knowledge
% of TeX increases, you will probably want to edit this, but
% it should be fine as is for beginners.

% almost certainly you want these
\usepackage{amssymb}
\usepackage{amsmath}
\usepackage{amsfonts}

% used for TeXing text within eps files
%\usepackage{psfrag}
% need this for including graphics (\includegraphics)
%\usepackage{graphicx}
% for neatly defining theorems and propositions
\usepackage{amsthm}
% making logically defined graphics
%%\usepackage{xypic}

% there are many more packages, add them here as you need them

% define commands here
\theoremstyle{remark}
\newtheorem*{example*}{Example}
\begin{document}
Let $\mathcal{C}$ be a category with a terminal object $1$.  A (weak) \emph{natural numbers object} consists of an object $N$ along with morphisms $z\colon 1\to N$ (``zero'') and $S\colon N\to N$ (``successor'') of $\mathcal{C}$ such that if $\xymatrix{1\ar[r]^{z'} & N'\ar[r]^{S'} & N}$ is a diagram in $\mathcal{C}$, then there exists a morphism $g\colon N\to N'$ such that the diagram
\[\xymatrix{
1\ar[r]^z\ar@{=}[d] & N\ar[r]^S\ar[d]^g  & N\ar[d]^g \\
1\ar[r]^{z'}        & N'\ar[r]^{S'}      & N'
}\]
is commutative.  The morphism $g$ is not required to be unique.  If we additionally require the morphism to be unique, we obtain a \emph{strong natural numbers object}.  Note that a strong natural numbers object can also be defined as an initial diagram of the form
\[\xymatrix{
1 \ar[r] & N\ar[r] & N.
}\]
%Not every category with a terminal object has a natural numbers object.

\begin{example*}
In the category $\mathbf{Set}$, the set $\mathbb{N}$ of natural numbers is a natural numbers object.  Using arithmetic notation for simplicity, the morphism $z\colon 1\to\mathbb{N}$ picks out the element zero, and the morphism $S\colon\mathbb{N}\to\mathbb{N}$ is defined by the formula $S(x) = x + 1$.  This object is the source of the name ``natural numbers object''.
\end{example*}

\begin{thebibliography}{99}
\bibitem{LaSc}
J.~Lambek and P.~J.~Scott. {\it Introduction to higher order categorical logic}. Cambridge University Press, 1986.
\end{thebibliography}

%%%%%
%%%%%
\end{document}
