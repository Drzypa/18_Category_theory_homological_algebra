\documentclass[12pt]{article}
\usepackage{pmmeta}
\pmcanonicalname{ExamplesOfMonics}
\pmcreated{2013-03-22 18:22:12}
\pmmodified{2013-03-22 18:22:12}
\pmowner{CWoo}{3771}
\pmmodifier{CWoo}{3771}
\pmtitle{examples of monics}
\pmrecord{7}{41011}
\pmprivacy{1}
\pmauthor{CWoo}{3771}
\pmtype{Example}
\pmcomment{trigger rebuild}
\pmclassification{msc}{18-00}
\pmclassification{msc}{18A20}

\usepackage{amssymb,amscd}
\usepackage{amsmath}
\usepackage{amsfonts}
\usepackage{mathrsfs}

% used for TeXing text within eps files
%\usepackage{psfrag}
% need this for including graphics (\includegraphics)
%\usepackage{graphicx}
% for neatly defining theorems and propositions
\usepackage{amsthm}
% making logically defined graphics
%%\usepackage{xypic}
\usepackage{pst-plot}

% define commands here
\newcommand*{\abs}[1]{\left\lvert #1\right\rvert}
\newtheorem{prop}{Proposition}
\newtheorem{thm}{Theorem}
\newtheorem{ex}{Example}
\newcommand{\real}{\mathbb{R}}
\newcommand{\pdiff}[2]{\frac{\partial #1}{\partial #2}}
\newcommand{\mpdiff}[3]{\frac{\partial^#1 #2}{\partial #3^#1}}
\begin{document}
This entry lists some common examples of monics (monomorphisms).  The examples also illustrate some of the techniques in finding monics.

\begin{enumerate}
\item In \textbf{Set}, the category of sets, the monics are exactly the one-to-one functions.  First, suppose $f:A\to B$ is one-to-one and that $g,h:C\to A$ are functions such that $f\circ g=f\circ h$.  Then for any $c\in C$, $f(g(c))=f(h(c))$.  Since $f$ is 1-1, $g(c)=h(c)$, or $g=h$, or $f$ is monic.  Conversely, suppose $f:A\to B$ is monic and that $f(x)=f(y)$.  Define functions $g,h:\lbrace z\rbrace \to A$ by $g(z)=x$ and $h(z)=y$.  Then $f\circ g(z)=f\circ h(z)$ by assumption.  But this means that $g(z)=h(z)$ since $f$ is monic.  Hence $x=y$, or $f$ is 1-1.
\item In \textbf{Grp}, the category of groups, the monics are exactly the injective group homomorphisms.  If $f:G\to H$ is 1-1 and $f\circ g=f\circ h$ for group homomorphisms $g,h:K\to G$.  Then for any $k\in K$, $f(g(k))=f(h(k))$, so that $g(k)=h(k)$.  Hence $f$ is monic.  Conversely, suppose that $f$ is monic and $f(x)=f(y)$.  Let $g,h:\mathbb{Z}\to G$ be group homomorphisms given by $g(1)=x$ and $h(1)=y$.  Then $f\circ g(1)=f(x)=f(y)=f\circ h(1)$.  This means that $x=g(1)=h(1)=y$, or $f$ is 1-1.
\item In \textbf{Rng}, the category of rings (with 1), the monics are exactly the injective ring homomorphisms.  If $f:R\to S$ is a 1-1 ring homomorphism, and $f\circ g=f\circ h$ for some ring homomorphisms $g,h:T\to R$, then for any $t\in T$, $f(g(t)=f(h(t))$, so that $g(t)=h(t)$, or $g=h$, so that $f$ is monic.  Conversely, let $f$ be monic and $f(x)=f(y)$.  Define $g,h:\mathbb{Z}[X]\to R$ be given by $g(X)=x$ and $h(X)=y$.  Once again, we see that $f(g(X))=f(x)=f(y)=f(h(X))$, so that $x=g(X)=h(X)=y$ since $f$ is monic.  Therefore, $f$ is 1-1.
\end{enumerate}

The technique above can be used to show that many of the morphisms whose underlying functions are injective, the morphisms themselves are monics.  However, monomorphisms are not synonymous with injections.  Below is a prototypical example:

In \textbf{DivAbGrp}, the category of divisible abelian groups, every injection is a monic, but not conversely.  If $f:A\to B$ is 1-1 and $f\circ g=f\circ h$ for $g,h:C\to A$, then $f\circ (g-h)=0$, so that $0=(g-h)(x)=g(x)-h(x)$ for all $x\in C$.  This means that $g=h$.  The counterexample comes from the following: let $f:\mathbb{Q}\to \mathbb{Q}/\mathbb{Z}$ be the canonical projection.  It is clear that $f$ is not 1-1.  Yet $f$ is a monomorphism: note first that $f\circ g=f\circ h$ is equivalent to $f\circ (g-h)=0$, so we may as well assume that $f\circ g=0$ and show that $g=0$ for any $g:C\to \mathbb{Q}$.  If not, then for some $a\in C$, $0\ne g(a)= p/q\in \mathbb{Q}$, where $p$ is a positive integer and $q\in \mathbb{Z}$.  If $q=1$, set $n=2p$; otherwise, set $n=p$.  Then $f\circ g(a/n)=f(1/2)=1/2$ if $q=1$, and $f\circ g(a/n)=f(1/q)=1/q$ otherwise.  In either case, $f\circ g (a/n) \ne 0$, a contradiction.


As another example where monomorphisms are not necessarily injective, it can be shown that in the category of pointed, connected topological spaces (with continuous functions preserving base points), the projection of the circular helix onto the unit circle (clearly not injective) is a monomorphism.
%%%%%
%%%%%
\end{document}
