\documentclass[12pt]{article}
\usepackage{pmmeta}
\pmcanonicalname{ImageOfAMorphism}
\pmcreated{2013-03-22 18:07:59}
\pmmodified{2013-03-22 18:07:59}
\pmowner{CWoo}{3771}
\pmmodifier{CWoo}{3771}
\pmtitle{image of a morphism}
\pmrecord{14}{40684}
\pmprivacy{1}
\pmauthor{CWoo}{3771}
\pmtype{Definition}
\pmcomment{trigger rebuild}
\pmclassification{msc}{18A05}
\pmsynonym{have images}{ImageOfAMorphism}
\pmsynonym{have coimages}{ImageOfAMorphism}
\pmsynonym{Ab2 axiom}{ImageOfAMorphism}
\pmrelated{ExactSequence2}
\pmrelated{CategoricalSequence}
\pmdefines{image}
\pmdefines{coimage}
\pmdefines{has images}
\pmdefines{has coimages}
\pmdefines{Ab2}

\usepackage{amssymb,amscd}
\usepackage{amsmath}
\usepackage{amsfonts}
\usepackage{mathrsfs}

% used for TeXing text within eps files
%\usepackage{psfrag}
% need this for including graphics (\includegraphics)
%\usepackage{graphicx}
% for neatly defining theorems and propositions
\usepackage{amsthm}
% making logically defined graphics
%%\usepackage{xypic}
\usepackage{pst-plot}

% define commands here
\newcommand*{\abs}[1]{\left\lvert #1\right\rvert}
\newtheorem{prop}{Proposition}
\newtheorem{thm}{Theorem}
\newtheorem{ex}{Example}
\newcommand{\real}{\mathbb{R}}
\newcommand{\pdiff}[2]{\frac{\partial #1}{\partial #2}}
\newcommand{\mpdiff}[3]{\frac{\partial^#1 #2}{\partial #3^#1}}
\newcommand{\goto}[1]{\buildrel #1 \over \longrightarrow}
\newcommand{\im}{\operatorname{im}}
\newcommand{\coim}{\operatorname{coim}}
\newcommand{\cok}{\operatorname{cok}}
\begin{document}
\subsubsection*{Image}

Let $\mathcal{C}$ be a category and a $f:A\to B$ a morphism from objects $A$ to $B$ in $\mathcal{C}$.  An \emph{image} of $f$ is a subobject $I$ of $B$ with a representing monomorphism $g:I\to B$, such that
\begin{itemize}
\item $f$ factors through $g$; i.e. there is a morphism $h:A\to I$ such that $f=g\circ h$:
$$A \goto{f} B = A \goto{h} I \goto{g} B$$
\item if $f$ factors through a monomorphism $s:J\to B$: 
$$A \goto{f} B = A \goto{t} J \goto{s} B,$$
then there is a morphism $x:I\to J$ such that
$$I \goto{g} B = I \goto{x} J \goto{s} B.$$
\end{itemize}

\textbf{Example}.  In the category of sets, the image of a function $f:A\to B$ is just the \PMlinkname{image}{Image} $f(A)$ with the canonical injection into $B$.

In the literature, the first bullet is equivalent to saying that the subobject $g: C\to B$ \emph{allows} $f$.  So the definition of the image of $f$ is the smallest subobject of $B$ that allows $f:A\to B$.

\subsubsection*{Coimage}

Dually, one defines the \emph{coimage} of $f:A\to B$ as a quotient object $C$ of $A$ with a representing epimorphism $h: A\to C$, such that
\begin{itemize}
\item $f$ factors through $h$; i.e. there is a morphism $g:C\to B$ such that $f=g\circ h$:
$$A \goto{f} B = A \goto{h} C \goto{g} B$$
\item if $f$ factors through an epimorphism $t:A\to D$: 
$$A \goto{f} B = A \goto{t} D \goto{s} B,$$
then there is a morphism $y:D\to C$ such that
$$A \goto{h} C = A \goto{t} D \goto{y} C.$$
\end{itemize}

\textbf{Remarks}.
\begin{enumerate}
\item In the definition of image, since $g$ is a monomorphism, $x$ is a monomorphism.  Furthermore, since $s$ is a monomorphism, $x$ is uniquely determined, and we have the following commutative diagram:
\begin{center}
$\xymatrix@R-=2pt{
& I \ar[dd]^x \ar[dr]^g & \\
A\ar[ur]^h \ar[dr]_t && B \\
& J \ar[ur]_s &
}$
\end{center}
\item Dually, $y$ is a uniquely determined epimorphism satisfying the following commutative diagram:
\begin{center}
$
\xymatrix@R-=2pt{
& D \ar[dd]^y \ar[dr]^g & \\
A\ar[ur]^t \ar[dr]_h && B \\
& C \ar[ur]_s &
}$
\end{center}
\item If $f:A\to B$ has an image (dually, coimage), it is unique up to isomorphism.  The image and coimage of $f$ are denoted by $\im(f)$ and $\coim(f)$ respectively.
\item Suppose that a category is Ab1.  If $\im(f)$ and $\coim(f)$ exist for $f:A\to B$, then there is a unique morphism $\overline{f}:\coim(f)\to \im(f)$ such that 
$$\xymatrix@C+=4pc{
{A}\ar[r]^{f}\ar[d]&{B}\\
{\coim(f)}\ar[r]^{\overline{f}}&{\im(f)}\ar[u]
}
$$
is commutative.  The \emph{Ab2 Axiom}, a la Grothendieck, is the statement that if $\overline{f}$ exists, it is an isomorphism.  A category is said to be \emph{Ab2} if it is Ab1, and every morphism satisfies the Ab2 Axiom.
\item A category $\mathcal{C}$ is said to \emph{have images} (dually, \emph{has coimages}) if the image (coimage) of any morphism exists.
\end{enumerate}

Every abelian category has images and coimages, and $$\im(f)=\ker(\cok(f))\qquad \mbox{ and }\qquad\coim(f)=\cok(\ker(f)),$$ where $\ker$ and $\cok$ are the kernel and cokernel operations.  In addition, we have the following important result:
\begin{quote}\emph{
if a morphism $f:A\to B$ can be factored as 
$$A \goto{f} B = A \goto{h} I \goto{g} B$$
with $g$ a monomorphism and $h$ an epimorphism, then $I$ (with $g$) is the image of $f$ and $I$ (with $h$) is the coimage of $f$.  }
\end{quote}
In other words, the factorization above is uniquely determined, up to isomorphism.
%%%%%
%%%%%
\end{document}
