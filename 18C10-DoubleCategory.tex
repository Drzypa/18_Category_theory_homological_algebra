\documentclass[12pt]{article}
\usepackage{pmmeta}
\pmcanonicalname{DoubleCategory}
\pmcreated{2013-03-22 19:21:55}
\pmmodified{2013-03-22 19:21:55}
\pmowner{bci1}{20947}
\pmmodifier{bci1}{20947}
\pmtitle{double category}
\pmrecord{26}{42318}
\pmprivacy{1}
\pmauthor{bci1}{20947}
\pmtype{Definition}
\pmcomment{trigger rebuild}
\pmclassification{msc}{18C10}
\pmclassification{msc}{18-00}
%\pmkeywords{double category}
%\pmkeywords{pseudo-double category of rings}
\pmrelated{FundamentalGroupoidFunctor}
\pmdefines{$Rng$}
\pmdefines{$2$-category}
\pmdefines{pseudo-double category of rings}
\pmdefines{$ObjRng$}
\pmdefines{$HorRng$}
\pmdefines{$VerRng$}
\pmdefines{higher -dimensional category}

\endmetadata

% almost certainly you want these
\usepackage{amssymb}
\usepackage{amsmath}
\usepackage{amsfonts}

% define commands here
\usepackage{amsmath, amssymb, amsfonts, amsthm, amscd, latexsym}
%%\usepackage{xypic}
\usepackage[mathscr]{eucal}
\theoremstyle{plain}
\newtheorem{lemma}{Lemma}[section]
\newtheorem{proposition}{Proposition}[section]
\newtheorem{theorem}{Theorem}[section]
\newtheorem{corollary}{Corollary}[section]

\theoremstyle{definition}
\newtheorem{definition}{Definition}[section]
\newtheorem{example}{Example}[section]
%\theoremstyle{remark}
\newtheorem{remark}{Remark}[section]
\newtheorem*{notation}{Notation}
\newtheorem*{claim}{Claim}

\renewcommand{\thefootnote}{\ensuremath{\fnsymbol{footnote%%@
}}}
\numberwithin{equation}{section}

\newcommand{\Ad}{{\rm Ad}}
\newcommand{\Aut}{{\rm Aut}}
\newcommand{\Cl}{{\rm Cl}}
\newcommand{\Co}{{\rm Co}}
\newcommand{\DES}{{\rm DES}}
\newcommand{\Diff}{{\rm Diff}}
\newcommand{\Dom}{{\rm Dom}}
\newcommand{\Hol}{{\rm Hol}}
\newcommand{\Mon}{{\rm Mon}}
\newcommand{\Hom}{{\rm Hom}}
\newcommand{\Ker}{{\rm Ker}}
\newcommand{\Ind}{{\rm Ind}}
\newcommand{\IM}{{\rm Im}}
\newcommand{\Is}{{\rm Is}}
\newcommand{\ID}{{\rm id}}
\newcommand{\GL}{{\rm GL}}
\newcommand{\Iso}{{\rm Iso}}
\newcommand{\Sem}{{\rm Sem}}
\newcommand{\St}{{\rm St}}
\newcommand{\Sym}{{\rm Sym}}
\newcommand{\SU}{{\rm SU}}
\newcommand{\Tor}{{\rm Tor}}
\newcommand{\U}{{\rm U}}

\newcommand{\A}{\mathcal A}
\newcommand{\Ce}{\mathcal C}
\newcommand{\D}{\mathcal D}
\newcommand{\E}{\mathcal E}
\newcommand{\F}{\mathcal F}
\newcommand{\G}{\mathcal G}
\newcommand{\Q}{\mathcal Q}
\newcommand{\R}{\mathcal R}
\newcommand{\cS}{\mathcal S}
\newcommand{\cU}{\mathcal U}
\newcommand{\W}{\mathcal W}

\newcommand{\bA}{\mathbb{A}}
\newcommand{\bB}{\mathbb{B}}
\newcommand{\bC}{\mathbb{C}}
\newcommand{\bD}{\mathbb{D}}
\newcommand{\bE}{\mathbb{E}}
\newcommand{\bF}{\mathbb{F}}
\newcommand{\bG}{\mathbb{G}}
\newcommand{\bK}{\mathbb{K}}
\newcommand{\bM}{\mathbb{M}}
\newcommand{\bN}{\mathbb{N}}
\newcommand{\bO}{\mathbb{O}}
\newcommand{\bP}{\mathbb{P}}
\newcommand{\bR}{\mathbb{R}}
\newcommand{\bV}{\mathbb{V}}
\newcommand{\bZ}{\mathbb{Z}}

\newcommand{\bfE}{\mathbf{E}}
\newcommand{\bfX}{\mathbf{X}}
\newcommand{\bfY}{\mathbf{Y}}
\newcommand{\bfZ}{\mathbf{Z}}

\renewcommand{\O}{\Omega}
\renewcommand{\o}{\omega}
\newcommand{\vp}{\varphi}
\newcommand{\vep}{\varepsilon}

\newcommand{\diag}{{\rm diag}}
\newcommand{\grp}{{\mathbb G}}
\newcommand{\dgrp}{{\mathbb D}}
\newcommand{\desp}{{\mathbb D^{\rm{es}}}}
\newcommand{\Geod}{{\rm Geod}}
\newcommand{\geod}{{\rm geod}}
\newcommand{\hgr}{{\mathbb H}}
\newcommand{\mgr}{{\mathbb M}}
\newcommand{\ob}{{\rm Ob}}
\newcommand{\obg}{{\rm Ob(\mathbb G)}}
\newcommand{\obgp}{{\rm Ob(\mathbb G')}}
\newcommand{\obh}{{\rm Ob(\mathbb H)}}
\newcommand{\Osmooth}{{\Omega^{\infty}(X,*)}}
\newcommand{\ghomotop}{{\rho_2^{\square}}}
\newcommand{\gcalp}{{\mathbb G(\mathcal P)}}

\newcommand{\rf}{{R_{\mathcal F}}}
\newcommand{\glob}{{\rm glob}}
\newcommand{\loc}{{\rm loc}}
\newcommand{\TOP}{{\rm TOP}}

\newcommand{\wti}{\widetilde}
\newcommand{\what}{\widehat}

\renewcommand{\a}{\alpha}
\newcommand{\be}{\beta}
\newcommand{\ga}{\gamma}
\newcommand{\Ga}{\Gamma}
\newcommand{\de}{\delta}
\newcommand{\del}{\partial}
\newcommand{\ka}{\kappa}
\newcommand{\si}{\sigma}
\newcommand{\ta}{\tau}
\newcommand{\lra}{{\longrightarrow}}
\newcommand{\ra}{{\rightarrow}}
\newcommand{\rat}{{\rightarrowtail}}
\newcommand{\oset}[1]{\overset {#1}{\ra}}
\newcommand{\osetl}[1]{\overset {#1}{\lra}}
\newcommand{\hr}{{\hookrightarrow}}

\begin{document}
\subsubsection{Background}

Charles Ehresmann defined in 1963 a {\em double category} $\mathcal{D}$ as an internal category in the category of small categories $\bf{Cat}$.

$$2-categories \to double~categories \to pseudo-algebras ~over~ the ~2-theory~of~ categories $$ and  
$$ crossed ~modules \to double~categories \to ...$$
\section{Double category definition}
\begin{definition}
A double category $\mathcal{D}$ consists of:
\begin{itemize}
\item a set of objects,
\item a set of horizontal morphisms $$f: A \to B,$$
\item a set of vertical morphisms $$j: A \to C,$$ and
\item a class of squares with source and target as shown in the following diagrams: $$\xymatrix{
{A}\ar[r]^{f}\ar[d]_{k}&{B}\ar[d]^{g}\\
{C}\ar[r]_{h}&{D}
}
$$
\end{itemize}

with compositions and units of the double category that satisfy the following axioms:
\begin{itemize}
\item \emph{i.} Horizontal:
\[
A\buildrel f_1 \over \longrightarrow
B \buildrel f_2 \over \longrightarrow
C = [f_1, f_2]= f_2 \circ f_1
\]

\[
A\buildrel 1^h_A \over \longrightarrow
A \buildrel f_1 \over \longrightarrow
B = A\buildrel f_1 \over \longrightarrow
B = A \buildrel f_1 \over \longrightarrow
B \buildrel 1^h_B \over \longrightarrow
B
\]

\item \emph{ii.} Vertical:
\[
[A\buildrel j_1 \over \longrightarrow
B \buildrel j_2 \over \longrightarrow
C]_{vert} = [j_1, j_2]_{vert.}= j_2 \circ j_1
\]

\[
[A\buildrel 1^v_A \over \longrightarrow
A \buildrel j_1 \over \longrightarrow
B = A\buildrel j_1 \over \longrightarrow
B = A \buildrel j_1 \over \longrightarrow
B \buildrel 1^v_B \over \longrightarrow
B]_{vert.}
\]
\emph{Compositions for square diagrams in a double category $\mathcal{D}$:}
\item \emph{iii.} Horizontal composition:
$$\xymatrix{
{A}\ar[r]^{f_1}\ar[d]_{j}&{B}\ar[d]^{k}\\
{D}\ar[r]_{g_1}&{E}}~~~~[\alpha]``\circ'' \xymatrix{
{B}\ar[r]^{f_2}\ar[d]_{k}&{C}\ar[d]^{l}\\
{E}\ar[r]_{g_2}&{F}}~~~~[\beta] = \xymatrix{
{A}\ar[r]^{[f_1f_2]}\ar[d]_{j}&{C}\ar[d]^{l}\\
{D}\ar[r]_{g_1g_2}&{F}} ~~~~[\alpha \beta].$$
\item \emph{iv.} Vertical composition of squares in $\mathcal{D}$:
${[\alpha \beta]}_{vert.}$ is expressed as
$$\xymatrix{
{A}\ar[r]^{f}\ar[d]_{[j_1 j_2]_v}&{B}\ar[d]^{[k_1 k_2]_v}\\
{E}\ar[r]_{h}&{F}}~~~~[\alpha \beta]_v.$$
\end{itemize}

\end{definition}

Moreover, all compositions are associative and unital, and also subject to the Interchange Law:

$$\xymatrix{
{[\alpha]}\ar[r]^{--}\ar[d]_{|}&{[\beta]}\ar[d]^{|}\\
{[\gamma]}\ar[r]_{--}&{[\delta]}
} =
{[ [\alpha \beta] ~~over~~ [\gamma \delta]]}_{vert.} = [\alpha \gamma]_v \circ [\beta \delta]_v.$$

Unit morphisms are also subject to the axioms of the double category. For further details on double categories and examples please see the related \PMlinkexternal{free download PDF file}{http://www-personal.umd.umich.edu/~tmfiore/1/fiorefolding.pdf}.

\subsection{Examples of Double Categories:}
\begin{itemize}
\item $Rng$:= pseudo double category or rings, bimodules, and equivariant maps: $Obj Rng$ := rings with identity; $HorRng$ := bimodules; $VerRng$ := homomorphisms of rings. 

\item $W$:= {\em pseudo double category of worldsheets} (A {\em worldsheet} is a real, compact, not necessarily connected, two dimensional, smooth manifold with complex structure and real analytically parametrized boundary components.)

\item Let $C_{Top}$ be a topological category, i.e. both $Obj C_{Top}$ and $Mor C_{Top}$ are topological spaces.
$P'OC$:= double category of Moore paths on $C_{Top}$.

\item The \PMlinkexternal{Double groupoid}{http://aux.planetmath.org/files/objects/12318/DG1.png}
\end{itemize}

\begin{definition}

The geometry of squares and their compositions lead to a common representation, or definition of a \emph{double groupoid} in the following form:


\begin{equation}
\label{squ} \D = \vcenter{\xymatrix @=3pc {S \ar @<1ex> [r] ^{s^1} \ar @<-1ex> [r]
_{t^1} \ar @<1ex> [d]^{\, t_2} \ar @<-1ex> [d]_{s_2} & H \ar[l]
\ar @<1ex> [d]^{\,t}
\ar @<-1ex> [d]_s \\
V \ar [u] \ar @<1ex> [r] ^s \ar @<-1ex> [r] _t & M \ar [l] \ar[u]}},
\end{equation}


where $M$ is a set of `points', $H,V$ are `horizontal' and `vertical' groupoids, and $S$ is a set of
`squares' with two compositions.

The laws for a double groupoid are also defined, more generally, for any topological space $\mathbb{T}$, and make it also describable as a groupoid internal to the category of groupoids.
\end{definition}
 

{\bf Remarks:}
Every $2$-category {\bf C} is a double category with {\em trivial} vertical morphisms
(See  Tom Fiore's : http://www-personal.umd.umich.edu/~tmfiore/1/fiorefolding.pdf)


Note the third set of arrows in the double groupoid diagram, making it a very special type of double category.


\begin{thebibliography}{9}

\bibitem{HKK}
K.A. Hardie, K.H. Kamps and R.W. Kieboom., A homotopy 2-groupoid of a Hausdorff 
\emph{Applied Categorical Structures}, \textbf{8} (2000): 209-234.

\bibitem{BHKP}
R. Brown, K.A. Hardie, K.H. Kamps  and T. Porter., 
\PMlinkexternal{A homotopy double groupoid of a Hausdorff space}{http://www.tac.mta.ca/tac/volumes/10/2/10-02.pdf} ,
{\it Theory and Applications of Categories} \textbf{10},(2002): 71-93.

\bibitem{Fiore98}
Tom Fiore.1998. \PMlinkexternal{Double Categories and Pseudo Algebras}{http://www-personal.umd.umich.edu/~tmfiore/1/fiorefolding.pdf} and $www.math.uchicago.edu/~Fiore/$

\end{thebibliography}

%%%%%
%%%%%
\end{document}
