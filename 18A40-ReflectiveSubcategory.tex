\documentclass[12pt]{article}
\usepackage{pmmeta}
\pmcanonicalname{ReflectiveSubcategory}
\pmcreated{2013-03-22 17:12:15}
\pmmodified{2013-03-22 17:12:15}
\pmowner{CWoo}{3771}
\pmmodifier{CWoo}{3771}
\pmtitle{reflective subcategory}
\pmrecord{4}{39524}
\pmprivacy{1}
\pmauthor{CWoo}{3771}
\pmtype{Definition}
\pmcomment{trigger rebuild}
\pmclassification{msc}{18A40}
\pmdefines{reflection functor}
\pmdefines{reflection}
\pmdefines{coreflective}
\pmdefines{coreflection}

\usepackage{amssymb,amscd}
\usepackage{amsmath}
\usepackage{amsfonts}
\usepackage{mathrsfs}

% used for TeXing text within eps files
%\usepackage{psfrag}
% need this for including graphics (\includegraphics)
%\usepackage{graphicx}
% for neatly defining theorems and propositions
\usepackage{amsthm}
% making logically defined graphics
%%\usepackage{xypic}
\usepackage{pst-plot}
\usepackage{psfrag}

% define commands here
\newtheorem{prop}{Proposition}
\newtheorem{thm}{Theorem}
\newtheorem{ex}{Example}
\newcommand{\real}{\mathbb{R}}
\newcommand{\pdiff}[2]{\frac{\partial #1}{\partial #2}}
\newcommand{\mpdiff}[3]{\frac{\partial^#1 #2}{\partial #3^#1}}
\begin{document}
Let $\mathcal{C}$ be a category and $\mathcal{D}$ a subcategory of $\mathcal{C}$.  $\mathcal{D}$ is called a \emph{reflective subcategory} of $\mathcal{C}$ if the inclusion functor $\operatorname{Inc}:\mathcal{D}\to \mathcal{C}$ has a left adjoint.  More explicitly, $\mathcal{D}$ in $\mathcal{C}$ is reflective iff for every object $A$ in $\mathcal{C}$, there is an object $B$ in $\mathcal{D}$ and a morphism $f:A\to B$ such that any morphism $g:A\to C$ can be uniquely factored through $f$; that is, there is a unique morphism $h:B\to C$ such that $g=h\circ f$.

The left adjoint is called the \emph{reflection functor} and the mapped objects and morphisms are called the \emph{reflections} (of the objects and morphisms being mapped by the reflection functor).

Some of the most common reflective subcategories are
\begin{itemize}
\item The subcategory of abelian groups in the category of groups.  The reflection functor is the abelianization functor.
\item The subcategory of fields in the category of integral domains.  The reflection of an integral domain is its field of fractions.
\item The subcategory of complete lattices in the category of lattices.  The reflection of a lattice is its lattice of ideals.
\end{itemize}

\textbf{Remark}.  If the inclusion functor has a right adjoint, then the subcategory is said to be \emph{coreflective}.  In other words, $\mathcal{D}$ in $\mathcal{C}$ is coreflective iff for any object $A\in \mathcal{C}$, there is an object $B\in \mathcal{D}$ and a morphism $f:B\to A$ such that any morphism $g:C\to A$ can be uniquely factored through $f$ (by a unique morphism $f:C\to B$).  For example, the subcategory of torsion abelian groups in the category of abelian groups is coreflective.  The coreflection of an abelian group is its torsion subgroup.
%%%%%
%%%%%
\end{document}
