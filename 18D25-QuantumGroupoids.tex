\documentclass[12pt]{article}
\usepackage{pmmeta}
\pmcanonicalname{QuantumGroupoids}
\pmcreated{2013-03-22 18:10:43}
\pmmodified{2013-03-22 18:10:43}
\pmowner{bci1}{20947}
\pmmodifier{bci1}{20947}
\pmtitle{quantum groupoids}
\pmrecord{75}{40751}
\pmprivacy{1}
\pmauthor{bci1}{20947}
\pmtype{Topic}
\pmcomment{trigger rebuild}
\pmclassification{msc}{18D25}
\pmclassification{msc}{18D20}
\pmclassification{msc}{18D05}
\pmclassification{msc}{18C10}
\pmclassification{msc}{18A40}
\pmclassification{msc}{18A30}
\pmclassification{msc}{18A25}
\pmclassification{msc}{28C10}
\pmclassification{msc}{55U40}
\pmclassification{msc}{22A22}
\pmclassification{msc}{55U35}
\pmclassification{msc}{22D25}
\pmsynonym{weak Hopf algebra}{QuantumGroupoids}
\pmsynonym{quantized locally compact groupoids with left Haar measure}{QuantumGroupoids}
%\pmkeywords{quantum groupoids}
%\pmkeywords{quantum groups}
%\pmkeywords{extended quantum symmetries}
%\pmkeywords{quantum spacetime structuresextended quantum symmetries}
%\pmkeywords{weak Hopf algebraa}
%\pmkeywords{quantized locally compact groupoids with left Haar measure}
\pmrelated{QuantumGroups}
\pmrelated{GroupoidCDynamicalSystem}
\pmrelated{HopfAlgebra}
\pmrelated{GroupoidAndGroupRepresentationsRelatedToQuantumSymmetries}
\pmrelated{QuantumCategory}
\pmrelated{QuantumAutomataAndQuantumComputation2}
\pmrelated{QuantumSpaceTimes}
\pmrelated{LocallyCompactGroupoids}
\pmrelated{QuantumFundamentalGroupoids}
\pmrelated{FunctionalBio}
\pmdefines{quantum groupoid}
\pmdefines{quantum group}
\pmdefines{extended quantum symmetries}
\pmdefines{quantum spacetime structure}

\endmetadata

% this is the default PlanetMath preamble.  
% almost certainly you want these
\usepackage{amssymb}
\usepackage{amsmath}
\usepackage{amsfonts}

% used for TeXing text within eps files
%\usepackage{psfrag}
% need this for including graphics (\includegraphics)
%\usepackage{graphicx}
% for neatly defining theorems and propositions
%\usepackage{amsthm}
% making logically defined graphics
%%%\usepackage{xypic}

% there are many more packages, add them here as you need them

% define commands here
\usepackage{amsmath, amssymb, amsfonts, amsthm, amscd, latexsym, enumerate}
\usepackage{xypic, xspace}
\usepackage[mathscr]{eucal}
\usepackage[dvips]{graphicx}
\usepackage[curve]{xy}

\setlength{\textwidth}{6.5in}
%\setlength{\textwidth}{16cm}
\setlength{\textheight}{9.0in}
%\setlength{\textheight}{24cm}

\hoffset=-.75in     %%ps format
%\hoffset=-1.0in     %%hp format
\voffset=-.4in


\theoremstyle{plain}
\newtheorem{lemma}{Lemma}[section]
\newtheorem{proposition}{Proposition}[section]
\newtheorem{theorem}{Theorem}[section]
\newtheorem{corollary}{Corollary}[section]

\theoremstyle{definition}
\newtheorem{definition}{Definition}[section]
\newtheorem{example}{Example}[section]
%\theoremstyle{remark}
\newtheorem{remark}{Remark}[section]
\newtheorem*{notation}{Notation}
\newtheorem*{claim}{Claim}

\renewcommand{\thefootnote}{\ensuremath{\fnsymbol{footnote}}}
\numberwithin{equation}{section}

\newcommand{\Ad}{{\rm Ad}}
\newcommand{\Aut}{{\rm Aut}}
\newcommand{\Cl}{{\rm Cl}}
\newcommand{\Co}{{\rm Co}}
\newcommand{\DES}{{\rm DES}}
\newcommand{\Diff}{{\rm Diff}}
\newcommand{\Dom}{{\rm Dom}}
\newcommand{\Hol}{{\rm Hol}}
\newcommand{\Mon}{{\rm Mon}}
\newcommand{\Hom}{{\rm Hom}}
\newcommand{\Ker}{{\rm Ker}}
\newcommand{\Ind}{{\rm Ind}}
\newcommand{\IM}{{\rm Im}}
\newcommand{\Is}{{\rm Is}}
\newcommand{\ID}{{\rm id}}
\newcommand{\grpL}{{\rm GL}}
\newcommand{\Iso}{{\rm Iso}}
\newcommand{\rO}{{\rm O}}
\newcommand{\Sem}{{\rm Sem}}
\newcommand{\SL}{{\rm Sl}}
\newcommand{\St}{{\rm St}}
\newcommand{\Sym}{{\rm Sym}}
\newcommand{\Symb}{{\rm Symb}}
\newcommand{\SU}{{\rm SU}}
\newcommand{\Tor}{{\rm Tor}}
\newcommand{\U}{{\rm U}}

\newcommand{\A}{\mathcal A}
\newcommand{\Ce}{\mathcal C}
\newcommand{\D}{\mathcal D}
\newcommand{\E}{\mathcal E}
\newcommand{\F}{\mathcal F}
%\newcommand{\grp}{\mathcal G}
\renewcommand{\H}{\mathcal H}
\renewcommand{\cL}{\mathcal L}
\newcommand{\Q}{\mathcal Q}
\newcommand{\R}{\mathcal R}
\newcommand{\cS}{\mathcal S}
\newcommand{\cU}{\mathcal U}
\newcommand{\W}{\mathcal W}

\newcommand{\bA}{\mathbb{A}}
\newcommand{\bB}{\mathbb{B}}
\newcommand{\bC}{\mathbb{C}}
\newcommand{\bD}{\mathbb{D}}
\newcommand{\bE}{\mathbb{E}}
\newcommand{\bF}{\mathbb{F}}
\newcommand{\bG}{\mathbb{G}}
\newcommand{\bK}{\mathbb{K}}
\newcommand{\bM}{\mathbb{M}}
\newcommand{\bN}{\mathbb{N}}
\newcommand{\bO}{\mathbb{O}}
\newcommand{\bP}{\mathbb{P}}
\newcommand{\bR}{\mathbb{R}}
\newcommand{\bV}{\mathbb{V}}
\newcommand{\bZ}{\mathbb{Z}}

\newcommand{\bfE}{\mathbf{E}}
\newcommand{\bfX}{\mathbf{X}}
\newcommand{\bfY}{\mathbf{Y}}
\newcommand{\bfZ}{\mathbf{Z}}

\renewcommand{\O}{\Omega}
\renewcommand{\o}{\omega}
\newcommand{\vp}{\varphi}
\newcommand{\vep}{\varepsilon}

\newcommand{\diag}{{\rm diag}}
\newcommand{\grp}{{\mathsf{G}}}
\newcommand{\dgrp}{{\mathsf{D}}}
\newcommand{\desp}{{\mathsf{D}^{\rm{es}}}}
\newcommand{\grpeod}{{\rm Geod}}
%\newcommand{\grpeod}{{\rm geod}}
\newcommand{\hgr}{{\mathsf{H}}}
\newcommand{\mgr}{{\mathsf{M}}}
\newcommand{\ob}{{\rm Ob}}
\newcommand{\obg}{{\rm Ob(\mathsf{G)}}}
\newcommand{\obgp}{{\rm Ob(\mathsf{G}')}}
\newcommand{\obh}{{\rm Ob(\mathsf{H})}}
\newcommand{\Osmooth}{{\Omega^{\infty}(X,*)}}
\newcommand{\grphomotop}{{\rho_2^{\square}}}
\newcommand{\grpcalp}{{\mathsf{G}(\mathcal P)}}

\newcommand{\rf}{{R_{\mathcal F}}}
\newcommand{\grplob}{{\rm glob}}
\newcommand{\loc}{{\rm loc}}
\newcommand{\TOP}{{\rm TOP}}

\newcommand{\wti}{\widetilde}
\newcommand{\what}{\widehat}

\renewcommand{\a}{\alpha}
\newcommand{\be}{\beta}
\newcommand{\grpa}{\grpamma}
%\newcommand{\grpa}{\grpamma}
\newcommand{\de}{\delta}
\newcommand{\del}{\partial}
\newcommand{\ka}{\kappa}
\newcommand{\si}{\sigma}
\newcommand{\ta}{\tau}

\newcommand{\med}{\medbreak}
\newcommand{\medn}{\medbreak \noindent}
\newcommand{\bign}{\bigbreak \noindent}

\newcommand{\lra}{{\longrightarrow}}
\newcommand{\ra}{{\rightarrow}}
\newcommand{\rat}{{\rightarrowtail}}
\newcommand{\ovset}[1]{\overset {#1}{\ra}}
\newcommand{\ovsetl}[1]{\overset {#1}{\lra}}
\newcommand{\hr}{{\hookrightarrow}}

\newcommand{\<}{{\langle}}

%\newcommand{\>}{{\rangle}}



%\usepackage{geometry, amsmath,amssymb,latexsym,enumerate}
%%%\usepackage{xypic}

\def\baselinestretch{1.1}


\hyphenation{prod-ucts}

%\grpeometry{textwidth= 16 cm, textheight=21 cm}

\newcommand{\sqdiagram}[9]{$$ \diagram  #1  \rto^{#2} \dto_{#4}&
#3  \dto^{#5} \\ #6    \rto_{#7}  &  #8   \enddiagram
\eqno{\mbox{#9}}$$ }

\def\C{C^{\ast}}

\newcommand{\labto}[1]{\stackrel{#1}{\longrightarrow}}

%\newenvironment{proof}{\noindent {\bf Proof} }{ \hfill $\Box$
%{\mbox{}}

\newcommand{\quadr}[4]
{\begin{pmatrix} & #1& \\[-1.1ex] #2 & & #3\\[-1.1ex]& #4&
 \end{pmatrix}}
\def\D{\mathsf{D}}
\begin{document}
This is a topic entry on quantum groupoids, related mathematical concepts
and their applications in modern quantum phyiscs. 

\subsubsection{Quantum groupoids and related groupoid C*--algebras}

\begin{definition} \emph{quantum groupoids}, $Q_{\grp}$' s, are currently defined either as quantized, \PMlinkname{locally compact groupoids}{LocallyCompactGroupoids} endowed with a left Haar measure system, $(\grp,\mu)$, or as \emph{weak Hopf algebras} (WHA). This concept is also an extension of the notion of quantum `group', which is sometimes represented by a Hopf algebra, $\mathsf{H}$. \emph{Quantum groupoid representations} define extended quantum symmetries beyond the `Standard Model' (SUSY) in Mathematical Physics or Noncommutative Geometry.
\end{definition}

  Quantum groupoid -- restricted here to a certain dual of a weak Hopf algebra-- and algebroid symmetries figure prominently both in the theory of dynamical deformations of quantum `groups' (that is, such groups that are the duals of Hopf algebras) and the quantum Yang--Baxter equations (Etingof et al., 1999,2001). On the other hand, one can also consider the natural extension of \emph{locally compact} (quantum) groups to locally compact (proper) \emph{groupoids} equipped with a Haar measure and a corresponding groupoid representation theory (Buneci, 2003) as a major, potentially interesting source for locally compact (but generally \emph{non-Abelian}) \emph{quantum groupoids}. The corresponding quantum groupoid representations on bundles of Hilbert spaces extend quantum symmetries well beyond those of quantum `groups'/Hopf algebras and simpler operator algebra representations, and are also consistent with the locally compact quantum group representations that were recently studied in some detail by Kustermans and Vaes (2000, and references cited therein).  The latter quantum groups are neither Hopf algebras, nor are they equivalent to Hopf algebras 
or their dual coalgebras. Quantum groupoid representations are, however, the next important 
step towards unifying quantum field theories with general relativity (GR) in a locally covariant 
and quantized form. Such representations need not however be restricted to weak Hopf algebra representations, as
the latter have no known connection to any type of GR theory and also appear to be inconsistent with GR.

 One is also motivated by numerous, important quantum physics examples to introduce a framework for quantum symmetry breaking in terms of either \emph{locally compact quantum groupoid}, or related algebroid, representations, such as those of \emph{weak Hopf C*-algebroids with convolution}; the latter are usually realized in the context
of \emph{rigged Hilbert spaces} (Bohm and Gadella, 1989).

 Furthermore, with regard to a unified and global framework for symmetry breaking, 
as well as higher order quantum symmetries, one needs to look towards the \emph{double groupoid}
structures of Brown and Spencer (1976), to enable one to introduce the concepts
of \emph{quantum and graded Lie bi--algebroids} which are expected to carry a distinctive C*--algebroid convolution structure. The extension to \emph{supersymmetry} leads then naturally to superalgebra, superfield symmetries and their involvement in supergravity or quantum gravity (QG) theories for intense
gravitational fields in fluctuating, quantized spacetimes. Their mathematical/quantum algebraic 
classification then involves superstructures with such supersymmetries that can only be
appropriately studied in (quantum) supercategories. 

 Thus, a natural extension of quantum symmetries leads one to Higher Dimensional Algebra (HDA) 
and may involve, for example,  both \textit{`quantum' double groupoids} defined as `locally compact'
double groupoids equipped with Haar measures via convolution, and an extension of superalgebra 
to double (super) algebroids, (that are naturally much more general than the Lie double algebroids defined in Mackenzie,
2004).

One can now proceed to formally define several quantum algebraic topology concepts that are 
needed to express the extended quantum symmetries in terms of proper quantum groupoid and algebroid
representations. `Hidden', higher dimensional quantum symmetries will then also emerge either 
\emph{via} generalized quantization procedures from higher dimensional algebra representations or be
determined as global or local invariants obtainable-- at least in principle-- through 
non-Abelian algebraic topology (NAAT) methods.

\subsubsection{Weak Hopf algebras}

 Let us begin by recalling the notion of a quantum `group' in relation to a Hopf algebra where the former is often realized as an automorphism group for a quantum space, that is, an object in a suitable category of generally noncommutative algebras. One of the most common guises of a quantum `group' is as the dual of a 
non-commutative, non-associative Hopf algebra. The Hopf algebras (cf. Chaician and Demichev 1996; 
Majid,1996), and their generalizations (Karaali, 2007), are some of the fundamental building blocks
of quantum operator algebra, even though they cannot be `integrated' to groups like the `integration' of Lie algebras to Lie groups; furthermore, the connection of Hopf algebras to quantum symmetries seems to be only indirect.

\begin{definition}
In order to define a \emph{weak Hopf algebra}, one can relax certain axioms of a Hopf algebra as follows~:
\begin{itemize}
\item[(1)]
 The comultiplication is not necessarily unit--preserving.
\end{itemize}

\begin{itemize}
\item[(2)]
 The counit $\vep$ is not necessarily a homomorphism of algebras.
\end{itemize}

\begin{itemize}
\item[(3)]
The axioms for the antipode map $S : A \lra A$ with respect to the
counit are as follows. For all $h \in H$,
\begin{equation}
\begin{aligned} m(\ID \otimes S) \Delta (h) &= (\vep \otimes
\ID)(\Delta (1) (h \otimes 1)) \\ m(S \otimes \ID) \Delta (h) &=
(\ID \otimes \vep)((1 \otimes h) \Delta(1)) \\ S(h) &= S(h_{(1)})
h_{(2)}  S(h_{(3)}) ~.
\end{aligned}
\end{equation}
\end{itemize}

 These axioms may be appended by the following commutative diagrams
\begin{equation}
{\begin{CD} A \otimes A @> S\otimes \ID >> A \otimes A
\\ @A \Delta AA   @VV m V
 \\ A @ > u \circ \vep >> A
\end{CD}} \qquad
{\begin{CD} A \otimes A @> \ID\otimes S >> A \otimes A
\\ @A \Delta AA   @VV m V
 \\ A @ > u \circ \vep >> A
\end{CD}}
\end{equation}
along with the counit axiom:
\begin{equation}
\xymatrix@C=3pc@R=3pc{ A \otimes A \ar[d]_{\vep \otimes 1} & A
\ar[l]_{\Delta} \ar[dl]_{\ID_A} \ar[d]^{\Delta}
\\ A  & A \otimes A \ar[l]^{1 \otimes \vep}}
\end{equation}

\end{definition}

  Several mathematicians substitute the term \emph{quantum `groupoid'} for a weak Hopf algebra, although this algebra in
itself is not a proper groupoid, but it may have a component
\emph{group} algebra as in the example of the quantum double
discussed next; nevertheless, weak Hopf algebras generalize Hopf
algebras --that with additional properties-- were previously
introduced as quantum `groups' by mathematical physicists. (The
latter--as already discussed--are not mathematical groups but algebras). 
As it will be shown in the next subsection, quasi--triangular quasi--Hopf algebras are
directly related to quantum symmetries in conformal (quantum)
field theories. Furthermore, weak C*--Hopf quantum algebras lead
to weak C*--Hopf algebroids that are linked to quasi--group
quantum symmetries, and also to certain Lie algebroids (and their
associated Lie--Weinstein groupoids) used to define Hamiltonian
(quantum) algebroids over the phase space of (quantum)
$W_N$--gravity.

\subsubsection{Examples of weak Hopf algebras.}

One can refer here to the example given by Bais et al. (2002). Let G be a non--Abelian group
and $H \subset G$ a discrete subgroup. Let {\em F(H)} denote the space of functions on H 
and $\bC H$ the group algebra (which consists of the linear span of group elements with the group structure).
\emph{The quantum double} \emph{D(H)} (Drinfel'd, 1987) is defined by the eqn :

$D(H) = F(H)~ \wti{\otimes}~ \bC H~$, where, for $x \in H$, the `twisted tensor product' 
is specified by the next eqn: 

$\wti{\otimes} \mapsto ~(f_1 \otimes h_1) (f_2 \otimes h_2)(x) =
f_1(x) f_2(h_1 x h_1^{-1}) \otimes h_1 h_2 ~$.
\
The physical interpretation given to this construction usually proceeds by considering $H$ 
as the `electric gauge group', and {\em F(H)} as the `magnetic symmetry' generated by 
$\{f \otimes e\}$~. In terms of the counit $\vep$, the double {\em D(H)} has a trivial representation 
given by $\vep(f \otimes h) = f(e)$~. there are several very interesting features of this construction.
For the purpose of braiding relations there is available an $R$ matrix, $R \in D(H) \otimes D(H)$, 
leading to the following operator: 

$\mathcal R \equiv \sigma \cdot (\Pi^A_{\a} \otimes \Pi^B_{\be})$ to be defined in terms of the 
Clebsch--Gordan series 
$\Pi^A_{\a} \otimes \Pi^B_{\be} \cong N^{AB \gamma}_{\a \be C}~ \Pi^C_{\gamma}$, and
where $\sigma$ denotes a flip operator. The operator $\R^2$ is sometimes called 
the \emph{monodromy} or \emph{Aharanov--Bohm phase factor}. In the case of a condensate in
a state $\vert v \rangle$ in the carrier space of some representation $\Pi^A_{\a}~$ 
one considers the maximal Hopf subalgebra $T$ of a Hopf algebra $A$ for which $\vert v \rangle$
is $T$--invariant; specifically ~:

$\Pi^A_{\a} (P)~\vert v \rangle = \vep(P) \vert v \rangle~,~
\forall P \in T~.$

For the second example, consider the example provided by Mack and Schomerus (1992)
using a more general notion of the Drinfel'd construction--the notion of a \emph{quasi
triangular quasi--Hopf algebra} (QTQHA) which was developed with the aim
of studying a range of essential symmetries with special properties, such as the 
\emph{quantum group algebra} $\U_q (\rm{sl}_2)$ with $\vert q \vert =1$~. If $q^p=1$, 
then it was shown that a QTQHA is canonically associated with $\U_q (\rm{sl}_2)$. Such QTQHAs are
claimed as the true symmetries of minimal conformal field theories.


\subsubsection{Weak Hopf C*--algebras in relation to quantum symmetry breaking.}

In our setting,a \emph{weak C*--Hopf algebra} is a weak *--Hopf
algebra which admits a faithful *--representation on a Hilbert
space. The weak C*--Hopf algebra is therefore much more likely to
be closely related to a `quantum groupoid' representation than any
weak Hopf algebra. However, one can argue that locally compact
groupoids equipped with a Haar measure (after quantization) come
even closer to defining quantum groupoids. There are already
several, significant examples that motivate the consideration of
weak C*--Hopf algebras which also deserve mentioning in the
context of `standard' quantum theories. Furthermore, notions such
as (proper) \emph{weak C*--algebroids} can provide the main
framework for symmetry breaking and quantum gravity that we are
considering here. Thus, one may consider the quasi-group
symmetries constructed by means of special transformations of the
`coordinate space' $M$. These transformations along with the
coordinate space $M$ define certain Lie groupoids, and also their
infinitesimal version - the Lie algebroids $\mathbf{A}$, when the
former are Weinstein groupoids. If one then lifts the algebroid
action from $M$ to the principal homogeneous space $\R$ over the
cotangent bundle $T^*M \lra M$, one obtains a physically
significant algebroid structure. The latter was called the
Hamiltonian algebroid, ${\mathcal A}^H$, related to the Lie
algebroid, $\mathbf{A}$. The Hamiltonian algebroid is an analog of
the Lie algebra of symplectic vector fields with respect to the
canonical symplectic structure on $\R$ or $T^*M$. In this recent
example, the Hamiltonian algebroid, $\mathcal{A}^H$ over $\R$, was
defined over the phase space of $W_N$--gravity, with the anchor
map to Hamiltonians of canonical transformations (Levin and
Olshanetsky, 2003,2008). Hamiltonian algebroids thus generalize
Lie algebras of canonical transformations; canonical
transformations of the Poisson sigma model phase space define a
\emph{Hamiltonian algebroid} with the Lie brackets related to such
a Poisson structure on the target space. The Hamiltonian algebroid
approach was utilized to analyze the symmetries of generalized
deformations of complex structures on Riemann surfaces
$\sum_{g,n}$ of genus $g $ with $n$ marked points. However, its
implicit algebraic connections to von Neumann *--algebras and/or
\emph{weak C*--algebroid representations} have not yet been
investigated. This example suggests that algebroid (quantum)
symmetries are implicated in the foundation of relativistic
quantum gravity theories and supergravity.


\begin{thebibliography}{99}

\bibitem{AABB70}
A. Abragam and B. Bleaney.: {\em Electron Paramagnetic Resonance of Transition Ions.}
Clarendon Press: Oxford, (1970).

\bibitem{AS}
E. M. Alfsen and F. W. Schultz: \emph{Geometry of State Spaces of Operator Algebras}, 
Birkh\"auser, Boston--Basel--Berlin (2003).

\bibitem{BLk3}
J.C. Baez and L. Langford. Higher-dimensional algebra. IV. 2-tangles. \textit{Adv. Math.}
\textbf{180} (no.2): 705--764 (2003).

\bibitem{ICB71}
I. Baianu : Categories, Functors and Automata Theory: A Novel Approach to Quantum Automata through Algebraic--Topological Quantum Computations., \textit{Proceed. 4th Intl. Congress LMPS}, (August-Sept. 1971).

\bibitem{ICB78}
I.C. Baianu: X--ray scattering by partially disordered membrane systems,
 \emph{Acta Cryst.} \textbf{A34}: 751--753 (1978).

 \bibitem{ICB6}
I.C. Baianu, K.A. Rubinson and  J. Patterson. 1979.  Ferromagnetic Resonance and Spin--Wave 
Excitations in Metallic Glasses., \emph{J. Phys. Chem. Solids}, \textbf{40}: 940--951.

\bibitem{ICB11}
I.C. Baianu, N. Boden,Y.K. Levine and D. Lightowlers. 1978. Dipolar Coupling between Groups of Three Spin--1/2 undergoing Hindered Reorientation in Solids., \emph{Solid-State Communications}. \textbf{27:} 474-478.

\bibitem{ICB12}
I.C. Baianu, N. Boden, Y.K.Levine and S.M.Ross. 1978. Quasi--quadrupolar NMR Spin--Echoes in
Solids Containing Dipolar Coupled Methyl Groups.,  \emph{J. Phys. (\textbf{C}) Solid State
Physics}., \textbf{11}: \textbf{L}37--41.

\bibitem{ICB9}
I.C. Baianu, K.A. Rubinson and J. Patterson. 1979. The Observation of Structural Relaxation in a FeNiPB Glass by X--ray Scattering and Ferromagnetic Resonance., \emph{Physica Status Solidi}(a), \textbf{53} \textbf{K}:133--135.

\bibitem{ICB8}
I.C. Baianu, N. Boden and D. Lightowlers.1981. NMR Spin--Echo Responses of Dipolar--Coupled
Spin--1/2 Triads in Solids., \emph{J. Magnetic Resonance}, \textbf{43}:101--111.

\bibitem{BGB07}
I. C. Baianu, J. F. Glazebrook and R. Brown.: A Non-Abelian, Categorical Ontology of Spacetimes and Quantum Gravity., \emph{Axiomathes} \textbf{17},(3-4): 353-408(2007).

\bibitem{BGBk8}
I.C.Baianu, J.F. Glazebrook, and R. Brown.: On Physical Applications of Non-Abelian Algebraic Topology in Quantum Field Theories \emph{in preparation}, (2008).

\bibitem{BBGGk8}
I.C.Baianu, R. Brown J.F. Glazebrook, and G. Georgescu, Towards Quantum Non-Abelian Algebraic Topology. \textit{in preparation}, (2008).

\bibitem{BSS}
F.A. Bais, B. J. Schroers and J. K. Slingerland: Broken quantum symmetry and confinement phases in planar physics, \emph{Phys. Rev. Lett.} \textbf{89} No. 18 (1--4): 181-201 (2002).

\bibitem{BJW}
J.W. Barrett.: Geometrical measurements in three-dimensional quantum gravity., in:
{\em Proceedings of the Tenth Oporto Meeting on Geometry, Topology and Physics (2001)}.
\textit{Intl. J. Modern Phys.} \textbf{A 18} , October, suppl., 97--113 (2003)

\bibitem{BA}
A. D. Blaom : Lie Algebroids and Cartan's Method of Equivalence., {\em arXiv:math/0509071v2} [math.DG], (12 Jan 2007).

\bibitem{BNS}
G. B\"ohm, F. Nill, and K. Szlach\'anyi: Weak Hopf algebras I, Integral theory and C*--structure, \emph{J. Algebra} \textbf{221} (2): 385--438 (1999).

\bibitem{BG}
A. Bohm and M. Gadella: Dirac kets, Gamow vectors and Gel'fand triples, \emph{Lect. Notes in Physics} \textbf{348} Springer Verlag, 1989.

\bibitem{BJ}
Borceux,F. and G. Janelidze: \emph{Galois Theories}, Cambridge Studies in Advanced Mathematics \textbf{72}, Cambridge (UK): Cambridge University Press, 2001.

\bibitem{BHP2k2}
R. Brown R, K. Hardie, H. Kamps, T. Porter T.: The homotopy double groupoid of a Hausdorff space., 
\emph{Theory Appl. Categories}, \textbf{10}:71--93 (2002).

\bibitem{BrownBook1}
R. Brown: \emph{Topology and Groupoids}, BookSurge LLC (2006).

\bibitem{BrownBook2}
R. Brown R, P.J. Higgins, and R. Sivera.: 2008. \textit{Non--Abelian Algebraic Topology}, 
\PMlinkexternal{PDF file download}{http://www.bangor.ac.uk/mas010/nonab--t/partI010604.pdf}.

\bibitem{BGB2}
R. Brown, J. F. Glazebrook and I. C. Baianu: A categorical and higher dimensional algebra framework for complex systems and spacetime structures, \emph{Axiomathes} \textbf{17}:409--493.
(2007).

\bibitem{BH87}
R. Brown and P. Higgins: tensor products and homotopies for $\omega$--groupoids and crossed complexes, \emph{J. Pure Appl. Algebra} \textbf{47} (1987), 1--33.

\bibitem{BJ2k4}
R. Brown and G. Janelidze.: Galois Theory and a new homotopy double groupoid of a map of spaces, {\em Applied Categorical Structures.} \textbf{12}: 63--80(2004).

\bibitem{BJ87}
R. Brown and G. Janelidze: Van Kampen theorems for categories of covering morphisms in lextensive categories, \textit{J. Pure Appl. Algebra}, \textbf{119}: 255---63, ISSN 0022--4049 (1997).

\bibitem{BL87}
R. Brown and J.-L. Loday: Homotopical excision, and Hurewicz theorems, for n--cubes of spaces, Proc. London Math. Soc., 54:(3), 176--192, (1987).

\bibitem{BL87}
R. Brown and J.-L. Loday: Van Kampen Theorems for diagrams of spaces, Topology, 26: 311--37 (1987).

\bibitem{BMos}
R. Brown and G. H. Mosa: Double algebroids and crossed modules of algebroids, University of Wales--Bangor, Maths Preprint, 1986.

\bibitem{BS1}
R. Brown  and C.B. Spencer: Double groupoids and crossed modules,
\emph{Cahiers Top. G\'eom.Diff.} \textbf{17} (1976), 343--362.

\bibitem{BRM03}
M. R. Buneci.: \emph{Groupoid Representations.}, Ed. Mirton: Timisoara (2003).

\bibitem{BCJI2k2}
J. Butterfield and C. J. Isham : A topos perspective on the Kochen--Specker theorem I - IV, Int. J. Theor. Phys, 37 (1998) No 11., 2669--2733 38 (1999) No 3., 827--859, 39 (2000) No 6., 1413--1436, 41 (2002) No 4., 613--639.

\bibitem{BCJI2k4}
J. Butterfield and C. J. Isham : Some possible roles for topos theory in quantum theory and quantum gravity, Foundations of Physics., 4, 820---837 (2004).

\bibitem{CS} 
J.S. Carter and M. Saito. Knotted surfaces, braid moves, and beyond. Knots and
quantum gravity (Riverside, CA, 1993), 191--229, Oxford Lecture Ser. Math. Appl.,
1, Oxford Univ. Press, New York, 1994.

\bibitem{Chaician}
M. Chaician and A. Demichev: \emph{Introduction to Quantum Groups}, World Scientific (1996).

\bibitem{Coleman}
Coleman and De Luccia: Gravitational effects on and of vacuum decay., \emph{Phys. Rev. D} \textbf{21}: 3305 (1980).

\bibitem{Connesbook}
A. Connes: \emph{Noncommutative Geometry}, Academic Press. 1994.

\bibitem{CF}
L. Crane and I.B. Frenkel. Four-dimensional topological quantum field theory, Hopf categories, and the canonical bases. Topology and physics. \textit{J. Math. Phys}. \textbf{35} (no. 10): 5136--5154 (1994).
 
\bibitem{CKY} 
L. Crane; L.H. Kauffman; D.N. Yetter. State-sum invariants of 4-manifolds. 
\textit{J. Knot Theory Ramifications} \textbf{6}, (no. 2): 177--234 1997).

\bibitem{DayStreet97}
Day, B. J. and Street, R.: Monoidal bicategories and Hopf algebroids. \textit{Advances in Mathematics}, \textbf{129}: 99--157 (1997)

\bibitem{DT96}
W. Drechsler and P. A. Tuckey:  On quantum and parallel transport in a Hilbert bundle over spacetime., Classical and Quantum Gravity, \textbf{13}:611--632 (1996). doi: 10.1088/0264--9381/13/4/004

\bibitem{Drinfeld}
V. G. Drinfel'd: Quantum groups, In \emph{Proc. Int. Cong. of Mathematicians, Berkeley 1986}, (ed. A. Gleason), Berkeley, 798--820 (1987).

\bibitem{Ellis}
G. J. Ellis: Higher dimensional crossed modules of algebras,
\emph{J. of Pure Appl. Algebra} \textbf{52} (1988), 277--282.

\bibitem{Etingof1}
P.. I. Etingof and A. N. Varchenko, Solutions of the Quantum Dynamical Yang-Baxter Equation and Dynamical Quantum Groups, Comm.Math.Phys., 196:  591--640 (1998)

\bibitem{Etingof2}
P. I. Etingof and A. N. Varchenko: Exchange dynamical quantum
groups, \emph{Commun. Math. Phys.} \textbf{205} (1): 19--52 (1999)

\bibitem{Etingof3}
P. I. Etingof and O. Schiffmann: Lectures on the dynamical Yang--Baxter equations, in \emph{Quantum Groups and Lie Theory (Durham, 1999)}, pp. 89--129, Cambridge University Press, Cambridge, 2001.


\bibitem{Fauser2002}
B. Fauser: A treatise on quantum Clifford Algebras. Konstanz, Habilitationsschrift., arXiv.math.QA/0202059 (2002).

\bibitem{Fauser2004}
B. Fauser: Grade Free product Formulae from Grassman-Hopf Gebras.
Ch. 18 in R. Ablamowicz, Ed., \emph{Clifford Algebras: Applications to Mathematics, Physics and Engineering},
Birkh\"{a}user: Boston, Basel and Berlin, (2004).

\bibitem{Fell}
J. M. G. Fell.: The Dual Spaces of  C*-Algebras, Transactions of the American
Mathematical Society, \textbf{94}: 365-403 (1960).

\bibitem{FernCastro}
F.M. Fernandez and E. A. Castro.:  \textit{(Lie) Algebraic Methods in Quantum Chemistry and Physics.}, Boca Raton: CRC Press, Inc  (1996).

\bibitem{Feynman}
 R. P. Feynman: Space--Time Approach to Non--Relativistic Quantum Mechanics, 
\emph{Reviews of Modern Physics}, 20: 367--387 (1948). [It is also reprinted in (Schwinger 1958).]

\bibitem{frohlich:nonab}
A.~Fr{\"o}hlich, {\em Non-{A}belian Homological Algebra. {I}.
{D}erived functors and satellites.\/}, Proc. London Math. Soc. (3), 11: 239--252 (1961).

\bibitem{GN}
Gel'fand, I. and Naimark, M., 1943, On the Imbedding of Normed Rings into the Ring of 
Operators in Hilbert Space, Recueil Math\'ematique [Matematicheskii Sbornik] Nouvelle Serie, \textbf{12} [54]: 197--213. [Reprinted in C*--algebras: 1943--1993, in the series Contemporary 
Mathematics, 167,  Providence, R.I. : American Mathematical Society, 1994.]

\bibitem{Georgescu06}
G. Georgescu: N--Valued Logics and \L ukasiewicz-Moisil Algebras.,
\textit{Axiomathes}, \textbf{16}(1-2): 123-136 (March 2006).

\bibitem{Georgescu68}
G. Georgescu, and D. Popescu.: On Algebraic Categories. \textit{Rev. Roum. Math. Pures et Appl}.,\textbf{24}(13):337--342 (1968).

\bibitem{GV70}
G. Georgescu, and C. Vraciu:  On the Characterization of \L ukasiewicz Algebras.,
\textit{J. Algebra}, 16(4): 486-495 (1970).

\bibitem{GR02}
R. Gilmore: \textit{``Lie Groups, Lie Algebras and Some of Their Applications.''},
Dover Publs., Inc.: Mineola and New York, 2005.

\bibitem{Grabowski}
J. Grabowski and G. Marmo: Generalized Lie bialgebroids and Jacobi
structures, \emph{J. Geom Phys.} \textbf{40}, no. 2, 176--199 (2001).

\bibitem{Grandis}
M. Grandis and L. Mauri: Cubical sets and their site, \emph{Theor.
Appl. Categories} \textbf{11} no. 38 (2003), 185--211.

\bibitem{Grisaru}
M. T. Grisaru and H. N. Pendleton: Some properties of scattering
amplitudes in supersymmetric theories, \emph{Nuclear Phys. B} \textbf{124}: 81--92(1977).


\bibitem{Alex2}
Grothendieck, A.: 1957, Sur quelque point d-alg\'{e}bre homologique. , \emph{Tohoku Math. J.}, \textbf{9:} 119-121.

\bibitem{Alex3}
Grothendieck, A. and J. Dieudon\'{e}.: 1960, El\'{e}ments de geometrie alg\'{e}brique., \emph{Publ. Inst. des Hautes Etudes de Science}, \textbf{4}; and
Vol. III, Publ. Math. Inst. des Hautes \'Etudes , 11: 1-167 (1961).


\bibitem{Grothendieck61}
A. Grothendieck: Technique de descente et th\'eor\`emes d'existence en g\'eom\'etrie alg\'ebrique. II, S\'eminaire Bourbaki 12 (1959/1960), exp. 195, Secr\'etariat Math\'ematique, Paris, 1961.


\bibitem{Grothendieck62}
A. Grothendieck: Technique de construction en g\'eom\'etrie analytique. 
IV. Formalisme g\'en\'eral des foncteurs repr\^esentables, 
S\'eminaire H. Cartan 13 (1960/1961), exp. 11, Secr\'etariat Math\'ematique, Paris, 1962.


\bibitem{Alex1}
Grothendieck, A.: 1971, Rev\^{e}tements \'Etales et Groupe Fondamental (SGA1),
chapter VI: Cat\'egories fibr\'ees et descente, \emph{Lecture Notes in Math.}
\textbf{224}, Springer--Verlag: Berlin.


\bibitem{Hahn1}
P. Hahn: Haar measure for measure groupoids., \textit{Trans. Amer. Math. Soc}. \textbf{242}: 1--33(1978).

\bibitem{Hahn2}
P. Hahn: The regular representations of measure groupoids., \textit{Trans. Amer. Math. Soc}. \textbf{242}:34--72(1978).

\bibitem{HeynLifsctz}
R. Heynman and S. Lifschitz. 1958. \emph{``Lie Groups and Lie Algebras"}., New York and %%@
London: Nelson Press.

\bibitem{HLS2k8}
C. Heunen, N. P. Landsman, B. Spitters.: A topos for algebraic quantum theory, (2008)   \\arXiv:0709.4364v2 [quant--ph]

\bibitem{Higgins2}P.~J. {H}iggins, `Presentations of Groupoids, with Applications to Groups', %%@
{\em Proc. {C}amb. Phil. Soc.}\textbf{60}: 7--20 (1964).

\bibitem{Higgins4}
P.~J. {H}iggins: {\em Categories and Groupoids\/}, {v}an Nostrand: New York (1971).

\bibitem{Hindeleh}
A. M. Hindeleh and R. Hosemann: Paracrystals representing the physical state of matter, \emph{Solid State Phys.} \textbf{21}: 4155--4170 (1988).

\bibitem{Hosemann1}
R. Hosemann and R. N. Bagchi: \emph{Direct Analysis of Diffraction
by Matter}, North--Holland Publs.: Amsterdam--New York (1962).

\bibitem{Hosemann2}
R. Hosemann, W. Vogel, D. Weick and F. J. Balta-Calleja: Novel
aspects of the real paracrystal, \emph{Acta Cryst.} \textbf{A37}: 85--91 (1981).

\bibitem{JG1}
G. Janelidze.: `Precategories and Galois theory', Springer Lecture Notes in Math. 1488:  157-173(1991).

\bibitem{JG2}
G. Janelidze.: `Pure Galois theory in categories', \textit{J. Algebra}, \textbf{132}: 270-286 (1990).

\bibitem{Kac}
V. Kac: Lie superalgebras, \emph{Advances in Math.} \textbf{26} (1): 8--96 (1977).

\bibitem{KP}
K.H. Kamps and T. Porter.: `A homotopy 2-groupoid from a fibration', \textit{Homotopy, homology and applications}, \textbf{1}: 79-93(1999) .

\bibitem{Karaali}
G. Karaali: On Hopf algebras and their generalizations. (2007)
\\arXiv:math/07/03441v2


\bibitem{KJVS}
J. Kustermans and S. Vaes: The operator algebra approach to quantum groups.
\textit{Proc. Natl. Acad. USA}, \textbf{97}, (2): 547-552, (2000).

\bibitem{Lance}
E. C. Lance: Hilbert C*--Modules. \emph{London Math. Soc. Lect.
Notes} \textbf{210}, \emph{Cambridge Univ. Press.}, (1995).

\bibitem{Land}
N. P. Landsman: Mathematical topics between classical and quantum
mechanics. \emph{Springer Verlag}, New York, (1998).

\bibitem{Land1}
N. P. Landsman: Compact quantum groupoids, in `Quantum Theory and
Symmetries' (Goslar, 18--22 July 1999) eds. H.-D. Doebner et al.,
World Scientific, (2000).
\\arXiv:math-ph/9912006


\bibitem{LandsRamaz2k1}
N. P. Landsman and B. Ramazan: Quantization of Poisson algebras associated to Lie algebroids. , In R. Kaminker et al, eds., \emph{Proc. Conf. on Groupoids in Physics, Analysis and Geometry}, Contemporary Mathematics series, \emph{E--print: math--ph/001005.}

\bibitem{Levin-Olshanetsky08}
A. Levin and M. Olshanetsky.: Hamiltonian Algebroids and deformations
of complex structures on Riemann curves., (2003, 2008). $arXiv:hep--th/0301078v1$


\bibitem{LJ}
J.L. Loday.: `Spaces with finitely many homotopy groups', \textit{J. Pure Appl. Algebra}, \textbf{24}: 179-201 (1982).

\bibitem{MP}
M. Mackaay and R. Picken.: `The holonomy of gerbes with connection', arXiv:math.DG/0007053.

\bibitem{MS} 
M. Mackaay. Spherical 2-categories and 4-manifold invariants., \textit{Adv. Math}. \textbf{143}
(no. 2): 288--348 (1999).
 
\bibitem{MS}
I. Moerdijk and J. A. Svensson: `Algebraic classification of equivariant 2-types', \textit{J. Pure Appl. Algebra} 89: 187-216 (1993).

\bibitem{Lu96}
J.-H. Lu: Hopf algebroids and quantum groupoids, Int. J. Math. \textbf{7}: 47--70 (1996).

\bibitem{Mackey}
G. W. Mackey: \emph{The Scope and History of Commutative and Noncommutative Harmonic Analysis}, Amer. Math. Soc., Providence, RI, (1992).

\bibitem{Mack}
G. Mack and V. Schomerus: Quasi Hopf quantum symmetry in quantum
theory, \emph{Nucl. Phys. B} \textbf{370} (1) (1992), 185--230.

\bibitem{Mack1}
K. C. H. Mackenzie: \emph{General Theory of Lie Groupoids and Lie
Algebroids}, London Math. Soc. Lecture Notes Series, \textbf{213},
Cambridge University Press, Cambridge, (2005).

\bibitem{MacLane65}
S. MacLane: \textit{``Categorical algebra.''},  Bull. Amer. Math. Soc. \textbf{71}, Number 1: 40-106(1965).

\bibitem{MM}
S. MacLane and I. Moerdijk: \textit{``Sheaves in Geometry and Logic - A
first Introduction to Topos Theory''}., Springer Verlag, New York
(1992).

\bibitem{Majid}
S. Majid: \emph{Foundations of Quantum Group Theory}, Cambridge
Univ. Press (1995).

\bibitem{Maltsiniotis}
G. Maltsiniotis.: Groupo$\ddot{i}$des quantiques., C. R. Acad. Sci. Paris, \textbf{314}:
249 -- 252.(1992)

\bibitem{May}
J.P. May: A Concise Course in Algebraic Topology. Chicago and London: The Chicago University
Press. (1999).

\bibitem{Mitchell}
B. Mitchell: \emph{The Theory of Categories}, Academic Press,
London, (1965).

\bibitem{Mitchell72}
B. Mitchell: Rings with several objects., \textit{Adv. Math}. \textbf{8}: 1 - 161 (1972).

\bibitem{Mitchell85}
B. Mitchell: Separable Algebroids., M\textit{emoirs American Math. Soc.} \textbf{333}:1x-1xx (1985).

\bibitem{Mo}
G. H. Mosa: \emph{Higher dimensional algebroids and Crossed
complexes}, PhD thesis, University of Wales, Bangor, (1986).

\bibitem{Moultaka}
G. Moultaka, M. Rausch de Traubenberg and A. Tanas\u a: Cubic
supersymmetry and abelain gauge invariance, \emph{Internat. J.
Modern Phys. A} \textbf{20} no. 25 (2005), 5779--5806.

\bibitem{Mrcun}
J. Mr\v cun : On spectral representation of coalgebras and Hopf
algebroids, (2002) preprint.\\ http://arxiv.org/abs/math/0208199.

\bibitem{MRW}
P. Muhli, J. Renault and D. Williams: Equivalence and isomorphism for groupoid C*--algebras., \textit{J. Operator Theory}, \textbf{17}: 3-22 (1987).

\bibitem{Nikshych}
D. A. Nikshych and L. Vainerman: \emph{J. Funct. Anal.}
\textbf{171} (2000) No. 2, 278--307

\bibitem{Nishimura96}
H. Nishimura.: Logical quantization of topos theory., \textit{International Journal of Theoretical Physics}, Vol. \textbf{35},(No. 12): 2555--2596 (1996).

\bibitem{NM}
M. Neuchl, PhD Thesis, University of Munich (1997).

\bibitem{OV06}
V. Ostrik.: Module Categories over Representations of \emph{SL-q(2)}
in the Non--simple Case. (2006). \\ http://arxiv.org/PS--cache/math/pdf/0509/0509530v2.pdf

\bibitem{PAk3}
A. L. T. Paterson: The Fourier-Stieltjes and Fourier algebras for locally
compact groupoids., \textit{Contemporary Mathematics} \textbf{321}: 223-237 (2003)

\bibitem{PLR}
R. J. Plymen and P. L. Robinson :  Spinors in Hilbert Space.
\emph{Cambridge Tracts in Math.} \textbf{114}, \emph{Cambridge Univ. Press} (1994).

\bibitem{NPopescu}
N. Popescu.: \emph{The Theory of Abelian Categories with Applications to Rings and
Modules.} New York and London: Academic Press (1968).

\bibitem{Prigogine}
I. Prigogine:  \textit{From being to becoming: Time and Complexity in the physical sciences}. W. H. Freeman and Co: San Francisco (1980).

\bibitem{RW1}
A. Ramsay: Topologies on measured groupoids., \textit{J. Functional Analysis}, \textbf{47}: 314-343 (1982).

\bibitem{RW2}
A. Ramsay and M. E. Walter: Fourier-Stieltjes algebras of locally compact groupoids.,
\textit{J. Functional Analysis}, \textbf{148}: 314-367 (1997).

\bibitem{RZ}
I. Raptis and R. R. Zapatrin : Quantisation of discretized spacetimes and the correspondence principle, \emph{Int. Jour.
Theor. Phys.} \textbf{39},(1), (2000).

\bibitem{REG} 
T. Regge.: General relativity without coordinates. \textit{Nuovo Cimento} (\textbf{10}) 19: 558--571 (1961).

\bibitem{Rehren}
H.--K. Rehren: Weak C*--Hopf symmetry, \emph{Quantum Group
Symposium at Group 21, Proceedings, Goslar (1996}, Heron Ptess, Sofia BG : 62--69(1997).

\bibitem{Renault1}
J. Renault: A groupoid approach to C*--algebras. Lecture Notes in Maths. \textbf{793}, Berlin: Springer--Verlag,(1980).

\bibitem{Renault2}
J. Renault: Representations de produits croises d'algebres de groupoides. ,
\textit{J. Operator Theory}, \textbf{18}:67--97 (1987).


\bibitem{Renault3}
J. Renault: The Fourier algebra of a measured groupoid and its multipliers.,
\textit{J. Functional Analysis}, \textbf{145}: 455-490 (1997).

\bibitem{Rieffel1}
M. A. Rieffel: Group C*--algebras as compact quantum metric spaces, \emph{Documenta Math.} \textbf{7}: 605--651 (2002).

\bibitem{Rieffel2}
M. A. Rieffel: Induced representations of  C*--algebras, \emph{Adv. in Math.} \textbf{13}: 176-254 (1974).

\bibitem{noncomm}
J. E. Roberts : More lectures on algebraic quantum field theory (in: A. Connes, et al., \textit{'Non--commutative Geometry'}), Springer: Berlin (2004).

\bibitem{RJ2} 
J. Roberts.: Skein theory and Turaev-Viro invariants. \textit{Topology} \textbf{34}( no. 4): 771--787 (1995).

\bibitem{RJ3} J. Roberts. Refined state-sum invariants of 3- and 4-manifolds. Geometric topology
(Athens, GA, 1993), 217--234, \textit{AMS/IP Stud. Adv. Math}., \textbf{2.1}, Amer. Math.Soc., Providence, RI, (1997).

\bibitem{RC98}
Rovelli, C.: 1998, Loop Quantum Gravity , in N. Dadhich, et al. {\em ``Living Reviews in Relativity''} \\
http:www.livingreviews.org/Articles/Volume1/1998--1--rovelli

\bibitem{Schwartz45}
Schwartz, L.:  Generalisation de la Notion de Fonction, de Derivation, de Transformation
de Fourier et Applications Mathematiques et Physiques., \textit{Annales de l'Universite de Grenoble},
\textbf{21}: 57--74 (1945).

\bibitem{Schwartz51-2}
Schwartz, L.: Theorie des Distributions, \emph{Publications de l'Institut de Mathematique
de l'Universit\'e de Strasbourg}, Vols 9--10, Paris: Hermann (1951--1952).

\bibitem{Seda1}
A. K. Seda: Haar measures for groupoids, \emph{Proc. Roy. Irish Acad.
Sect. A} \textbf{76} No. 5, 25--36 (1976).

\bibitem{Seda1}
A. K. Seda: On the Continuity of Haar measures on topological groupoids, \emph{Proc. Amer Mat. Soc.} \textbf{96}: 115--120 (1986).

\bibitem{Seda2}
A. K. Seda: Banach bundles of continuous functions and an integral
representation theorem, \emph{Trans. Amer. Math. Soc.} \textbf{270} No.1 : 327--332(1982).


\bibitem{Segal47a}
I.E. Segal.:  Irreducible Representations of Operator Algebras. , 
{\em Bulletin of the American Mathematical Society}, \textbf{53}: 73--88 (1947a).

\bibitem{Segal47b}
I. E. Segal.: Postulates for General Quantum Mechanics, \emph{Annals of Mathematics}, \textbf{4}:
930--948 (1947b).

\bibitem{Sklyanin83}
E.K. Sklyanin: Some Algebraic Structures Connected with the Yang-Baxter equation, Funct. Anal.Appl., \textbf{16}: 263--270 (1983).

\bibitem{Sklyanin84}
Some Algebraic Structures Connected with the Yang-Baxter equation. Representations of Quantum Algebras, Funct. Anal.Appl., 17: 273--284 (1984).

\bibitem{Szlach}
K. Szlach\'anyi: The double algebraic view of finite quantum
groupoids, \emph{J. Algebra} \textbf{280} (1) , 249--294 (2004).

\bibitem{Sweedler96}
M. E. Sweedler: \textit{Hopf algebras.} W.A. Benjamin, INC., New
York (1996).

\bibitem{Tanasa}
A. Tanas\u a: Extension of the Poincar\'e symmetry and its field
theoretical interpretation, \emph{SIGMA Symmetry Integrability
Geom. Methods Appl.} \textbf{2} (2006), Paper 056, 23 pp.
(electronic).

\bibitem{taylorj:88}
J.~{T}aylor, {\em Quotients of Groupoids by the Action of a
Group\/}, Math.  Proc. {C}amb. Phil. Soc., 103, (1988), 239--249.

\bibitem{tonksthesis}
A.~P. Tonks, 1993, {\em Theory and applications of crossed
complexes\/}, Ph.D. thesis, University of Wales, Bangor.

\bibitem{TV}
V.G. Turaev and O.Ya. Viro. State sum invariants of 3--manifolds and quantum
6j--symbols. \textit{Topology} \textbf{31} (no. 4): 865--902 (1992).

\bibitem{Varilly}
J. C. V\'arilly: An introduction to noncommutative geometry, (1997)
arXiv:physics/9709045

\bibitem{van kampen1}
E.~H.~van {Kampen}, {\em On the Connection Between the Fundamental
Groups of some Related Spaces\/}, Amer. J. Math., 55, (1933), 261--267.

\bibitem{Xu}
P. Xu.: Quantum groupoids and deformation quantization. (1997). $arxiv.org/pdf/q--alg/9708020.pdf$.

\bibitem{Y} 
D.N. Yetter.: TQFT' s from homotopy 2-types. \textit{J. Knot Theor}. \textbf{2}: 113--123(1993).

\bibitem{Weinberg}
S. Weinberg.:  \emph{The Quantum Theory of Fields}. Cambridge, New York and Madrid: %%@
Cambridge University Press, Vols. 1 to 3, (1995--2000).

\bibitem{Weinstein}
A. Weinstein : Groupoids: unifying internal and external symmetry,
\emph{Notices of the Amer. Math. Soc.} \textbf{43} (7): 744--752 (1996).

\bibitem{WB}
J. Wess and J. Bagger: \emph{Supersymmetry and Supergravity},
Princeton University Press, (1983).

\bibitem{WJ1}
J. Westman: Harmonic analysis on groupoids, \textit{Pacific J. Math.} \textbf{27}: 621-632. (1968).


\bibitem{WJ1}
J. Westman: Groupoid theory in algebra, topology and analysis., \textit{University of California at Irvine} (1971).

\bibitem{Wickra}
S. Wickramasekara and A. Bohm: Symmetry representations in the rigged Hilbert space formulation of quantum mechanics, \emph{J. Phys. A} \textbf{35} (2002), no. 3, 807--829.


\bibitem{Wightman1}
Wightman, A. S., 1956, Quantum Field Theory in Terms of Vacuum Expectation Values., Physical Review, \textbf{101}: 860--866.

\bibitem{Wightman2}
Wightman, A.S.: 1976, Hilbert's Sixth Problem: Mathematical Treatment of the Axioms of Physics., In: 
\textit{Proceedings of Symposia in Pure Mathematics}, \textbf{28}: 147--240.

\bibitem{Wightman--Garding3}
Wightman, A.S. and G\"arding, L., 1964, {\em Fields as Operator--Valued Distributions in 
Relativistic Quantum Theory} , Arkiv fur Fysik, textbf{28}: 129--184.


\bibitem{Woronowicz1}
S. L. Woronowicz: Twisted \emph{SU(2)} group : An example of a non--commutative differential 
calculus, RIMS, Kyoto University \textbf{23} (1987), 613--665.

\end{thebibliography}
%%%%%
%%%%%
\end{document}
