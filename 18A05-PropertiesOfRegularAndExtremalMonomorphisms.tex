\documentclass[12pt]{article}
\usepackage{pmmeta}
\pmcanonicalname{PropertiesOfRegularAndExtremalMonomorphisms}
\pmcreated{2013-03-22 16:03:48}
\pmmodified{2013-03-22 16:03:48}
\pmowner{kompik}{10588}
\pmmodifier{kompik}{10588}
\pmtitle{properties of regular and extremal monomorphisms}
\pmrecord{12}{38117}
\pmprivacy{1}
\pmauthor{kompik}{10588}
\pmtype{Theorem}
\pmcomment{trigger rebuild}
\pmclassification{msc}{18A05}
\pmrelated{ExtremalMonomorphism}
\pmrelated{RegularMonomorphism}
\pmrelated{Equalizer}
\pmrelated{StrongMonomorphism}

\endmetadata

% this is the default PlanetMath preamble. as your knowledge
% of TeX increases, you will probably want to edit this, but
% it should be fine as is for beginners.

% almost certainly you want these
\usepackage{amssymb}
\usepackage{amsmath}
\usepackage{amsfonts}
\usepackage{amsthm}

% used for TeXing text within eps files
%\usepackage{psfrag}
% need this for including graphics (\includegraphics)
%\usepackage{graphicx}
% for neatly defining theorems and propositions
%
% making logically defined graphics
%%\usepackage{xypic}

% there are many more packages, add them here as you need them

% define commands here

\newcommand{\sR}[0]{\mathbb{R}}
\newcommand{\sC}[0]{\mathbb{C}}
\newcommand{\sN}[0]{\mathbb{N}}
\newcommand{\sZ}[0]{\mathbb{Z}}

\newcommand{\R}[0]{\mathbb{R}}
\newcommand{\C}[0]{\mathbb{C}}
\newcommand{\N}[0]{\mathbb{N}}
\newcommand{\Z}[0]{\mathbb{Z}}


%\usepackage{bbm}
%\newcommand{\N}{\mathbbmss{N}}
%\newcommand{\Z}{\mathbbmss{Z}}
%\newcommand{\C}{\mathbbmss{C}}
%\newcommand{\R}{\mathbbmss{R}}
%\newcommand{\Q}{\mathbbmss{Q}}



\newcommand*{\norm}[1]{\lVert #1 \rVert}
\newcommand*{\abs}[1]{| #1 |}

\newcommand{\Map}[3]{#1:#2\to#3}
\newcommand{\Emb}[3]{#1:#2\hookrightarrow#3}
\newcommand{\Mor}[3]{#2\overset{#1}\to#3}

\newcommand{\Cat}[1]{\mathcal{#1}}
\newcommand{\Kat}[1]{\mathbf{#1}}
\newcommand{\Func}[3]{\Map{#1}{\Cat{#2}}{\Cat{#3}}}
\newcommand{\Funk}[3]{\Map{#1}{\Kat{#2}}{\Kat{#3}}}

\newcommand{\intrv}[2]{\langle #1,#2 \rangle}

\newcommand{\vp}{\varphi}
\newcommand{\ve}{\varepsilon}

\newcommand{\Invimg}[2]{\inv{#1}(#2)}
\newcommand{\Img}[2]{#1[#2]}
\newcommand{\ol}[1]{\overline{#1}}
\newcommand{\ul}[1]{\underline{#1}}
\newcommand{\inv}[1]{#1^{-1}}
\newcommand{\limti}[1]{\lim\limits_{#1\to\infty}}

\newcommand{\Ra}{\Rightarrow}

%fonts
\newcommand{\mc}{\mathcal}

%shortcuts
\newcommand{\Ob}{\mathrm{Ob}}
\newcommand{\Hom}{\mathrm{hom}}
\newcommand{\homs}[2]{\mathrm{hom(}{#1},{#2}\mathrm )}
\newcommand{\Eq}{\mathrm{Eq}}
\newcommand{\Coeq}{\mathrm{Coeq}}

%theorems
\newtheorem{THM}{Theorem}
\newtheorem{DEF}{Definition}
\newtheorem{PROP}{Proposition}
\newtheorem{LM}{Lemma}
\newtheorem{COR}{Corollary}
\newtheorem{EXA}{Example}

%categories
\newcommand{\Top}{\Kat{Top}}
\newcommand{\Haus}{\Kat{Haus}}
\newcommand{\Set}{\Kat{Set}}

%diagrams
\newcommand{\UnimorCD}[6]{
\xymatrix{ {#1} \ar[r]^{#2} \ar[rd]_{#4}& {#3} \ar@{-->}[d]^{#5} \\
& {#6} } }

\newcommand{\RovnostrCD}[6]{
\xymatrix@C=10pt@R=17pt{
& {#1} \ar[ld]_{#2} \ar[rd]^{#3} \\
{#4} \ar[rr]_{#5} && {#6} } }

\newcommand{\RovnostrCDii}[6]{
\xymatrix@C=10pt@R=17pt{
{#1} \ar[rr]^{#2} \ar[rd]_{#4}&& {#3} \ar[ld]^{#5} \\
& {#6} } }

\newcommand{\RovnostrCDiiop}[6]{
\xymatrix@C=10pt@R=17pt{
{#1}  && {#3} \ar[ll]_{#2}  \\
& {#6} \ar[lu]^{#4} \ar[ru]_{#5} } }

\newcommand{\StvorecCD}[8]{
\xymatrix{
{#1} \ar[r]^{#2} \ar[d]_{#4} & {#3} \ar[d]^{#5} \\
{#6} \ar[r]_{#7} & {#8}
}
}

\newcommand{\TriangCD}[6]{
\xymatrix{ {#1} \ar[r]^{#2} \ar[rd]_{#4}&
{#3} \ar[d]^{#5} \\
& {#6} } }
\begin{document}
We will denote the equalizer of $f$ and $g$ by $e=\Eq(f,g)$. 
%Coequalizers will be denoted as $\Coeq(f,g)$.

\begin{PROP}
Every regular monomorphism is a monomorphism. (Every regular epimorphism is an epimorphism.)
\end{PROP}

\begin{proof}
Let $f=\Eq(r,s)$. Let $f\circ g=f\circ h$. Then $r\circ(f\circ g)=s\circ(f\circ g)$ and by
the definition of the equalizer there exists a unique morphism $h$ such that $f\circ g=f\circ
h$, thus $g=h$.
\end{proof}

\begin{PROP} \label{GFJEEXMONOIMPFEXJEMONO}
If $g\circ f$ is an extremal monomorphism, then $f$ is an extremal monomorphism.\\
If $g\circ f$ is an extremal epimorphism, then $g$ is an extremal epimorphism.
\end{PROP}

\begin{proof}
Since $g\circ f$ is a monomorphism, $f$ is a monomorphism too. Let $f=h\circ e$ and $e$ be an
epimorphism. Then $g\circ f=g\circ h\circ e$, but $g\circ f$ is an extremal monomorphism,
thus $e$ is an isomorphism.

The second part of the proposition is \PMlinkname{dual}{DualityPrinciple}  to the first part.
\end{proof}

\begin{PROP} \label{HIERMOR}
If $\Map fXY$ is a morphism then each of the following conditions implies the next one:
  \begin{enumerate}
%  \renewcommand{\labelenumi}{$(\mathrm{\roman{enumi}})$}
  \renewcommand{\labelenumi}{(\roman{enumi})}
    \item $f$ is an isomorphism
    \item $f$ is a section
    \item $f$ is a regular monomorphism
    \item $f$ is an extremal monomorphism
    \item $f$ is a monomorphism.
  \end{enumerate}

(Dual claim: $f$ is an isomorphism $\Rightarrow$ retraction $\Rightarrow$ regular epimorphism
$\Rightarrow$ extremal epimorphism $\Rightarrow$ epimorphism.)
\end{PROP}


\begin{proof}
(i)$\Rightarrow$(ii) straightforward from the definition.

(ii)$\Rightarrow$(iii) Let $g\circ f=id_A$, we will show that $f=\Eq(id_B,f\circ g)$. It
holds $(f\circ g)\circ f=f\circ(g\circ f)=f\circ id_A=f=id_B\circ f$. If $(f\circ g)\circ
h=h$ then $h=f\circ(g\circ h)$ and there is unique such morphism, since $f$ is a monomorphism
(every section is a monomorphism).

(iii)$\Rightarrow$(iv) Let $f=\Eq(r,s)$ and $f=g\circ e$ with $e$ an epimorphism. It holds:
$(r\circ g)\circ e=r\circ(g\circ e)=r\circ f=s\circ f=s\circ (g\circ e)=(s\circ g)\circ e$,
thus it holds $r\circ g=s\circ g$ as well (since $e$ is an epimorphism). By the universal
property in the definition of equalizer there exists a unique morphism $e'$ such that
$g=f\circ e'$. Thus we get $f\circ id_A=f=g\circ e=f\circ e'\circ e$ and $f$ is a
monomorphism, hence $e'\circ e=id_A$, i.e., $e$ is a section. Moreover $id_E\circ e=e=e\circ
id_A=e\circ(e'\circ e)=(e\circ e')\circ e$ and $e$ is an epimorphism, hence $id_E=e\circ e'$,
i.e., $e$ is a section. The morphism $e$ is a retraction and a section too, thus $e$ is an
isomorphism.

(iv)$\Rightarrow$(v) Follows easily from the definition.
\end{proof}

The implication retraction $\Ra$ regular epimorphism can be interpreted in the category of
topological spaces $\Top$ as the well-known fact that each retraction is a quotient map.

\begin{PROP} \label{EKVEPODMPREIZOM}
Let $\Map fAB$ be a morphism. The following conditions are equivalent:
  \begin{enumerate}
  \renewcommand{\labelenumi}{(\roman{enumi})}
    \item $f$ is an isomorphism
    \item $f$ is an epimorphism and a section
    \item $f$ is an epimorphism and an extremal monomorphism
    \item $f$ is a monomorphism and a retraction
    \item $f$ is a monomorphism and an extremal epimorphism.
  \end{enumerate}
\end{PROP}

\begin{proof}
Thanks to the duality principle, it suffices to prove the equivalence of the first three conditions.

(i) $\Rightarrow$ (ii) follows directly from the definition and (ii) $\Rightarrow$ (iii) is
an easy consequence of the above proposition. (iii) $\Rightarrow$ (i): $f=id_B\circ f$ and
$f$ an epimorphism and  extremal monomorphism. This implies that $f$ is an isomorphism.
\end{proof}
%%%%%
%%%%%
\end{document}
