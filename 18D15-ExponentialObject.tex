\documentclass[12pt]{article}
\usepackage{pmmeta}
\pmcanonicalname{ExponentialObject}
\pmcreated{2013-03-22 16:36:20}
\pmmodified{2013-03-22 16:36:20}
\pmowner{CWoo}{3771}
\pmmodifier{CWoo}{3771}
\pmtitle{exponential object}
\pmrecord{9}{38801}
\pmprivacy{1}
\pmauthor{CWoo}{3771}
\pmtype{Definition}
\pmcomment{trigger rebuild}
\pmclassification{msc}{18D15}
\pmrelated{CompleteCategory}
\pmdefines{has exponentials}
\pmdefines{exponential functor}

\endmetadata

\usepackage{amssymb,amscd}
\usepackage{amsmath}
\usepackage{amsfonts}

% used for TeXing text within eps files
%\usepackage{psfrag}
% need this for including graphics (\includegraphics)
%\usepackage{graphicx}
% for neatly defining theorems and propositions
%\usepackage{amsthm}
% making logically defined graphics
%%\usepackage{xypic}
\usepackage{pst-plot}
\usepackage{psfrag}

% define commands here

\begin{document}
Let $\mathcal{C}$ be a category with finite products and $A,B$ be objects in $\mathcal{C}$.  An object $E$ in $\mathcal{C}$ is called an \emph{exponential object} from $A$ to $B$ if it satisfies the following conditions:
\begin{itemize}
\item there is a morphism $f:E\times A\to B$, called an \emph{evaluation morphism}
\item for any morphism $g:C\times A\to B$, there is a unique morphism $h:C\to E$ such that $f\circ (h\times 1_A)=g$, where $h\times 1_A:C\times A\to E\times A$ is the product morphism of $h$ and the identity morphism on $A$.
\end{itemize}
The two conditions can be summarized by the following commutative diagram:
\begin{center}
$
\xymatrix@R-=20pt{
E\times A\ar[dr]^f\\
&B\\
C\times A\ar[ur]_g\ar[uu]^{h\times 1_A}
}
$
\end{center}
where $h$ is uniquely determined by $g$.  It is easy to see that any two exponential objects from $A$ to $B$ are isomorphic, hence the existence of an exponential objects between two objects is a universal property.  We may write $B^A (\cong E$ above) \emph{the} exponential object from $A$ to $B$.

For example, in the category of sets, $\textbf{Set}$, where products clearly exist (empty set is a set!) between pairs of objects (sets), the exponential from $A$ to $B$ is the set $B^A$, which is defined as the set of all functions from $A$ to $B$.  The evaluation morphism is the function $ev: B^A\times A\to B$ given by $ev(f,a)=f(a)$, where $f\in B^A$ and $a\in A$.  If $g:C\times A\to B$ is any function, then we define $h:C\to B^A$ by $h(c)(a)=g(c,a)$.  Then $ev\circ (h\times 1_A)(c,a)=ev(h(c),a)=h(c)(a)=g(c,a)$, and $ev$ is universal (in the sense of the second condition above).

Since each $h$ is uniquely determined by $g$ in the above definition, and conversely every $h$ determines a $g$ by the formula $g=f\circ (h\times 1_A)$, we have a bijection $$\hom(C\times A,B)\cong \hom(C,B^A).$$

If an exponential object exists between every pair of objects in category $C$ with finite products, then we say that $C$ \emph{has exponentials}.  According to the bijection above, we see that the functor $\cdot\times A:\mathcal{C}\to \mathcal{C}$ has a right adjoint, namely $\cdot ^A:\mathcal{C}\to\mathcal{C}$, called the \emph{exponential functor}.

It can be seen that a category $C$ with finite products has exponentials iff the covariant function $\cdot\times A:\mathcal{C}\to\mathcal{C}$ has a right adjoint for every object $A$ in $\mathcal{C}$.
%%%%%
%%%%%
\end{document}
