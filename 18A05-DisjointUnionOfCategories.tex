\documentclass[12pt]{article}
\usepackage{pmmeta}
\pmcanonicalname{DisjointUnionOfCategories}
\pmcreated{2013-03-22 18:27:22}
\pmmodified{2013-03-22 18:27:22}
\pmowner{CWoo}{3771}
\pmmodifier{CWoo}{3771}
\pmtitle{disjoint union of categories}
\pmrecord{5}{41119}
\pmprivacy{1}
\pmauthor{CWoo}{3771}
\pmtype{Example}
\pmcomment{trigger rebuild}
\pmclassification{msc}{18A05}
\pmrelated{CategoricalDirectSum}
\pmrelated{ProductOfCategories}
\pmdefines{disjoint union}

\endmetadata

\usepackage{amssymb,amscd}
\usepackage{amsmath}
\usepackage{amsfonts}
\usepackage{mathrsfs}

% used for TeXing text within eps files
%\usepackage{psfrag}
% need this for including graphics (\includegraphics)
%\usepackage{graphicx}
% for neatly defining theorems and propositions
\usepackage{amsthm}
% making logically defined graphics
%%\usepackage{xypic}
\usepackage{pst-plot}

% define commands here
\newcommand*{\abs}[1]{\left\lvert #1\right\rvert}
\newtheorem{prop}{Proposition}
\newtheorem{thm}{Theorem}
\newtheorem{ex}{Example}
\newcommand{\real}{\mathbb{R}}
\newcommand{\pdiff}[2]{\frac{\partial #1}{\partial #2}}
\newcommand{\mpdiff}[3]{\frac{\partial^#1 #2}{\partial #3^#1}}
\begin{document}
Let $\lbrace \mathcal{C}_i\rbrace$ be a collection of categories, indexed by a set $I$.  The \PMlinkescapetext{\emph{disjoint union}} $\mathcal{C}$ of these categories is defined as follows:
\begin{enumerate}
\item the class of objects of $\mathcal{C}$ is the disjoint union of classes of objects, $\operatorname{Ob}(\mathcal{C}_i)$, for every $i\in I$,
\item the class of morphisms of $\mathcal{C}$ is the disjoint union of classes of morphisms, $\operatorname{Mor}(\mathcal{C}_i)$, for every $i\in I$.
\item for objects $A,B$ in $\mathcal{C}$, if they are objects of $\mathcal{C}_i$, then $\hom(A,B)$ is the set of morphisms from $A$ to $B$ in $\mathcal{C}_i$, otherwise, $\hom(A,B):=\varnothing$.
\item given $\hom(A,B)$ and $\hom(B,C)$, the composition of morphisms is defined so that, if $A,B,C$ are all objects of some $\mathcal{C}_i$, the composition is the same as the composition of morphisms defined in $\mathcal{C}_i$.  Otherwise, it is defined as $\varnothing$.
\end{enumerate}

With the above conditions, one immediately sees that $\mathcal{C}$ is a category, as each $\hom(A,B)$ is a set, associativity of morphism composition and identity morphisms all inherit from the individual categories $\mathcal{C}_i$.

\textbf{Remark}.  If each $\mathcal{C}_i$ is small, so is their disjoint union.  In fact, in \textbf{Cat}, the category of small categories, the disjoint union of these categories is their coproduct.
%%%%%
%%%%%
\end{document}
