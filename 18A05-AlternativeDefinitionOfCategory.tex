\documentclass[12pt]{article}
\usepackage{pmmeta}
\pmcanonicalname{AlternativeDefinitionOfCategory}
\pmcreated{2013-03-22 16:38:31}
\pmmodified{2013-03-22 16:38:31}
\pmowner{rspuzio}{6075}
\pmmodifier{rspuzio}{6075}
\pmtitle{alternative definition of category}
\pmrecord{9}{38845}
\pmprivacy{1}
\pmauthor{rspuzio}{6075}
\pmtype{Definition}
\pmcomment{trigger rebuild}
\pmclassification{msc}{18A05}

\endmetadata

% this is the default PlanetMath preamble.  as your knowledge
% of TeX increases, you will probably want to edit this, but
% it should be fine as is for beginners.

% almost certainly you want these
\usepackage{amssymb}
\usepackage{amsmath}
\usepackage{amsfonts}

% used for TeXing text within eps files
%\usepackage{psfrag}
% need this for including graphics (\includegraphics)
%\usepackage{graphicx}
% for neatly defining theorems and propositions
%\usepackage{amsthm}
% making logically defined graphics
%%\usepackage{xypic}
\xyoption{all}

% there are many more packages, add them here as you need them

% define commands here
\newcommand{\comp}{\circ}
\newcommand{\src}{S}
\newcommand{\tgt}{T}
\newcommand{\Hom}{\mathrm{Hom}}
\begin{document}
When compared with set theory, category theory may seem ontologically
extravagant; after all, set theory postulates the existence of only
one type of entity, the class, while category theory postulates the
existence of at least two types, objects and morphisms.  However, it
is possible to dispense with objects and deal only with morphisms.
The basic idea behind this reduction is the observation that objects
appear in a one-to-one correspondence with their identity morphisms.
Thus to say what a category is, we need only specify the axioms the
morphisms must obey; ultimately this will allow us to present a
category as a sort of partial algebraic system, with a partial 
binary operation representing composition of morphisms and two 
unary operations representing the domain and codomain of a morphism.

Let $M$ be a class.  Its members will be called \emph{morphisms}.
Then a \emph{category} on $M$ consists of the following ingredients:

\begin{itemize}
\item
a partial binary operation $\comp\colon M\times M\to M$, called
\emph{composition};

\item
a unary operation $\src\colon M\to M$, called \emph{source} or \emph{domain}; and

\item
a unary operation $\tgt\colon M\to M$, called \emph{target} or \emph{codomain}.
\end{itemize}

The above operations are required to satisfy the following axioms.

\begin{enumerate}
\item
{\bf (Right absorption.)} 
For any morphism $f$,
\[
\src\src f = \tgt\src f = \src f \text{\quad and\quad}
\tgt\src f = \tgt\tgt f = \tgt f.
\]

\item
{\bf (Existence of composite morphisms.)}  
The composite morphism $g\comp f$ is defined if and only if $\tgt g =
\src f$; the morphism $g\comp f$ has source $\src(g\comp f) = \src f$
and target $\tgt(g\comp f) = \tgt f$.
\[\xymatrix{
\bullet\ar@(ul,dl)[]_{\src(g\comp f) = \src f}\ar[r]^f\ar@{.>}[rd]_{\exists !\, g\comp f} & \bullet\ar[d]^g \\
                                                 & \bullet\ar@(dr,ur)_{\tgt (g\comp f) = \tgt g}
}\]

\item
{\bf (Existence of identity morphisms.)}
For any morphism $f$, $\tgt f\comp f = f\comp\src f = f$.
\[\xymatrix{
  \bullet\ar[r]^f\ar[d]_{\src f}\ar[rd]^f 
& \bullet\ar[d]^{\tgt f} \\
  \bullet\ar[r]_f 
& \bullet
}\]


\item
{\bf (Associativity of composition.)} 
Whenever $h\comp g$ and $g\comp f$ are both defined, then $h\comp
(g\comp f) = (h\comp g)\comp f$.
\[\xymatrix{
  \bullet\ar[r]^f\ar[rd]_{g\comp f} 
& \bullet\ar[d]^g\ar[rd]^{h\comp g}
& \\
& \bullet\ar[r]_h 
& \bullet
}\]
\end{enumerate}

Using this definition, we can define an \emph{object} to be a morphism
in the image of $\src$; by the right absorption law, this is
equivalent to being in the image of $\tgt$.  If we also define
$\Hom(A,B)$ to be the collection of morphisms $f$ such that $\src f =
A$ and $\tgt f = B$, then we recover the rest of the ordinary
definition of category.

\PMlinkescapeword{absorption}
\PMlinkescapeword{collection}
\PMlinkescapeword{composite}
\PMlinkescapeword{observation}
\PMlinkescapeword{postulates}
\PMlinkescapeword{reduction}
\PMlinkescapeword{right}
\PMlinkescapeword{satisfy}
\PMlinkescapeword{source}
\PMlinkescapeword{type}

\begin{thebibliography}{8}
\bibitem{pfas} P. Freyd, A. Scedrov, {\em Categories, Allegories}, North-Holland, New York (1989).
\end{thebibliography}
%%%%%
%%%%%
\end{document}
