\documentclass[12pt]{article}
\usepackage{pmmeta}
\pmcanonicalname{Source}
\pmcreated{2013-03-22 16:03:25}
\pmmodified{2013-03-22 16:03:25}
\pmowner{kompik}{10588}
\pmmodifier{kompik}{10588}
\pmtitle{source}
\pmrecord{10}{38109}
\pmprivacy{1}
\pmauthor{kompik}{10588}
\pmtype{Definition}
\pmcomment{trigger rebuild}
\pmclassification{msc}{18A05}
\pmrelated{UniversalProperty}
\pmdefines{source}
\pmdefines{monosource}
\pmdefines{extremal monosource}
\pmdefines{sink}
\pmdefines{extremal episink}

\endmetadata

% this is the default PlanetMath preamble. as your knowledge
% of TeX increases, you will probably want to edit this, but
% it should be fine as is for beginners.

% almost certainly you want these
\usepackage{amssymb}
\usepackage{amsmath}
\usepackage{amsfonts}
\usepackage{amsthm}

% used for TeXing text within eps files
%\usepackage{psfrag}
% need this for including graphics (\includegraphics)
%\usepackage{graphicx}
% for neatly defining theorems and propositions
%
% making logically defined graphics
%%%\usepackage{xypic}

% there are many more packages, add them here as you need them

% define commands here

\newcommand{\sR}[0]{\mathbb{R}}
\newcommand{\sC}[0]{\mathbb{C}}
\newcommand{\sN}[0]{\mathbb{N}}
\newcommand{\sZ}[0]{\mathbb{Z}}
\newcommand{\N}[0]{\mathbb{N}}


\usepackage{bbm}
\newcommand{\Z}{\mathbbmss{Z}}
\newcommand{\C}{\mathbbmss{C}}
\newcommand{\R}{\mathbbmss{R}}
\newcommand{\Q}{\mathbbmss{Q}}



\newcommand*{\norm}[1]{\lVert #1 \rVert}
\newcommand*{\abs}[1]{| #1 |}

\newcommand{\Map}[3]{#1:#2\to#3}
\newcommand{\Emb}[3]{#1:#2\hookrightarrow#3}
\newcommand{\Mor}[3]{#2\overset{#1}\to#3}

\newcommand{\Cat}[1]{\mathcal{#1}}
\newcommand{\Kat}[1]{\mathbf{#1}}
\newcommand{\Func}[3]{\Map{#1}{\Cat{#2}}{\Cat{#3}}}
\newcommand{\Funk}[3]{\Map{#1}{\Kat{#2}}{\Kat{#3}}}

\newcommand{\intrv}[2]{\langle #1,#2 \rangle}

\newcommand{\vp}{\varphi}
\newcommand{\ve}{\varepsilon}

\newcommand{\Invimg}[2]{\inv{#1}(#2)}
\newcommand{\Img}[2]{#1[#2]}
\newcommand{\ol}[1]{\overline{#1}}
\newcommand{\ul}[1]{\underline{#1}}
\newcommand{\inv}[1]{#1^{-1}}
\newcommand{\limti}[1]{\lim\limits_{#1\to\infty}}

\newcommand{\Ra}{\Rightarrow}

%fonts
\newcommand{\mc}{\mathcal}

%shortcuts
\newcommand{\Ob}{\mathrm{Ob}}
\newcommand{\Hom}{\mathrm{hom}}
\newcommand{\homs}[2]{\mathrm{hom(}{#1},{#2}\mathrm )}
\newcommand{\Eq}{\mathrm{Eq}}
\newcommand{\Coeq}{\mathrm{Coeq}}

%theorems
\newtheorem{THM}{Theorem}
\newtheorem{DEF}{Definition}
\newtheorem{PROP}{Proposition}
\newtheorem{LM}{Lemma}
\newtheorem{COR}{Corollary}
\newtheorem{EXA}{Example}

%categories
\newcommand{\Top}{\Kat{Top}}
\newcommand{\Haus}{\Kat{Haus}}
\newcommand{\Set}{\Kat{Set}}

%diagrams
\newcommand{\UnimorCD}[6]{
\xymatrix{ {#1} \ar[r]^{#2} \ar[rd]_{#4}& {#3} \ar@{-->}[d]^{#5} \\
& {#6} } }

\newcommand{\RovnostrCD}[6]{
\xymatrix@C=10pt@R=17pt{
& {#1} \ar[ld]_{#2} \ar[rd]^{#3} \\
{#4} \ar[rr]_{#5} && {#6} } }

\newcommand{\RovnostrCDii}[6]{
\xymatrix@C=10pt@R=17pt{
{#1} \ar[rr]^{#2} \ar[rd]_{#4}&& {#3} \ar[ld]^{#5} \\
& {#6} } }

\newcommand{\RovnostrCDiiop}[6]{
\xymatrix@C=10pt@R=17pt{
{#1}  && {#3} \ar[ll]_{#2}  \\
& {#6} \ar[lu]^{#4} \ar[ru]_{#5} } }

\newcommand{\StvorecCD}[8]{
\xymatrix{
{#1} \ar[r]^{#2} \ar[d]_{#4} & {#3} \ar[d]^{#5} \\
{#6} \ar[r]_{#7} & {#8}
}
}

\newcommand{\TriangCD}[6]{
\xymatrix{ {#1} \ar[r]^{#2} \ar[rd]_{#4}&
{#3} \ar[d]^{#5} \\
& {#6} } }

\begin{document}
In the whole entry we suppose we are given a category $\Kat A$. By an object we always mean an object in $\Kat A$ and by a morphisms an $\Kat A$-morphism.

\begin{DEF}
A \emph{source} in a category $\Kat A$ is a pair $(A,(f_i)_{i\in I})$ where $A$ is an object and $\Map{f_i}A{A_i}$ are morphisms indexed by a class $I$.

The object $A$ is called the \emph{domain of the source} and the family $(A_i)_{i\in I}$ is called the codomain of the source.
\end{DEF}

A \emph{sink} is a pair $((f_i)_{i\in I},A)$ where $A$ is an object and $\Map{f_i}{A_i}A$ are
morphisms.

Sources can be composed with morphisms. If $\mathcal S=(A,(f_i)_{i\in I}$ is a source and
$\Map fBA$ is a morphism, we use the notation $(B,(f_i\circ f)_{i\in I})=\mathcal S\circ f$. 
Similarly, for sinks, we use the notation 
$f\circ\mathcal S=((f\circ f_i)_{i\in I},B)$ if
$\mathcal S=((f_i)_{i\in I}, A)$ is a sink and $\Map fAB$ is a morphism.

\begin{DEF}
A source $\mathcal S=(A,(f_i)_{i\in I})$ is called a \emph{monosource} if for any pair
$\Map{r,s}BA$ of morphisms from the equality $\mathcal S\circ r=\mathcal S\circ s$ follows
$r=s$.

A sink $\mathcal S=((f_i)_{i\in I},A)$ is called an \emph{episink} if for any pair
$\Map{r,s}AB$ of morphisms $r=s$ whenever $r\circ\mathcal S=s\circ\mathcal S$.

A monosource $\mc S$ is called \emph{extremal monosource}, if the following holds: Whenever
$\mathcal S=\overline{\mathcal S}\circ e$ for an epimorphism $e$, then $e$ is an isomorphism.

An episink $\mc S$ is called \emph{extremal episink} if the following holds: Whenever
$\mathcal S=m\circ\overline{\mathcal S}$ pre for a monomorphism $m$, tak $m$ is an isomorphism.
\end{DEF}

Every limit is an extremal monosource, a colimit is an extremal episink.

\begin{thebibliography}{1}

\bibitem{ahs}
J.~Ad\'amek, H.~Herrlich, and G.~Strecker.
\newblock {\em Abstract and Concrete Categories}.
\newblock Wiley, New York, 1990.

\end{thebibliography}

%%%%%
%%%%%
\end{document}
