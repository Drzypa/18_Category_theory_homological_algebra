\documentclass[12pt]{article}
\usepackage{pmmeta}
\pmcanonicalname{LimitOfAFunctor}
\pmcreated{2013-03-22 14:11:23}
\pmmodified{2013-03-22 14:11:23}
\pmowner{CWoo}{3771}
\pmmodifier{CWoo}{3771}
\pmtitle{limit of a functor}
\pmrecord{28}{35619}
\pmprivacy{1}
\pmauthor{CWoo}{3771}
\pmtype{Definition}
\pmcomment{trigger rebuild}
\pmclassification{msc}{18A30}
\pmsynonym{inverse limit}{LimitOfAFunctor}
\pmsynonym{projective limit}{LimitOfAFunctor}
\pmsynonym{left limit}{LimitOfAFunctor}
\pmsynonym{left root}{LimitOfAFunctor}
\pmsynonym{root}{LimitOfAFunctor}
\pmsynonym{direct limit}{LimitOfAFunctor}
\pmsynonym{inductive limit}{LimitOfAFunctor}
\pmsynonym{right limit}{LimitOfAFunctor}
\pmsynonym{right root}{LimitOfAFunctor}
\pmsynonym{coroot}{LimitOfAFunctor}
\pmrelated{Category}
\pmrelated{Limit}
\pmrelated{InverseLimit}
\pmrelated{CategoryAssociatedToAPartialOrder}
\pmrelated{RepresentableFunctor}
\pmrelated{Small}
\pmrelated{GrothendieckCategory}
\pmrelated{LimitingCone}
\pmdefines{limit}
\pmdefines{colimit}
\pmdefines{index category}

\endmetadata

% this is the default PlanetMath preamble.  as your knowledge
% of TeX increases, you will probably want to edit this, but
% it should be fine as is for beginners.

% almost certainly you want these
\usepackage{amssymb}
\usepackage{amsmath}
\usepackage{amsfonts}

% used for TeXing text within eps files
%\usepackage{psfrag}
% need this for including graphics (\includegraphics)
%\usepackage{graphicx}
% for neatly defining theorems and propositions
\usepackage{amsthm}
% making logically defined graphics
%%\usepackage{xypic}

% there are many more packages, add them here as you need them

\usepackage{mathrsfs}

% define commands here

\newtheorem{theorem}{Theorem}
\newtheorem{defn}{Definition}
\newtheorem{prop}{Proposition}
\newtheorem{lemma}{Lemma}
\newtheorem{cor}{Corollary}
\newtheorem*{warning}{Warning}

\newcommand{\Univ}{\mathscr{U}}
\DeclareMathOperator{\liminv}{\varprojlim}
\DeclareMathOperator{\limdir}{\varinjlim}
\DeclareMathOperator{\Funct}{Funct}
\DeclareMathOperator{\Hom}{Hom}
\begin{document}
\PMlinkescapeword{Fix}
\PMlinkescapeword{fix}
%\section*{General definition}

Let $G$ be a functor from categories $\mathcal{I}$ to $\mathcal{C}$.  A \emph{limit} of $G$ is a pair $(L,\tau)$ where
\begin{enumerate}
\item $L:\mathcal{I}\to \mathcal{C}$ is a constant functor,
\item $\tau: L\to G$ is a natural transformation,
\end{enumerate}
such that it is universal among all pairs satisfying (1) and (2).  In other words, for any pair $(M,\varphi)$, where $M:\mathcal{I}\to \mathcal{C}$ is a constant functor, and $\varphi: M\to G$ is a natural transformation, there is a unique natural transformation $\phi:M\to L$ with the following commutative diagram:
\begin{center}
$
\xymatrix@C+=64pt{
M\ar[dr]^{\varphi}\ar@{.>}[dd]_{\phi}\\
&G\\
L\ar[ur]_{\tau}
}
$
\end{center}

Since we may identify a constant functor with its value, say an object $A$, in $\mathcal{C}$, a limit of $G$ may be viewed as an object $A$ in $\mathcal{C}$, together with a collection of morphisms $A \to G(I)$, or $G_I$, for each object $I$ in $\mathcal{I}$ such that 
\begin{alignat*}{2}
\mbox{if }I\to J\mbox{ is a morphism, then }A\to G_I\to G_J = A \to G_J. & \qquad (*)
\end{alignat*}
Furthermore, if another object $B$ in $\mathcal{C}$ satisfies $(*)$, then there is a unique morphism $B\to A$ such that $$B\to A\to G_I = B\to G_I$$ for all objects $I$ in $\mathcal{I}$.  

%Fix a universe $\Univ$.  Let $\mathcal{I}$ be a \PMlinkname{$\Univ$-small}{Small} category (call it an index category) and $\mathcal{C}$ a $\Univ$-category.  For an arbitrary functor $G:\mathcal{I\to C}$, the \emph{inverse limit} of $G$ is the contravariant functor $L:\mathcal{C}\to \Univ$-Set (the category of small sets) given by:
%$$X \mapsto \Hom(k_X, G),$$
%where 
%\begin{enumerate}
%\item $k_X:\mathcal{I\to C}$ is the constant functor whose fixed value is $X$, and
%\item $\Hom(k_X,G)$ is the set of morphisms from $k_X$ to $G$ in $\mathcal{C^I}$ (viewed as a functor category).
%\end{enumerate}

\textbf{Remarks}.
\begin{itemize}
\item
A limit of a functor may or may not exist.  If it exists, then it is unique up to natural equivalence.  In other words, we may call it \emph{the} limit of $G$.
\item
In the literature, the limit of $G$ is variously known as the \emph{inverse limit}, \emph{left limit}, \emph{projective limit}, \emph{root}, or \emph{left root} of $G$.  and is generally written $\liminv G$.
\item
The category $\mathcal{I}$ above is usually called the \emph{index category}, and the functor $G$ a \emph{diagram} in $\mathcal{C}$.
\item
The most common index categories are finite categories, discrete categories, partially ordered categories, and more specifically, directed categories.
\item
On the other hand, it turns out that if, in a category, if equalizers of any pairs of morphisms, and arbitrary products of any collections of objects exist, then it can be shown that every functor into this category has a limit.
%\item
%It often occurs that the functor $L$ is representable, that is, $L=\Hom(-,L')$ for some object $L'\in \mathcal{C}$.  If this occurs, we write 
%$$\liminv G = L'$$
%and say that the object $L'$ is the \emph{inverse limit} of $G$.
%\item
%It can be checked that $\liminv:\mathcal{C^I}\to ^{\Univ\text{-Set}}\mathcal{C}$ is a functor, where $^{\Univ\text{-Set}}\mathcal{C}$ is the category of contravariant functors from $\mathcal{C}$ to $\Univ$-Set.
\end{itemize}

By reversing all the arrows in $\mathcal{C}$, we arrive at the dual concept of a limit, that of a \emph{colimit}.  Precisely, the \emph{colimit} of $G:\mathcal{I}\to \mathcal{C}$ is a pair $(R,\tau)$ where $R:\mathcal{I}\to \mathcal{C}$ is a constant functor and $\tau: G\to R$ is a natural transformation, such that for any pair $(M,\varphi)$ of constant functor $M:\mathcal{I}\to \mathcal{C}$ and natural transformation $\varphi: G\to M$, there is a unique natural transformation $\phi: R\to M$ such that the following diagram
\begin{center}
$
\xymatrix@C+=64pt{
&R\ar@{.>}[dd]^{\phi}\\
G\ar[ur]^{\tau} \ar[dr]_{\varphi}\\
&M
}
$
\end{center}
The colimit is also known as the \emph{direct limit}, \emph{right limit}, \emph{inductive limit}, \emph{coroot}, or \emph{right root}, and is written $\limdir$.

\textbf{Examples}.  Some common examples of limits (inverse limits) are products, terminal objects, pullbacks, equalizers, kernels, and kernel pairs.  These examples can be readily verified.  Let us verify that a terminal object is a limit:  

Let $G$ be the empty functor (from the empty category) into an arbitrary category $\mathcal{C}$.  So the limit of $G$ is just an object $C$ in $\mathcal{C}$, and that's it, as there are no objects in the empty category, there are no morphisms from $C$ in the limit of $G$.  If $A$ is any object in $\mathcal{C}$, then there is a unique morphism $A\to C$, and that's it.  But this means that $C$ is just a terminal object of $\mathcal{C}$.

Please see the verification of some of these examples in the attachments below.

Some examples of colimits (direct limits) are coproducts, initial objects, pushouts, coequalizers, cokernels, and cokernel pairs.

%\begin{warning}
%This definition requires that the index category be $\Univ$-small.  So, for example, the category of all $\Univ$-sets is too big to take an inverse limit over.  This can pose problems from time to time.  For example, when one wants to define stalks for sheaves on the crystalline site, one wishes to take an inverse limit over all infinitesimal thickenings of a neighborhood, but one can readily construct non-isomorphic infinitesimal thickenings in too great a number to be contained in a $\Univ$-set.  Some careful argument is required to resolve this issue; in \cite{sga4} it is treaded using the theory of topoi.
%\end{warning}

%The canonical reference for this very general treatment is \cite{sga4}, Tome 1.

%\section*{Limits in practice}

%In practice, this very general definition is rather difficult to work with.
%One can prove the following:
%\begin{theorem}
%Let $L\in \mathcal{C}$ be the inverse limit of $G$.  Then:
%\begin{enumerate}
%\item\label{item:1} Suppose we have an object $A$, and for each $i\in I$ we are given a morphism $f_i\colon A\to G(i)$ such that the diagram
%\[
%\xymatrix{
%& A\ar[dl]_{f_i}\ar[dr]^{f_j} & \\
%G(i)\ar[rr]^{G(h)} & & G(j) \\
%}
%\]
%commutes for every $h\colon i\to j$.  Then there exists a unique morphism $f\colon A \to L$.
%\item\label{item:2} The object $L$ has a natural family of maps $\pi_i$ of the type just described.
%\end{enumerate}
%\end{theorem}
%\begin{proof}
%The proof is essentially formal from the definition.  We begin with (\ref{item:1}). Let $\{f_i\}$ be a collection of morphisms as in the theorem.  Then together the $\{f_i\}$ form a natural transformation from $k_A$ to $G$.  But a natural transformation of functors is exactly what we call a morphism in the category $\mathcal{C^I}$.  So the $\{f_i\}$ form an element of $\Hom(k_A,G)$.  But the theorem requires that the functor $X\mapsto\Hom(k_X,G)$ be representable by $L$, so that $\Hom(k_A,G) = \Hom(A,L)$.  Since the $\{f_i\}$ correspond to a unique element on the left-hand side, they correspond to a unique element on the right-hand side, that is, a unique morphism $A\to L$.  But since 
%\[
%\left(X\mapsto\Hom(k_X,G)\right) = \Hom(-,L)
%\]
%as functors, the diagrams commute.

%To verify (\ref{item:2}), simply observe that if $A=L$, we have a canonical element of $\Hom(L,L)$, namely the identity arrow; it corresponds to a natural transformation $\{\pi_i\}$ as required.
%\end{proof}



%\textbf{Remark}.  Very common index categories include finite categories with only the identity arrows, positive integers, or more generally directed sets with the \PMlinkname{category obtained from their order}{CategoryAssociatedToAPartialOrder}.  In both these cases, limits take on a much more comprehensible form.

%\begin{thebibliography}{9}
%\bibitem[SGA4]{sga4}
%Grothendieck et al. \emph{Seminaires en Geometrie Algebrique 4}, Tome 1, Expos\'e 1 (or the appendix to Expos\'e 1, by N. Bourbaki for more detail and a large number of results there described as ``ne pouvant servir \`a rien''). SGA4 is \PMlinkexternal{available}{http://modular.fas.harvard.edu/sga/sga/pdf/index.html} on the Web. (It is in French.)
%\end{thebibliography}
%%%%%
%%%%%
\end{document}
