\documentclass[12pt]{article}
\usepackage{pmmeta}
\pmcanonicalname{Syzygy}
\pmcreated{2013-03-22 14:51:07}
\pmmodified{2013-03-22 14:51:07}
\pmowner{CWoo}{3771}
\pmmodifier{CWoo}{3771}
\pmtitle{syzygy}
\pmrecord{7}{36524}
\pmprivacy{1}
\pmauthor{CWoo}{3771}
\pmtype{Definition}
\pmcomment{trigger rebuild}
\pmclassification{msc}{18G10}
\pmclassification{msc}{16E05}
\pmclassification{msc}{13D02}

\endmetadata

% this is the default PlanetMath preamble.  as your knowledge
% of TeX increases, you will probably want to edit this, but
% it should be fine as is for beginners.

% almost certainly you want these
\usepackage{amssymb,amscd}
\usepackage{amsmath}
\usepackage{amsfonts}

% used for TeXing text within eps files
%\usepackage{psfrag}
% need this for including graphics (\includegraphics)
%\usepackage{graphicx}
% for neatly defining theorems and propositions
%\usepackage{amsthm}
% making logically defined graphics
%%\usepackage{xypic}

% there are many more packages, add them here as you need them

% define commands here
\begin{document}
Let $\mathcal{A}$ be an abelian category.  Let $M\in\operatorname{Ob}(\mathcal{A})$.  If there is an exact sequence
$$\xymatrix{0\ar[r]&S\ar[r]&P_n\ar[r]&{\cdots}\ar[r]&P_1\ar[r]&P_0\ar[r]&M\ar[r]&0},$$
where each $P_i$ is a projective object in $\mathcal{A}$, then we call $S$ an $n$th \emph{syzygy} of $M$.

If $S$ is itself projective, then the projective dimension of $M$, $\operatorname{pd}(M)$, is less than or equal to $n$.

\textbf{Remark}.  
\begin{enumerate}
\item The word ``syzygy'' is used in astromony to describe three celestial objects (usually the Sun, Moon and Earth, or the Sun, Earth and Moon) being collinear. 
\item Any two $n$th syzygies of a given object are projectively equivalent.
\end{enumerate}
%%%%%
%%%%%
\end{document}
