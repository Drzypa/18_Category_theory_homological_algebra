\documentclass[12pt]{article}
\usepackage{pmmeta}
\pmcanonicalname{GeneralizedSchanuelsLemma}
\pmcreated{2013-03-22 17:38:53}
\pmmodified{2013-03-22 17:38:53}
\pmowner{CWoo}{3771}
\pmmodifier{CWoo}{3771}
\pmtitle{generalized Schanuel's lemma}
\pmrecord{12}{40079}
\pmprivacy{1}
\pmauthor{CWoo}{3771}
\pmtype{Theorem}
\pmcomment{trigger rebuild}
\pmclassification{msc}{18G10}
\pmclassification{msc}{18G20}
\pmclassification{msc}{16E10}

\endmetadata

\usepackage{amssymb,amscd}
\usepackage{amsmath}
\usepackage{amsfonts}
\usepackage{mathrsfs}

% for neatly defining theorems and propositions
\usepackage{amsthm}
% making logically defined graphics
%%\usepackage{xypic}

% define commands here
\newcommand*{\abs}[1]{\left\lvert #1\right\rvert}
\newtheorem{prop}{Proposition}
\newtheorem{thm}{Theorem}
\newtheorem{ex}{Example}
\newcommand{\real}{\mathbb{R}}
\newcommand{\pdiff}[2]{\frac{\partial #1}{\partial #2}}
\newcommand{\mpdiff}[3]{\frac{\partial^#1 #2}{\partial #3^#1}}
\newcommand{\kernel}{\operatorname{Ker}}
\newcommand{\image}{\operatorname{Im}}
\begin{document}
\begin{thm} Let $R$ be a ring, $A$ and $B$ are projectively equivalent $R$-modules with two projective resolutions:
\begin{center}
$P:\xymatrix{\ldots \ar[r]^{\alpha_2}&P_1\ar[r]^{\alpha_1}&P_0\ar[r]^{\alpha_0}&A\ar[r]&0}$
\end{center}
\begin{center}
$Q:\xymatrix{\ldots \ar[r]^{\beta_2}&Q_1\ar[r]^{\beta_1}&Q_0\ar[r]^{\beta_0}&B\ar[r]&0}$
\end{center}
Define $[P,Q]_n:=P_n\oplus Q_{n-1}\oplus P_{n-2}\oplus Q_{n-3}\oplus\cdots$.  Then $$\kernel(\alpha_n)\oplus[Q,P]_n\cong\kernel(\beta_n)\oplus[P,Q]_n,$$
for all $n\geq0$.  In other words, $\kernel(\alpha_n)$ is projective equivalent to $\kernel(\beta_n)$.
\end{thm}

\begin{proof}
We proceed by induction on $n$.  If $n=0$, then $[P,Q]_0=P_0$ and $[Q,P]_0=Q_0$.  We have the following two short exact sequences:
\begin{center}
$\xymatrix{0\ar[r]&{\kernel(P_0\rightarrow A)}\ar[r]&P_0\ar[r]&A\ar[r]&0}$
\end{center}
\begin{center}
$\xymatrix{0\ar[r]&{\kernel(Q_0\rightarrow B)}\ar[r]&Q_0\ar[r]&B\ar[r]&0}$
\end{center}
Therefore, by Schanuel's Lemma,
\begin{eqnarray*}
\kernel(\alpha_0)\oplus[Q,P]_0&=&\kernel(\alpha_0)\oplus Q_0 \\
&=&\kernel(P_0\rightarrow A)\oplus Q_0 \\
&\cong& \kernel(Q_0\rightarrow B)\oplus P_0 \\
&=&\kernel(\beta_0)\oplus P_0 \\
&=&\kernel(\beta_0)\oplus[P,Q]_0,
\end{eqnarray*}
since $A$ and $B$ are projectively equivalent.

Next, assume the induction step is true for $n=k$.  To begin, we first notice two short exact sequences:
\begin{equation}
\xymatrix{0\ar[r]&{\kernel(\alpha_{k+1})}\ar[r]&P_{k+1}\ar[r]&{\image(\alpha_{k+1})}\ar[r]&0}
\end{equation}
\begin{equation}
\xymatrix{0\ar[r]&{\kernel(\beta_{k+1})}\ar[r]&Q_{k+1}\ar[r]&{\image(\beta_{k+1})}\ar[r]&0}
\end{equation}
Since $P$ and $Q$ are exact, $\image(\alpha_{k+1})=\kernel(\alpha_k)$ and $\image(\beta_{k+1})=\kernel(\beta_k)$ and the
two short exact sequences (4) and (5) can be rewritten as:
\begin{equation}
\xymatrix{0\ar[r]&{\kernel(\alpha_{k+1})}\ar[r]&P_{k+1}\ar[r]&{\kernel(\alpha_k)}\ar[r]&0}
\end{equation}
\begin{equation}
\xymatrix{0\ar[r]&{\kernel(\beta_{k+1})}\ar[r]&Q_{k+1}\ar[r]&{\kernel(\beta_k)}\ar[r]&0}
\end{equation}
Taking direct sums of the above modules with appropriate $[,]_k$'s, we now have two further short exact sequences:
\begin{center}
$\xymatrix{0\ar[r]&{\kernel(\alpha_{k+1})\oplus[0,0]_k}\ar[r]&
{P_{k+1}\oplus[Q,P]_k}\ar[r]&
{\kernel(\alpha_k)\oplus[Q,P]_k}\ar[r]&0}$
\end{center}
\begin{center}
$\xymatrix{0\ar[r]&{\kernel(\beta_{k+1})\oplus[0,0]_k}\ar[r]&
{Q_{k+1}\oplus[P,Q]_k}\ar[r]&
{\kernel(\beta_k)\oplus[P,Q]_k}\ar[r]&0}$
\end{center}
where the $0$ in $[0,0]_k$ denotes the zero resolution.  First,
$$\kernel(\alpha_{k+1})\oplus[0,0]_k\cong\kernel(\alpha_{k+1})\mbox{ and }
\kernel(\beta_{k+1})\oplus[0,0]_k\cong\kernel(\beta_{k+1}).$$  Second,
$$P_{k+1}\oplus[Q,P]_k=[P,Q]_{k+1}\mbox{ and }Q_{k+1}\oplus[P,Q]_k=[Q,P]_{k+1}$$ are both projective.  Finally,
$$\kernel(\alpha_k)\oplus[Q,P]_k\cong\kernel(\beta_k)\oplus[P,Q]_k$$ by the induction hypothesis.  Hence, again, by
Schanuel's Lemma,
\begin{eqnarray*}
\kernel(\alpha_{k+1})\oplus[Q,P]_{k+1}&\cong&\kernel(\alpha_{k+1})\oplus[0,0]_k\oplus[Q,P]_{k+1}\\
&=&\kernel(\alpha_{k+1})\oplus[0,0]_k\oplus Q_{k+1}\oplus[P,Q]_k\\
&\cong&\kernel(\beta_{k+1})\oplus[0,0]_k\oplus P_{k+1}\oplus[Q,P]_k\\
&=&\kernel(\beta_{k+1})\oplus[0,0]_k\oplus[P,Q]_{k+1}\\
&\cong&\kernel(\beta_{k+1})\oplus[P,Q]_{k+1}
\end{eqnarray*}
\end{proof}
%%%%%
%%%%%
\end{document}
