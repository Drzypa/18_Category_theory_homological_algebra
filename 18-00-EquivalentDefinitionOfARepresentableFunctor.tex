\documentclass[12pt]{article}
\usepackage{pmmeta}
\pmcanonicalname{EquivalentDefinitionOfARepresentableFunctor}
\pmcreated{2013-03-22 15:49:18}
\pmmodified{2013-03-22 15:49:18}
\pmowner{CWoo}{3771}
\pmmodifier{CWoo}{3771}
\pmtitle{equivalent definition of a representable functor}
\pmrecord{9}{37788}
\pmprivacy{1}
\pmauthor{CWoo}{3771}
\pmtype{Result}
\pmcomment{trigger rebuild}
\pmclassification{msc}{18-00}

\endmetadata

% this is the default PlanetMath preamble.  as your knowledge
% of TeX increases, you will probably want to edit this, but
% it should be fine as is for beginners.

% almost certainly you want these
\usepackage{amssymb}
\usepackage{amsmath}
\usepackage{amsfonts}

% used for TeXing text within eps files
%\usepackage{psfrag}
% need this for including graphics (\includegraphics)
%\usepackage{graphicx}
% for neatly defining theorems and propositions
%\usepackage{amsthm}
% making logically defined graphics
%%%\usepackage{xypic}

% there are many more packages, add them here as you need them

% define commands here
\begin{document}
We provide an equivalent, motivating, way of defining a representable functor.

Let $\mathcal{C}$ be a category and $F : \mathcal{C} \rightarrow Set$ be a covariant functor and $A \in \mathcal{C}$. Then the following are equivalent

\begin{enumerate}
\item $\mathcal{C}(A,-)$ is naturally isomorphic to $F$ (or, isomorphic in the appropriate category of functors)
\item There exists an element $i \in F(A)$ such that for every $B \in \mathcal{C}, r \in F(B)$ there exists a unique $f \in \mathcal{C}(A,B)$ such that $F(f)(i) = r$
\end{enumerate}


To illustrate the significance of this, consider the category $\mathcal{C} = \bf{Vect}_k$ of vector spaces over a field $k$. For arbitrary vector spaces $V, W$ consider the functor $F : \mathcal{C} \rightarrow Set$ determined by

$$ F(U) = \operatorname{Bilin}(V \times W, U) $$

Where this denotes the set of maps which are linear in both entries. This is a covariant functor in the obvious way. Then one may define $V \otimes W$ as the object which represents $F$ (if it exists). The significance of the result is it shows this is equivalent to the 'usual' definition: there is a bilinear map $i : V \times W \rightarrow V \otimes W$ through which all bilinear maps from $V \times W$ (these are quantified by r in the theorem) factor uniquely. This is because $r : V \times W \rightarrow U$ factors through $i$ exactly when there is an $f \in \mathcal{C}(V \otimes W, U)$ such that $F(f)(i) = r$.

Such universal constructions can be shown to be functorial in the basic objects. For instance the tensor product may be shown to be a functor

$$ {\bf{Vect}_k} \times {\bf{Vect}_k} \rightarrow {\bf{Vect}_k} $$

To generalise this suppose that $\mathcal{D}$ is a category (roughly representing ${\bf{Vect}_k} \times {\bf{Vect}_k}$ in our case) and we have a functor

$$ F : \mathcal{D}^{op} \times \mathcal{C} \rightarrow Set $$

such that $F(d,-) : \mathcal{C} \rightarrow Set$ is isomorphic to $\mathcal{C}(G(d),-)$ for some object $G(d)$. Then one may show that $G$ extends to a functor in such a way that $F(-,-)$ is naturally isomorphic to $\mathcal{C}(G(-),-)$.

We may show further that if $F,F^\prime$ are isomorphic functors and $G,G^\prime$ are functors which represent them respectively, then there is a natural isomorphism between $G$ and $G^\prime$.
%%%%%
%%%%%
\end{document}
