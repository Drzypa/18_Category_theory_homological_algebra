\documentclass[12pt]{article}
\usepackage{pmmeta}
\pmcanonicalname{CohomologyOfACochainComplex}
\pmcreated{2013-03-22 19:03:46}
\pmmodified{2013-03-22 19:03:46}
\pmowner{rm50}{10146}
\pmmodifier{rm50}{10146}
\pmtitle{cohomology of a cochain complex}
\pmrecord{6}{41946}
\pmprivacy{1}
\pmauthor{rm50}{10146}
\pmtype{Definition}
\pmcomment{trigger rebuild}
\pmclassification{msc}{18G35}
\pmrelated{Tor}

\endmetadata

\usepackage{amssymb}
\usepackage{amsmath}
\usepackage{amsfonts}

% used for TeXing text within eps files
%\usepackage{psfrag}
% need this for including graphics (\includegraphics)
%\usepackage{graphicx}
% for neatly defining theorems and propositions
%\usepackage{amsthm}
% making logically defined graphics
%%%\usepackage{xypic}

% there are many more packages, add them here as you need them

% define commands here
\DeclareMathOperator{\im}{im}
\begin{document}
\PMlinkescapeword{exact}
\PMlinkescapeword{objects}

If $(\mathcal{A},d)$ is a \PMlinkname{cochain complex}{CochainComplex2}
\[
  \cdots \xrightarrow{d_{n-1}} A^{n-1} \xrightarrow{d_{n}} A^n \xrightarrow {d_{n+1}}  
      A^{n+1} \xrightarrow{d_{n+2}} \cdots
\]
then the $n^{\mathrm{th}}$ \emph{cohomology group} (or \emph{cohomology module})
$H^n(\mathcal{A},d)$ of $(\mathcal{A},d)$
is the quotient module
\[
 H^n(\mathcal{A},d)=\frac{\ker d_{n+1}}{\im d_n}.
\]

The cochain complex is an \PMlinkname{exact sequence}{ExactSequence} if and only if
all of the cohomology groups are trivial.
The cohomology groups can therefore be thought of
as measuring the extent to which the cochain complex fails to be exact.

Cohomology groups of other objects are defined as the cohomology groups of an associated cochain complex. (For example, see the entry on the \PMlinkname{cohomology of simplicial complexes}{SimplicialComplex}.)

[Compare this entry with the entry on homology of a chain complex.]
%%%%%
%%%%%
\end{document}
