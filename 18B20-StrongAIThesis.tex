\documentclass[12pt]{article}
\usepackage{pmmeta}
\pmcanonicalname{StrongAIThesis}
\pmcreated{2013-03-22 17:07:11}
\pmmodified{2013-03-22 17:07:11}
\pmowner{dankomed}{17058}
\pmmodifier{dankomed}{17058}
\pmtitle{strong AI thesis}
\pmrecord{23}{39422}
\pmprivacy{1}
\pmauthor{dankomed}{17058}
\pmtype{Topic}
\pmcomment{trigger rebuild}
\pmclassification{msc}{18B20}
\pmclassification{msc}{68T40}
\pmclassification{msc}{68T27}
\pmclassification{msc}{03D05}
\pmclassification{msc}{03D80}
\pmclassification{msc}{68T01}
\pmsynonym{strong artificial intelligence thesis}{StrongAIThesis}
%\pmkeywords{artificial intelligence}
%\pmkeywords{automaton}
%\pmkeywords{weak AI thesis}
%\pmkeywords{computer simulation}
%\pmkeywords{neural networks}
%\pmkeywords{G\"odel's incompleteness theorem}
\pmrelated{WeakAIThesis}
\pmrelated{ArtificialInteglligence}
\pmrelated{Automaton}
\pmrelated{UniversalTuringMachine}
\pmrelated{UltraComplexSystems}
\pmrelated{HighlyComplexSystems}
\pmdefines{consistent algorithm}

\endmetadata

% this is the default PlanetMath preamble.  as your knowledge
% of TeX increases, you will probably want to edit this, but
% it should be fine as is for beginners.

% almost certainly you want these
\usepackage{amssymb}
\usepackage{amsmath}
\usepackage{amsfonts}

% used for TeXing text within eps files
%\usepackage{psfrag}
% need this for including graphics (\includegraphics)
%\usepackage{graphicx}
% for neatly defining theorems and propositions
%\usepackage{amsthm}
% making logically defined graphics
%%%\usepackage{xypic}

% there are many more packages, add them here as you need them

% define commands here

\begin{document}
\begin{verse}Strong AI (artificial intelligence) thesis: the \PMlinkname{Mind}{UltraComplexSystems} is assumed, or postulated, to be a consistent algorithm, and therefore if properly programmed, a digital computer can, in principle, mimick the mind, provided the basic assumption about the Mind is correct. A proper computer program can be intelligent, understand, perceive, have beliefs, learn, adapt, and exhibit other cognitive \PMlinkescapetext{states} normally ascribed to human mind.
\end{verse}

The idea that the algorithm should be consistent stems from earlier discussions by Turing, G\"odel, Church, Lucas, Penrose and others, and reflects the fact that the artificial mind must be able to do mathematics in order to be acknowledged to have status identical to that of a human mind. Since humans do not find the equation "0=1" to be true, then it is essential to be assumed that the algorithm that is claimed to simulate the human mind shall be consistent, because if it were inconsistent, it would be able to prove "0=1", something that human mathematicians don't accept. (Comment: from inconsistency follows everything, therefore in inconsistent formal system (algorithm) every formula will be provable).

According to Penrose's second G\"odelian argument, strong AI is false, because if we were minds that are consistent algorithms we would not be able to know that. Instead when being asked about our own consistency we would have entered in perpetual looping.

Strong AI advocates usually drop the consistency requirement from the definition of the strong AI thesis (Mooney, 1999), however this is of no great benefit, because taken together Penrose's second G\"odelian argument and the "weakened" strong AI thesis will imply that we are inconsistent algorithms. Some authors readily accept this latter claim (Makey, 1995), yet it is highly surprizing that intelligence benefits from making errors, or more explictly that to be mind requires to be inconsistent at first \PMlinkescapetext{place}.

According to minority of strong AI advocates, the thesis of consistency is not part of the proper definition of the strong AI thesis. Yet, it seems futile to keep the name "strong AI thesis" for thesis which is actually "weakened" by dropping the consistency requirement. Even if this part of the definition is omitted, then it will necessarily follow that we are inconsistent algorithms. One cannot even entertain the possibility that "human minds could be in principle consistent algorithms, but we cannot know that", simply because consistent algorithms do no not provide answer "I don't know" when being asked about their own consistency, instead consistent algorithms enter into perpetual looping, something that humans don't do.

Additional caution can be paid to the usage of phrases with undefined meaning such as "if properly programmed", because this hardly makes the thesis "strong". It is exactly on the contrary, such a thesis is immunized in Popper's sense, and is so weak that it cannot be even in principle disproved - simply every assumption on the programming that leads to conclusion "not mind", could be met by objection "the considered programming is not properly chosen". Indeed if the notion of what the "proper programming should be" is not constructible at first place, it is inconsistent to define the mind as "algorithm".

From the preceding discussion it follows that the strong AI thesis is false, and artificial intelligence research should be developed from consistent but weaker AI thesis alternatives.

References

1. Makey (1995) \PMlinkexternal{G\"odel's Incompleteness Theorem is Not an Obstacle to Artificial Intelligence.}{http://www.sdsc.edu/~jeff/Godel\%5Fvs\%5FAI.html}

2. Mooney RJ (1999) \PMlinkexternal{Philosophical Arguments Against AI.}{http://www.cs.utexas.edu/~mooney/cs343/slide-handouts/philosophy.4.pdf}

3. Searle JR. (1980) \PMlinkexternal{Minds, brains, and programs. Behavioral and Brain Sciences 3(3): 417-457.}{http://members.aol.com/NeoNoetics/MindsBrainsPrograms.html}

4. Baianu, I.C. 2004. Computer Models and Automata Theory in Biology and Medicine: Computer Simulation and 
Computability of Biological Systems:                           
\PMlinkexternal{An Updated PDF download}{http://aux.planetmath.org/files/lec/70/COMPUTER_SIMULATIONCOMPUTABILITYBIOSYSTEMSrefnew.pdf}
\\
in \emph{Mathematical Modelling}, Vol. {\bf 7}, pp. 1513-1577. (p.1 of the Updated doc. file: ``{\em A. Are biological systems recursively computable ?}''). 
%%%%%
%%%%%
\end{document}
